\PassOptionsToPackage{usenames}{color}
\pdfoutput=1 % ensure pdflatex (for arXiv)
\documentclass[11pt,letterpaper]{article}
\usepackage{comment}
\usepackage{relsize} % relative font sizes (e.g. \smaller). must precede ACL style
%\usepackage{style/acl2012}

\usepackage{hyperref}

\usepackage[round]{natbib}
\begin{comment}
\usepackage[style=authoryear-comp,natbib=true,hyperref=true]{biblatex}

% tell biblatex not to quote titles in the bibliography
\DeclareFieldFormat{title}{#1} % don't italicize titles by default
\DeclareFieldFormat[book]{title}{\mkbibemph{#1}\isdot} % but do italicize books
\DeclareFieldFormat[article]{title}{#1\isdot}
\DeclareFieldFormat[inbook]{title}{#1\isdot}
\DeclareFieldFormat[incollection]{title}{#1\isdot}
\DeclareFieldFormat[inproceedings]{title}{#1\isdot}
\DeclareFieldFormat[patent]{title}{#1\isdot}
\DeclareFieldFormat[thesis]{title}{#1\isdot}
\DeclareFieldFormat[unpublished]{title}{#1\isdot}
% ...and for articles, to use number(issue) instead of number.issue
\renewbibmacro*{journal+issuetitle}{%
  \usebibmacro{journal}%
  \setunit*{\addspace}%
  \iffieldundef{series}
    {}
    {\newunit
     \printfield{series}%
     \setunit{\addspace}}%
  \printfield{volume}%
%  \setunit*{\adddot}%
  \printfield{number}%
  \setunit{\addcomma\space}%
  \printfield{eid}%
  \setunit{\addspace}%
  \usebibmacro{issue+date}%
  \newunit\newblock
  \usebibmacro{issue}%
  \newunit}
\DeclareFieldFormat[article]{number}{\mkbibparens{#1}}
% use ``pages'' instead of ``pp.''
\DefineBibliographyStrings{english}{%
    pages  =  {pages} % for multiple page numbers
}
% but for articles, just use a colon
\DeclareFieldFormat[article]{pages}{:#1} %TODO. check with LeCun citation
% and don't put a colon after In
\renewbibmacro*{in:}{%
  \bibstring{in}%\addcolon
  \setunit{\space}}

\DeclareFieldFormat{label}{#1\isdot}
\renewbibmacro*{year+labelyear}{%
  \iffieldundef{year}
    {}
    {\printtext{%
       \printfield{year}%
       \printfield{labelyear}%
       \setunit{\adddot}}}}

\bibliography{features.bib}
\end{comment}


%\usepackage{times}
%\usepackage{latexsym}


\usepackage[boxed]{algorithm2e}
\renewcommand\AlCapFnt{\small}
\usepackage[small,bf,skip=5pt]{caption}
\usepackage{sidecap} % side captions
\usepackage{rotating}	% sideways

% customize \paragraph spacing
\makeatletter
\renewcommand{\paragraph}{%
  \@startsection{paragraph}{4}%
  {\z@}{3.25ex \@plus 1ex \@minus .2ex}{-1em}% reduce 3.25 to .2 to minimize space
  {\normalfont\normalsize\bfseries}%
}
\makeatother

% Italicize subparagraph headings
\usepackage[nobottomtitles*]{titlesec}
\titleformat*{\subparagraph}{\itshape}
\titlespacing{\subparagraph}{%
  1em}{%              left margin
  0pt}{% space before (vertical)
  1em}%               space after (horizontal)


% MOVE SECTION NUMBERS INTO LEFT MARGIN
% They will be roman, right-aligned and separated from the heading text by .2cm
% adapted from http://tex.stackexchange.com/a/311712
\titleformat{\section}[block]
  {\Large\bfseries}
  {}
  {0pt}
  {\hspace{-1.2cm}% Move into margin
   \makebox[1cm][r]{\normalfont\thesection}\hspace{.2cm}}% Set number + title
\titleformat{\subsection}[block]
  {\large\bfseries}
  {}
  {0pt}
  {\hspace{-1.2cm}% Move into margin
   \makebox[1cm][r]{\normalfont\thesubsection}\hspace{.2cm}}% Set number + title
\titleformat{\subsubsection}[block]
  {\normalsize\bfseries}
  {}
  {0pt}
  {\hspace{-1.2cm}% Move into margin
   \makebox[1cm][r]{\normalfont\thesubsubsection}\hspace{.2cm}}% Set number + title

%\usepackage{lingmacros}
% Lists

\usepackage{enumitem} % customizable lists
\setitemize{noitemsep,topsep=0em} %,leftmargin=*
\setenumerate{noitemsep,leftmargin=0em,itemindent=13pt,topsep=0em}

\usepackage{adjustbox}
\newcommand{\choices}[1]{\adjustbox{stack=ct}{#1}}  % for placing alternatives 
% inline with an example. e.g. \choices{to\\from\\with}
% ct = horizontally-centered, top

\usepackage{textcomp}
% \usepackage{arabtex} % must go after xparse, if xparse is used!
%\usepackage{utf8}
% \setcode{utf8} % use UTF-8 Arabic
% \newcommand{\Ar}[1]{\RL{\novocalize #1}} % Arabic text

\usepackage[procnames]{listings}

\usepackage{amssymb}	%amsfonts,eucal,amsbsy,amsthm,amsopn
\usepackage{amsmath}

%\usepackage{mathptmx}	% txfonts
\usepackage{fourier}
\usepackage[scaled=.87]{helvet}
\usepackage[scaled=.8]{beramono}
\usepackage[T1]{fontenc}
\usepackage[utf8x]{inputenc}

\usepackage{MnSymbol}	% must be after mathptmx

\usepackage{latexsym}





% Tables
\usepackage{array}
\usepackage{multirow}
\usepackage{booktabs} % pretty tables
\usepackage{multicol}
\usepackage{footnote}


\usepackage{url}
\usepackage[usenames]{color}
\usepackage{xcolor}

\definecolor{darkblue}{rgb}{0, 0, 0.5}
\hypersetup{colorlinks=true,citecolor=darkblue, linkcolor=., urlcolor=darkblue}

% colored frame box
\newcommand{\cfbox}[2]{%
    \colorlet{currentcolor}{.}%
    {\color{#1}%
    \fbox{\color{currentcolor}#2}}%
}

\usepackage[normalem]{ulem} % \uline
\usepackage{colortbl}
\usepackage{graphicx}
\usepackage{subcaption}
\usepackage{mdframed}

%\usepackage{tikz-dependency}
\usepackage{tikz}
\usepackage[edges]{forest}
%\usepackage{tree-dvips}
\usetikzlibrary{arrows,positioning,calc} 

\DeclareMathOperator*{\argmax}{arg\,max}
\DeclareMathOperator*{\argmin}{arg\,min}



% Author comments
\usepackage{color}
\newcommand\bmmax{0} % magic to avoid 'too many math alphabets' error
\usepackage{bm}
\definecolor{orange}{rgb}{1,0.5,0}
\definecolor{mdgreen}{rgb}{0,0.6,0}
\definecolor{mdblue}{rgb}{0,0,0.7}
\definecolor{dkblue}{rgb}{0,0,0.5}
\definecolor{dkgray}{rgb}{0.3,0.3,0.3}
\definecolor{slate}{rgb}{0.25,0.25,0.4}
\definecolor{gray}{rgb}{0.5,0.5,0.5}
\definecolor{ltgray}{rgb}{0.7,0.7,0.7}
\definecolor{ltltgray}{rgb}{0.9,0.9,0.9}
\definecolor{purple}{rgb}{0.7,0,1.0}
\definecolor{lavender}{rgb}{0.65,0.55,1.0}

% Settings for algorithm listings
\makeatletter
\lst@AddToHook{EveryPar}{%
  \label{lst:\thelstnumber}% make a label for each line number except the first (assumes only one listing in the document)
}
\makeatother
\lstset{
% basicstyle=\rmshape,
  numbers=left,
  numberstyle=\tt\color{gray},
  firstnumber=2,
  stepnumber=5,
  xleftmargin=3em,
  language=Python,
  upquote=true,
  showstringspaces=false,
  formfeed=\newpage,
  tabsize=1,
  stringstyle=\color{mdgreen},
  commentstyle=\itshape\color{lavender},
  basicstyle=\small\smaller\ttfamily,
  morekeywords={lambda,with,as,assert},
  keywordstyle=\bfseries\color{magenta},
  procnamekeys={def},
  procnamestyle=\bfseries\color{orange},
  aboveskip=0.5cm,
  belowskip=0.5cm
}
\renewcommand{\lstlistingname}{Algorithm}


\newcommand{\ensuretext}[1]{#1}
\newcommand{\cjdmarker}{\ensuretext{\textcolor{green}{\ensuremath{^{\textsc{CJ}}_{\textsc{D}}}}}}
\newcommand{\nssmarker}{\ensuretext{\textcolor{magenta}{\ensuremath{^{\textsc{NS}}_{\textsc{S}}}}}}
\newcommand{\nasmarker}{\ensuretext{\textcolor{red}{\ensuremath{^{\textsc{NA}}_{\textsc{S}}}}}}
\newcommand{\jbmarker}{\ensuretext{\textcolor{orange}{\ensuremath{^{\textsc{J}}_{\textsc{B}}}}}}
\newcommand{\abmarker}{\ensuretext{\textcolor{purple}{\ensuremath{^{\textsc{A}}_{\textsc{B}}}}}}
\newcommand{\jhmarker}{\ensuretext{\textcolor{cyan}{\ensuremath{^{\textsc{JD}}_{\textsc{H}}}}}}
\newcommand{\ajbmarker}{\ensuretext{\textcolor{blue}{\ensuremath{^{\textsc{AJ}}_{\textsc{B}}}}}}
\newcommand{\arkcomment}[3]{\ensuretext{\textcolor{#3}{[#1 #2]}}}
%\newcommand{\arkcomment}[3]{}
\newcommand{\cjd}[1]{\arkcomment{\cjdmarker}{#1}{green}}
\newcommand{\nss}[1]{\arkcomment{\nssmarker}{#1}{magenta}}
\newcommand{\nas}[1]{\arkcomment{\nasmarker}{#1}{red}}
\newcommand{\jb}[1]{\arkcomment{\jbmarker}{#1}{orange}}
\newcommand{\jh}[1]{\arkcomment{\jhmarker}{#1}{cyan}}
\newcommand{\ab}[1]{\arkcomment{\abmarker}{#1}{purple}}
\newcommand{\ajb}[1]{\arkcomment{\ajbmarker}{#1}{blue}}
\newcommand{\params}{\mathbf{\theta}}
\newcommand{\wts}{\mathbf{w}}
\newcommand{\g}{\mathbf{g}}
\newcommand{\f}{\mathbf{f}}
\newcommand{\x}{\mathbf{x}}
\newcommand{\y}{\mathbf{y}}
\newcommand{\overbar}[1]{\mkern 1.5mu\overline{\mkern-1.5mu#1\mkern-1.5mu}\mkern 1.5mu} % \bar is too narrow in math
\newcommand{\cost}{c}

\newcommand{\citeposs}[2][]{\citeauthor{#2}'s (\citeyear[#1]{#2})}
\newcommand{\Citeposs}[2][]{\Citeauthor{#2}'s (\citeyear[#1]{#2})}

\usepackage{nameref}
\usepackage{cleveref}

% use \S for all references to all kinds of sections, and \P to paragraphs
% (sadly, we cannot use the simpler \crefname{} macro because it would insert a space after the symbol)
\crefformat{part}{\S#2#1#3}
\crefformat{chapter}{\S#2#1#3}
\crefformat{section}{\S#2#1#3}
\crefformat{subsection}{\S#2#1#3}
\crefformat{subsubsection}{\S#2#1#3}
\crefformat{paragraph}{\P#2#1#3}
\crefformat{subparagraph}{\P#2#1#3}
%\crefmultiformat{part}{\S#2#1#3}{ and~\S#2#1#3}{, \S#2#1#3}{, and~\S#2#1#3}
%\crefmultiformat{chapter}{\S#2#1#3}{ and~\S#2#1#3}{, \S#2#1#3}{, and~\S#2#1#3}
\crefmultiformat{section}{\S#2#1#3}{ and~\S#2#1#3}{, \S#2#1#3}{, and~\S#2#1#3}
\crefmultiformat{subsection}{\S#2#1#3}{ and~\S#2#1#3}{, \S#2#1#3}{, and~\S#2#1#3}
\crefmultiformat{subsubsection}{\S#2#1#3}{ and~\S#2#1#3}{, \S#2#1#3}{, and~\S#2#1#3}
\crefmultiformat{paragraph}{\P\P#2#1#3}{ and~#2#1#3}{, #2#1#3}{, and~#2#1#3}
\crefmultiformat{subparagraph}{\P\P#2#1#3}{ and~#2#1#3}{, #2#1#3}{, and~#2#1#3}
%\crefrangeformat{part}{\mbox{\S\S#3#1#4--#5#2#6}}
%\crefrangeformat{chapter}{\mbox{\S\S#3#1#4--#5#2#6}}
\crefrangeformat{section}{\mbox{\S\S#3#1#4--#5#2#6}}
\crefrangeformat{subsection}{\mbox{\S\S#3#1#4--#5#2#6}}
\crefrangeformat{subsubsection}{\mbox{\S\S#3#1#4--#5#2#6}}
\crefrangeformat{paragraph}{\mbox{\P\P#3#1#4--#5#2#6}}
\crefrangeformat{subparagraph}{\mbox{\P\P#3#1#4--#5#2#6}}
% for \label[appsec]{...}
\crefname{part}{Part}{Parts}
\Crefname{part}{Part}{Parts}
\crefname{chapter}{ch.}{ch.}
\Crefname{chapter}{Ch.}{Ch.}
\crefname{figure}{figure}{figures}
\crefname{subfigure}{figure}{figures}
\Crefname{subfigure}{Figure}{Figures}
\crefname{appsec}{appendix}{appendices}
\Crefname{appsec}{Appendix}{Appendices}
\crefname{algocf}{algorithm}{algorithms}
\Crefname{algocf}{Algorithm}{Algorithms}
\crefname{enums}{example}{examples}
\Crefname{enums}{Example}{Examples}
\crefname{enumsi}{example}{examples}
\Crefname{enumsi}{Example}{Examples}
\crefname{}{example}{examples} % lingmacros \toplabel has no internal name for the kind of label
\Crefname{}{Example}{Examples}
\crefformat{enums}{(#2#1#3)}
\crefformat{enumsi}{(#2#1#3)}
\crefformat{}{(#2#1#3)}
\crefname{xnumi}{example}{examples} % gb4e
\crefname{xnumi}{example}{examples} % gb4e
\Crefname{xnumii}{Example}{Examples} % gb4e
\Crefname{xnumii}{Example}{Examples} % gb4e
\crefformat{xnumi}{(#2#1#3)} % gb4e
\crefformat{xnumii}{(#2#1#3)} % gb4e
\crefrangeformat{enums}{\mbox{(#3#1#4--#5#2#6)}}
\crefrangeformat{enumsi}{\mbox{(#3#1#4--#5#2#6)}}
\crefrangeformat{xnumi}{\mbox{(#3#1#4--#5#2#6)}} % gb4e
\crefrangeformat{xnumii}{\mbox{(#3#1#4--#5#2#6)}} % gb4e

\ifx\creflastconjunction\undefined%
\newcommand{\creflastconjunction}{, and\nobreakspace} % Oxford comma for lists
\else%
\renewcommand{\creflastconjunction}{, and\nobreakspace} % Oxford comma for lists
\fi%

\newcommand*{\Fullref}[1]{\hyperref[{#1}]{\Cref*{#1}: \nameref*{#1}}}
\newcommand*{\fullref}[1]{\hyperref[{#1}]{\cref*{#1}: \nameref{#1}}}
\newcommand{\fnref}[1]{fn.~\ref{#1}} % don't use \cref{} due to bug in (now out-of-date) cleveref package w.r.t. footnotes
\newcommand{\Fnref}[1]{Fn.~\ref{#1}}

%\captionsetup[subfigure]{labelformat=simple}
\renewcommand\thesubfigure{(\alph{subfigure})}

% Space savers
% From http://www.eng.cam.ac.uk/help/tpl/textprocessing/squeeze.html
%\addtolength{\dbltextfloatsep}{-.5cm} % space between last top float or first bottom float and the text.
%\addtolength{\intextsep}{-.5cm} % space left on top and bottom of an in-text float.
%\addtolength{\abovedisplayskip}{-.5cm} % space before maths
%\addtolength{\belowdisplayskip}{-.5cm} % space after maths
%\addtolength{\topsep}{-.5cm} %space between first item and preceding paragraph
%\setlength{\belowcaptionskip}{-.25cm}

\usepackage{gb4e} % linguistic examples. put after all other package imports






% Special macros
\newcommand{\w}[1]{\textit{#1}}	% word
\newcommand{\p}[1]{\textbf{\textsf{#1}}} % preposition type
\newcommand{\lbl}[1]{\textsc{#1}} % class label
\newcommand{\sst}[1]{\lbl{#1}} % supersense tag label
\newcommand{\nsst}[1]{\sst{n:#1}} % noun supersense tag label
\newcommand{\vsst}[1]{\sst{v:#1}} % verb supersense tag label
\newcommand{\psst}[1]{\textcolor{mdgreen}{\hyperref[sec:#1]{\sst{#1}}}} % preposition supersense tag label
\newcommand{\olbl}[1]{\textcolor{purple}{\textrm{#1}}} % other label: `i, `d, etc.
%\newcommand{\nsst}[1]{\sst{#1~\textroundcap{\vphantom{-}}~}} % noun supersense tag label
%\newcommand{\vsst}[1]{\sst{#1\raisebox{-1.5pt}{\textasciicaron}}} % verb supersense tag label
%\newcommand{\psst}[1]{\sst{#1\raisebox{2pt}{\rotatebox{180}{\textsublhalfring{\phantom{.}}}}}} %\textcorner % preposition supersense tag label

\newcommand{\rf}[2]{\psst{#1}$\leadsto$\psst{#2}}
\newcommand{\rff}[3]{\psst{#1}$\leadsto$\psst{#2}$\leadsto$\psst{#3}}


\newcommand{\tg}[1]{\texttt{#1}}	% supersense tag name
\newcommand{\gfl}[1]{%\renewcommand\texttildelow{{\lower.74ex\hbox{\texttt{\char`\~}}}} % http://latex.knobs-dials.com/
\mbox{\textsmaller{\texttt{#1}}}}	% supersense tag symbol
 \newcommand{\tagdef}[1]{#1\hfill} % tag definition
\newcommand{\tagt}[2]{\ensuremath{\underset{\textrm{\textlarger{\tg{#2}}}\strut}{\w{#1}\rule[-.3\baselineskip]{0pt}{0pt}}}} % tag text (a word or phrase) with an SST. (second arg is the tag)
\newcommand{\glosst}[2]{\ensuremath{\underset{\textrm{#2}}{\textrm{#1}}}} % gloss text (a word or phrase) (second arg is the gloss)
\newcommand{\AnnA}[0]{\mbox{\textbf{Ann-A}}} % annotator A
\newcommand{\AnnB}[0]{\mbox{\textbf{Ann-B}}} % annotator B
\newcommand{\sys}[1]{\mbox{\textbf{#1}}}   % name of a system (one of our experimental conditions)
\newcommand{\dataset}[1]{\mbox{\textsc{#1}}}	% one of the datasets in our experiments
\newcommand{\datasplit}[1]{\mbox{\textbf{#1}}}	% portion one of the datasets in our experiments

\newcommand{\fnf}[1]{\textsc{\textsf{#1}}} % FrameNet frame
\newcommand{\fnr}[1]{\textbf{\textsf{#1}}} % FrameNet role (frame element name)
\newcommand{\fnrel}[1]{\textsl{#1}} % FrameNet frame relation type
\newcommand{\fnst}[1]{\textsl{#1}} % FrameNet semantic type
\newcommand{\fnlu}[1]{\textsf{#1}} % FrameNet lexical unit (predicate)
\newcommand{\pbf}[1]{\mbox{\textsf{#1}}} % PropBank frame (roleset)
\newcommand{\pbr}[1]{\textbf{\textsf{#1}}} % PropBank role (numbered or modifier argument label)
\newcommand{\vpred}[1]{\textbf{#1}} % verb predicate


%\newcommand{\lex}[1]{\textsmaller{\textsf{\textcolor{slate}{\textbf{#1}}}}}	% example lexical item
\newcommand{\lex}[1]{\textit{#1}} % lexical item/lexical example
\newcommand{\pex}[1]{\textit{#1}} % phrasal example - don't index by default

%\newcommand{\w}[1]{\textit{#1}}	% word
\newcommand{\gap}[0]{\ \ } % space around gap contents
\newcommand{\tat}[0]{\textasciitilde}

\newcommand{\shortlong}[2]{#1} % short vs. long version of the paper
\newcommand{\confversion}[1]{#1}
%\newcommand{\finalversion}[1]{#1}
\newcommand{\finalversion}[1]{}
\newcommand{\futureversion}[1]{}
\newcommand{\shortversion}[1]{#1}
\newcommand{\considercutting}[1]{#1}
\newcommand{\longversion}[1]{} % ...if only there were more space...
\newcommand{\subversion}[1]{#1} % for the submission version only
\newcommand{\draftnotice}[1]{} % for the draft version only
%\newcommand{\subversion}[1]{}

\newenvironment{ggroup}{{}}{{}}

\newcommand{\shortdef}[1]{\begin{mdframed}\noindent\textlarger{#1}\end{mdframed}}

\newenvironment{history}{\begin{mdframed}[linecolor=ltltgray,backgroundcolor=ltltgray]\small\noindent\textit{History.}}{\end{mdframed}}



\newcommand{\hierA}[1]{\textcolor{red}{\hyperref[sec:#1]{#1}}}
\newcommand{\hierB}[1]{\textcolor{blue}{\hyperref[sec:#1]{#1}}}
\newcommand{\hierC}[1]{\textcolor{mdgreen}{\hyperref[sec:#1]{#1}}}
\newcommand{\hierD}[1]{\textcolor{orange}{\hyperref[sec:#1]{#1}}}

\newcommand{\hierAdef}[1]{\section{\psst{#1}}\label{sec:#1}}
\newcommand{\hierBdef}[1]{\subsection{\psst{#1}}\label{sec:#1}}
\newcommand{\hierCdef}[1]{\subsubsection{\psst{#1}}\label{sec:#1}}
\newcommand{\hierDdef}[1]{\paragraph{\psst{#1}}\label{sec:#1}}

\hyphenation{WordNet}
\hyphenation{WordNets}
\hyphenation{FrameNet}
\hyphenation{SemCor}
\hyphenation{SemEval}
\hyphenation{ParsedSemCor}
\hyphenation{VerbNet}
\hyphenation{PennConverter}
\hyphenation{an-aly-sis}
\hyphenation{an-aly-ses}
\hyphenation{news-text}
\hyphenation{base-line}
\hyphenation{de-ve-lop-ed}
\hyphenation{comb-over}
\hyphenation{per-cept}
\hyphenation{per-cepts}
\hyphenation{post-edit-ing}
\hyphenation{shriv-eled}
\hyphenation{Huddle-ston}

\title{\draftnotice{{\it\small WORKING DRAFT}\\[5pt] 
}Adposition and Case Supersenses v2}

\newcommand{\emldisplay}[2]{\texttt{\href{mailto:#1}{#2}}}
\newcommand{\eml}[1]{\textsmaller{\emldisplay{#1}{#1}}}

\author{\hspace{.2cm}\textbf{Nathan Schneider}\hspace{.2cm} \\ 
  \hspace{.2cm}Georgetown University\hspace{.2cm} \\
     \hspace{.2cm}\eml{nathan.schneider@georgetown.edu}\hspace{.2cm} \and
\textbf{Jena D. Hwang} \quad
\textbf{Archna Bhatia} \\
 	IHMC \\
     {\smaller \{\emldisplay{jhwang@ihmc.us}{jhwang},\emldisplay{abhatia@ihmc.us}{abhatia}\}\texttt{@ihmc.us}} \and
\textbf{Na-Rae Han} \\
	\hspace{.75cm}University of Pittsburgh\hspace{.75cm} \\
    \eml{naraehan@pitt.edu} \and 
\textbf{Vivek Srikumar} \\
	\hspace{1.25cm}University of Utah\hspace{1.25cm} \\
    \eml{svivek@cs.utah.edu} \and
\textbf{Tim O'Gorman} \\
  \hspace{.1cm}University of Colorado Boulder\hspace{.1cm} \\
    \eml{timothy.ogorman@colorado.edu} \and
\textbf{Omri Abend} \\
  \hspace{.1cm}Hebrew University of Jerusalem\hspace{.1cm} \\
    \eml{oabend@cs.huji.ac.il} \and
\textbf{Austin Blodgett} \\
  Georgetown University \\
    \eml{ajb341@georgetown.edu}
}

\date{\nss{\today{} draft; insert arXiv submission date here}}

\begin{document}
\maketitle
\begin{abstract}
\noindent 
This document describes an inventory of 50~semantic labels 
designed to characterize the use of adpositions and case markers 
at a somewhat coarse level of granularity. 
Version~2 is a revision of the supersense inventory proposed for English by 
\citet{schneider-15,schneider-16} and documented in PrepWiki\footnote{\url{http://tiny.cc/prepwiki}} 
(henceforth ``v1''), which in turn was based on previous schemes.
The present inventory was developed after extensive review of the 
v1 corpus annotations for English, as well as consideration of adposition 
and case phenomena in Hebrew, Hindi, and Korean.
It also adds examples for English \p{'s} (genitive case) possessives, 
as described in \nss{\citet{blodgett-??}}.
Examples in this document are limited to English; 
a multilingual and more detailed online lexical resource is forthcoming.
\end{abstract}

\section{Overview}

\nss{v1 is documented in PrepWiki (\url{http://tiny.cc/prepwiki}).
v2 will be documented in Xposition (URL TBD).}

\subsection{What counts as an adposition?}

``Adposition'' is the cover term for prepositions and postpositions. 
Briefly, we consider an affix, word, or multiword expression to be an adposition if it:
\begin{itemize}
  \item Mediates a semantically asymmetric figure--ground relation between two concepts
  \item Is a grammatical item that can mark an NP, and in some cases may mark clauses (as a subordinator) 
  or be intransitive. 
  We also include always-intransitive grammatical items whose core meaning is spatial and highly schematic, 
  like English \p{together}, \p{apart}, and \p{away}.
  \item Is not a differential object marker (e.g., Hebrew \p{'et}, which marks direct objects if and only if 
  they are definite).
\end{itemize}
\ab{differential object markers are ignored because they don't have any lexical semantic content, is that the criterion?}

\nss{What about a word that matches the above criteria where it is used as an intransitive 
predicate, e.g. \pex{She is \p{out}/\p{away}}?}\ab{I think that is an adverbial usage instead of adpositional.}

\subsection{Inventory}

The v2 hierarchy is a tree with 50~labels.
They are organized into three major subhierarchies: 
\psst{Circumstance} (18~labels), \psst{Participant} (14~labels), 
and \psst{Configuration} (18~labels). 

\begin{minipage}{\textwidth}
\vspace{.2cm}
\begin{multicols}{3}
\begin{ggroup}
  \sffamily\color{gray}
\begin{forest}
  for tree={%
    folder,
    grow'=0,
    fit=band,
    inner ysep=.75,
  }
  [{\hierA{Circumstance}}
    [{\hierB{Temporal}}
      [{\hierC{Time}}
        [{\hierD{StartTime}}]
        [{\hierD{EndTime}}]
      ]
      [{\hierC{Frequency}}]
      [{\hierC{Duration}}]
      [{\hierC{Interval}}]
    ]
    [{\hierB{Locus}}
      [{\hierC{Source}}]
      [{\hierC{Goal}}]
    ]
    [{\hierB{Path}}
      [{\hierC{Direction}}]
      [{\hierC{Extent}}]
    ]
    [{\hierB{Means}}]
    [{\hierB{Manner}}]
    [{\hierB{Explanation}}
      [{\hierC{Purpose}}]
    ]
  ]
\end{forest}
\columnbreak

\begin{forest}
  for tree={%
    folder,
    grow'=0,
    fit=band,
    inner ysep=.75,
  }
  [{\hierA{Participant}}
    [{\hierB{Causer}}
      [{\hierC{Agent}}
        [{\hierD{Co-Agent}}]
      ]
    ]
    [{\hierB{Theme}}
      [{\hierC{Co-Theme}}]
      [{\hierC{Topic}}]
    ]
    [{\hierB{Stimulus}}]
    [{\hierB{Experiencer}}]
    [{\hierB{Originator}}]
    [{\hierB{Recipient}}]
    [{\hierB{Cost}}]
    [{\hierB{Beneficiary}}]
    [{\hierB{Instrument}}]
  ]
\end{forest}
\columnbreak

\begin{forest}
  for tree={%
    folder,
    grow'=0,
    fit=band,
    inner ysep=.75,
  }
  [{\hierA{Configuration}}
    [{\hierB{Identity}}]
    [{\hierB{Species}}]
    [{\hierB{Gestalt}}
      [{\hierC{Possessor}}]
      [{\hierC{Whole}}]
    ]
    [{\hierB{Characteristic}}
      [{\hierC{Possession}}]
      [{\hierC{Part/Portion}}
        [{\hierD{Stuff}}]
      ]
    ]
    [{\hierB{Accompanier}}]
    [{\hierB{InsteadOf}}]
    [{\hierB{ComparisonRef}}]
    [{\hierB{RateUnit}}]
    [{\hierB{Quantity}}
      [{\hierC{Approximator}}]
    ]
    [{\hierB{SocialRel}}
      [{\hierC{OrgRole}}]
    ]
  ]
\end{forest}
\end{ggroup}
\end{multicols}
\end{minipage}

\begin{itemize}
\item Items in the \psst{Circumstance} subhierarchy are prototypically 
expressed as adjuncts of time, place, manner, purpose, etc.\ 
elaborating an event or entity.
\item Items in the \psst{Participant} subhierarchy are prototypically 
entities functioning as arguments to an event.
\item Items in the \psst{Configuration} subhierarchy are prototypically
entities or properties in a static relationship to some entity.
\end{itemize}

\subsection{Limitations}

This inventory is only designed to capture semantic relations 
with a figure--ground asymmetry. This excludes:
\begin{itemize}
  \item The semantics of coordination, where the two sides of the relation 
are on equal footing, is not captured here. (Note that sometimes a morpheme can 
have symmetric as well as asymmetric interpretations: e.g., Korean \p{-wa}.)
  \item Aspects of meaning that pertain to information structure, discourse, 
or pragmatics.
\end{itemize}
Moreover, this inventory only captures semantic distinctions 
that tend to correlate with major differences in syntactic distribution. 
Thus, while there are labels for locative (\psst{Locus}), ablative (\psst{Source}), 
allative (\psst{Goal}), and \psst{Path} semantics---and analogous temporal categories---%
finer-grained details of spatiotemporal meaning are for the most part lexical 
(viz.: the difference between \pex{\p{in} the box} and \pex{\p{on} the box}, 
or temporal \p{at}, \p{before}, \p{during}, and \p{after}) and are not represented here.\ab{how does this relate to semantic distinctions correlating with syntactic distribution?}\footnote{This is not to claim
that all members of a category can be grammatical in all the same contexts: 
\pex{\p{on} Saturday} and \pex{\p{at} 5:00} are both labeled \psst{Time}, 
though the prepositions are by no means interchangeable in American English. 
We are simply asserting that the different constructions specific to days of the week 
versus times of the day are minor aspects of the grammar of English.}

\subsection{Major changes from v1}

Changes that affect only a single label are explained below the relevant 
v2 labels.

\begin{itemize}
  \item \textbf{Removed multiple inheritance.} 
  The v1 network was quite tangled. The structure is greatly simplified 
  by analyzing some tokens as \emph{construals} \citep{hwang-17}.
  \item \textbf{Revised and expanded the \psst{Configuration} subhierarchy.}
  \item \textbf{Removed the locative concreteness distinction.}
  In v1, labels \sst{Location}, \sst{InitialLocation}, and \sst{Destination} 
  were reserved for concrete locations, and the respective supertypes 
  \sst{Locus}, \sst{Source}, and \sst{Goal} used to cover abstract locations.
  This distinction was found to be difficult and without apparent
  relevance to the English preposition system.\ab{probably also because we did not find it relevant for the other three languages we applied the hierarchy to!}\ab{although I'm thinking for many of these changes- is the hierarchy supposed to be English-focused?} The concrete labels were thus removed.
  \item \textbf{Removed the location/state/value distinction.}
  The v1 scheme attempted to make an elaborate distinction between 
  values, states, and other kinds of abstract locations. 
  However, the English preposition system does not seem particularly 
  sensitive to these distinctions. (We are not aware of any prepositions 
  that mark primarily values or primarily states; rather, productive 
  metaphors allow locative prepositions to be extended to cover these, 
  and there are cases where teasing apart abstract location vs.~state vs.~value 
  is difficult.) Therefore, \sst{State}, \sst{InitialState}, \sst{EndState}, 
  \sst{Value}, and \sst{ValueComparison} were removed.
  \item \textbf{Revised the treatment of comparison and related notions.} 
  \nss{\sst{Comparison/Contrast}, \sst{Scalar/Rank}, \sst{ValueComparison}; 
  moved \psst{Approximator} under \psst{Quantity}}
  \item \textbf{Greatly simplified the \psst{Path} subhierarchy.} See \cref{sec:Path}.
  \item \textbf{Simplified the \psst{Temporal} subhierarchy.} See \cref{sec:Temporal}.
  \item \nss{removed \sst{Activity}, \sst{Material}, \sst{Reciprocation}}
\end{itemize}

\subsection{Major changes from earlier versions of this document}

\begin{itemize}
  \item \emph{Since the \textbf{April 7, 2017} version:}
  \begin{itemize}
    \item Broadened and clarified \sst{DeicticTime}, moved it up a level in the hierarchy, 
      and renamed it to \psst{Interval}
    \item Clarified \psst{Locus}, \psst{Source}, \psst{Goal}, \psst{Path}, and 
      \psst{Direction}, especially with regard to intransitive prepositions
  \end{itemize}
\end{itemize}

\hierAdef{Circumstance}

\shortdef{Macrolabel for labels pertaining to space and time; 
abstract\slash metaphoric locations such as states;
and other categories that usually constitute semantically non-core properties of events.}

Rarely, \psst{Circumstance} is used directly for:
\begin{itemize}
  \item \textbf{Contextualization}
\begin{exe}
    \ex \p{In} arguing for tax reform, the president claimed that loopholes allow 
    big corporations to profit from moving their headquarters overseas.
    \ex Bipartisan compromise is unlikely \p{with} the election just around the corner.
  \end{exe}
  For these cases, the preposition helps situate 
  the background context in which the main event takes place. 
  The background context is often realized as a subordinate clause 
  preceding the main clause. 
  \item \textbf{Occasions}
  \begin{exe}
    \ex I bought her a bike \p{for} Christmas.
    \ex I had peanut butter \p{for} lunch.
  \end{exe}
  These simultaneously express a \psst{Time} and some element of causality 
  similar to \psst{Purpose}.
  But the PP is not exactly answering a \pex{Why?} or \pex{When?} question. 
  Instead, the sentence most naturally answers a question like \pex{On what occasion was X done?}
  or \pex{Under what circumstances did X happen?}.
  \item Any other descriptions of event/state properties that are \textbf{insufficiently specified} 
  to fall under spatial, temporal, causal, or other subtypes like \psst{Manner}. E.g.:
  \begin{exe}
    \ex\label{ex:over-lunch} Let's discuss the matter \p{over} lunch. [compare \cref{ex:at-lunch}]
  \end{exe}
\end{itemize}

\hierBdef{Temporal}

\shortdef{Abstract supercategory for temporal descriptions: 
\textbf{when}, \textbf{for how long}, \textbf{how often}, \textbf{how many times}, 
etc.\ something happened or will happen.}

\begin{history}
  The v1 category \sst{Age} (e.g., \pex{a child \p{of} five}) 
  was a mutual subtype of \psst{Temporal} and \sst{Attribute}. 
  Being quite specific and rare, for v2 it was merged with \psst{Characteristic}. 
  Combined with the changes to \psst{Time} subcategories (see below), 
  this reduced by~3 the number of labels in the \psst{Temporal} subtree, 
  bringing it to 7.
\end{history}

\hierCdef{Time}

\shortdef{\textbf{When} something happened or will happen, in relation to an 
explicit or implicit reference time or event.}

\begin{exe}
  \ex We ate \choices{\p{in} the afternoon\\\p{at} 2:00\\\p{on} Friday}.
  \ex\label{ex:at-lunch} Let's talk \choices{\p{at}\\\p{during}} lunch. [compare \cref{ex:over-lunch}]
  \ex I will finish \p{after} \choices{tomorrow\\lunch\\you (do)}.
  \ex I will finish \p{by} \choices{tomorrow\\lunch}.
\end{exe}

The preposition \p{since} is ambiguous:
\begin{exe}
  \ex {} [`after'] I grew a beard---that was \p{since} the breakup. (\psst{Time})
  \ex {} [`ever since'] I have loved you \p{since} the party where we met. (\psst{StartTime}) %\ab{are 9 \& 10 really different?\nss{yes, 10 cannot be paraphrased with `after'}}
  \ex {} [`because'] I'll try not to whistle \p{since} I know that gets on your nerves. (\psst{Explanation})
\end{exe}

\psst{Time} is also used if the reference time is implicit and determined from 
the discourse:
\begin{exe}
  \ex We broke up last year, and I haven't seen her \p{since}. [since we broke up]
\end{exe}

However, \psst{Interval} is used for adpositions whose complement (object) 
is the amount of time between two reference points:
\begin{exe}
  \ex We left the party \p{after} an hour. [an hour after it started] (\psst{Interval})
  \ex We left the party an hour \p{ago}. [an hour before now] (\psst{Interval})
\end{exe}

\begin{history}
  In v1, point-like temporal prepositions (\p{at}, \p{on}, \p{in}, \p{as}) 
  were distinguished from displaced temporal prepositions (\p{before}, \p{after}, etc.) 
  which present the two times in the relation as unequal. 
  \sst{RelativeTime} inherited from \psst{Time} and was reserved for the 
  displaced temporal prepositions, as well as subclasses \psst{StartTime}, 
  \psst{EndTime}, \sst{DeicticTime}, and \sst{ClockTimeCxn}. 
  
  For v2, \sst{RelativeTime} was merged into \psst{Time}: the distinction 
  was found to be entirely lexical and lacked parallelism with the spatial hierarchy. 
  \sst{ClockTimeCxn} was also merged with \sst{Time}, the usages covered by the former 
  (expressions of clock time like \pex{ten \p{of} seven})
  being exceedingly rare and not very different semantically from 
  prepositions like \p{before}.
  \sst{DeicticTime} became \psst{Interval}.
\end{history}

\hierDdef{StartTime}

\shortdef{When the event denoted by the governor begins.}

Prototypical prepositions are \p{from} and \p{since} (but see note under \psst{Time} 
about the ambiguity of \p{since}):
\begin{exe}\ex\begin{xlist}
  \ex The show will run \p{from} 10 a.m. to 2 p.m.
  \ex a document dating \p{from} the thirteenth century
\end{xlist}\end{exe}

Note that simple \psst{Time} is used with verbs like \w{start} and \w{begin}: 
the event directly described by the PP is the starting, not the thing that started.
\begin{exe}
  \ex The show will start \p{at} 10 a.m. (\psst{Time})
\end{exe}

\hierDdef{EndTime}

\shortdef{When the event denoted by the governor finishes.}

Prototypical prepositions are \p{to}, \p{until}, \p{till}, \p{up\_to}, and \p{through}:
\begin{exe}
  \ex The show will run from 10 a.m. \p{to} 2 p.m.
  \ex Add the cider and boil \p{until} the liquid has reduced by half.
  \ex If we have survived \p{up\_to} now what is stopping us from surviving in the future?
  \ex They will be in London from March 24 \p{through} May 7.
\end{exe}

Note that simple \psst{Time} is used with verbs like \w{end} and \w{finish}: 
the event directly described by the PP is the ending, not the thing that ended.
\begin{exe}
  \ex The show will end \p{at} 2~p.m. (\psst{Time})
\end{exe}


\hierCdef{Frequency}

\shortdef{\textbf{At what rate} something happens or continues, 
or the instance of repetition that the event represents.}

\begin{exe}
  \ex Guests were arriving \p{at} a steady clip.
  \ex The risk becomes worse \p{by} the day.
  \ex The camcorder failed \p{for} the third time.
\end{exe}

Contrast: \psst{RateUnit}

\hierCdef{Duration}

\shortdef{Indication of \textbf{how long} an event or state lasts
(with reference to an amount of time or 
time period\slash larger event that it spans).}

\begin{exe}
  \ex\label{ex:forDuration} I walked \choices{\p{for}\\\#\p{in}} 20~minutes.
  \ex\label{ex:GoalDuration} I walked to$_{\psst{Goal}}$ the store \choices{\p{in}/\p{within}\\\#\p{for}} 20~minutes. [see \cref{ex:inDuration}]
  \ex\label{ex:ExtentDuration} I walked a mile \choices{\p{in}/\p{within}\\\#\p{for}} 20~minutes.
  \ex\label{ex:AmbigDuration} I mowed the lawn \choices{\p{for}\\\p{in}/\p{within}} an hour.
\end{exe}
Note that the presence of a goal \cref{ex:GoalDuration} or 
extent of an event (\pex{a mile} in \cref{ex:ExtentDuration}) 
can affect the choice \psst{Duration} preposition, blocking \p{for}.
\Cref{ex:AmbigDuration} shows a direct object which can be interpreted 
either as something against which partial progress is made---licensing \p{for} 
and the inference that some of the lawn was not reached---or 
as defining the complete scope of progress, licensing \p{in}/\p{within} 
and the inference that the lawn was covered in its entirety.

The object of a \psst{Duration} preposition can also be a reference event 
or time period used as a yardstick for the extent of the main event:
\begin{exe}
  \ex\label{ex:EventDuration} I walked \p{for} the entire race. [the entire time of the race]
  \ex I walked \choices{\p{throughout}\\\p{through}\\well \p{into}} the night.
\end{exe}

Some \p{for}-\psst{Duration}s measure the length of the specified event's \emph{result}:
\begin{exe}\ex \begin{xlist}
  \ex John went to the store \p{for} an hour. [he spent an hour at the store, not an hour going there]\footnote{This stands 
in contrast with \pex{John walked to the store \p{for} an hour}, where the most natural reading is that it took an hour to get to the store \citep[p.~230]{chang-98}.}
  \ex John left the party \p{for} an hour. [he spent an hour away from the party before returning]
\end{xlist}\end{exe}

A \psst{Duration} may be a stretch of time in which a simple event is repeated 
iteratively or habitually:
\begin{exe}\ex\begin{xlist}
  \ex I lifted weights \p{for} an hour. [many individual lifting acts collectively lasting an hour]
  \ex I walked to the store \p{for} a year. [over the course of a year, habitually went to the store by walking]
\end{xlist}\end{exe}

See further discussion at \psst{Interval}.

\hierCdef{Interval}

\shortdef{The amount of time between a reference time and 
(the nearest boundary of) an event or state.}

This applies to adpositions whose \emph{complement} (object) 
is the amount of intervening time:
\begin{exe}
  \ex I ate \choices{10~minutes \p{ago}\\a while \p{back}}. [before now]\footnote{While 
  \pex{a while \p{back}} and \pex{a few generations \p{back}} are generally accepted, 
  %\p{back} with smaller measurements,
  the use of \p{back} rather than \p{ago} for nearer and more precise temporal references,
  e.g.~\pex{10~minutes \p{back}}, appears to be especially associated with Indian English \citep[p.~7]{yadurajan-01}.}
  \ex\label{ex:inAmbiguousTime} I will eat \p{in} 10~minutes.
    \begin{xlist}
      \ex\label{ex:inDuration} {} [`for no more than 10~minutes' reading]: \psst{Duration}
      \ex\label{ex:inInterval} {} [`10~minutes from now' reading]: \psst{Interval}
    \end{xlist}
  \ex\begin{xlist}
    \ex What are the revenue projections 6~months \p{out}?
    \ex I've started watching a new TV series and am 3~episodes \p{in}.\nss{``3 episodes into the show''?}
  \end{xlist}
  \ex\label{ex:AfterObj} The game started at 7:00, but I arrived \choices{\p{after}\\\p{within}} 20~minutes.
\end{exe}
Some adpositions license a temporal difference measure in \emph{modifier} position, which does not qualify:
\begin{exe}
  \ex To beat the crowds, I will arrive \uline{a while} \choices{\p{before} (it starts)\\\p{beforehand}}. (\psst{Time})
  \ex\label{ex:AfterMod} The game started at 7:00, but I arrived \uline{20~minutes} \choices{\p{after} (it started)\\\p{afterward}}. (\psst{Time})
\end{exe}
The preposition \p{after} can be used either way---contrast \cref{ex:AfterMod} with \cref{ex:AfterObj}.

Note that having \psst{Interval} as a separate category allows us to distinguish the sense of \p{in} 
in \cref{ex:inInterval} from both the \psst{Duration} sense \cref{ex:inDuration} 
and the \psst{Time} sense (\pex{\p{in} the morning}).

\paragraph{Versus \psst{Duration}.} 
The prepositions \p{in} and \p{within} are ambiguous between \psst{Interval} and \psst{Duration}.\footnote{By contrast, 
\p{afer} seems to strongly favor \psst{Interval}. 
\pex{\p{After} a week, I had climbed all the way to the summit} is possible, 
but the conclusion that the climbing took a week may be an inference 
rather than something that is directly expressed.}
The distinction can be subtle.
With a verb which refers to a punctual moment of culmination, 
we use \psst{Interval} even though there may be an implicit 
preparatory process with a duration:
\begin{exe}\ex\begin{xlist}
  \ex I reached the summit \p{in} 3~days. [3 days from some salient prior time, e.g., when I started climbing]
  \ex I was at the summit \p{within} 3~days.
  \ex I finished climbing \p{in} 3~days.
  \ex I had the engine fixed \p{in} 3~days.
\end{xlist}\end{exe}
Whereas if the verb can be made progressive and the preposition replaced with \p{for}, 
it is a \psst{Duration}:
\begin{exe}\ex\begin{xlist}
  \ex I had climbed to the summit \p{in} 3~days.\\
  $\rightarrow$ I had been climbing to the summit \p{for} 3~days. (\psst{Duration})
  \ex I fixed the engine \p{in} 3~days.\\
  $\rightarrow$ I was fixing the engine \p{for} 3~days. (\psst{Duration})
\end{xlist}\end{exe}

Though \p{for} generally marks \psst{Duration}s, it can mark an \psst{Interval} under negation:
\begin{exe}
  \ex I haven't eaten \choices{\p{in}\\\p{for}} hours. [hours have passed since the last time I ate]
\end{exe}

% Some temporal prepositions may be intransitive \cref{ex:IntrRefTime}, 
% or may take a temporal difference measure as the object \cref{ex:TimeDiff},
% provided that a reference time is salient in the discourse. 
% With such prepositions, the reference time may default to the speech time. 
% However, we limit \psst{DeicticTime} to those prepositions 
% whose \emph{inherent} reference point is the speech time. 
% The following contexts allow a \psst{Time} but not a \psst{DeicticTime}: 
% \begin{exe}
%   %\ex\label{ex:IntrRefTime} To beat the crowds at the game, I will arrive an hour \choices{\p{before}\\\p{beforehand}\\ \#\p{ago}}. (\psst{Time})
%   \ex\label{ex:IntrRefTime} To beat the crowds at the game, I will arrive a while \choices{\p{before}\\\p{beforehand}\\\#\p{ago}\\\#\p{back}}. (\psst{Time})
%   \ex\label{ex:TimeDiff} I took a seat in the waiting room. \choices{\p{After}\\\p{Within}\\??\p{In}} 15 minutes, the doctor saw me for an hour. (\psst{Time})
% \end{exe}
% To summarize: \psst{DeicticTime} covers preposition usages where `now' is inherent,
% and the more general \psst{Time} applies if the reference time is given in the 
% direct object or depends on the discourse.

\begin{history}
  Version~1 featured a label called \sst{DeicticTime}, under \psst{RelativeTime}, 
  which was meant to cover \p{ago} and temporal usages of other adpositions 
  (such as \p{in}) whose reference point is the utterance time or deictic center. 
  This concept proved difficult to apply and was (without good justification) 
  used as a catch-all for intransitive usages of temporal prepositions. 
  For v2, the new concept of \psst{Interval} is broader in that it drops the deictic 
  requirement (also covering \p{within}), while \psst{Time} has been clarified to include 
  intransitive usages of prepositions like \p{before} where the reference time 
  can be recovered from discourse context.
\end{history}


\hierBdef{Locus}

\shortdef{Location, condition, or value. May be abstract.}

\begin{exe}
  \ex I like to sing \choices{\p{at} the gym\\\p{in} the shower\\\p{on} Main St.}.
  \ex The cat is \choices{\p{on\_top\_of}\\\p{off}\\\p{beside}\\\p{near}} the dog.
  %\ex the wheels \p{on} the bus % Whole ~> Locus
  \ex I read it \choices{\p{in} a book\\\p{on} a website}.
  \ex The charge is \p{on} my credit card.
  \ex We met \p{on} a trip to Paris.
  \ex The Dow is \p{at} \choices{a new high\\20,000}.
  \ex I am now \p{off} work.
  \ex She was \p{in} a coma.
\end{exe}
Words that incorporate a kind of reference point are \psst{Locus} 
even without an overt object:
\begin{exe}
  \ex\begin{xlist}
    \ex The cat is \p{inside} the house.
    \ex The cat is \p{inside}.
  \end{xlist}
  \ex\begin{xlist}
    \ex All passengers are \p{aboard} the ship.
    \ex All passengers are \p{aboard}.
  \end{xlist}
\end{exe}
\psst{Locus} also applies to \p{in}, \p{out}, \p{off}, \p{away}, \p{back}, 
etc.\ when used to describe a location without an overt object:
\begin{exe}
  \ex\begin{xlist}
    \ex The doctor is \choices{\p{in}\\\p{out\_of}\\\p{away\_from}} the office.
    \ex The doctor is \choices{\p{in}\\\p{out}\\\p{away}}.
  \end{xlist}
\end{exe}
And to \p{around} meaning `nearby' or `in the area':
\begin{exe}
  \ex Will you be \p{around} in the afternoon?
  \ex She's the best doctor \p{around}!
\end{exe}

In a phenomenon called \textbf{fictive motion} \citep{talmy-96}, 
dynamic language may be used to describe static scenes. 
We use construal for these:
\begin{exe}
  \ex A road runs \p{through} my property. (\rf{Locus}{Path})
  \ex The road extends \p{to} the river. (\rf{Locus}{Goal})
  \ex I saw him \p{from} the roof. (\rf{Locus}{Source})
  \ex We're just \p{across} the street from$_{\text{\rf{Locus}{Source}}}$ you. (\rf{Locus}{Path})
\end{exe}

\hierCdef{Source}

\shortdef{Initial location, condition, or value. May be abstract.}

For motion events, the initial location is where the thing in motion 
(the figure) starts out.
\psst{Source} also applies to abstract or metaphoric initial locations, 
including initial states in a dynamic event.
%and simple \psst{Source} is used for adpositions that generically mark starting points.

In English, a prototypical \psst{Source} preposition is \p{from}:
\begin{exe}
  \ex\label{ex:catBox} The cat jumped \choices{\p{from}\\\p{out\_of}} the box.
  \ex\label{ex:catLedge} The cat jumped \choices{\p{from}\\\p{off\_of}\\\p{off}} the ledge.
  \ex\label{ex:internet} I got it \choices{\p{from}\\\p{off}} the internet.
  \ex people \p{from} France
  \ex The temperature is rising \p{from} a low of 30 degrees.
  \ex I have arrived \p{from} work.
  \ex\label{ex:coma} She \choices{awoke \p{from}\\came \p{out\_of}} a coma.
  \ex We are moving \p{off\_of} that strategy.
\end{exe}
The \psst{Source} use of \p{from} can combine with a specific locative PP:
\begin{exe}
  \ex I took the cat \p{from} behind$_{\psst{Locus}}$ the couch.
\end{exe}
Note that \p{away\_from} is ambiguous between marking a starting point (\psst{Source}) 
and a separate orientational reference point (\psst{Direction}):
\begin{exe}
  \ex At the sound of the gun, the sprinters ran \choices{\p{away\_from}\\\p{from}} the starting line. (\psst{Source})
  \ex The bikers ride parallel to the river for several miles, then 
  head east, \choices{\p{away\_from}\\\#\p{from}} the river. 
  (\psst{Direction}: bikers are never at the river)
\end{exe}
%
%Other prepositions specify more information about the nature of the figure's 
%position and trajectory vis-\`{a}-vis its initial location: 
%e.g., \p{off} and \p{off\_of} for motion away from the top of a surface, 
%and \p{out\_of} for motion through the boundary of a container. 
%We say this involves construal of a \psst{Source} as a \psst{Direction}:
% \begin{exe}
%   \exp{ex:catBox} The cat jumped \p{out\_of} the box. (\rf{Source}{Direction})
%   \exp{ex:catLedge} The cat jumped  the ledge. (\rf{Source}{Direction})
% \end{exe}
% \nss{possible objection: we don't distinguish \p{at} (generic \psst{Locus}) 
% vs. \p{on} and \p{in} (surface and container, resp.). So why distinguish their \psst{Source} counterparts?}
%
% If figurative motion language is entrenched and bleached such that the 
% image of a specific kind of motion is no longer salient, simple \psst{Source} is used. 
% The construal as \psst{Direction} is maintained only if the figurative path
% is somewhat salient:
%
Note, too, that \p{off(\_of)} and \p{out(\_of)} can also mark simple states:
\begin{exe}
  \ex I am \p{off} \choices{medications\\work}. (\psst{Locus})
  \ex The lights are \choices{\p{off}\\\p{out}}. (\psst{Locus})
  \ex We are \p{out\_of} toilet paper. (\psst{Locus})
\end{exe}

Sometimes a specific \psst{Source} is implicit, and the preposition is intransitive. 
But if no specific referent is implied, another label may be more appropriate:
\begin{exe}
  \ex The cat was sitting on the ledge, then jumped \p{off}. (\psst{Source}: implicit `(of) it')
  \ex He was offered the deal, but walked \p{away}. (\psst{Source}: implicit `from it')
  \ex The bird flew \choices{\p{away}\\\p{off}}. (\psst{Direction}: vaguely away from the viewpoint)
\end{exe}

\psst{Source} is prototypically inanimate, 
though it can be used to construe animate \psst{Participant}s 
(especially \psst{Originator} and \psst{Causer}).
Contrasts with \psst{Goal}.

\hierCdef{Goal}

\shortdef{Final location (destination), condition, or value. May be abstract.}

Prototypical prepositions include \p{to}, \p{into}, and \p{onto}:
\begin{exe}
  \ex I ran \p{to} the store.
  \ex The cat jumped \p{onto} the ledge.
  \ex The temperature is rising \p{to} a high of 40 degrees.
  \ex She slipped \p{into} a coma.
\end{exe}
For motion events, a \psst{Goal} must have been reached if the event 
has progressed to completion (was not interrupted).
\psst{Direction} is used instead for \p{toward(s)} and \p{for}, 
which mark an intended destination that is not necessarily reached:
\begin{exe}
  \ex\begin{xlist}
    \ex I headed \p{to} work. (\psst{Goal})
    \ex I headed \choices{\p{towards}\\\p{for}\\\#\p{to}} work but never made it there. (\psst{Direction})
  \end{xlist}
\end{exe}

\paragraph{\emph{go to}.} A conventional way to express one's status as a student at some school is 
with the expression \pex{go \p{to} (name or kind of school)}.
Construal is used when \pex{go \p{to}} indicates student status, rather than 
(or in addition to) physical attendance:
\begin{exe}
  \ex\label{ex:student} I went \p{to} (school at$_{\psst{Locus}}$) UC Berkeley. (\rf{OrgRole}{Goal})
  \exp{ex:student} I went \p{to} UC Berkeley for the football game. (\psst{Goal})
\end{exe}
Going to a business as a customer, going to an attorney as a client, 
going to a doctor as a patient, etc.\ can also convey long-term status, 
but there is considerable gray area between habitual going and 
being in a professional relationship, so we simply use \psst{Goal}:
\begin{exe}
  \ex I go \p{to} Dr.~Smith for my allergies. (\psst{Goal})
\end{exe}

\paragraph{Locative as destination.}
English regularly allows canonically static locative prepositions to mark 
goals with motion verbs like \pex{put}.\ab{Again is the hierarchy supposed to be English-focused? Adpositions in some other languages, e.g., German prepositions, differentiate senses that English does not. Do we ignore those distinctions for now only because we have not applied the hierarchy to those languages or do we do it because the hierarchy is to be faithful to English?} 
We use the construal \rf{Goal}{Locus}:
\begin{exe}
  \ex I put the lamp \p{next\_to} the chair.
  \ex I'll just hop \p{in} the shower.
  \ex I put my CV \p{on} the internet.
  \ex The cat jumped \p{on} my face.
\end{exe}

% Examined COCA first 100 results for "for London" and "for Paris".
% 'leave' is the dominant verb
%leave/flee/depart/embark/take off/set out/set sail/... for, board a plane for; head/make for; bound for; train/bus for

\psst{Goal} is prototypically inanimate, though it can be used to construe animate \psst{Participant}s 
(especially \psst{Recipient}).
Contrasts with \psst{Source}.

\hierBdef{Path}

\shortdef{The ground that must be covered in order for the motion to be complete.}

The ground covered is often a linear extent with or without 
specific starting and ending points:
\begin{exe}
  \ex The bird flew \p{over} the building.
  \ex The sun traveled \p{across} the sky.
  \ex Hot water is running \p{through} the pipes.
\end{exe}

It can also be a waypoint\slash something that must be passed or encircled. 
\begin{exe}
  \ex We flew to Rome \p{via} Paris.
  \ex I go \p{by} that coffee shop every morning.
  \ex The earth has completed another orbit \p{around} the sun.
\end{exe}
If this is a portal in the boundary of a container, 
it is often construed as \psst{Source}, \psst{Goal}, or \psst{Locus}:
\begin{exe}
  \ex The bird flew \p{in} the window. (\rf{Path}{Locus})
  \ex The bird flew \p{out} the window. (\rf{Path}{Source})
  \ex A cool breeze blew \p{into} the window. (\rf{Path}{Goal})
\end{exe}

The prepositions \p{around} and \p{throughout} can mark a region in which motion 
that follows an aimless or complex trajectory is contained. 
%and which it roughly ``covers'' via an . 
Construal is used for these, whether or not the region is explicit:
\begin{exe}\ex \rf{Locus}{Path}:\begin{xlist}
  \ex The kids ran \p{around}.
  \ex The kids ran \choices{\p{around}\\\p{throughout}} the kitchen.
  \ex The kids ran \p{around} in the kitchen.
\end{xlist}\end{exe}

See also: \psst{Instrument}, \psst{Manner}

\begin{history}
  The v1 hierarchy distinguished many different subcategories of path descriptions. 
  The labels \sst{Traversed}, \sst{1DTrajectory}, \sst{2DArea}, \sst{3DMedium}, 
  \sst{Contour}, \sst{Via}, \sst{Transit}, and \sst{Course} have all been merged
  with \psst{Path} for v2.
\end{history}

\hierCdef{Direction}

\shortdef{How motion or an object is aimed\slash oriented.}

A \psst{Direction} expresses the orientation of a stationary figure or of a figure's motion.
Prototypical markers\footnote{Known variously as \emph{adverbs}, \emph{particles}, 
and \emph{intransitive prepositions}.}
are \p{away} and \p{back}; \p{up} and \p{down}; 
\p{off}; and \p{out},
provided that no specific \psst{Source} or \psst{Goal} is salient:
\begin{exe}
  \ex The bird flew \choices{\p{up}\\\p{out}\\\p{away}\\\p{off}}.
  \ex I walked \p{over} to where they were sitting.
  \ex The price shot \p{up}.
\end{exe}

In addition, transitive \p{toward(s)}, \p{for}, and \p{at} can 
indicate where something is aimed or directed (but see discussion at \psst{Goal}):
\begin{exe}
  \ex The camera is aimed \p{at} the subject.
  \ex The toddler kicked \p{at} the wall.
\end{exe}

See discussion of \p{away\_from} at \psst{Source}.


\hierCdef{Extent}

\shortdef{The size of a path.}

This can be the physical distance traversed or the amount of change on a scale:
\begin{exe}
  \ex We ran \p{for} miles.
  \ex The price shot up \p{by} 10\%.
\end{exe}

\hierBdef{Means}

\shortdef{Secondary action or event that characterizes \textbf{how} 
the main event happens or is achieved.}

Prototypically a volitional action, though not necessarily \cref{ex:chlorophyll}. 
A volitional \psst{Means} will often modify an intended result, 
though the outcome can be unintended as well \cref{ex:oops}.
\begin{exe}
  \ex Open the door \p{by} turning the knob.
  \ex They retaliated \choices{\p{by} shooting\\\p{with} shootings}.
  \ex\label{ex:oops} The owners destroyed the company \p{by} growing it too fast.
  \ex\label{ex:chlorophyll} Chlorophyll absorbs the light \p{by} transfer of electrons.
\end{exe}

\psst{Means} is similar to \psst{Instrument}, which is used for causally supporting entities 
and is a kind of \psst{Participant}.

Contrast with \psst{Explanation}, which characterizes \textbf{why} 
something happens. I.e., an \psst{Explanation} portrays the secondary event 
as the causal \emph{instigator} of the main event, whereas \psst{Means} 
portrays it merely as a \emph{facilitator}.

\begin{history}
  In v1, \psst{Means} was a subtype of \psst{Instrument}, 
  but with the removal of multiple inheritance for v2, 
  the former was moved directly under \psst{Circumstance} 
  and the latter directly under \psst{Participant}.
\end{history}

\hierBdef{Manner}

\shortdef{Description of \textbf{how} something happens or exists
that does not directly invoke a location, path, or temporal or causal relation. 
Often the ``style''\nss{or shape?} of something.}

% FN definition of Manner under Expend_resource: 
%Any description of the intentional act which is not covered by more specific FEs, 
%including secondary effects (quietly, loudly), and general descriptions comparing events 
%(the same way). In addition, it may indicate salient characteristics of an Agent 
%that also affect the action ( deliberately, eagerly, carefully). 

\begin{exe}
  \ex The toddler is old enough to eat \p{by} herself.
  \ex The people shouted \p{with} pleasure.
  \ex\label{ex:pathmanner} They dance \p{in} a circle. (\rf{Path}{Manner})
  \ex The sand is \p{in} a pyramid shape.\nss{Characteristic as Manner, or just one? same for below}
  \ex It was written \p{in} French.
  \ex music \p{in} C major
  %\ex You can use it \p{as} a hammer. (\rf{Manner}{Identity}\nss{?})
\end{exe}

\nss{relation to Characteristic?}

\begin{history}
  In v1, \psst{Manner} was positioned as an ancestor of all 
  categories that license a \emph{How?} question, including 
  \psst{Instrument}, \psst{Means}, and \sst{Contour}, as in \cref{ex:pathmanner}. 
  This criterion was deemed too broad, so \psst{Manner} has no 
  subtypes in v2.
\end{history}

\hierBdef{Explanation}

\shortdef{Assertion of \textbf{why} something happens or is the case.}

This marks a secondary event that is asserted as the reason for the main event or state. 

\begin{exe}
\ex  I went outside \p{because\_of} the smell. %(Why did you go outside?)
\ex  The rain is \p{due\_to} a cold front. %(Why is there rain?)
\end{exe}

When a preposition like \p{after} is used and the relation is temporal as well as causal, 
construal captures the overlap. While \p{since} and \p{as} can also be temporal, 
there are tokens where they cannot be paraphrased respectively with \p{after} and \emph{when}:
\begin{exe}
  \ex I joined a protest \p{after} the shameful vote in Congress. (\rf{Explanation}{Time})
  \ex Her popularity has grown \p{since} she announced a bid for president. (\rf{Explanation}{Time})
  \ex I will appoint him \choices{\p{since}\\\p{as}\\\#\p{after}\\\#when} he is most qualified for the job. (\psst{Explanation})
\end{exe}


\nss{TODO: Function}

Question test: \psst{Explanation} and its subtype \psst{Purpose} license
\pex{Why?} questions.

\hierCdef{Purpose}

\shortdef{Something that somebody wants to bring about,
asserted to be why something was done, is the case, or exists.}

Central usages of \psst{Purpose} explain the motivation behind an action.
Typically the governing event serves as a means for achieving or facilitating the \psst{Purpose}. 
Prototypical markers include \p{for} and infinitive marker \p{to}:
\begin{exe}
  \ex\begin{xlist}
    \ex He rose \p{to} make a grand speech.
    \ex surgery \p{to} treat a leg injury
  \end{xlist}
  \ex\begin{xlist} 
    \ex He rose \p{for} a grand speech.
    \ex We hired a caterer \p{for} (the party) tonight.
  \end{xlist}
\end{exe}
Something directly manipulated/affected can stand in metonymically 
for the desired event:
\begin{exe}\ex\begin{xlist}
  \ex I went to the store \p{for} eggs. [understood: `to acquire/buy eggs']
  \ex surgery \p{for} a leg injury [understood: `to treat a leg injury']
\end{xlist}\end{exe}

In contrast to the above, where the governor denotes an \emph{event}, 
an \emph{entity} can be modified to explicate an intended use or affordance. 
Because this can be understood as a static property of the entity---why it was created 
or what it is useful for (part of its qualia structure)---we
use the construal \rf{Characteristic}{Purpose}:\footnote{In FrameNet 
as of v1.7, these sorts of purposes are labeled as \sst{Inherent\_purpose}. 
See, e.g., the example ``MONEY [to support yourself and your family]'' in the \textbf{Money} frame 
(\url{https://framenet2.icsi.berkeley.edu/fnReports/data/lu/lu13361.xml?mode=annotation}).}
\begin{exe}
  \ex\label{ex:charPurp} \rf{Characteristic}{Purpose}:
  \begin{xlist}
    \ex a shoulder \p{to} cry on
    \ex The noose \p{for} the prisoner was too loose. [understood: `for use on the prisoner']
    \ex a good store \p{for} eggs [understood: `for acquiring/buying eggs']
    \ex a good book \p{to} give to young readers
    \ex a good book \p{for} young readers [understood: `for giving to young readers']
    %\ex physical therapy \p{for} a leg injury. [understood: `treating a leg injury']
  \end{xlist}
\end{exe}

Question test: \psst{Explanation} and its subtype \psst{Purpose}, 
when used adverbially, license \pex{Why?} questions. 
\psst{Purpose} usually licenses an \pex{in order to} 
or \pex{for the purpose of} paraphrase.

\nss{overlap with Circumstance-occasion: for dinner, for my birthday}
\nss{closeness to Beneficiary}

\begin{history}
  In v1, the usages illustrated in \cref{ex:charPurp} were assigned a separate label, 
  \sst{Function}, which inherited from both \sst{Attribute} and \sst{Purpose}.
  The ability to use construal removes the need for a separate label.
\end{history}

\hierAdef{Participant}

\shortdef{Thing, usually an entity, that plays a causal role in an event.}

Not used directly---see subtypes.

\hierBdef{Causer}

\shortdef{Instigator of, and a core participant in, an event.}

\psst{Causer} is applied directly to inanimate things or forces conceptualized as entities. 
Prototypical prepositions are \p{by} (prominently including passive-\p{by}), \p{of}, and \p{'s}:
\begin{exe}
  \ex the devastation of$_{\psst{Theme}}$ the town wreaked \p{by} the fire
  \ex\begin{xlist} 
    \ex the devastation \p{of} the fire
    \ex the fire\p{'s} devastation of$_{\psst{Theme}}$ the town
  \end{xlist}
\end{exe}
The \psst{Causer} is sometimes construed as a \psst{Source}:
\begin{exe}
  \ex \begin{xlist}
    \ex the devastation \p{from} the fire (\rf{Causer}{Source})
    \ex fatalities \p{from} cancer (\rf{Causer}{Source})
    \ex FDR suffered \p{from} polio. (\rf{Causer}{Source})
  \end{xlist}
\end{exe}

See also: \psst{Instrument}

\hierCdef{Agent}

\shortdef{Animate instigator of an action (typically volitional).}

Prototypical prepositions are \p{by} (prominently including passive-\p{by}), \p{of}, and \p{'s}:
\begin{exe}
  \ex\begin{xlist}
    \ex the decisive vote \choices{\p{by}\\\p{of}} the City Council
    \ex the City Council\p{'s} decisive vote 
  \end{xlist}
  \ex they needed Joan\p{'s} help
  %\ex ?they needed help of hers
  %(\rf{Agent}{Possessor})
\end{exe}
When two symmetric \psst{Agent}s are collected in a single NP 
functioning as a set, it is marked as a \psst{Whole} construal:
\begin{exe}
  \ex There was a war \p{between} France and Spain. (\rf{Agent}{Whole})
  \ex This is a discussion \p{among} friends. (\rf{Agent}{Whole})
%  \ex Please talk \p{amongst} yourselves. \rf{Agent}{Whole}
% reflexives are weird. maybe Co-Agent

\end{exe}

Compare: \psst{Co-Agent}; 
see also: \psst{OrgRole}, \psst{Originator}, \psst{Stimulus}


\hierDdef{Co-Agent}

\shortdef{Second semantically core participant that would otherwise be labeled \psst{Agent}, 
but which is adpositionally marked in contrast with an \psst{Agent} 
occupying a non-oblique syntactic position (subject or object).
Typically, the \psst{Agent} and \psst{Co-Agent} engage in the event 
in a reciprocal fashion.}

\begin{exe}
  \ex I fought in a war \p{against} the Germans.
  \ex I \choices{talked\\argued} \p{with} my roommate about cleaning duties.
\end{exe}

See also: \psst{Accompanier}, \psst{SocialRel}

\hierBdef{Theme}

\shortdef{Undergoer that is a semantically core participant in an event or state, 
and that does not meet the criteria for any other label.}

Prototypical \psst{Theme}s undergo (nonagentive) motion, are transferred, 
or undergo an internal change of state (sometimes called \emph{patients}).
Adpositional \psst{Theme}s are usually construed as something else:
\begin{exe}
  \ex Fill the bowl \p{with} water. (\rf{Theme}{Instrument})
  \ex The mechanic made a repair \p{to} the engine. (\rf{Theme}{Goal})
  \ex \begin{xlist}
      \ex Sheldukher \choices{searched\\fumbled} \p{for} his laser pistol. (\rf{Theme}{Goal})
      \ex There is a significant demand \p{for} new housing. (\rf{Theme}{Goal})
      \ex They charge higher prices \p{for} goods bought by credit card. (\rf{Theme}{Goal})
    \end{xlist}
  \ex\begin{xlist}
      \ex\begin{xlist} 
        \ex the price \p{of} tea (\rf{Theme}{Gestalt})\nss{?}
        \ex the tea\p{'s} price\nss{or is this just \psst{Characteristic}?}
      \end{xlist}
      \ex\begin{xlist} 
        \ex the approach \p{of} the waves
        \ex the wave\p{s'} approach
      \end{xlist}
      \ex\begin{xlist}
        \ex the \choices{death\\murder} \p{of} a salesman
        \ex the salesman\p{'s} \choices{death\\murder}
      \end{xlist}
    \end{xlist}
  \ex \begin{xlist}
      \ex The mechanic worked \p{on} the engine.
      \ex We noshed \p{on} snacks.
      \ex Students spend a lot of money \p{on} textbooks.
    \end{xlist}
  \ex \begin{xlist}
      \ex There was an increase \p{in} oil prices.
      \ex I'm covered \p{in} bees! (\rf{Theme}{Locus})
    \end{xlist}
  \ex \begin{xlist}
      \ex The training saved us \p{from} almost certain death. (\rf{Theme}{Source})
      \ex They prevented us \p{from} boarding the plane. (\rf{Theme}{Source})
    \end{xlist}
\end{exe}
When two symmetric undergoers are collected in a single NP 
functioning as a set, it is marked as a \psst{Whole} construal:
\begin{exe}
  \ex There was a collision in mid-air \p{between} two light aircraft. (\rf{Theme}{Whole})
  \ex Links \p{between} science and industry are important. (\rf{Locus}{Whole})
\end{exe}

\begin{history}
  In v1, following many thematic role inventories, 
  \sst{Patient} was a distinct label for undergoers that were 
  affected (undergoing an internal change of state). 
  It was merged into \psst{Theme} for v2 because the affectedness criterion can be subtle 
  and difficult to apply.
\end{history}

Compare: \psst{Co-Theme}

\hierCdef{Co-Theme}

\shortdef{Second semantically core undergoer that would otherwise be labeled \psst{Theme}, 
but which is adpositionally marked in contrast with a \psst{Theme} 
occupying a non-oblique syntactic position (subject or object).}

\begin{exe}
  \ex They replaced my old tires \p{with} new ones.
\end{exe}

\begin{history}
  In v1, \sst{Co-Patient} was a distinct label, and the two shared a common supertype, 
  \sst{Co-Participant}. 
  See note at \psst{Theme}.
\end{history}

See also: \psst{InsteadOf}, \psst{Co-Agent}

\hierCdef{Topic}

\shortdef{Information content or subject matter in communication or cognition.}

Prototypical prepositions are \p{about} and \p{on}:

\begin{exe}
  \ex I gave a presentation \choices{\p{about}\\\p{on}} politics.
  \ex Try not to think \p{about} it.
\end{exe}

Less prototypical \psst{Topic} markers include:

\begin{exe}
  \ex Are you interested \p{in} politics?
  \ex I was accused \p{of} treason.
  \ex I'm \choices{an expert\\talented} \p{at} cooking.
\end{exe}

See also: \psst{Stimulus}

\hierBdef{Stimulus}%
%
\shortdef{That which is perceived or experienced (bodily, perceptually, or emotionally).}

\psst{Stimulus} does not seem to have any prototypical adposition 
in the languages we have looked at. In English, it can be construed in several ways:
\begin{exe}
  \ex My affection \p{for} you (\rf{Stimulus}{Beneficiary})
  \ex Scared \p{by} the bear (\rf{Stimulus}{Causer})
  \ex I startled \p{at} the noise (\rf{Stimulus}{Goal})
  \ex I care \p{about} you (\rf{Stimulus}{Topic})
\end{exe}

Counterpart: \psst{Experiencer}

\hierBdef{Experiencer}

\shortdef{Animate who is aware of a bodily experience, perception, emotion, or mental state.}

\psst{Experiencer} does not seem to have any prototypical adposition 
in the languages we have looked at. In English, it can be construed in several ways:
\begin{exe}
  \ex\begin{xlist}
    \ex The anger \p{of} the students (\rf{Experiencer}{Possessor})
    \ex The student\p{s'} anger (\rf{Experiencer}{Possessor})
  \end{xlist}
  \ex Running is enjoyable \p{for} me (\rf{Experiencer}{Beneficiary})
  \ex It feels hot \p{to} me (\rf{Experiencer}{Goal})
\end{exe}

Elsewhere, the term \emph{cognizer} is sometimes used for one whose 
mental state is described.

Counterpart: \psst{Stimulus}

\hierBdef{Originator}

\shortdef{Animate who is the initial possessor or creator/producer of something,
including the speaker/communicator of information. 
Excludes events where transfer/communication is not framed as unidirectional.}

A ``source'' in the broadest sense of a starting point/condition. 
Contrasts with \psst{Recipient} if there is transfer/communication.

English construals:\footnote{If we consider subject position as an \psst{Agent} construal 
and direct object position as a \psst{Theme} construal, then we can add examples like 
\pex{\uline{She} talked to her editor} (\rf{Originator}{Agent}) and 
\pex{They robbed \uline{her} of her life savings} (\rf{Originator}{Theme}).
\psst{Originator} does not apply to the subject of events like \pex{exchange} or \pex{talk/chat (with)}, 
which involve a back-and-forth between 
\psst{Agent} and \psst{Co-Agent} (or a plural \psst{Agent}).}
\begin{exe}
  \ex \rf{Originator}{Agent} (passive-\p{by} or adnominal \p{by}):
  \begin{xlist}
    \ex\label{ex:worksBy} works \p{by} Shakespeare [cf.~\cref{ex:worksOf}]
    \ex The telephone was invented \p{by} AlexanderGraham Bell.
    \ex The story was \choices{given\\told} to$_{\text{\rf{Recipient}{Goal}}}$ her \p{by} her editor.
  \end{xlist}
  \ex \rf{Originator}{Source}:
  \begin{xlist}
    \ex\label{ex:worksOf} works \p{of} Shakespeare [cf.~\cref{ex:worksBy}]
    \ex The story was obtained \p{from} an anonymous White House employee.
    \ex I heard the news \p{from} Larry.
  \end{xlist}
  \nss{Shakespeare\p{'s} works: Agent or Source?}
\end{exe}



% \nss{Do we need to limit further the kinds of events that have an Originator 
% and Recipient? Only communicative events where the message is core? 
% (``Say'', ``tell'', ``inform'', but not ``talk'' or ``negotiate''. 
% Otherwise we have ``talk with him'' as \rf{Originator}{Co-Agent}; 
% ``negotations by the parties'' as \rf{Orignator}{Agent}; 
% and ``negotations between the parties'' as Originator \tat> Agent \tat> Whole!)}

\begin{history}
  \psst{Originator} merges v1 labels \sst{Donor/Speaker} and \sst{Creator}, 
  which were difficult to distinguish in the case of authorship.
  %(e.g., \pex{the operas \p{of} Puccini}).
  \sst{Donor/Speaker} was a subtype of \sst{InitialLocation}, which 
  inherited from \sst{Location} and \sst{Source}. 
  \sst{Creator} was a subtype of \psst{Agent}.
  Moving \psst{Originator} directly under \psst{Participant} 
  puts it in a neutral position with respect to its possible construals.
\end{history}

\hierBdef{Recipient}

\shortdef{Animate who is the (actual or intended) final possessor of a thing or message.
Excludes events where transfer/communication is not framed as unidirectional.}
A ``goal'' in the broadest sense of an ending point/condition. 
Contrasts with \psst{Originator}.

English construals:\footnote{If subject position is viewed as an \psst{Agent} construal, 
then active subject with a transfer verbs like \pex{get} or \pex{receive} is \rf{Recipient}{Agent}.
If direct object position is viewed as a \psst{Theme} construal, 
then \pex{She informed \uline{her editor}} are \rf{Recipient}{Theme}.}
\begin{exe}
  \ex She \choices{gave the story\\spoke} \p{to} her editor. (\rf{Recipient}{Goal})
  \ex The news was not well received \p{by} the White House. (\rf{Recipient}{Agent})
\end{exe}

\psst{Originator} does not apply to events like \pex{exchange/talk/chat (\p{with})}, 
which involve a back-and-forth between 
\psst{Agent} and \psst{Co-Agent} (or a plural \psst{Agent} subject):
\begin{exe}
  \ex She \choices{swapped stories\\chatted} \p{with} her friends. (\psst{Co-Agent})
\end{exe}

\begin{history}
  In v1, \psst{Recipient} was the counterpart to \sst{Donor/Speaker}:
  \psst{Recipient} was a subtype of \sst{Destination}, which 
  inherited from \sst{Location} and \sst{Goal}. 
  Moving \psst{Recipient} directly under \psst{Participant} 
  puts it in a neutral position with respect to its possible construals.
\end{history}

\hierBdef{Cost}

\shortdef{An amount (typically of money) that is linked to an item or service 
that it pays for\slash could pay for, or given as the amount earned or owed.} 

The governor may be an explicit commercial scenario:
\begin{exe}
  %\ex I paid/owed John \$10 for the book. %#nonprep
  \ex I \choices{bought\\sold} the book \p{for} \$10.
  \ex The book is \choices{priced\\valued} \p{at} \$10.
  \ex I got a refund \p{of} \$10.
\end{exe}
Or the \psst{Cost} may be specified as an adjunct with a non-commerical governor:
\begin{exe}
  \ex You can ride the bus \p{for} \choices{free\\\$1}.
\end{exe}
\psst{Cost} is \emph{not} used with general scenes of possession or transfer, 
even if the thing possessed or transferred happens to be an amount of money:
\begin{exe}
  \ex I bestowed the winner \p{with} \$100. (\psst{Co-Theme})
\end{exe}

\begin{history}
  This category was not present in v1, which had the broader category \sst{Value}. 
  VerbNet \citep{verbnet,palmer-17} has a similar category called \sst{Asset}; we chose the name 
  \psst{Cost} to emphasize that it describes a relation rather than an entity type 
  (it does not apply to money with a verb like \pex{possess} or \pex{transfer}, 
  for instance).
\end{history}

\hierBdef{Beneficiary}

\shortdef{Animate or personified undergoer that is (potentially) 
advantaged or disadvantaged by the event or state.}

This label does not distinguish the polarity of the relation 
(helping or hurting, which is sometimes termed \emph{maleficiary}).

\begin{exe}
  \ex Vote \choices{\p{for}\\\p{against}} Pedro!
  \ex These are \choices{clothes \p{for} children\\children\p{'s} clothes}.\nss{s-genitive simple Beneficiary or \rf{Beneficiary}{Possessor}?}
  \ex Junk food is bad \p{for} your health.
  \ex My parrot died \p{on} me.
\end{exe}

\hierBdef{Instrument}

\shortdef{An entity that facilitates an action by applying intermediate causal force.}

Prototypically, an \psst{Agent} intentionally applies the \psst{Instrument} 
with the purpose of achieving a result:
\begin{exe}\ex\begin{xlist}
  \ex I broke the window \p{with} a hammer.
  \ex I destroyed the argument \p{with} my words.
\end{xlist}\end{exe}
Less prototypically, the action could be unintentional:
\begin{exe}
  \ex I accidentally poked myself in the eye \p{with} a stick.
\end{exe}
The key is that the \psst{Instrument} is not sufficiently ``independently causal'' 
to instigate the event.

However, to downplay the agency of the individual operating the instrument, 
the instrument can be placed in a passive \p{by}-phrase, 
which construes it as the instigator:
\begin{exe}\ex\label{ex:passiveInstrument}\begin{xlist}
  \ex The window was broken \p{by} the hammer. (\rf{Instrument}{Causer})
  \ex My headache was alleviated \p{by} aspirin. (\rf{Instrument}{Causer})
\end{xlist}\end{exe}
Note that the examples in \cref{ex:passiveInstrument} can be rephrased 
in active voice with the \psst{Instrument} as the subject.

A device serving as a mode of transportation or medium of communication 
counts as an \psst{Instrument}, but is often construed as a \psst{Locus} or \psst{Path}:
\begin{exe}
  \ex Communicate \p{by} \choices{phone\\email}. (\psst{Instrument})
  \ex Talk \p{on} the phone. (\rf{Instrument}{Locus})
  \ex Send it \choices{\p{over}\\\p{via}} email. (\rf{Instrument}{Path})
  \ex Travel \p{by} train. (\psst{Instrument})
  \ex Escape \p{with} a getaway car. (\psst{Instrument})
  \ex Escape \p{in} the getaway car. (\rf{Instrument}{Locus})
\end{exe}
This includes some expressions which incorporate the \psst{Instrument} 
in a noun:
\begin{exe}
  \ex ride \p{on} horseback (\rf{Instrument}{Locus})
  \ex hold \p{at} knifepoint (\rf{Instrument}{Locus})
\end{exe}
Other non-prototypical instruments that can be construed as paths 
include waypoints from \psst{Source} to \psst{Goal}, 
and people that serve as intermediaries:
\begin{exe}
  \ex We flew to London \p{via} Paris. (\rf{Instrument}{Path})
  \ex I found out the news \p{via} Sharon. (\rf{Instrument}{Path})
\end{exe}

Conversely, roadways count as \psst{Path}s but can be construed as \psst{Instrument}s:
\begin{exe}
  \ex Escape \p{through} the tunnel. (\psst{Path})
  \ex Escape \p{by} tunnel. (\rf{Path}{Instrument})
\end{exe}

Compare \psst{Means}, which is used for facilitative events rather than entities.

\hierAdef{Configuration}

\shortdef{Thing, usually an entity or property, that is involved 
in a static relationship to some other entity.}

Not used directly---see subtypes.

\hierBdef{Identity}

\shortdef{A category being ascribed to something, 
or something belonging to the category denoted by the governor.}

Prototypical prepositions are \p{of} (where the governor is the category) 
and \p{as} (where the object is the category):
\begin{exe}
  \ex\label{ex:stateof} the state \p{of} Washington [as opposed to the city]
  \ex The liberal state \p{of} Washington has not been receptive to Trump's message.
  \ex \p{As} a liberal state, Washington has not been receptive to Trump's message.
  \ex\label{ex:ascolleague} I like Bob \p{as} a colleague. [but not as a friend]
  \ex What a gem \p{of} a restaurant! [exclamative idiom: both NPs are indefinite]
  \ex the \choices{idea\\task\\hassle} \p{of} opening a new business
  \ex\label{ex:shell} the \choices{topic\\issue} \p{of} semantics
\end{exe}
Something may be specified with a category in order to disambiguate it \cref{ex:stateof}, 
or to provide an interpretation or frame of reference with which that entity is to be considered.
In some cases, like \cref{ex:shell}, the category is a \emph{shell noun} \citep{schmid-00} 
requiring further specification.

Categorizations may be situational rather than permanent/definitional:
\begin{exe}\ex\label{ex:assituational}\begin{xlist}
  \ex She appears \p{as} Ophelia in \emph{Hamlet}.
  \ex He is usually a bartender, but today he is working \p{as} a waiter.
\end{xlist}\end{exe}

Paraphrase test: ``(thing) IS (category) [in the context of the event]'': 
``Washington is a liberal state'', ``opening a new business is a hassle'', 
``She is Ophelia'', etc. Note that \p{as}+category may attach syntactically 
to a verb, as in \cref{ex:ascolleague} and  \cref{ex:assituational}, 
rather than being governed by the item it describes.

\begin{history}
  Generalized from v1, where it was called \sst{Instance} and restricted 
  to the ``(category) \p{of} (thing)'' formulation. 
  The relevant usages of \p{as} were labeled \sst{Attribute}.
\end{history}

\hierBdef{Species}

\shortdef{A category qualified by \w{sort}, \w{type}, \w{kind}, \w{species}, \w{breed}, etc. 
Includes \w{variety}, \w{selection}, \w{range}, \w{assortment}, etc.\ 
meaning `many different kinds'.}

\begin{exe}
  \ex that sort \p{of} business
  \ex A good type \p{of} ant to keep is the red ant .
  \ex certain strains \p{of} \emph{Escherichia coli}
  \ex Modern breeds \p{of} these homing pigeons return reliably
  \ex Some poor sap applied the wrong brand \p{of} paint
  \ex This store offers a wide selection \p{of} footstools
\end{exe}

\psst{Species} is \emph{not} used if the sort/variety noun 
is the object rather than the governor:
\begin{exe}
  \ex a business \p{of} that sort (\psst{Characteristic})
\end{exe}

\hierBdef{Gestalt}

\shortdef{Generalized notion of ``whole'' understood with reference to 
a component part, possession, set member, or characteristic. 
See \psst{Characteristic}.}

\psst{Gestalt} applies directly to:
\begin{itemize}
\item	The holder of a property if the property is the governor:
\begin{exe}
  \ex \begin{xlist} 
      \ex\begin{xlist}
        \ex the blueness \p{of} the sky
        \ex the sky\p{'s} blueness\nss{\rf{Gestalt}{Possessor}?}
      \end{xlist}
      \ex\begin{xlist}
        \ex the wisdom \p{of} the crowd
        \ex the crowd\p{'s} wisdom\nss{\rf{Gestalt}{Possessor}? also, why is this not Experiencer, like anger?}
      \end{xlist}
      \ex\begin{xlist}
        \ex the start time \p{of} the party
        \ex the party\p{'s} start time\nss{\rf{Gestalt}{Possessor}?}
      \end{xlist}
      \ex\label{ex:amountGestalt} the amount \p{of} time allowed [but see \cref{ex:QuantityGestalt}]
    \end{xlist}
  \end{exe}
\item	The wearer of attire:
\begin{exe}
  \ex the uniforms \p{of} the children
  \ex the shirt \p{on} him (\rf{Gestalt}{Locus})
\end{exe}
\item	A referent temporarily associated with another referent in the discourse 
and used to help identify it: 
\begin{exe}
  \ex Sam\p{'s} dog (= the dog that Sam mentioned seeing earlier in the conversation)
\end{exe}
\item	Anything that is borderline between subcategories \psst{Possessor} and \psst{Whole}
\item The construal \rf{Locus}{Gestalt} is used for a container denoted by the governor:
\begin{exe}
\ex the room\p{'s} 2 beds (\rf{Locus}{Gestalt})
\end{exe}
\end{itemize}

See also: \psst{Quantity}

\hierCdef{Possessor}

\shortdef{Animate who \textbf{has} something (the \psst{Possession}) 
which is not part of their body 
or inherent to their identity/character but could, in principle, be taken away.}

Prototypically expressed with \p{'s} (the \emph{s-genitive}, a case clitic) 
and \p{of} (the \emph{of-genitive}):

\begin{exe}
{\setlength\multicolsep{0pt}%
\begin{multicols}{2}
  \ex\begin{xlist}
  % of-genitive
  \ex the house \p{of} the Smith family
  \ex the corgis \p{of} Queen Elizabeth
  % s-genitive
  \sn the Smith family\p{'s} house
  \sn Queen Elizabeth\p{'s} corgis
  % \ex the rich \p{'s} money}
  % \ajb{\ex \choices{John \p{'s}\\\p{our}} dog
  % \ex ?the dog \p{of} John
  % \ex the dog \p{of} ours
\end{xlist}
\end{multicols}}
\end{exe}

See \psst{SocialRel}.

\hierCdef{Whole}

\shortdef{Something described with respect to its part, portion, subevent, subset, 
or set element. See \psst{Part/Portion}.}

\begin{exe}
  {\setlength\multicolsep{0pt}%
  \begin{multicols}{2}
  \ex \begin{xlist}
    % of-genitive
    \ex	the new engine \p{of} the car
    \ex	the flaxen hair \p{of} the girl
    \ex\label{ex:layers}	the 3 layers \p{of} the cake
    \ex\label{ex:prongs}	the 3 prongs \p{of} the strategy
    \ex the tastiest bit \p{of} the cake
    \ex the southern tip \p{of} the island
    \ex the interior \p{of} the shopping bag
    \ex the end \p{of} the journey
    \ex the 14~episodes \p{of} a TV series
    
    % s-genitive
    \sn the car\p{'s} new engine
    \sn the girl\p{'s} flaxen hair
    \sn the cake\p{'s} 3 layers
    \sn the strategy\p{'s} 3 prongs
    \sn the cake\p{'s} tastiest bit
    \sn the island\p{'s} southern tip
    \sn the shopping bag\p{'s} interior
    \sn the journey\p{'s} end
    \sn a TV series\p{'s} 14~episodes
  \end{xlist}
  \end{multicols}}
  \ex	the south \p{of} France
    %\ex 2 \p{of} my 5 daughters % Quantity ~> Whole
  \ex\label{ex:rest} The \choices{remainder\\rest} \p{of} the cake %\nss{Maybe this should be 
    % \rf{Quantity}{Whole} after all, even though ``the rest of it'' is a dubious way to answer 
    % ``How much of it?''. ``The remaining 6 ounces of cake'' certainly specifies a quantity.\ab{but is that because of "6 ounces" or "remainder (\/remaining)"?   Since "remaining of the cake" is cake too (a physical object in this case), it has the property of having quantity, but "remaining of ..." does not have "quantity" as the sense in the foreground, it's part-whole relationship.}}
    %\ex	the tennis matches \p{of} a series
    %\ex	the beginning \p{of} the party
  \ex \rf{Whole}{Locus}: \begin{xlist}
    \ex	the 14~episodes \p{in} a TV series
    \ex	the new engine \p{in} the car
    \ex the escape key \p{on} the keyboard
    \ex the flaxen hair \p{on} the girl
  \end{xlist}
  \ex	the clothes are \p{in} a pile (\rf{Whole}{Manner})
  \ex Sets and ratios:
    \begin{xlist}
      \ex This is one \p{of} the \choices{worst\\better} retaurants in town. (\psst{Whole})
      \ex 2 \p{in} 10 American children are redheads. (\rf{Whole}{Locus})
      \ex 2 \p{out\_of} 10 American children are redheads. (\rf{Whole}{Source}) %\nss{should this be \psst{RateUnit}?}\ab{maybe}
      \ex \p{Out\_of} the 10 children in the class, only Mary is a redhead. (\rf{Whole}{Source})
      \ex\label{ex:amongSet} \p{Among} the 10 children in the class, only Mary is a redhead. (\psst{Whole})
    \end{xlist}
\end{exe}

If the governor narrows the reference to a certain amount of the \psst{Whole}, 
the construal \rf{Quantity}{Whole} is used---see \cref{ex:QuantityWhole}. 
Note that this only applies if the governor is a measure term; 
it does not apply to distinctive parts like ``layers'' \cref{ex:layers} 
and ``prongs'' \cref{ex:prongs}, even if a count is specified.

Used to construe geographic and temporal ``containers'':
\begin{exe}
  \ex	Famous castles \p{of} the valley (\rf{Locus}{Whole})
  \ex \begin{xlist}
    \ex the \choices{15th\\Ides} \p{of} March (\rf{Time}{Whole})
    \ex March \p{of} 44~BC (\rf{Time}{Whole})
  \end{xlist}
\end{exe}

The prepositions \p{between} and \p{among} can impose \psst{Whole} construals 
by combining two or more items in the object NP (contrast with \cref{ex:amongSet}):
\begin{exe}
  \ex\label{ex:betweenParties}  The negotiations \choices{\p{between}\\\p{among}} the parties went well. (\rf{Agent}{Whole})
  \exp{ex:betweenParties} The negotiations \p{by} the parties went well. (\psst{Agent})
\end{exe}

\hierBdef{Characteristic}

\shortdef{Generalized notion of a part, feature, possession, 
or the contents or composition of something, 
understood with respect to that thing (the \psst{Gestalt}).}

Can be used to construe person-to-person relationships such as kinship, 
whose scene role should be \psst{SocialRel}. 
Labels \psst{Possession}, \psst{Part/Portion}, and its subtype \psst{Stuff} 
are defined for some important subclasses.

\psst{Characteristic} applies directly to:
\begin{itemize}
\item	A property value: 
\begin{exe} \ex \begin{xlist}
  \ex a car \p{of} high quality
  \ex a man \p{of} honor
  \ex a business \p{of} that sort [contrast with \psst{Species}, \cref{sec:Species}]
\end{xlist}\end{exe}
\item	Attire:
\begin{exe}
  \ex the kid \p{with} a vest (on)
  \ex the kid \p{in} a vest (\rf{Characteristic}{Locus})
\end{exe}
\item	Role of a complex framal \psst{Gestalt} that has no obvious decomposition into parts: 
\begin{exe}\ex \begin{xlist}
  \ex the restaurant \p{with} \choices{a convenient location\\an extensive menu}
  \ex a party \p{with} great music
\end{xlist}\end{exe}
\item	That which is located in a container denoted by the governor: 
\begin{exe}
  \ex a room \p{with} 2 beds
\end{exe}
\item	Anything that is borderline between subcategories \psst{Possession} and \psst{Part/Portion}
\end{itemize}

Typically, one of ``\psst{Gestalt} \{HAS, CONTAINS\} \psst{Characteristic}'' is entailed. 
This does not help to distinguish subtypes.

\begin{history}
  The v1 label \sst{Attribute} was intended to apply to features of something, 
  but was rather squishy. \nss{...}
\end{history}

\hierCdef{Possession}

\shortdef{That which some \psst{Possessor} (animate or personified, e.g.~an institution) 
\textbf{has}, and which is not part of their body or inherent to their identity/character 
but could, in principle, be taken away.}

Sometimes called \emph{alienable} possession. 
The possession may be concrete or abstract, and temporary or permanent.
Excludes attire: see \psst{Characteristic}. 

Prototypical prepositions are \p{with} and \p{without}:
\begin{exe}
\ex	People \p{with} money
\end{exe}

Immediate concrete possession uses an \psst{Accompanier} construal:
\begin{exe}
  \ex Hagrid exited the shop \p{with} (= carrying) a snowy owl. (\rf{Possessor}{Accompanier})
\end{exe}

Paraphrase test: ``\psst{Possessor} POSSESSES \psst{Possession}'', 
or ``\psst{Possessor} is IN POSSESSION OF \psst{Possession}''. 
The latter is especially appropriate for immediate concrete possession.

\hierCdef{Part/Portion}

\shortdef{A part, portion, subevent, subset, or set element (e.g., an example or exception) 
of some \psst{Whole}.}

Anything directly labeled with \psst{Part/Portion} 
is understood to be \textbf{incomplete} relative to the \psst{Whole}.
This includes body parts and partial food ingredients.

Prototypical prepositions include \p{with}, \p{without};
\p{such\_as}, \p{like} for exemplification; 
and \p{but}, \p{except}, \p{except\_for} for exceptions:
\begin{exe}
  \ex \begin{xlist}
    \ex	A car \p{with} a new engine
    \ex	A strategy \p{with} 3 prongs
    \ex	The girl \p{with} flaxen hair
    \ex	A man \p{with} a wooden leg named Smith
    \ex	A valley \p{with} a castle
    \ex	A quintet \p{with} 2 cellos
    \ex	A performance \p{with} a guitar solo
    \ex	A cake \p{with} 3 layers
    \ex	A sandwich \p{with} wheat bread
    \ex	Soup \p{with} carrots (in it)
    \ex	A chicken sandwich \p{with} ketchup (on it)
  \end{xlist}
  \ex	Bread \p{without} gluten
  \ex	Strategies \p{such\_as} divide-and-conquer
  \ex Everyone \p{except} Bob plays trombone.
\end{exe}

Some can be paraphrased with INCLUDES, but this is not determinative.


\hierDdef{Stuff}

\shortdef{The members comprising a group/ensemble, 
or the material comprising some unit of substance. 
\psst{Stuff} is distinguished from other instances of \psst{Part/Portion}
in fully covering (or ``summarizing'') the aggregate whole.}

Paraphrase test: ``\psst{Whole} CONSISTS OF \psst{Stuff}''

\begin{exe}
  \ex	\begin{xlist}
    \ex A flock \p{of} birds
    \ex	A throng \p{of} tourists
    \ex	A clump \p{of} sand
    \ex	A piece \p{of} wood
    \ex	A series \p{of} tennis matches
    \ex	An evening \p{of} Brahms
    \ex	A meal \p{of} salmon
  \end{xlist}
  \ex	A salad \choices{\p{of}\\\p{with}} mixed greens
  \ex\label{ex:bottleStuff} This bottle is \p{of} beer (and that one is of wine). [but see \cref{ex:bottleQuantity}]
  % \ex	\rf{Quantity}{Stuff}: see \cref{ex:QuantityStuff}
  %   \begin{xlist}
  %     \ex A bottle('s worth) \p{of} beer
  %     \ex A bag('s worth) \p{of} chips
  %   \end{xlist}
  \ex \rf{OrgRole}{Stuff}:
  	\begin{xlist}
      \ex An order \p{of} nuns
      \ex	A chamber group \choices{\p{of}\\\p{with}} 5 players
    \end{xlist}
\end{exe}

See also: \psst{Quantity}

\psst{Stuff} has no specific counterpart under \psst{Whole}.

\hierBdef{Accompanier}

\shortdef{Entity that another entity is together with.}

Sometimes called \emph{comitative}.

Prototypical prepositions are \p{with}, \p{without}, \p{along\_with}, 
\p{together\_with}, and \p{in\_addition\_to}:
\begin{exe}
  \ex I'll have soup \choices{\p{with}\\\p{without}} salad.
  \ex She'll be \p{with} us in spirit.
\end{exe}

For an ``extra participant'' in an activity, 
where two parties perform the activity together 
(but the nature of the activity would not fundamentally 
change if they each performed it independently), 
a \psst{Co-Agent} construal is used:
\begin{exe}
  \ex Do you want to walk \p{with} me? (\rf{Accompanier}{Co-Agent})
\end{exe}
By contrast, if the nature of the scene fundamentally requires multiple participants, 
simple \psst{Co-Agent} is used. Often there is ambiguity:\footnote{Adding \p{together} 
seems to favor the (b)~readings: \pex{I fought \p{together\_with} them}, \pex{We fought \p{together}} 
can only mean we were on the same side. Contrastive stress can also force one reading: 
\pex{I fought \p{WITH} them (not \p{AGAINST} them)}.}
\begin{exe}
  \ex Do you want to talk \p{with} me? 
  \begin{xlist}
    \ex {}[\emph{The reading:} Should we have a conversation?] (\psst{Co-Agent})
    \ex {}[\emph{The reading:} Do you want to join me in talking to a third party?] 
      (\rf{Accompanier}{Co-Agent})
  \end{xlist}
  \ex I fought \p{with} them to reform the regulation.
  \begin{xlist}
    \ex {}[\emph{The reading:} I fought against them.] (\psst{Co-Agent})
    \ex {}[\emph{The reading:} I was on the same side as them.] (\rf{Accompanier}{Co-Agent})
  \end{xlist}
\end{exe}

If the object denotes a item that the governor has on hand in their possession, 
then the construal \rf{Possession}{Accompanier} is used:
\begin{exe}
  \ex I walked in \p{with} an umbrella. (\rf{Possession}{Accompanier})
\end{exe}

See also: \psst{Instrument}, \psst{Manner}

\hierBdef{InsteadOf}

\shortdef{A default or already established thing for which something else stands in 
or is chosen as an alternative.}

\begin{exe}
  \ex I ordered soup \choices{\p{instead\_of}\\\p{rather\_than}} salad.
  \ex \p{Instead\_of} ordering salad, I ordered soup.
  %\ex They replaced \uline{my old tires} with new ones. %#nonprep
  \ex The new shirts were gray \p{instead\_of} black.
  \ex They \choices{substituted\\swapped} my old tires \p{for} new ones.
\end{exe}
May be construed spatially:
\begin{exe}
  \ex I chose soup \p{over} salad. (\rf{InsteadOf}{Locus})
\end{exe}

See also: \psst{Accompanier}, \psst{ComparisonRef}, \psst{Co-Theme}

\hierBdef{ComparisonRef}

\shortdef{The reference point in an explicit comparison (or contrast), i.e., 
an expression indicating that something is 
\textbf{similar/analogous to}, \textbf{different from}, or \textbf{the same as}
something else.}

The marker of the ``something else'' (the ground in the figure–ground relationship) 
is given the label \psst{ComparisonRef}:
\begin{exe}
  \ex \begin{xlist}
    \ex She is taller \p{than} me.
    \ex She is taller \p{than} I am.
    \ex She is taller \p{than} she is wide.
    \ex She is better at math \p{than} at drawing.
    \ex The shirt is more gray \p{than} black.
    %\ex She is greater in height \p{than} me.
  \end{xlist}
  \ex \begin{xlist}
    \ex She is as tall \p{as} I am.
    \ex Your face is as$_{\text{\psst{Characteristic}}}$ red \p{as} a rose.
    \ex Your face is red \p{as} a rose.
    \ex Your surname is the\_same \p{as} mine.
  \end{xlist}
  \ex Harry had never met anyone quite \p{like} Luna.
  \ex It was \choices{\p{as\_if}\\\p{like}} he had insulted my mother.
\end{exe}

The comparison is often made with respect to some dimension or attribute, the \psst{Characteristic}, 
which may or may not be scalar. 
The comparison may be figurative, employing simile, hyperbole, or spatial metaphor 
(\pex{close to} in the sense of `similar to'). 
The \psst{ComparisonRef} may even be a desirable or hypothetical/irrealis 
event or state (\pex{It was \p{as} it should have been}).

Prototypical prepositions include \p{than}, \p{as} (including the second item 
in the \p{as}--\p{as} construction), \p{like}, \p{unlike}. 
Prominent construals are \p{to} (\psst{Goal} for similar-thing) 
and \p{from} (\psst{Source} for dissimilar-thing).

\hierBdef{RateUnit}

\shortdef{Unit of measure in a rate expression.}

The prototypical preposition in \p{per}:

\begin{exe} \ex \begin{xlist}
  \ex The cost is \$10 \p{per} item.
  \ex A fuel efficiency of 40 miles \p{per} gallon (of gas)
\end{xlist}\end{exe}

Paraphrase: The adposition can be paraphrased as ``for each/every''.

\begin{history}
  In v1, this fell under \sst{Value}.
\end{history}

\hierBdef{Quantity}

\shortdef{Something measured by a quantity denoted by the governor.}

The governor may be a precise or vague count/measurement. 
This includes nouns like ``lack'', ``dearth'', ``shortage'', ``excess'', or ``surplus''
(meaning a too-small or too-large amount).

Question test: the governor answers ``How much/many of (object)?''

The main preposition is \p{of}.

\begin{itemize}
\item Simple \psst{Quantity}:
\begin{exe}
  \ex\label{ex:bottleQuantity}	Pour me a bottle('s worth) \p{of} beer. [but see \cref{ex:bottleStuff}]
  \ex	I have 2 years \p{of} training.
  \ex	\begin{xlist}
    \ex I ate \choices{6 ounces\\a piece} \p{of} cake.
    \ex	An ounce \p{of} compassion
  \end{xlist}
  \ex	There's a dearth \p{of} cake in the house.
  \ex	This cake has thousands \p{of} sprinkles.
  \ex They number in the tens \p{of} thousands.
  \ex	\begin{xlist}
    \ex\label{ex:anumber} I have a \choices{number\\handful} \p{of} students.
    \ex	I have a lot \p{of} students.
    \ex	We did a lot \p{of} traveling.
    \ex	There is a lot \p{of} wet sand on the beach.
  \end{xlist}
  \ex	A pair \p{of} shoes
\end{exe}

\item If the measure includes a word like ``amount'', ``quantity'', or ``number'',\footnote{Excluding 
the expression ``a number'' meaning `several', as in \cref{ex:anumber}.} 
the construal \rf{Quantity}{Gestalt} is used 
(because the amount of something can be viewed as an attribute):
\begin{exe}
  \ex\label{ex:QuantityGestalt} \rf{Quantity}{Gestalt}:
  \begin{xlist}
    \ex	A generous amount \p{of} time
    \ex A large number \p{of} students
  \end{xlist}
\end{exe}
But if ``amount'', ``quantity'', etc. is used without a measure as its modifier, 
it is simply \psst{Gestalt}: see \cref{ex:amountGestalt}.

\item If the governor is a \textbf{collective noun}, 
the construal \rf{Quantity}{Stuff} is used 
(note that a ``consisting of'' paraphrase is possible):
\begin{exe}
  \ex\label{ex:QuantityStuff} \rf{Quantity}{Stuff}:
  \begin{xlist}
    \ex Can you outrun a herd \p{of} wildebeest?
    \ex Put 3 bales \p{of} hay on the truck.
    \ex	\choices{A group\\2 groups\\A throng} \p{of} vacationers just arrived.
  \end{xlist}
\end{exe}

\item Otherwise, if the object refers to \textbf{a specific item or set}, 
and the quantity measures a portion of that item 
(whether a quantifier, absolute measure, or fractional measure),
the construal \rf{Quantity}{Whole} is used:
\begin{exe}
  \ex\label{ex:QuantityWhole} \rf{Quantity}{Whole}:
  \begin{xlist}
    \ex	I ate 6 ounces \p{of} the cake in the refrigerator.
    \ex	I ate \choices{half\\50\%} \p{of} the cake.
    \ex	\choices{All/many/lots/a lot/\\some/few/both/none} \p{of} the town's residents 
    are students.
    \ex	I have seen all \p{of} the city. (= the whole city)
    \ex	A lot \p{of} the sand on the beach is wet.
    \ex	2 \p{of} the children are redheads.
    \ex 2 \p{of} the 10 children in the class are redheads.
  \end{xlist}
\end{exe}
However, simple \psst{Whole} is used if the portion is specified as 
``the rest'', ``the remainder'', etc., as in \cref{ex:rest}. %\nss{Reconsidering this: see \cref{ex:rest}}\ab{I see "half of it", "remainder of it" do have some shared meaning. Why the three "5 ounces of it", "half of it" and "remainder of it" seem different is that in the 1st case, the quantity of the proportion itself is specified, for the 2nd case, it is unspecified but can be determined from knowledge about the quantity of the whole thing and in the 3rd case, it is unspecified and cannot be %determined from knowing the quantity of the whole thing only, you also need to know how much has already been removed.}
\end{itemize}

\hierCdef{Approximator}

\shortdef{An ``operator'' that semantically takes a measurement, 
quantity, or range as an argument and ``transforms'' it in some way 
into a new measurement, quantity, or range.}

For instance:
\begin{exe}
  \ex We have \p{about} 3 eggs left.
  \ex We have \p{in\_the\_vicinity\_of} 3 eggs left.
  \ex We have \p{over} 3 eggs left.
  \ex We have \p{between} 3 and 6 eggs left.
\end{exe}
Similarly for \p{around}, \p{under}, \p{more\_than}, \p{less\_than}, \p{greater\_than}, 
\p{fewer\_than}, \p{at\_least}, and \p{at\_most}.\footnote{These constructions are 
markedly different from most PPs; it is even questionable whether these usages 
should count as prepositions. Without getting into the details here, 
even if their syntactic status is in doubt, 
we deem it practical to assign them with a semantic label in our inventory because they 
overlap lexically with ``true'' prepositions.}

\hierBdef{SocialRel}

\shortdef{Entity, such as an institution or another individual, 
with which an individual has a stable affiliation.}

Typically, \psst{SocialRel} applies directly to relations between 
individuals.
It does not have any prototypical adpositions. 
Construals include:
\begin{exe}
  \ex \begin{xlist}
      \ex\label{ex:workwithSR} I work \p{with} Michael. (\rf{SocialRel}{Co-Agent})
      \ex Joan has a class \p{with} Miss Zarves. (\rf{SocialRel}{Co-Agent})
    \end{xlist}
  \ex\rf{SocialRel}{Possessor} \begin{xlist}
      \ex Joan is the \choices{sister\\wife} \p{of} John.
      \ex Joan is a student \p{of} Miss Zarves.
      \ex the family \p{of} Miss Zarves
      \ajb{
      Compare \p{of} and \p{'s}:
      \ex Joan is John \p{'s} \choices{sister\\wife}.
      \ex Joan is Miss Zarves \p{'s} student.
      \ex Miss Zarves \p{'} family 
      }
    \end{xlist}
  \ex Joan is studying \p{under} Prof.~Smith. (\rf{SocialRel}{Locus})
  \ex Joan is married \p{to} John. (\rf{SocialRel}{Co-Theme})
  \ex Joan is divorced \p{from} John. (\rf{SocialRel}{Co-Theme})
\end{exe}

Note, however, that \emph{work \p{with}} is ambiguous between 
being in an established professional relationship \cref{ex:workwithSR}, 
and engaging temporarily in a joint productive activity:
\begin{exe}
  \ex\label{ex:workwithCA} I was working \p{with} Michael after lunch. (\psst{Co-Agent})
\end{exe}
It is up to annotators to decide from context which interpretation 
better fits the context.

\begin{history}
  Renamed from v1 label \sst{ProfessionalAspect}, which was borrowed from 
  \citet{srikumar-13,srikumar-13-inventory}.
  The name \psst{SocialRel} reflects
  a broader set of stative relations involving an individual 
  in a social context, including kinship and friendship.
  See also note under \psst{OrgRole}.
\end{history}

\hierCdef{OrgRole}

\shortdef{Organization or institution with which an individual 
has a stable affiliation, such as membership or a business relationship.}

Like its supertype \psst{SocialRel}, \psst{OrgRole} 
lacks any prototypical adposition, but participates in numerous construals:

\begin{exe}
  \ex \rf{OrgRole}{Gestalt}:
  \begin{xlist}
      \ex the chairman \p{of} the board
      \ex the president \p{of} the U.S.
      \ex I am a loyal customer \p{of} Graeter's.
      \ex employees \p{of} Grunnings
      \ajb{
      Compare \p{of} and \p{'s}:
      \ex the board \p{'s} chairman
      \ex the U.S. \p{'} president
      \ex I am Graeter \p{'s} loyal customer
      \ex Grunnings \p{'} employees
      (\rf{OrgRole}{Possessor})}
    \end{xlist}
  \ex Mr. Dursley works \p{for} Grunnings. (\rf{OrgRole}{Beneficiary})
  \ex Mr. Dursley works \p{at} Grunnings. (\rf{OrgRole}{Locus})
  \ex Mr. Dursley is \p{from} Grunnings. (\rf{OrgRole}{Source})
  \ex Mr. Dursley is \p{with} Grunnings. (\rf{OrgRole}{Accompanier})
  \ex Mr. Dursley is employed \p{by} Grunnings. (\rf{OrgRole}{Agent}) %\nss{or do we say `employ' is just a regular Agent/Theme verb?}
  \ex I bank \p{with} TSB. (\rf{OrgRole}{Accompanier})
  \ex I serve \p{on} the committee. (\rf{OrgRole}{Locus})
\end{exe}

A family counts as an institution 
construed as a \psst{Whole} (set of its members) 
or as a \psst{Locus}:
\begin{exe}
  \ex I am the baby \p{of} the family. (\rf{OrgRole}{Whole})
  \ex people \p{in} my family (\rf{OrgRole}{Whole})
\end{exe}

For a relation between a unit and a larger institution, 
use \psst{Whole}:
\begin{exe}
  \ex the Principals Committee \p{of} the National Security Council (\psst{Whole})
\end{exe}

See also: \psst{Stuff}

\begin{history}
  \psst{OrgRole} is now distinguished within the broader \psst{SocialRel} category 
  following the precedent of the Abstract Meaning Representation \citep[AMR;][]{amr,amr-guidelines}. 
  In AMR, \texttt{have-org-role-91} captures relations between 
  an individual and an institution (such as an organization or family),
  whereas \texttt{have-rel-role-91} is used for relations between two individuals.
\end{history}


\bibliographystyle{plainnat}
\bibliography{psst2.bib}


%\printbibliography[maxnames=99]


\end{document}
