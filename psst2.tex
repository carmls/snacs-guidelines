\PassOptionsToPackage{usenames}{color}
\pdfoutput=1 % ensure pdflatex (for arXiv)
\documentclass[11pt,letterpaper]{article}
\usepackage{comment}
\usepackage{relsize} % relative font sizes (e.g. \smaller). must precede ACL style
%\usepackage{style/acl2012}

\usepackage[linkcolor=blue]{hyperref}

\usepackage[round]{natbib}
\begin{comment}
\usepackage[style=authoryear-comp,natbib=true,hyperref=true]{biblatex}

% tell biblatex not to quote titles in the bibliography
\DeclareFieldFormat{title}{#1} % don't italicize titles by default
\DeclareFieldFormat[book]{title}{\mkbibemph{#1}\isdot} % but do italicize books
\DeclareFieldFormat[article]{title}{#1\isdot}
\DeclareFieldFormat[inbook]{title}{#1\isdot}
\DeclareFieldFormat[incollection]{title}{#1\isdot}
\DeclareFieldFormat[inproceedings]{title}{#1\isdot}
\DeclareFieldFormat[patent]{title}{#1\isdot}
\DeclareFieldFormat[thesis]{title}{#1\isdot}
\DeclareFieldFormat[unpublished]{title}{#1\isdot}
% ...and for articles, to use number(issue) instead of number.issue
\renewbibmacro*{journal+issuetitle}{%
  \usebibmacro{journal}%
  \setunit*{\addspace}%
  \iffieldundef{series}
    {}
    {\newunit
     \printfield{series}%
     \setunit{\addspace}}%
  \printfield{volume}%
%  \setunit*{\adddot}%
  \printfield{number}%
  \setunit{\addcomma\space}%
  \printfield{eid}%
  \setunit{\addspace}%
  \usebibmacro{issue+date}%
  \newunit\newblock
  \usebibmacro{issue}%
  \newunit}
\DeclareFieldFormat[article]{number}{\mkbibparens{#1}}
% use ``pages'' instead of ``pp.''
\DefineBibliographyStrings{english}{%
    pages  =  {pages} % for multiple page numbers
}
% but for articles, just use a colon
\DeclareFieldFormat[article]{pages}{:#1} %TODO. check with LeCun citation
% and don't put a colon after In
\renewbibmacro*{in:}{%
  \bibstring{in}%\addcolon
  \setunit{\space}}

\DeclareFieldFormat{label}{#1\isdot}
\renewbibmacro*{year+labelyear}{%
  \iffieldundef{year}
    {}
    {\printtext{%
       \printfield{year}%
       \printfield{labelyear}%
       \setunit{\adddot}}}}

\bibliography{features.bib}
\end{comment}


%\usepackage{times}
%\usepackage{latexsym}


\usepackage[boxed]{algorithm2e}
\renewcommand\AlCapFnt{\small}
\usepackage[small,bf,skip=5pt]{caption}
\usepackage{sidecap} % side captions
\usepackage{rotating}	% sideways

% customize \paragraph spacing
\makeatletter
\renewcommand{\paragraph}{%
  \@startsection{paragraph}{4}%
  {\z@}{3.25ex \@plus 1ex \@minus .2ex}{-1em}% reduce 3.25 to .2 to minimize space
  {\normalfont\normalsize\bfseries}%
}
\makeatother

% Italicize subparagraph headings
\usepackage[nobottomtitles*]{titlesec}
\titleformat*{\subparagraph}{\itshape}
\titlespacing{\subparagraph}{%
  1em}{%              left margin
  0pt}{% space before (vertical)
  1em}%               space after (horizontal)


% MOVE SECTION NUMBERS INTO LEFT MARGIN
% They will be roman, right-aligned and separated from the heading text by .2cm
% adapted from http://tex.stackexchange.com/a/311712
\titleformat{\section}[block]
  {\Large\bfseries}
  {}
  {0pt}
  {\hspace{-1.2cm}% Move into margin
   \makebox[1cm][r]{\normalfont\thesection}\hspace{.2cm}}% Set number + title
\titleformat{\subsection}[block]
  {\large\bfseries}
  {}
  {0pt}
  {\hspace{-1.2cm}% Move into margin
   \makebox[1cm][r]{\normalfont\thesubsection}\hspace{.2cm}}% Set number + title
\titleformat{\subsubsection}[block]
  {\normalsize\bfseries}
  {}
  {0pt}
  {\hspace{-1.2cm}% Move into margin
   \makebox[1cm][r]{\normalfont\thesubsubsection}\hspace{.2cm}}% Set number + title

%\usepackage{lingmacros}
% Lists

\usepackage{enumitem} % customizable lists
\setitemize{noitemsep,topsep=0em} %,leftmargin=*
\setenumerate{noitemsep,leftmargin=0em,itemindent=13pt,topsep=0em}

\usepackage{adjustbox}
\newcommand{\choices}[1]{\adjustbox{stack=ct}{#1}}  % for placing alternatives 
% inline with an example. e.g. \choices{to\\from\\with}
% ct = horizontally-centered, top

\usepackage{textcomp}
% \usepackage{arabtex} % must go after xparse, if xparse is used!
%\usepackage{utf8}
% \setcode{utf8} % use UTF-8 Arabic
% \newcommand{\Ar}[1]{\RL{\novocalize #1}} % Arabic text

\usepackage[procnames]{listings}

\usepackage{amssymb}	%amsfonts,eucal,amsbsy,amsthm,amsopn
\usepackage{amsmath}

%\usepackage{mathptmx}	% txfonts
\usepackage{fourier}
\usepackage[scaled=.87]{helvet}
\usepackage[scaled=.8]{beramono}
\usepackage[T1]{fontenc}
\usepackage[utf8x]{inputenc}

\usepackage{MnSymbol}	% must be after mathptmx

\usepackage{latexsym}





% Tables
\usepackage{array}
\usepackage{multirow}
\usepackage{booktabs} % pretty tables
\usepackage{multicol}
\usepackage{footnote}


\usepackage{url}
\usepackage[usenames]{color}
\usepackage{xcolor}

% colored frame box
\newcommand{\cfbox}[2]{%
    \colorlet{currentcolor}{.}%
    {\color{#1}%
    \fbox{\color{currentcolor}#2}}%
}

\usepackage[normalem]{ulem} % \uline
\usepackage{colortbl}
\usepackage{graphicx}
\usepackage{subcaption}
\usepackage{mdframed}

%\usepackage{tikz-dependency}
\usepackage{tikz}
\usepackage[edges]{forest}
%\usepackage{tree-dvips}
\usetikzlibrary{arrows,positioning,calc} 

\DeclareMathOperator*{\argmax}{arg\,max}
\DeclareMathOperator*{\argmin}{arg\,min}



% Author comments
\usepackage{color}
\newcommand\bmmax{0} % magic to avoid 'too many math alphabets' error
\usepackage{bm}
\definecolor{orange}{rgb}{1,0.5,0}
\definecolor{mdgreen}{rgb}{0,0.6,0}
\definecolor{mdblue}{rgb}{0,0,0.7}
\definecolor{dkblue}{rgb}{0,0,0.5}
\definecolor{dkgray}{rgb}{0.3,0.3,0.3}
\definecolor{slate}{rgb}{0.25,0.25,0.4}
\definecolor{gray}{rgb}{0.5,0.5,0.5}
\definecolor{ltgray}{rgb}{0.7,0.7,0.7}
\definecolor{ltltgray}{rgb}{0.9,0.9,0.9}
\definecolor{purple}{rgb}{0.7,0,1.0}
\definecolor{lavender}{rgb}{0.65,0.55,1.0}

% Settings for algorithm listings
\makeatletter
\lst@AddToHook{EveryPar}{%
  \label{lst:\thelstnumber}% make a label for each line number except the first (assumes only one listing in the document)
}
\makeatother
\lstset{
% basicstyle=\rmshape,
  numbers=left,
  numberstyle=\tt\color{gray},
  firstnumber=2,
  stepnumber=5,
  xleftmargin=3em,
  language=Python,
  upquote=true,
  showstringspaces=false,
  formfeed=\newpage,
  tabsize=1,
  stringstyle=\color{mdgreen},
  commentstyle=\itshape\color{lavender},
  basicstyle=\small\smaller\ttfamily,
  morekeywords={lambda,with,as,assert},
  keywordstyle=\bfseries\color{magenta},
  procnamekeys={def},
  procnamestyle=\bfseries\color{orange},
  aboveskip=0.5cm,
  belowskip=0.5cm
}
\renewcommand{\lstlistingname}{Algorithm}


\newcommand{\ensuretext}[1]{#1}
\newcommand{\cjdmarker}{\ensuretext{\textcolor{green}{\ensuremath{^{\textsc{CJ}}_{\textsc{D}}}}}}
\newcommand{\nssmarker}{\ensuretext{\textcolor{magenta}{\ensuremath{^{\textsc{NS}}_{\textsc{S}}}}}}
\newcommand{\nasmarker}{\ensuretext{\textcolor{red}{\ensuremath{^{\textsc{NA}}_{\textsc{S}}}}}}
\newcommand{\bomarker}{\ensuretext{\textcolor{blue}{\ensuremath{^{\textsc{B}}_{\textsc{O}}}}}}
\newcommand{\jbmarker}{\ensuretext{\textcolor{orange}{\ensuremath{^{\textsc{J}}_{\textsc{B}}}}}}
\newcommand{\dbmarker}{\ensuretext{\textcolor{purple}{\ensuremath{^{\textsc{D}}_{\textsc{B}}}}}}
\newcommand{\arkcomment}[3]{\ensuretext{\textcolor{#3}{[#1 #2]}}}
%\newcommand{\arkcomment}[3]{}
\newcommand{\cjd}[1]{\arkcomment{\cjdmarker}{#1}{green}}
\newcommand{\nss}[1]{\arkcomment{\nssmarker}{#1}{magenta}}
\newcommand{\nas}[1]{\arkcomment{\nasmarker}{#1}{red}}
\newcommand{\bo}[1]{\arkcomment{\bomarker}{#1}{blue}}
\newcommand{\jb}[1]{\arkcomment{\jbmarker}{#1}{orange}}
\newcommand{\db}[1]{\arkcomment{\dbmarker}{#1}{purple}}
\newcommand{\params}{\mathbf{\theta}}
\newcommand{\wts}{\mathbf{w}}
\newcommand{\g}{\mathbf{g}}
\newcommand{\f}{\mathbf{f}}
\newcommand{\x}{\mathbf{x}}
\newcommand{\y}{\mathbf{y}}
\newcommand{\overbar}[1]{\mkern 1.5mu\overline{\mkern-1.5mu#1\mkern-1.5mu}\mkern 1.5mu} % \bar is too narrow in math
\newcommand{\cost}{c}

\newcommand{\citeposs}[2][]{\citeauthor{#2}'s (\citeyear[#1]{#2})}
\newcommand{\Citeposs}[2][]{\Citeauthor{#2}'s (\citeyear[#1]{#2})}

\usepackage{nameref}
\usepackage{cleveref}

% use \S for all references to all kinds of sections, and \P to paragraphs
% (sadly, we cannot use the simpler \crefname{} macro because it would insert a space after the symbol)
\crefformat{part}{\S#2#1#3}
\crefformat{chapter}{\S#2#1#3}
\crefformat{section}{\S#2#1#3}
\crefformat{subsection}{\S#2#1#3}
\crefformat{subsubsection}{\S#2#1#3}
\crefformat{paragraph}{\P#2#1#3}
\crefformat{subparagraph}{\P#2#1#3}
%\crefmultiformat{part}{\S#2#1#3}{ and~\S#2#1#3}{, \S#2#1#3}{, and~\S#2#1#3}
%\crefmultiformat{chapter}{\S#2#1#3}{ and~\S#2#1#3}{, \S#2#1#3}{, and~\S#2#1#3}
\crefmultiformat{section}{\S#2#1#3}{ and~\S#2#1#3}{, \S#2#1#3}{, and~\S#2#1#3}
\crefmultiformat{subsection}{\S#2#1#3}{ and~\S#2#1#3}{, \S#2#1#3}{, and~\S#2#1#3}
\crefmultiformat{subsubsection}{\S#2#1#3}{ and~\S#2#1#3}{, \S#2#1#3}{, and~\S#2#1#3}
\crefmultiformat{paragraph}{\P\P#2#1#3}{ and~#2#1#3}{, #2#1#3}{, and~#2#1#3}
\crefmultiformat{subparagraph}{\P\P#2#1#3}{ and~#2#1#3}{, #2#1#3}{, and~#2#1#3}
%\crefrangeformat{part}{\mbox{\S\S#3#1#4--#5#2#6}}
%\crefrangeformat{chapter}{\mbox{\S\S#3#1#4--#5#2#6}}
\crefrangeformat{section}{\mbox{\S\S#3#1#4--#5#2#6}}
\crefrangeformat{subsection}{\mbox{\S\S#3#1#4--#5#2#6}}
\crefrangeformat{subsubsection}{\mbox{\S\S#3#1#4--#5#2#6}}
\crefrangeformat{paragraph}{\mbox{\P\P#3#1#4--#5#2#6}}
\crefrangeformat{subparagraph}{\mbox{\P\P#3#1#4--#5#2#6}}
% for \label[appsec]{...}
\crefname{part}{Part}{Parts}
\Crefname{part}{Part}{Parts}
\crefname{chapter}{ch.}{ch.}
\Crefname{chapter}{Ch.}{Ch.}
\crefname{figure}{figure}{figures}
\crefname{subfigure}{figure}{figures}
\Crefname{subfigure}{Figure}{Figures}
\crefname{appsec}{appendix}{appendices}
\Crefname{appsec}{Appendix}{Appendices}
\crefname{algocf}{algorithm}{algorithms}
\Crefname{algocf}{Algorithm}{Algorithms}
\crefname{enums}{example}{examples}
\Crefname{enums}{Example}{Examples}
\crefname{enumsi}{example}{examples}
\Crefname{enumsi}{Example}{Examples}
\crefname{}{example}{examples} % lingmacros \toplabel has no internal name for the kind of label
\Crefname{}{Example}{Examples}
\crefformat{enums}{(#2#1#3)}
\crefformat{enumsi}{(#2#1#3)}
\crefformat{}{(#2#1#3)}
\crefname{xnumi}{example}{examples} % gb4e
\crefname{xnumi}{example}{examples} % gb4e
\Crefname{xnumii}{Example}{Examples} % gb4e
\Crefname{xnumii}{Example}{Examples} % gb4e
\crefformat{xnumi}{(#2#1#3)} % gb4e
\crefformat{xnumii}{(#2#1#3)} % gb4e
\crefrangeformat{enums}{\mbox{(#3#1#4--#5#2#6)}}
\crefrangeformat{enumsi}{\mbox{(#3#1#4--#5#2#6)}}
\crefrangeformat{xnumi}{\mbox{(#3#1#4--#5#2#6)}} % gb4e
\crefrangeformat{xnumii}{\mbox{(#3#1#4--#5#2#6)}} % gb4e

\ifx\creflastconjunction\undefined%
\newcommand{\creflastconjunction}{, and\nobreakspace} % Oxford comma for lists
\else%
\renewcommand{\creflastconjunction}{, and\nobreakspace} % Oxford comma for lists
\fi%

\newcommand*{\Fullref}[1]{\hyperref[{#1}]{\Cref*{#1}: \nameref*{#1}}}
\newcommand*{\fullref}[1]{\hyperref[{#1}]{\cref*{#1}: \nameref{#1}}}
\newcommand{\fnref}[1]{fn.~\ref{#1}} % don't use \cref{} due to bug in (now out-of-date) cleveref package w.r.t. footnotes
\newcommand{\Fnref}[1]{Fn.~\ref{#1}}

%\captionsetup[subfigure]{labelformat=simple}
\renewcommand\thesubfigure{(\alph{subfigure})}

% Space savers
% From http://www.eng.cam.ac.uk/help/tpl/textprocessing/squeeze.html
%\addtolength{\dbltextfloatsep}{-.5cm} % space between last top float or first bottom float and the text.
%\addtolength{\intextsep}{-.5cm} % space left on top and bottom of an in-text float.
%\addtolength{\abovedisplayskip}{-.5cm} % space before maths
%\addtolength{\belowdisplayskip}{-.5cm} % space after maths
%\addtolength{\topsep}{-.5cm} %space between first item and preceding paragraph
%\setlength{\belowcaptionskip}{-.25cm}

\usepackage{gb4e} % linguistic examples. put after all other package imports






% Special macros
\newcommand{\w}[1]{\textit{#1}}	% word
\newcommand{\p}[1]{\textbf{\textsf{#1}}} % preposition type
\newcommand{\lbl}[1]{\textsc{#1}} % class label
\newcommand{\sst}[1]{\lbl{#1}} % supersense tag label
\newcommand{\nsst}[1]{\sst{n:#1}} % noun supersense tag label
\newcommand{\vsst}[1]{\sst{v:#1}} % verb supersense tag label
\newcommand{\psst}[1]{\textcolor{mdgreen}{\hyperref[sec:#1]{\sst{#1}}}} % preposition supersense tag label
\newcommand{\olbl}[1]{\textcolor{purple}{\textrm{#1}}} % other label: `i, `d, etc.
%\newcommand{\nsst}[1]{\sst{#1~\textroundcap{\vphantom{-}}~}} % noun supersense tag label
%\newcommand{\vsst}[1]{\sst{#1\raisebox{-1.5pt}{\textasciicaron}}} % verb supersense tag label
%\newcommand{\psst}[1]{\sst{#1\raisebox{2pt}{\rotatebox{180}{\textsublhalfring{\phantom{.}}}}}} %\textcorner % preposition supersense tag label

\newcommand{\rf}[2]{\psst{#1}$\leadsto$\psst{#2}}
\newcommand{\rff}[3]{\psst{#1}$\leadsto$\psst{#2}$\leadsto$\psst{#3}}


\newcommand{\tg}[1]{\texttt{#1}}	% supersense tag name
\newcommand{\gfl}[1]{%\renewcommand\texttildelow{{\lower.74ex\hbox{\texttt{\char`\~}}}} % http://latex.knobs-dials.com/
\mbox{\textsmaller{\texttt{#1}}}}	% supersense tag symbol
 \newcommand{\tagdef}[1]{#1\hfill} % tag definition
\newcommand{\tagt}[2]{\ensuremath{\underset{\textrm{\textlarger{\tg{#2}}}\strut}{\w{#1}\rule[-.3\baselineskip]{0pt}{0pt}}}} % tag text (a word or phrase) with an SST. (second arg is the tag)
\newcommand{\glosst}[2]{\ensuremath{\underset{\textrm{#2}}{\textrm{#1}}}} % gloss text (a word or phrase) (second arg is the gloss)
\newcommand{\AnnA}[0]{\mbox{\textbf{Ann-A}}} % annotator A
\newcommand{\AnnB}[0]{\mbox{\textbf{Ann-B}}} % annotator B
\newcommand{\sys}[1]{\mbox{\textbf{#1}}}   % name of a system (one of our experimental conditions)
\newcommand{\dataset}[1]{\mbox{\textsc{#1}}}	% one of the datasets in our experiments
\newcommand{\datasplit}[1]{\mbox{\textbf{#1}}}	% portion one of the datasets in our experiments

\newcommand{\fnf}[1]{\textsc{\textsf{#1}}} % FrameNet frame
\newcommand{\fnr}[1]{\textbf{\textsf{#1}}} % FrameNet role (frame element name)
\newcommand{\fnrel}[1]{\textsl{#1}} % FrameNet frame relation type
\newcommand{\fnst}[1]{\textsl{#1}} % FrameNet semantic type
\newcommand{\fnlu}[1]{\textsf{#1}} % FrameNet lexical unit (predicate)
\newcommand{\pbf}[1]{\mbox{\textsf{#1}}} % PropBank frame (roleset)
\newcommand{\pbr}[1]{\textbf{\textsf{#1}}} % PropBank role (numbered or modifier argument label)
\newcommand{\vpred}[1]{\textbf{#1}} % verb predicate


%\newcommand{\lex}[1]{\textsmaller{\textsf{\textcolor{slate}{\textbf{#1}}}}}	% example lexical item
\newcommand{\lex}[1]{\textit{#1}} % lexical item/lexical example
\newcommand{\pex}[1]{\textit{#1}} % phrasal example - don't index by default

%\newcommand{\w}[1]{\textit{#1}}	% word
\newcommand{\gap}[0]{\ \ } % space around gap contents
\newcommand{\tat}[0]{\textasciitilde}

\newcommand{\shortlong}[2]{#1} % short vs. long version of the paper
\newcommand{\confversion}[1]{#1}
%\newcommand{\finalversion}[1]{#1}
\newcommand{\finalversion}[1]{}
\newcommand{\futureversion}[1]{}
\newcommand{\shortversion}[1]{#1}
\newcommand{\considercutting}[1]{#1}
\newcommand{\longversion}[1]{} % ...if only there were more space...
\newcommand{\genitiveversion}[1]{}
\newcommand{\subversion}[1]{#1} % for the submission version only
\newcommand{\draftnotice}[1]{#1} % for the draft version only
%\newcommand{\subversion}[1]{}

\newenvironment{ggroup}{{}}{{}}

\newcommand{\shortdef}[1]{\begin{mdframed}\noindent\textlarger{#1}\end{mdframed}}

\newenvironment{history}{\begin{mdframed}[linecolor=ltltgray,backgroundcolor=ltltgray]\small\noindent\textit{History.}}{\end{mdframed}}



\newcommand{\hierA}[1]{\textcolor{red}{\hyperref[sec:#1]{#1}}}
\newcommand{\hierB}[1]{\textcolor{blue}{\hyperref[sec:#1]{#1}}}
\newcommand{\hierC}[1]{\textcolor{mdgreen}{\hyperref[sec:#1]{#1}}}
\newcommand{\hierD}[1]{\textcolor{orange}{\hyperref[sec:#1]{#1}}}

\newcommand{\hierAdef}[1]{\section{\psst{#1}}\label{sec:#1}}
\newcommand{\hierBdef}[1]{\subsection{\psst{#1}}\label{sec:#1}}
\newcommand{\hierCdef}[1]{\subsubsection{\psst{#1}}\label{sec:#1}}
\newcommand{\hierDdef}[1]{\paragraph{\psst{#1}}\label{sec:#1}}

\hyphenation{WordNet}
\hyphenation{WordNets}
\hyphenation{FrameNet}
\hyphenation{SemCor}
\hyphenation{SemEval}
\hyphenation{ParsedSemCor}
\hyphenation{VerbNet}
\hyphenation{PennConverter}
\hyphenation{an-aly-sis}
\hyphenation{an-aly-ses}
\hyphenation{news-text}
\hyphenation{base-line}
\hyphenation{de-ve-lop-ed}
\hyphenation{comb-over}
\hyphenation{per-cept}
\hyphenation{per-cepts}
\hyphenation{post-edit-ing}
\hyphenation{shriv-eled}
\hyphenation{Huddle-ston}

\title{\draftnotice{{\it\small WORKING DRAFT}\\[5pt] 
Adposition Supersenses v2}}

\newcommand{\emldisplay}[2]{\texttt{\href{mailto:#1}{#2}}}
\newcommand{\eml}[1]{\textsmaller{\emldisplay{#1}{#1}}}

\author{\hspace{.2cm}\textbf{Nathan Schneider}\hspace{.2cm} \\ 
  \hspace{.2cm}Georgetown University\hspace{.2cm} \\
     \hspace{.2cm}\eml{nathan.schneider@georgetown.edu}\hspace{.2cm} \and
\textbf{Jena D. Hwang} \quad
\textbf{Archna Bhatia} \\
 	IHMC \\
     {\smaller \{\emldisplay{jhwang@ihmc.us}{jhwang},\emldisplay{abhatia@ihmc.us}{abhatia}\}\texttt{@ihmc.us}} \and
\textbf{Na-Rae Han} \\
	\hspace{.75cm}University of Pittsburgh\hspace{.75cm} \\
    \eml{naraehan@pitt.edu} \and 
\textbf{Vivek Srikumar} \\
	\hspace{1.25cm}University of Utah\hspace{1.25cm} \\
    \eml{svivek@cs.utah.edu} \and
\textbf{Tim O'Gorman} \\
  \hspace{.1cm}University of Colorado Boulder\hspace{.1cm} \\
    \eml{timothy.ogorman@colorado.edu} \and
\textbf{Omri Abend} \\
  \hspace{.1cm}Hebrew University of Jerusalem\hspace{.1cm} \\
    \eml{oabend@cs.huji.ac.il}
}

\date{}

\begin{document}
\maketitle
\begin{abstract}
\noindent 
This document describes an inventory of 50~semantic labels 
designed to characterize the use of adpositions and case markers 
at a somewhat coarse level of granularity. 
Version~2 is a revision of the supersense inventory proposed for English by 
\citet{schneider-15,schneider-16} (henceforth ``v1''), which in turn was based on previous schemes.
The present inventory was developed after extensive review of the 
v1 corpus annotations for English, as well as consideration of adposition 
and case phenomena in Hebrew, Hindi, and Korean.
Examples in this document are limited to English; 
a multilingual and more detailed online lexical resource is forthcoming.
\end{abstract}

\section{Overview}

v1 is documented in PrepWiki (\url{http://tiny.cc/prepwiki}).

v2 will be documented in Xposition (URL TBD).

\subsection{What counts as an adposition?}

``Adposition'' is the cover term for prepositions and postpositions. 
Briefly, we consider an affix, word, or multiword expression to be an adposition if it:
\begin{itemize}
  \item Mediates a semantically asymmetric figure--ground relation between two concepts
  \item Is a grammatical item that can mark an NP, and in some cases may mark clauses (as a subordinator) 
  or be intransitive. 
  We also include always-intransitive grammatical items whose core meaning is spatial and highly schematic, 
  like English \p{together}, \p{apart}, and \p{away}.
  \item Is not a differential object marker (e.g., Hebrew \p{'et}, which marks direct objects if and only if 
  they are definite).
\end{itemize}
\nss{What about a word that matches the above criteria where it is used as an intransitive 
predicate, e.g. \pex{She is \p{out}/\p{away}}?}

\subsection{Inventory}

The v2 hierarchy is a tree with 50~labels.
They are organized into three major subhierarchies: 
\psst{Circumstance} (18~labels), \psst{Participant} (14~labels), 
and \psst{Configuration} (18~labels). 

\begin{minipage}{\textwidth}
\begin{multicols}{3}
\begin{ggroup}
  \sffamily\color{gray}
\begin{forest}
  for tree={%
    folder,
    grow'=0,
    fit=band,
    inner ysep=.75,
  }
  [{\hierA{Circumstance}}
    [{\hierB{Temporal}}
      [{\hierC{Time}}
        [{\hierD{StartTime}}]
        [{\hierD{EndTime}}]
        [{\hierD{DeicticTime}}]
      ]
      [{\hierC{Frequency}}]
      [{\hierC{Duration}}]
    ]
    [{\hierB{Locus}}
      [{\hierC{Source}}]
      [{\hierC{Goal}}]
    ]
    [{\hierB{Path}}
      [{\hierC{Direction}}]
      [{\hierC{Extent}}]
    ]
    [{\hierB{Means}}]
    [{\hierB{Manner}}]
    [{\hierB{Explanation}}
      [{\hierC{Purpose}}]
    ]
  ]
\end{forest}
\columnbreak

\begin{forest}
  for tree={%
    folder,
    grow'=0,
    fit=band,
    inner ysep=.75,
  }
  [{\hierA{Participant}}
    [{\hierB{Causer}}
      [{\hierC{Agent}}
        [{\hierD{Co-Agent}}]
      ]
    ]
    [{\hierB{Theme}}
      [{\hierC{Co-Theme}}]
      [{\hierC{Topic}}]
    ]
    [{\hierB{Stimulus}}]
    [{\hierB{Experiencer}}]
    [{\hierB{Originator}}]
    [{\hierB{Recipient}}]
    [{\hierB{Cost}}]
    [{\hierB{Beneficiary}}]
    [{\hierB{Instrument}}]
  ]
\end{forest}
\columnbreak

\begin{forest}
  for tree={%
    folder,
    grow'=0,
    fit=band,
    inner ysep=.75,
  }
  [{\hierA{Configuration}}
    [{\hierB{Identity}}]
    [{\hierB{Species}}]
    [{\hierB{Gestalt}}
      [{\hierC{Possessor}}]
      [{\hierC{Whole}}]
    ]
    [{\hierB{Characteristic}}
      [{\hierC{Possession}}]
      [{\hierC{Part/Portion}}
        [{\hierD{Stuff}}]
      ]
    ]
    [{\hierB{Accompanier}}]
    [{\hierB{InsteadOf}}]
    [{\hierB{ComparisonRef}}]
    [{\hierB{RateUnit}}]
    [{\hierB{Quantity}}
      [{\hierC{Approximator}}]
    ]
    [{\hierB{SocialRel}}
      [{\hierC{OrgRole}}]
    ]
  ]
\end{forest}
\end{ggroup}
\end{multicols}
\end{minipage}

\begin{itemize}
\item Items in the \psst{Circumstance} subhierarchy are prototypically 
expressed as adjuncts of time, place, manner, purpose, etc.\ 
elaborating an event or entity.
\item Items in the \psst{Participant} subhierarchy are prototypically 
entities functioning as arguments to an event.
\item Items in the \psst{Configuration} subhierarchy are prototypically
entities or properties in a static relationship to some entity.
\end{itemize}

\subsection{Limitations}

This inventory is only designed to capture semantic relations 
with a figure--ground asymmetry. This excludes:
\begin{itemize}
  \item The semantics of coordination, where the two sides of the relation 
are on equal footing, are not captured here. (Note that sometimes a morpheme can 
have symmetric as well as asymmetric interpretations: e.g., Korean \p{-wa}).
  \item Aspects of meaning that pertain to information structure, discourse, 
or pragmatics.
\end{itemize}
Moreover, this inventory only captures semantic distinctions 
that tend to correlate with major differences in syntactic distribution. 
Thus, while there are labels for locative (\psst{Locus}), ablative (\psst{Source}), 
allative (\psst{Goal}), and \psst{Path} semantics---and analogous temporal categories---%
finer-grained details of spatiotemporal meaning are for the most part lexical 
(viz.: the difference between \pex{\p{in} the box} and \pex{\p{on} the box}, 
or temporal \p{at}, \p{before}, \p{during}, and \p{after}) and are not represented here.\footnote{This is not to claim
that all members of a category can be grammatical in all the same contexts: 
\pex{\p{on} Saturday} and \pex{\p{at} 5:00} are both labeled \psst{Time}, 
though the prepositions are by no means interchangeable in American English. 
We are simply asserting that the different constructions specific to days of the week 
versus times of the day are minor aspects of the grammar of English.}

\subsection{Major changes from v1}

Changes that affect only a single label are explained below the relevant 
v2 labels.

\begin{itemize}
  \item \textbf{Removed multiple inheritance.} 
  The v1 network was quite tangled. The structure is greatly simplified 
  by analyzing some tokens as \emph{construals} \citep{hwang-17}.
  \item \textbf{Revised and expanded the \psst{Configuration} subhierarchy.}
  \item \textbf{Removed the locative concreteness distinction.}
  In v1, labels \sst{Location}, \sst{InitialLocation}, and \sst{Destination} 
  were reserved for concrete locations, and the respective supertypes 
  \sst{Locus}, \sst{Source}, and \sst{Goal} used to cover abstract locations.
  This distinction was found to be difficult and without apparent
  relevance to the English preposition system. The concrete labels were thus removed.
  \item \textbf{Removed the location/state/value distinction.}
  The v1 scheme attempted to make an elaborate distinction between 
  values, states, and other kinds of abstract locations. 
  However, the English preposition system does not seem particularly 
  sensitive to these distinctions. (We are not aware of any prepositions 
  that mark primarily values or primarily states; rather, productive 
  metaphors allow locative prepositions to be extended to cover these, 
  and there are cases where teasing apart abstract location vs.~state vs.~value 
  is difficult.) Therefore, \sst{State}, \sst{InitialState}, \sst{EndState}, 
  \sst{Value}, and \sst{ValueComparison} were removed.
  \item \textbf{Revised the treatment of comparison and related notions.} 
  \sst{Comparison/Contrast}, \sst{Scalar/Rank}, \sst{ValueComparison}; 
  moved \psst{Approximator} under \psst{Quantity}
  \item \textbf{Greatly simplified the \psst{Path} subhierarchy.} See \cref{sec:Path}.
  \item \textbf{Simplified the \psst{Temporal} subhierarchy.} See \cref{sec:Temporal}.
\end{itemize}

\hierAdef{Circumstance}

\hierBdef{Temporal}

\hierCdef{Time}

\hierDdef{StartTime}

\hierDdef{EndTime}

\hierDdef{DeicticTime}

\hierCdef{Frequency}

\hierCdef{Duration}

\hierBdef{Locus}

\hierCdef{Source}

\shortdef{Initial location, condition, or value. May be abstract.}

Prototypically inanimate, though it can be used to construe animate \psst{Participant}s 
(especially \psst{Originator} and \psst{Causer}).
Contrasts with \psst{Goal}.

\hierCdef{Goal}

\shortdef{Final location, condition, or value. May be abstract.}

Prototypically inanimate, though it can be used to construe animate \psst{Participant}s 
(especially \psst{Recipient}).
Contrasts with \psst{Source}.

\hierBdef{Path}

\hierCdef{Direction}

\hierCdef{Extent}

\hierBdef{Means}

\hierBdef{Manner}

\hierBdef{Explanation}

\hierCdef{Purpose}




\hierAdef{Participant}

\hierBdef{Causer}

\shortdef{Instigator of, and a core participant in, an event.}

\psst{Causer} is applied directly to inanimate things or forces conceptualized as entities. 
Prototypical prepositions are \p{by} (prominently including passive-\p{by}) and \p{of}:
\begin{exe}
  \ex the devastation of the town wreaked \p{by} the fire
  \ex the devastation \p{of} the fire
\end{exe}
The \psst{Causer} is sometimes construed as a \psst{Source}:
\begin{exe}
  \ex \begin{xlist}
    \ex the devastation \p{from} the fire (\rf{Causer}{Source})
    \ex fatalities \p{from} cancer (\rf{Causer}{Source})
    \ex FDR suffered \p{from} polio. (\rf{Causer}{Source})
  \end{xlist}
\end{exe}

See also: \psst{Instrument}

\hierCdef{Agent}

\shortdef{Animate instigator of an action (typically volitional).}

Prototypical prepositions are \p{by} (prominently including passive-\p{by}) and \p{of}:
\begin{exe}
  \ex the decisive vote \choices{\p{by}\\\p{of}} the City Council
\end{exe}
When two symmetric \psst{Agent}s are collected in a single NP 
functioning as a set, it is marked as a \psst{Whole} construal:
\begin{exe}
  \ex There was a war \p{between} France and Spain. (\rf{Agent}{Whole})
  \ex This is a discussion \p{among} friends. (\rf{Agent}{Whole})
%  \ex Please talk \p{amongst} yourselves. \rf{Agent}{Whole}
% reflexives are weird. maybe Co-Agent
\end{exe}

Compare: \psst{Co-Agent}; 
see also: \psst{OrgRole}, \psst{Originator}, \psst{Stimulus}


\hierDdef{Co-Agent}

\shortdef{Second semantically core participant that would otherwise be labeled \psst{Agent}, 
but which is adpositionally marked in contrast with an \psst{Agent} 
occupying a non-oblique syntactic position (subject or object).
Typically, the \psst{Agent} and \psst{Co-Agent} engage in the event 
in a reciprocal fashion.}

\begin{exe}
  \ex I fought in a war \p{against} the Germans.
  \ex I \choices{talked\\argued} \p{with} my roommate about cleaning duties.
\end{exe}

See also: \psst{Accompanier}, \psst{SocialRel}

\hierBdef{Theme}

\shortdef{Undergoer that is a semantically core participant in an event or state, 
and that does not meet the criteria for any other label.}

Prototypical \psst{Theme}s undergo (nonagentive) motion, are transferred, 
or undergo an internal change of state (sometimes called \emph{patients}).
Adpositional \psst{Theme}s are usually construed as something else:
\begin{exe}
  \ex Fill the bowl \p{with} water. (\rf{Theme}{Instrument})
  \ex The mechanic made a repair \p{to} the engine. (\rf{Theme}{Goal})
  \ex \begin{xlist}
      \ex Sheldukher \choices{searched\\fumbled} \p{for} his laser pistol. (\rf{Theme}{Goal})
      \ex There is a significant demand \p{for} new housing. (\rf{Theme}{Goal})
      \ex They charge higher prices \p{for} goods bought by credit card. (\rf{Theme}{Goal})
    \end{xlist}
  \ex \begin{xlist}
      \ex the price \p{of} tea in China (\rf{Theme}{Gestalt})
      \ex the approach \p{of} the waves
      \ex the \choices{death\\murder} \p{of} a salesman
    \end{xlist}
  \ex \begin{xlist}
      \ex The mechanic worked \p{on} the engine.
      \ex We noshed \p{on} snacks.
      \ex Students spend a lot of money \p{on} textbooks.\nss{multiple construal: Theme as Goal as Locus?}
    \end{xlist}
  \ex \begin{xlist}
      \ex There was an increase \p{in} oil prices.
      \ex I'm covered \p{in} bees! (\rf{Theme}{Locus})
    \end{xlist}
  \ex \begin{xlist}
      \ex The training saved us \p{from} almost certain death. (\rf{Theme}{Source})
      \ex They prevented us \p{from} boarding the plane. (\rf{Theme}{Source})
    \end{xlist}
\end{exe}
When two symmetric undergoers are collected in a single NP 
functioning as a set, it is marked as a \psst{Whole} construal:
\begin{exe}
  \ex There was a collision in mid-air \p{between} two light aircraft. (\rf{Theme}{Whole})
  \ex Links \p{between} science and industry are important. (\rf{Theme}{Whole})\nss{or \rf{Locus}{Whole}?}
\end{exe}

\begin{history}
  In v1, following many thematic role inventories, 
  \sst{Patient} was a distinct label for undergoers that were 
  affected (undergoing an internal change of state). 
  It was merged into \psst{Theme} for v2 because the affectedness criterion can be subtle 
  and difficult to apply.
\end{history}

Compare: \psst{Co-Theme}

\hierCdef{Co-Theme}

\shortdef{Second semantically core undergoer that would otherwise be labeled \psst{Theme}, 
but which is adpositionally marked in contrast with a \psst{Theme} 
occupying a non-oblique syntactic position (subject or object).}

\begin{exe}
  \ex They replaced my old tires \p{with} new ones.
\end{exe}

\begin{history}
  In v1, \sst{Co-Patient} was a distinct label, and the two shared a common supertype, 
  \sst{Co-Participant}. 
  See note at \psst{Theme}.
\end{history}

See also: \psst{InsteadOf}, \psst{Co-Agent}

\hierCdef{Topic}

\shortdef{Information content or subject matter in communication or cognition.}

Prototypical prepositions are \p{about} and \p{on}:

\begin{exe}
  \ex I gave a presentation \choices{\p{about}\\\p{on}} politics.
  \ex Try not to think \p{about} it.
\end{exe}

Less prototypical \psst{Topic} markers include:

\begin{exe}
  \ex Are you interested \p{in} politics?
  \ex I was accused \p{of} treason.
  \ex I'm \choices{an expert\\talented} \p{at} cooking.
\end{exe}

See also: \psst{Stimulus}

\hierBdef{Stimulus}%
%
\shortdef{That which is perceived or experienced (bodily, perceptually, or emotionally).}

\psst{Stimulus} does not seem to have any prototypical adposition 
in the languages we have looked at. In English, it can be construed in several ways:
\begin{exe}
  \ex My affection \p{for} you (\rf{Stimulus}{Beneficiary})
  \ex Scared \p{by} the bear (\rf{Stimulus}{Causer})
  \ex I startled \p{at} the noise (\rf{Stimulus}{Goal})
  \ex I care \p{about} you (\rf{Stimulus}{Topic})
\end{exe}

Counterpart: \psst{Experiencer}

\hierBdef{Experiencer}

\shortdef{Animate who is aware of a bodily experience, perception, emotion, or mental state.}

\psst{Experiencer} does not seem to have any prototypical adposition 
in the languages we have looked at. In English, it can be construed in several ways:
\begin{exe}
  \ex The anger \p{of} the students (\rf{Experiencer}{Possessor})
  \ex Running is enjoyable \p{for} me (\rf{Experiencer}{Beneficiary})
  \ex It feels hot \p{to} me (\rf{Experiencer}{Goal})
\end{exe}

Elsewhere, the term \emph{cognizer} is sometimes used for one whose 
mental state is described.

Counterpart: \psst{Stimulus}

\hierBdef{Originator}

\shortdef{Animate who is the initial possessor or creator/producer of something,
including the speaker/communicator of information. 
Excludes events where transfer/communication is not framed as unidirectional.}

A ``source'' in the broadest sense of a starting point/condition. 
Contrasts with \psst{Recipient} if there is transfer/communication.

Typically construed as \psst{Agent} (with \pex{give}, \pex{tell}, \pex{talk \p{to}}, \pex{create}: 
subject or passive-\p{by}; adnominal \p{by} as in \pex{works \p{by} Shakespeare}) 
or \psst{Source} (\pex{obtain/hear \p{from}}; adnominal \p{of} as in \pex{works \p{of} Shakespeare}). 
Occasionally construed as \psst{Theme} (\pex{rob \uline{her} of her life savings}: direct object).

Does not apply to events like \pex{exchange}, \pex{talk/chat \p{with}}, 
or \pex{negotiate}, which involve a back-and-forth between 
\psst{Agent} and \psst{Co-Agent} (or a plural \psst{Agent}).\nss{OK?}

% \nss{Do we need to limit further the kinds of events that have an Originator 
% and Recipient? Only communicative events where the message is core? 
% (``Say'', ``tell'', ``inform'', but not ``talk'' or ``negotiate''. 
% Otherwise we have ``talk with him'' as \rf{Originator}{Co-Agent}; 
% ``negotations by the parties'' as \rf{Orignator}{Agent}; 
% and ``negotations between the parties'' as Originator \tat> Agent \tat> Whole!)}

\begin{history}
  \psst{Originator} merges v1 labels \sst{Donor/Speaker} and \sst{Creator}, 
  which were difficult to distinguish in the case of authorship.
  %(e.g., \pex{the operas \p{of} Puccini}).
  \sst{Donor/Speaker} was a subtype of \sst{InitialLocation}, which 
  inherited from \sst{Location} and \sst{Source}. 
  \sst{Creator} was a subtype of \psst{Agent}.
  Moving \psst{Originator} directly under \psst{Participant} 
  puts it in a neutral position with respect to its possible construals.
\end{history}

\hierBdef{Recipient}

\shortdef{Animate who is the (actual or intended) final possessor of a thing or message.
Excludes events where transfer/communication is not framed as unidirectional.}
A ``goal'' in the broadest sense of an ending point/condition. 
Contrasts with \psst{Originator}.

Typically construed as \psst{Goal} (\pex{give/talk \p{to}}), 
\psst{Agent} (with \pex{receive}: subject or passive-\p{by}), 
or \psst{Theme} (with \pex{inform}: direct object).

Does not apply to events like \pex{exchange}, \pex{talk/chat \p{with}}, 
or \pex{negotiate}, which involve a back-and-forth between 
\psst{Agent} and \psst{Co-Agent} (or a plural \psst{Agent}).\nss{OK?}

\begin{history}
  In v1, \psst{Recipient} was the counterpart to \sst{Donor/Speaker}:
  \psst{Recipient} was a subtype of \sst{Destination}, which 
  inherited from \sst{Location} and \sst{Goal}. 
  Moving \psst{Recipient} directly under \psst{Participant} 
  puts it in a neutral position with respect to its possible construals.
\end{history}

\hierBdef{Cost}

\shortdef{An amount (typically of money) that is linked to an item or service 
that it pays for\slash could pay for, or given as the amount earned or owed.} 

The governor may be an explicit commercial scenario:
\begin{exe}
  %\ex I paid/owed John \$10 for the book. %#nonprep
  \ex I \choices{bought\\sold} the book \p{for} \$10.
  \ex The book is \choices{priced\\valued} \p{at} \$10.
  \ex I got a refund \p{of} \$10.
\end{exe}
Or the \psst{Cost} may be specified as an adjunct with a non-commerical governor:
\begin{exe}
  \ex You can ride the bus \p{for} \choices{free\\\$1}.
\end{exe}
\psst{Cost} is \emph{not} used with general scenes of possession or transfer, 
even if the thing possessed or transferred happens to be an amount of money:
\begin{exe}
  \ex I bestowed the winner \p{with}$_{\text{\psst{Co-Theme}}}$ \$100.
\end{exe}

\begin{history}
  This category was not present in v1, which had the broader category \sst{Value}. 
  VerbNet \citep{verbnet,palmer-17} has a similar category called \sst{Asset}; we chose the name 
  \psst{Cost} to emphasize that it describes a relation rather than an entity type 
  (it does not apply to money with a verb like \pex{possess} or \pex{transfer}, 
  for instance).
\end{history}

\hierBdef{Beneficiary}

\shortdef{Animate or personified undergoer that is (potentially) 
advantaged or disadvantaged by the event or state.}

This label does not distinguish the polarity of the relation 
(helping or hurting, which is sometimes termed \emph{maleficiary}).

\begin{exe}
  \ex Vote \choices{\p{for}\\\p{against}} Pedro!
  \ex Junk food is bad \p{for} your health.
  \ex My parrot died \p{on} me.
\end{exe}

\hierBdef{Instrument}

\shortdef{An entity that facilitates an action by applying intermediate causal force.}

Prototypically, an \psst{Agent} manipulates the \psst{Instrument} with the purpose of achieving a result.
This includes a device serving as a mode of transportation or medium of communication 
(often construed as a \psst{Locus} or \psst{Path}).
Less prototypically, the action could be unintentional 
(\pex{I accidentally poked myself in the eye \p{with} a stick}). 
The key is that the \psst{Instrument} is not sufficiently ``independently causal'' to instigate the event, 
though it can be construed as the instigator (\pex{The window was broken \p{by} the hammer}: \rf{Instrument}{Causer}).
Other non-prototypical instruments include waypoints from \psst{Source} to \psst{Goal}, 
and people that relay information from \psst{Originator} to \psst{Recipient}.

Compare \psst{Means}, which is used for facilitative events rather than entities.

\hierAdef{Configuration}

\hierBdef{Identity}

\shortdef{A category being ascribed to something, 
or something belonging to the category denoted by the governor.}

Prototypical prepositions are \p{of} (where the governor is the category) 
and \p{as} (where the object is the category):
\begin{exe}
  \ex\label{ex:stateof} the state \p{of} Washington [as opposed to the city]
  \ex The liberal state \p{of} Washington has not been receptive to Trump's message.
  \ex \p{As} a liberal state, Washington has not been receptive to Trump's message.
  \ex\label{ex:ascolleague} I like Bob \p{as} a colleague. [but not as a friend]
  \ex What a gem \p{of} a restaurant! [exclamative idiom: both NPs are indefinite]
  \ex the \choices{idea\\task\\hassle} \p{of} opening a new business
  \ex\label{ex:shell} the \choices{topic\\issue} \p{of} semantics
\end{exe}
Something may be specified with a category in order to disambiguate it \cref{ex:stateof}, 
or to provide an interpretation or frame of reference with which that entity is to be considered.
In some cases, like \cref{ex:shell}, the category is a \emph{shell noun} \citep{schmid-00} 
requiring further specification.

Categorizations may be situational rather than permanent/definitional:
\begin{exe}\ex\label{ex:assituational}\begin{xlist}
  \ex She appears \p{as} Ophelia in \emph{Hamlet}.
  \ex He is usually a bartender, but today he is working \p{as} a waiter.
\end{xlist}\end{exe}

Paraphrase test: ``(thing) IS (category) [in the context of the event]'': 
``Washington is a liberal state'', ``opening a new business is a hassle'', 
``She is Ophelia'', etc. Note that \p{as}+category may attach syntactically 
to a verb, as in \cref{ex:ascolleague} and  \cref{ex:assituational}, 
rather than being governed by the item it describes.

\begin{history}
  Generalized from v1, where it was called \sst{Instance} and restricted 
  to the ``(category) \p{of} (thing)'' formulation. 
  The relevant usages of \p{as} were labeled \sst{Attribute}.
\end{history}

\hierBdef{Species}

\shortdef{A category qualified by \w{sort}, \w{type}, \w{kind}, \w{species}, \w{breed}, etc. 
Includes \w{variety}, \w{selection}, \w{range}, \w{assortment}, etc.\ 
meaning `many different kinds'.}

\begin{exe}
  \ex that sort \p{of} business
  \ex A good type \p{of} ant to keep is the red ant .
  \ex certain strains \p{of} \emph{Escherichia coli}
  \ex Modern breeds \p{of} these homing pigeons return reliably
  \ex Some poor sap applied the wrong brand \p{of} paint
  \ex This store offers a wide selection \p{of} footstools
\end{exe}

\psst{Species} is \emph{not} used if the sort/variety noun 
is the object rather than the governor:
\begin{exe}
  \ex a business \p{of}$_{\text{\psst{Characteristic}}}$ that sort
\end{exe}

\hierBdef{Gestalt}

\shortdef{Generalized notion of ``whole'' understood with reference to 
a component part, possession, set member, or characteristic. 
See \psst{Characteristic}.}

\psst{Gestalt} applies directly to:
\begin{itemize}
\item	The holder of a property if the property is the governor:
\begin{exe}
  \ex \begin{xlist} 
      \ex the blueness \p{of} the sky
      \ex the wisdom \p{of} the crowd
      \ex the time \p{of} the party
      \ex\label{ex:amountGestalt} the amount \p{of} time allowed [but see \cref{ex:QuantityGestalt}]
    \end{xlist}
  \end{exe}
\item	The wearer of attire:
\begin{exe}
  \ex the uniforms \p{of} the children
  \ex the shirt \p{on} him (\rf{Gestalt}{Locus})
\end{exe}
\genitiveversion{\item	A referent temporarily associated with another referent in the discourse 
and used to help identify it: 
\begin{exe}
  \ex Sam\p{'s} dog (= the dog that Sam mentioned seeing earlier in the conversation)
\end{exe}}
\item	Anything that is borderline between subcategories \psst{Possessor} and \psst{Whole}
\end{itemize}
\genitiveversion{
Used to construe \psst{Locus} for a container denoted by the governor:
\begin{exe}
\ex the room\p{'s} 2 beds (\rf{Locus}{Gestalt})
\end{exe}
}

See also: \psst{Quantity}

\hierCdef{Possessor}

\shortdef{Animate who \textbf{has} something (the \psst{Possession}) 
which is not part of their body 
or inherent to their identity/character but could, in principle, be taken away.}

Prototypically expressed with \p{of}\genitiveversion{or genitive marking}:

\begin{exe}
  \ex the money \p{of} the rich
\end{exe}

See \psst{SocialRel}.

\hierCdef{Whole}

\shortdef{Something described with respect to its part, portion, subevent, subset, 
or set element. See \psst{Part/Portion}.}

\begin{exe}
  \ex \begin{xlist}
    \ex	The new engine \p{of} the car
    \ex	The flaxen hair \p{of} the girl
    \ex\label{ex:layers}	The 3 layers \p{of} the cake
    \ex\label{ex:prongs}	The 3 prongs \p{of} the strategy
    %\ex 2 \p{of} my 5 daughters % Quantity ~> Whole
    \ex\label{ex:rest} The \choices{remainder\\rest} \p{of} the cake\nss{Maybe this should be 
    \rf{Quantity}{Whole} after all, even though ``the rest of it'' is a dubious way to answer 
    ``How much of it?''. ``The remaining 6 ounces of cake'' certainly specifies a quantity.}
    \ex The tastiest bit \p{of} the cake
    \ex	The tennis matches \p{of} a series
    \ex	The interior \p{of} the shopping bag
    \ex	The south (region) \p{of} France
    \ex	The beginning \p{of} the party
  \end{xlist}
  \ex \begin{xlist}
    \ex	The tennis matches \p{in} a series (\rf{Whole}{Locus})
    \ex	The new engine \p{in} the car (\rf{Whole}{Locus})
    \ex the escape key \p{on} the keyboard (\rf{Whole}{Locus})
    \ex	The clothes \p{in} the pile (\rf{Whole}{Manner})\nss{or Locus?}
  \end{xlist}
  \ex Sets and ratios:
    \begin{xlist}
      \ex This is one \p{of} the \choices{worst\\better} retaurants in town. (\psst{Whole})
      \ex 2 \p{in} 10 American children are redheads. (\rf{Whole}{Locus})
      \ex 2 \p{out\_of} 10 American children are redheads. (\rf{Whole}{Source})\nss{should this be \psst{RateUnit}?}
      \ex \p{Out\_of} the 10 children in the class, only Mary is a redhead. (\rf{Whole}{Source})
      \ex\label{ex:amongSet} \p{Among} the 10 children in the class, only Mary is a redhead. (\psst{Whole})
    \end{xlist}
\end{exe}

If the governor narrows the reference to a certain amount of the \psst{Whole}, 
the construal \rf{Quantity}{Whole} is used---see \cref{ex:QuantityWhole}. 
Note that this only applies if the governor is a measure term; 
it does not apply to distinctive parts like ``layers'' \cref{ex:layers} 
and ``prongs'' \cref{ex:prongs}, even if a count is specified.

Used to construe geographic and temporal ``containers'':
\begin{exe}
  \ex	Famous castles \p{of} the valley (\rf{Locus}{Whole})
  \ex \begin{xlist}
    \ex the \choices{15th\\Ides} \p{of} March (\rf{Time}{Whole})
    \ex March \p{of} 44~BC (\rf{Time}{Whole})
  \end{xlist}
\end{exe}

The prepositions \p{between} and \p{among} can impose \psst{Whole} construals 
by combining two or more items in the object NP (contrast with \cref{ex:amongSet}):
\begin{exe}
  \ex\label{ex:betweenParties}  The negotiations \choices{\p{between}\\\p{among}} the parties went well. (\rf{Agent}{Whole})
  \exp{ex:betweenParties} The negotiations \p{by} the parties went well. (\psst{Agent})
\end{exe}

\hierBdef{Characteristic}

\shortdef{Generalized notion of a part, feature, possession, 
or the contents or composition of something, 
understood with respect to that thing (the \psst{Gestalt}).}

Can be used to construe person-to-person relationships such as kinship, 
whose scene role should be \psst{SocialRel}. 
Labels \psst{Possession}, \psst{Part/Portion}, and its subtype \psst{Stuff} 
are defined for some important subclasses.

\psst{Characteristic} applies directly to:
\begin{itemize}
\item	A property value: 
\begin{exe} \ex \begin{xlist}
  \ex a car \p{of} high quality
  \ex a man \p{of} honor
  \ex a business \p{of} that sort [contrast with \psst{Species}, \cref{sec:Species}]
\end{xlist}\end{exe}
\item	Attire:
\begin{exe}
  \ex the kid \p{with} a vest (on)
  \ex the kid \p{in} a vest (\rf{Characteristic}{Locus})
\end{exe}
\item	Role of a complex framal \psst{Gestalt} that has no obvious decomposition into parts: 
\begin{exe}\ex \begin{xlist}
  \ex the restaurant \p{with} \choices{a convenient location\\an extensive menu}
  \ex a party \p{with} great music
\end{xlist}\end{exe}
\item	That which is located in a container denoted by the governor: 
\begin{exe}
  \ex a room \p{with} 2 beds
\end{exe}
\item	Anything that is borderline between subcategories \psst{Possession} and \psst{Part/Portion}
\end{itemize}

Typically, one of ``\psst{Gestalt} \{HAS, CONTAINS\} \psst{Characteristic}'' is entailed. 
This does not help to distinguish subtypes.

\begin{history}
  The v1 label \sst{Attribute} was intended to apply to features of something, 
  but was rather squishy. \nss{...}
\end{history}

\hierCdef{Possession}

\shortdef{That which some \psst{Possessor} (animate or personified, e.g.~an institution) 
\textbf{has}, and which is not part of their body or inherent to their identity/character 
but could, in principle, be taken away.}

Sometimes called \emph{alienable} possession. 
The possession may be concrete or abstract, and temporary or permanent.
Excludes attire: see \psst{Characteristic}. 

Prototypical prepositions are \p{with} and \p{without}:
\begin{exe}
\ex	People \p{with} money
\end{exe}

Immediate concrete possession uses an \psst{Accompanier} construal:
\begin{exe}
  \ex Hagrid exited the shop \p{with} (= carrying) a snowy owl. (\rf{Possessor}{Accompanier})
\end{exe}

Paraphrase test: ``\psst{Possessor} POSSESSES \psst{Possession}'', 
or ``\psst{Possessor} is IN POSSESSION OF \psst{Possession}''. 
The latter is especially appropriate for immediate concrete possession.

\hierCdef{Part/Portion}

\shortdef{A part, portion, subevent, subset, or set element (e.g., an example or exception) 
of some \psst{Whole}.}

Anything directly labeled with \psst{Part/Portion} 
is understood to be \textbf{incomplete} relative to the \psst{Whole}.
This includes body parts and partial food ingredients.

Prototypical prepositions include \p{with}, \p{without};
\p{such as}, \p{like} for exemplification; 
and \p{but}, \p{except}, \p{except\_for} for exceptions:
\begin{exe}
  \ex \begin{xlist}
    \ex	A car \p{with} a new engine
    \ex	A strategy \p{with} 3 prongs
    \ex	The girl \p{with} flaxen hair
    \ex	A man \p{with} a wooden leg named Smith
    \ex	A valley \p{with} a castle
    \ex	A quintet \p{with} 2 cellos
    \ex	A performance \p{with} a guitar solo
    \ex	A cake \p{with} 3 layers
    \ex	A sandwich \p{with} wheat bread
    \ex	Soup \p{with} carrots (in it)
    \ex	A chicken sandwich \p{with} ketchup (on it)
  \end{xlist}
  \ex	Bread \p{without} gluten
  \ex	Strategies \p{such as} divide-and-conquer
  \ex Everyone \p{except} Bob plays trombone.
\end{exe}

Some can be paraphrased with INCLUDES, but this is not determinative.


\hierDdef{Stuff}

\shortdef{The members comprising a group/ensemble, 
or the material comprising some unit of substance. 
\psst{Stuff} is distinguished from other instances of \psst{Part/Portion}
in fully covering (or ``summarizing'') the aggregate whole.}

Paraphrase test: ``\psst{Whole} CONSISTS OF \psst{Stuff}''

\begin{exe}
  \ex	\begin{xlist}
    \ex A flock \p{of} birds
    \ex	A throng \p{of} tourists
    \ex	A clump \p{of} sand
    \ex	A piece \p{of} wood
    \ex	A series \p{of} tennis matches
    \ex	An evening \p{of} Brahms
    \ex	A meal \p{of} salmon
  \end{xlist}
  \ex	A salad \choices{\p{of}\\\p{with}} mixed greens
  \ex\label{ex:bottleStuff} This bottle is \p{of} beer (and that one is of wine). [but see \cref{ex:bottleQuantity}]
  % \ex	\rf{Quantity}{Stuff}: see \cref{ex:QuantityStuff}
  %   \begin{xlist}
  %     \ex A bottle('s worth) \p{of} beer
  %     \ex A bag('s worth) \p{of} chips
  %   \end{xlist}
  \ex \rf{OrgRole}{Stuff}:
  	\begin{xlist}
      \ex An order \p{of} nuns
      \ex	A chamber group \choices{\p{of}\\\p{with}} 5 players
    \end{xlist}
\end{exe}

See also: \psst{Quantity}

\psst{Stuff} has no specific counterpart under \psst{Whole}.

\hierBdef{Accompanier}

\shortdef{Entity that another entity is together with.}

Sometimes called \emph{comitative}.

Prototypical prepositions are \p{with}, \p{without}, \p{along\_with}, 
\p{together\_with}, and \p{in\_addition\_to}:
\begin{exe}
  \ex I'll have soup \choices{\p{with}\\\p{without}} salad.
  \ex She'll be \p{with} us in spirit.
\end{exe}

For an ``extra participant'' in an activity, 
where two parties perform the activity together 
(but the nature of the activity would not fundamentally 
change if they each performed it independently), 
a \psst{Co-Agent} construal is used:
\begin{exe}
  \ex Do you want to walk \p{with} me? (\rf{Accompanier}{Co-Agent})
\end{exe}
By contrast, if the nature of the scene fundamentally requires multiple participants, 
simple \psst{Co-Agent} is used. Often there is ambiguity:\footnote{Adding \p{together} 
seems to favor the (b)~readings: \pex{I fought \p{together\_with} them}, \pex{We fought \p{together}} 
can only mean we were on the same side. Contrastive stress can also force one reading: 
\pex{I fought \p{WITH} them (not \p{AGAINST} them)}.}
\begin{exe}
  \ex Do you want to talk \p{with} me? 
  \begin{xlist}
    \ex {}[\emph{The reading:} Should we have a conversation?] (\psst{Co-Agent})
    \ex {}[\emph{The reading:} Do you want to join me in talking to a third party?] 
      (\rf{Accompanier}{Co-Agent})
  \end{xlist}
  \ex I fought \p{with} them to reform the regulation.
  \begin{xlist}
    \ex {}[\emph{The reading:} I fought against them.] (\psst{Co-Agent})
    \ex {}[\emph{The reading:} I was on the same side as them.] (\rf{Accompanier}{Co-Agent})
  \end{xlist}
\end{exe}

If the object denotes a item that the governor has on hand in their possession, 
then the construal \rf{Possession}{Accompanier} is used:
\begin{exe}
  \ex I walked in \p{with} an umbrella. (\rf{Possession}{Accompanier})
\end{exe}

See also: \psst{Instrument}, \psst{Manner}

\hierBdef{InsteadOf}

\shortdef{A default or already established thing for which something else stands in 
or is chosen as an alternative.}

\begin{exe}
  \ex I ordered soup \choices{\p{instead\_of}\\\p{rather\_than}} salad.
  \ex \p{Instead\_of} ordering salad, I ordered soup.
  %\ex They replaced \uline{my old tires} with new ones. %#nonprep
  \ex The new shirts were gray \p{instead\_of} black.
  \ex They \choices{substituted\\swapped} my old tires \p{for} new ones.
\end{exe}
May be construed spatially:
\begin{exe}
  \ex I chose soup \p{over} salad. (\rf{InsteadOf}{Locus})
\end{exe}

See also: \psst{Accompanier}, \psst{ComparisonRef}, \psst{Co-Theme}

\hierBdef{ComparisonRef}

\shortdef{The reference point in an explicit comparison (or contrast), i.e., 
an expression indicating that something is 
\textbf{similar/analogous to}, \textbf{different from}, or \textbf{the same as}
something else.}

The marker of the ``something else'' (the ground in the figure–ground relationship) 
is given the label \psst{ComparisonRef}:
\begin{exe}
  \ex \begin{xlist}
    \ex She is taller \p{than} me.
    \ex She is taller \p{than} I am.
    \ex She is taller \p{than} she is wide.
    \ex She is better at math \p{than} at drawing.
    \ex The shirt is more gray \p{than} black.
    %\ex She is greater in height \p{than} me.
  \end{xlist}
  \ex \begin{xlist}
    \ex She is as tall \p{as} I am.
    \ex Your face is as$_{\text{\psst{Characteristic}}}$ red \p{as} a rose.
    \ex Your face is red \p{as} a rose.
    \ex Your surname is the\_same \p{as} mine.
  \end{xlist}
  \ex Harry had never met anyone quite \p{like} Luna.
  \ex It was \choices{\p{as\_if}\\\p{like}} he had insulted my mother.
\end{exe}

The comparison is often made with respect to some dimension or attribute, the \psst{Characteristic}, 
which may or may not be scalar. 
The comparison may be figurative, employing simile, hyperbole, or spatial metaphor 
(\pex{close to} in the sense of `similar to'). 
The \psst{ComparisonRef} may even be a desirable or hypothetical/irrealis 
event or state (\pex{It was \p{as} it should have been}).

Prototypical prepositions include \p{than}, \p{as} (including the second item 
in the \p{as}--\p{as} construction), \p{like}, \p{unlike}. 
Prominent construals are \p{to} (\psst{Goal} for similar-thing) 
and \p{from} (\psst{Source} for dissimilar-thing).

\hierBdef{RateUnit}

\shortdef{Unit of measure in a rate expression.}

The prototypical preposition in \p{per}:

\begin{exe} \ex \begin{xlist}
  \ex The cost is \$10 \p{per} item.
  \ex A fuel efficiency of 40 miles \p{per} gallon (of gas)
\end{xlist}\end{exe}

Paraphrase: The adposition can be paraphrased as ``for each/every''.

\begin{history}
  In v1, this fell under \sst{Value}.
\end{history}

\hierBdef{Quantity}

\shortdef{Something measured by a quantity denoted by the governor.}

The governor may be a precise or vague count/measurement. 
This includes nouns like ``lack'', ``dearth'', ``shortage'', ``excess'', or ``surplus''
(meaning a too-small or too-large amount).

Question test: the governor answers ``How much/many of (object)?''

The main preposition is \p{of}.

\begin{itemize}
\item Simple \psst{Quantity}:
\begin{exe}
  \ex\label{ex:bottleQuantity}	Pour me a bottle('s worth) \p{of} beer. [but see \cref{ex:bottleStuff}]
  \ex	I have 2 years \p{of} training.
  \ex	\begin{xlist}
    \ex I ate \choices{6 ounces\\a piece} \p{of} cake.
    \ex	An ounce \p{of} compassion
  \end{xlist}
  \ex	There's a dearth \p{of} cake in the house.
  \ex	This cake has thousands \p{of} sprinkles.
  \ex They number in the tens \p{of} thousands.
  \ex	\begin{xlist}
    \ex\label{ex:anumber} I have a \choices{number\\handful} \p{of} students.
    \ex	I have a lot \p{of} students.
    \ex	We did a lot \p{of} traveling.
    \ex	There is a lot \p{of} wet sand on the beach.
  \end{xlist}
  \ex	A pair \p{of} shoes
\end{exe}

\item If the measure includes a word like ``amount'', ``quantity'', or ``number'',\footnote{Excluding 
the expression ``a number'' meaning `several', as in \cref{ex:anumber}.} 
the construal \rf{Quantity}{Gestalt} is used 
(because the amount of something can be viewed as an attribute):
\begin{exe}
  \ex\label{ex:QuantityGestalt} \rf{Quantity}{Gestalt}:
  \begin{xlist}
    \ex	A generous amount \p{of} time
    \ex A large number \p{of} students
  \end{xlist}
\end{exe}
But if ``amount'', ``quantity'', etc. is used without a measure as its modifier, 
it is simply \psst{Gestalt}: see \cref{ex:amountGestalt}.

\item If the governor is a \textbf{collective noun}, 
the construal \rf{Quantity}{Stuff} is used 
(note that a ``consisting of'' paraphrase is possible):
\begin{exe}
  \ex\label{ex:QuantityStuff} \rf{Quantity}{Stuff}:
  \begin{xlist}
    \ex Can you outrun a herd \p{of} wildebeest?
    \ex Put 3 bales \p{of} hay on the truck.
    \ex	\choices{A group\\2 groups\\A throng} \p{of} vacationers just arrived.
  \end{xlist}
\end{exe}

\item Otherwise, if the object refers to \textbf{a specific item or set}, 
and the quantity measures a portion of that item 
(whether a quantifier, absolute measure, or fractional measure),
the construal \rf{Quantity}{Whole} is used:
\begin{exe}
  \ex\label{ex:QuantityWhole} \rf{Quantity}{Whole}:
  \begin{xlist}
    \ex	I ate 6 ounces \p{of} the cake in the refrigerator.
    \ex	I ate \choices{half\\50\%} \p{of} the cake.
    \ex	\choices{All/many/lots/a lot/\\some/few/both/none} \p{of} the town's residents 
    are students.
    \ex	I have seen all \p{of} the city. (= the whole city)
    \ex	A lot \p{of} the sand on the beach is wet.
    \ex	2 \p{of} the children are redheads.
    \ex 2 \p{of} the 10 children in the class are redheads.
  \end{xlist}
\end{exe}
However, simple \psst{Whole} is used if the portion is specified as 
``the rest'', ``the remainder'', etc., as in \cref{ex:rest}.\nss{Reconsidering this: see \cref{ex:rest}}
\end{itemize}

\hierCdef{Approximator}

\hierBdef{SocialRel}

\shortdef{Entity, such as an institution or another individual, 
with which an individual has a stable affiliation.}

Typically, \psst{SocialRel} applies directly to relations between 
individuals.
It does not have any prototypical adpositions. 
Construals include:
\begin{exe}
  \ex \begin{xlist}
      \ex\label{ex:workwithSR} I work \p{with} Michael. (\rf{SocialRel}{Co-Agent})
      \ex Joan has a class \p{with} Miss Zarves. (\rf{SocialRel}{Co-Agent})
    \end{xlist}
  \ex \begin{xlist} 
      \ex Joan is the \choices{sister\\wife} \p{of} John. (\rf{SocialRel}{Possessor})
      \ex Joan is a student \p{of} Miss Zarves. (\rf{SocialRel}{Possessor})
    \end{xlist}
  \ex Joan is studying \p{under} Prof.~Smith. (\rf{SocialRel}{Locus})
  \ex Joan is married \p{to} John. (\rf{SocialRel}{Goal})
\end{exe}

Note, however, that \emph{work \p{with}} is ambiguous between 
being in an established professional relationship \cref{ex:workwithSR}, 
and engaging temporarily in a joint productive activity:
\begin{exe}
  \ex\label{ex:workwithCA} I was working \p{with}$_{\text{\psst{Co-Agent}}}$ Michael after lunch.
\end{exe}
It is up to annotators to decide from context which interpretation 
better fits the context.

\begin{history}
  Renamed from v1 label \sst{ProfessionalAspect}, which was borrowed from 
  \citet{srikumar-13,srikumar-13-inventory}.
  The name \psst{SocialRel} reflects
  a broader set of stative relations involving an individual 
  in a social context, including kinship and friendship.
  See also note under \psst{OrgRole}.
\end{history}

\hierCdef{OrgRole}

\shortdef{Organization or institution with which an individual 
has a stable affiliation, such as membership or a business relationship.}

Like its supertype \psst{SocialRel}, \psst{OrgRole} 
lacks any prototypical adposition, but participates in numerous construals:

\begin{exe}
  \ex \begin{xlist}
      \ex the chairman \p{of} the board (\rf{OrgRole}{Gestalt})
      \ex the president \p{of} the U.S. (\rf{OrgRole}{Gestalt})
      \ex I am a loyal customer \p{of} Graeter's. (\rf{OrgRole}{Gestalt})
      \ex employees \p{of} Grunnings (\rf{OrgRole}{Gestalt})
    \end{xlist}
  \ex Mr. Dursley works \p{for} Grunnings. (\rf{OrgRole}{Beneficiary})
  \ex Mr. Dursley works \p{at} Grunnings. (\rf{OrgRole}{Locus})
  \ex Mr. Dursley is \p{from} Grunnings. (\rf{OrgRole}{Source})
  \ex Mr. Dursley is \p{with} Grunnings. (\rf{OrgRole}{Accompanier})
  \ex Mr. Dursley is employed \p{by} Grunnings. (\rf{OrgRole}{Agent})\nss{or do we say `employ' is just a regular Agent/Theme verb?}
  \ex I bank \p{with} TSB. (\rf{OrgRole}{Accompanier})
  \ex I serve \p{on} the committee. (\rf{OrgRole}{Locus})
\end{exe}

A family counts as an institution 
construed as a \psst{Whole} (set of its members) 
or as a \psst{Locus}:\nss{what about `the family of the patient'?}
\begin{exe}
  \ex I am the baby \p{of} the family. (\rf{OrgRole}{Whole})
  \ex people \p{in} my family (\rf{OrgRole}{Locus})\nss{double construal?}
\end{exe}

For a relation between a unit and a larger institution, 
use \psst{Whole}:
\begin{exe}
  \ex the Principals Committee \p{of}$_{\text{\psst{Whole}}}$ the National Security Council
\end{exe}

See also: \psst{Stuff}

\begin{history}
  \psst{OrgRole} is now distinguished within the broader \psst{SocialRel} category 
  following the precedent of the Abstract Meaning Representation \citep[AMR;][]{amr,amr-guidelines}. 
  In AMR, \texttt{have-org-role-91} captures relations between 
  an individual and an institution (such as an organization or family),
  whereas \texttt{have-rel-role-91} is used for relations between two individuals.
\end{history}


\bibliographystyle{plainnat}
\bibliography{psst2.bib}


%\printbibliography[maxnames=99]


\end{document}
