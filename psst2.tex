\PassOptionsToPackage{usenames}{color}
\pdfoutput=1 % ensure pdflatex (for arXiv)
\documentclass[11pt,letterpaper]{article}
\usepackage{comment}
\usepackage{relsize} % relative font sizes (e.g. \smaller). must precede ACL style
%\usepackage{style/acl2012}



\usepackage[round]{natbib}
\begin{comment}
\usepackage[style=authoryear-comp,natbib=true,hyperref=true]{biblatex}

% tell biblatex not to quote titles in the bibliography
\DeclareFieldFormat{title}{#1} % don't italicize titles by default
\DeclareFieldFormat[book]{title}{\mkbibemph{#1}\isdot} % but do italicize books
\DeclareFieldFormat[article]{title}{#1\isdot}
\DeclareFieldFormat[inbook]{title}{#1\isdot}
\DeclareFieldFormat[incollection]{title}{#1\isdot}
\DeclareFieldFormat[inproceedings]{title}{#1\isdot}
\DeclareFieldFormat[patent]{title}{#1\isdot}
\DeclareFieldFormat[thesis]{title}{#1\isdot}
\DeclareFieldFormat[unpublished]{title}{#1\isdot}
% ...and for articles, to use number(issue) instead of number.issue
\renewbibmacro*{journal+issuetitle}{%
  \usebibmacro{journal}%
  \setunit*{\addspace}%
  \iffieldundef{series}
    {}
    {\newunit
     \printfield{series}%
     \setunit{\addspace}}%
  \printfield{volume}%
%  \setunit*{\adddot}%
  \printfield{number}%
  \setunit{\addcomma\space}%
  \printfield{eid}%
  \setunit{\addspace}%
  \usebibmacro{issue+date}%
  \newunit\newblock
  \usebibmacro{issue}%
  \newunit}
\DeclareFieldFormat[article]{number}{\mkbibparens{#1}}
% use ``pages'' instead of ``pp.''
\DefineBibliographyStrings{english}{%
    pages  =  {pages} % for multiple page numbers
}
% but for articles, just use a colon
\DeclareFieldFormat[article]{pages}{:#1} %TODO. check with LeCun citation
% and don't put a colon after In
\renewbibmacro*{in:}{%
  \bibstring{in}%\addcolon
  \setunit{\space}}

\DeclareFieldFormat{label}{#1\isdot}
\renewbibmacro*{year+labelyear}{%
  \iffieldundef{year}
    {}
    {\printtext{%
       \printfield{year}%
       \printfield{labelyear}%
       \setunit{\adddot}}}}

\bibliography{features.bib}
\end{comment}


%\usepackage{times}
%\usepackage{latexsym}

%\usepackage{makeidx} % index
\usepackage{imakeidx} % allows multiple indices
\makeindex % regular index
\makeindex[name=construals,title={Index of Construals by Scene Role}]
\makeindex[name=revconstruals,title={Index of Construals by Function}]

\usepackage[boxed]{algorithm2e}
\renewcommand\AlCapFnt{\small}
\usepackage[small,bf,skip=5pt]{caption}
\usepackage{sidecap} % side captions
\usepackage{rotating}	% sideways

% customize \paragraph spacing
\makeatletter
\renewcommand{\paragraph}{%
  \@startsection{paragraph}{4}%
  {\z@}{3.25ex \@plus 1ex \@minus .2ex}{-1em}% reduce 3.25 to .2 to minimize space
  {\normalfont\normalsize\bfseries}%
}
\makeatother

% Italicize subparagraph headings
\usepackage[nobottomtitles*]{titlesec}
\titleformat*{\subparagraph}{\itshape}
\titlespacing{\subparagraph}{%
  1em}{%              left margin
  0pt}{% space before (vertical)
  1em}%               space after (horizontal)


% MOVE SECTION NUMBERS INTO LEFT MARGIN
% They will be roman, right-aligned and separated from the heading text by .2cm
% adapted from http://tex.stackexchange.com/a/311712
\titleformat{\section}[block]
  {\Large\bfseries}
  {}
  {0pt}
  {\hspace{-1.2cm}% Move into margin
   \makebox[1cm][r]{\normalfont\thesection}\hspace{.2cm}}% Set number + title
\titleformat{name=\section,numberless}[block] % \section*
  {\Large\bfseries}
  {}
  {0pt}
  {\hspace{-1.2cm}% Move into margin
   \makebox[1cm][r]{\normalfont}\hspace{.2cm}}% Set  title
\titleformat{\subsection}[block]
  {\large\bfseries}
  {}
  {0pt}
  {\hspace{-1.2cm}% Move into margin
   \makebox[1cm][r]{\normalfont\thesubsection}\hspace{.2cm}}% Set number + title
\titleformat{\subsubsection}[block]
  {\normalsize\bfseries}
  {}
  {0pt}
  {\hspace{-1.2cm}% Move into margin
   \makebox[1cm][r]{\normalfont\thesubsubsection}\hspace{.2cm}}% Set number + title

%\usepackage{lingmacros}
% Lists

\usepackage{enumitem} % customizable lists
\setitemize{noitemsep,topsep=0em} %,leftmargin=*
\setenumerate{noitemsep,leftmargin=0em,itemindent=13pt,topsep=0em}

\usepackage{adjustbox}
\newcommand{\choices}[1]{\adjustbox{stack=ct}{#1}}  % for placing alternatives 
% inline with an example. e.g. \choices{to\\from\\with}
% ct = horizontally-centered, top

\usepackage{textcomp}
% \usepackage{arabtex} % must go after xparse, if xparse is used!
%\usepackage{utf8}
% \setcode{utf8} % use UTF-8 Arabic
% \newcommand{\Ar}[1]{\RL{\novocalize #1}} % Arabic text

\usepackage[procnames]{listings}

\usepackage{amssymb}	%amsfonts,eucal,amsbsy,amsthm,amsopn
\usepackage{amsmath}

%\usepackage{mathptmx}	% txfonts
\usepackage{fourier}
\usepackage[scaled=.87]{helvet}
\usepackage[scaled=.8]{beramono}
\usepackage[T1]{fontenc}
\usepackage[utf8x]{inputenc}

\usepackage{MnSymbol}	% must be after mathptmx

\usepackage{latexsym}





% Tables
\usepackage{array}
\usepackage{multirow}
\usepackage{booktabs} % pretty tables
\usepackage{multicol}
\usepackage{footnote}

\usepackage{hyperref}
\usepackage{url}
\usepackage[usenames]{color}
\usepackage{xcolor}

\definecolor{darkblue}{rgb}{0, 0, 0.5}
\hypersetup{colorlinks=true,citecolor=darkblue, linkcolor=., urlcolor=darkblue}

% colored frame box
\newcommand{\cfbox}[2]{%
    \colorlet{currentcolor}{.}%
    {\color{#1}%
    \fbox{\color{currentcolor}#2}}%
}

\usepackage[normalem]{ulem} % \uline
\usepackage{colortbl}
\usepackage{graphicx}
\usepackage{subcaption}
\usepackage{mdframed}

%\usepackage{tikz-dependency}
\usepackage{tikz}
\usepackage[edges]{forest}
%\usepackage{tree-dvips}
\usetikzlibrary{arrows,positioning,calc} 

\DeclareMathOperator*{\argmax}{arg\,max}
\DeclareMathOperator*{\argmin}{arg\,min}



% Author comments
\usepackage{color}
\newcommand\bmmax{0} % magic to avoid 'too many math alphabets' error
\usepackage{bm}
\definecolor{orange}{rgb}{1,0.5,0}
\definecolor{mdgreen}{rgb}{0,0.6,0}
\definecolor{mdblue}{rgb}{0,0,0.7}
\definecolor{dkblue}{rgb}{0,0,0.5}
\definecolor{dkgray}{rgb}{0.3,0.3,0.3}
\definecolor{slate}{rgb}{0.25,0.25,0.4}
\definecolor{gray}{rgb}{0.5,0.5,0.5}
\definecolor{ltgray}{rgb}{0.7,0.7,0.7}
\definecolor{ltltgray}{rgb}{0.9,0.9,0.9}
\definecolor{purple}{rgb}{0.7,0,1.0}
\definecolor{lavender}{rgb}{0.65,0.55,1.0}

% Settings for algorithm listings
\makeatletter
\lst@AddToHook{EveryPar}{%
  \label{lst:\thelstnumber}% make a label for each line number except the first (assumes only one listing in the document)
}
\makeatother
\lstset{
% basicstyle=\rmshape,
  numbers=left,
  numberstyle=\tt\color{gray},
  firstnumber=2,
  stepnumber=5,
  xleftmargin=3em,
  language=Python,
  upquote=true,
  showstringspaces=false,
  formfeed=\newpage,
  tabsize=1,
  stringstyle=\color{mdgreen},
  commentstyle=\itshape\color{lavender},
  basicstyle=\small\smaller\ttfamily,
  morekeywords={lambda,with,as,assert},
  keywordstyle=\bfseries\color{magenta},
  procnamekeys={def},
  procnamestyle=\bfseries\color{orange},
  aboveskip=0.5cm,
  belowskip=0.5cm
}
\renewcommand{\lstlistingname}{Algorithm}


\newcommand{\ensuretext}[1]{#1}
\newcommand{\cjdmarker}{\ensuretext{\textcolor{green}{\ensuremath{^{\textsc{CJ}}_{\textsc{D}}}}}}
\newcommand{\nssmarker}{\ensuretext{\textcolor{magenta}{\ensuremath{^{\textsc{NS}}_{\textsc{S}}}}}}
\newcommand{\nasmarker}{\ensuretext{\textcolor{red}{\ensuremath{^{\textsc{NA}}_{\textsc{S}}}}}}
\newcommand{\jbmarker}{\ensuretext{\textcolor{orange}{\ensuremath{^{\textsc{J}}_{\textsc{B}}}}}}
\newcommand{\abmarker}{\ensuretext{\textcolor{purple}{\ensuremath{^{\textsc{A}}_{\textsc{B}}}}}}
\newcommand{\jhmarker}{\ensuretext{\textcolor{cyan}{\ensuremath{^{\textsc{JD}}_{\textsc{H}}}}}}
\newcommand{\ajbmarker}{\ensuretext{\textcolor{blue}{\ensuremath{^{\textsc{AJ}}_{\textsc{B}}}}}}
\newcommand{\arkcomment}[3]{\ensuretext{\textcolor{#3}{[#1 #2]}\index{#1@\textcolor{#3}{#1}}}}
%\newcommand{\arkcomment}[3]{}
\newcommand{\cjd}[1]{\arkcomment{\cjdmarker}{#1}{green}}
\newcommand{\nss}[1]{\arkcomment{\nssmarker}{#1}{magenta}}
\newcommand{\nas}[1]{\arkcomment{\nasmarker}{#1}{red}}
\newcommand{\jb}[1]{\arkcomment{\jbmarker}{#1}{orange}}
\newcommand{\jh}[1]{\arkcomment{\jhmarker}{#1}{cyan}}
\newcommand{\ab}[1]{\arkcomment{\abmarker}{#1}{purple}}
\newcommand{\ajb}[1]{\arkcomment{\ajbmarker}{#1}{blue}}
\newcommand{\params}{\mathbf{\theta}}
\newcommand{\wts}{\mathbf{w}}
\newcommand{\g}{\mathbf{g}}
\newcommand{\f}{\mathbf{f}}
\newcommand{\x}{\mathbf{x}}
\newcommand{\y}{\mathbf{y}}
\newcommand{\overbar}[1]{\mkern 1.5mu\overline{\mkern-1.5mu#1\mkern-1.5mu}\mkern 1.5mu} % \bar is too narrow in math
\newcommand{\cost}{c}

\newcommand{\citeposs}[2][]{\citeauthor{#2}'s (\citeyear[#1]{#2})}
\newcommand{\Citeposs}[2][]{\Citeauthor{#2}'s (\citeyear[#1]{#2})}

%\usepackage{nameref}
\usepackage{cleveref}

% use \S for all references to all kinds of sections, and \P to paragraphs
% (sadly, we cannot use the simpler \crefname{} macro because it would insert a space after the symbol)
\crefformat{part}{\S#2#1#3}
\crefformat{chapter}{\S#2#1#3}
\crefformat{section}{\S#2#1#3}
\crefformat{subsection}{\S#2#1#3}
\crefformat{subsubsection}{\S#2#1#3}
\crefformat{paragraph}{\P#2#1#3}
\crefformat{subparagraph}{\P#2#1#3}
%\crefmultiformat{part}{\S#2#1#3}{ and~\S#2#1#3}{, \S#2#1#3}{, and~\S#2#1#3}
%\crefmultiformat{chapter}{\S#2#1#3}{ and~\S#2#1#3}{, \S#2#1#3}{, and~\S#2#1#3}
\crefmultiformat{section}{\S#2#1#3}{ and~\S#2#1#3}{, \S#2#1#3}{, and~\S#2#1#3}
\crefmultiformat{subsection}{\S#2#1#3}{ and~\S#2#1#3}{, \S#2#1#3}{, and~\S#2#1#3}
\crefmultiformat{subsubsection}{\S#2#1#3}{ and~\S#2#1#3}{, \S#2#1#3}{, and~\S#2#1#3}
\crefmultiformat{paragraph}{\P\P#2#1#3}{ and~#2#1#3}{, #2#1#3}{, and~#2#1#3}
\crefmultiformat{subparagraph}{\P\P#2#1#3}{ and~#2#1#3}{, #2#1#3}{, and~#2#1#3}
%\crefrangeformat{part}{\mbox{\S\S#3#1#4--#5#2#6}}
%\crefrangeformat{chapter}{\mbox{\S\S#3#1#4--#5#2#6}}
\crefrangeformat{section}{\mbox{\S\S#3#1#4--#5#2#6}}
\crefrangeformat{subsection}{\mbox{\S\S#3#1#4--#5#2#6}}
\crefrangeformat{subsubsection}{\mbox{\S\S#3#1#4--#5#2#6}}
\crefrangeformat{paragraph}{\mbox{\P\P#3#1#4--#5#2#6}}
\crefrangeformat{subparagraph}{\mbox{\P\P#3#1#4--#5#2#6}}
% for \label[appsec]{...}
\crefname{part}{Part}{Parts}
\Crefname{part}{Part}{Parts}
\crefname{chapter}{ch.}{ch.}
\Crefname{chapter}{Ch.}{Ch.}
\crefname{figure}{figure}{figures}
\crefname{subfigure}{figure}{figures}
\Crefname{subfigure}{Figure}{Figures}
\crefname{appsec}{appendix}{appendices}
\Crefname{appsec}{Appendix}{Appendices}
\crefname{algocf}{algorithm}{algorithms}
\Crefname{algocf}{Algorithm}{Algorithms}
\crefname{enums}{example}{examples}
\Crefname{enums}{Example}{Examples}
\crefname{enumsi}{example}{examples}
\Crefname{enumsi}{Example}{Examples}
\crefname{}{example}{examples} % lingmacros \toplabel has no internal name for the kind of label
\Crefname{}{Example}{Examples}
\crefformat{enums}{(#2#1#3)}
\crefformat{enumsi}{(#2#1#3)}
\crefformat{}{(#2#1#3)}
\crefname{xnumi}{example}{examples} % gb4e
\crefname{xnumi}{example}{examples} % gb4e
\Crefname{xnumii}{Example}{Examples} % gb4e
\Crefname{xnumii}{Example}{Examples} % gb4e
\crefformat{xnumi}{(#2#1#3)} % gb4e
\crefformat{xnumii}{(#2#1#3)} % gb4e
\crefrangeformat{enums}{\mbox{(#3#1#4--#5#2#6)}}
\crefrangeformat{enumsi}{\mbox{(#3#1#4--#5#2#6)}}
\crefrangeformat{xnumi}{\mbox{(#3#1#4--#5#2#6)}} % gb4e
\crefrangeformat{xnumii}{\mbox{(#3#1#4--#5#2#6)}} % gb4e
\crefmultiformat{enumsi}{(#2#1#3}{, #2#1#3)}{, #2#1#3}{, #2#1#3)}
\crefmultiformat{xnumi}{(#2#1#3}{, #2#1#3)}{, #2#1#3}{, #2#1#3)} % gb4e
\crefmultiformat{xnumii}{(#2#1#3}{, #2#1#3)}{, #2#1#3}{, #2#1#3)} % gb4e
\crefrangemultiformat{enumsi}{(#3#1#4--#5#2#6}{, #3#1#4--#5#2#6)}{, #3#1#4--#5#2#6}{, #3#1#4--#5#2#6)}
\crefrangemultiformat{xnumi}{(#3#1#4--#5#2#6}{, #3#1#4--#5#2#6)}{, #3#1#4--#5#2#6}{, #3#1#4--#5#2#6)} % gb4e
\crefrangemultiformat{xnumii}{(#3#1#4--#5#2#6}{, #3#1#4--#5#2#6)}{, #3#1#4--#5#2#6}{, #3#1#4--#5#2#6)} % gb4e


\ifx\creflastconjunction\undefined%
\newcommand{\creflastconjunction}{, and\nobreakspace} % Oxford comma for lists
\else%
\renewcommand{\creflastconjunction}{, and\nobreakspace} % Oxford comma for lists
\fi%

\newcommand*{\Fullref}[1]{\hyperref[{#1}]{\Cref*{#1}: \nameref*{#1}}}
\newcommand*{\fullref}[1]{\hyperref[{#1}]{\cref*{#1}: \nameref{#1}}}
\newcommand{\fnref}[1]{fn.~\ref{#1}} % don't use \cref{} due to bug in (now out-of-date) cleveref package w.r.t. footnotes
\newcommand{\Fnref}[1]{Fn.~\ref{#1}}

%\captionsetup[subfigure]{labelformat=simple}
\renewcommand\thesubfigure{(\alph{subfigure})}

% Space savers
% From http://www.eng.cam.ac.uk/help/tpl/textprocessing/squeeze.html
%\addtolength{\dbltextfloatsep}{-.5cm} % space between last top float or first bottom float and the text.
%\addtolength{\intextsep}{-.5cm} % space left on top and bottom of an in-text float.
%\addtolength{\abovedisplayskip}{-.5cm} % space before maths
%\addtolength{\belowdisplayskip}{-.5cm} % space after maths
%\addtolength{\topsep}{-.5cm} %space between first item and preceding paragraph
%\setlength{\belowcaptionskip}{-.25cm}

\usepackage{suffix} % for \WithSuffix

\usepackage{gb4e} % linguistic examples. put after all other package imports
\newenvironment{xexe}{\begin{exe}\ex\begin{xlist}}{\end{xlist}\end{exe}} % single numbered example with sub-examples


\newcommand{\backtick}[0]{\textasciigrave}


% Special macros
\newcommand{\w}[1]{\textit{#1}}	% word
\newcommand{\p}[1]{\textbf{\textsf{#1}}\index{#1@\textbf{\textsf{#1}}}} % preposition type
\WithSuffix\newcommand\p*[2]{\textbf{\textsf{#1}}\index{#2@\textbf{\textsf{#2}}}} % preposition token and lemma
\newcommand{\lbl}[1]{\textsc{#1}} % class label
\newcommand{\sst}[1]{\lbl{#1}\index{#1@\lbl{#1}}} % OLD supersense tag label
\newcommand{\nsst}[1]{\sst{n:#1}} % noun supersense tag label
\newcommand{\vsst}[1]{\sst{v:#1}} % verb supersense tag label
\newcommand{\psst}[1]{\psstX{#1}{#1}} % preposition supersense tag label
\newcommand{\psstX}[2]{\textcolor{mdgreen}{\hyperref[sec:#1]{\lbl{#2}}}\index{#1@\textcolor{mdgreen}{\textbf{\textsc{#1}}}}} % preposition supersense tag label
\newcommand{\psstdef}[1]{\textcolor{mdgreen}{\hyperref[sec:#1]{\lbl{#1}}}\index{#1@\textcolor{mdgreen}{\textbf{\textsc{#1}}}|textbf}} % preposition supersense tag label
\newcommand{\olbl}[1]{\textcolor{purple}{\textrm{#1}}} % other label: `i, `d, etc.
%\newcommand{\nsst}[1]{\sst{#1~\textroundcap{\vphantom{-}}~}} % noun supersense tag label
%\newcommand{\vsst}[1]{\sst{#1\raisebox{-1.5pt}{\textasciicaron}}} % verb supersense tag label
%\newcommand{\psst}[1]{\sst{#1\raisebox{2pt}{\rotatebox{180}{\textsublhalfring{\phantom{.}}}}}} %\textcorner % preposition supersense tag label

\newcommand{\rf}[2]{\psst{#1}$\leadsto$\psst{#2}\index[construals]{\protect\psst{#1}$\leadsto$\protect\psst{#2}}\index[revconstruals]{#2 #1@\protect\psst{#1}$\leadsto$\protect\psst{#2}}}
\newcommand{\rff}[3]{\psst{#1}$\leadsto$\psst{#2}$\leadsto$\psst{#3}\index[construals]{\psst{#1}$\leadsto$\psst{#2}$\leadsto$\psst{#3}}}
\newcommand{\backc}[0]{\hyperref[sec:coord]{\olbl{\backtick c}}\index{\backtick c@\olbl{\backtick c}}\xspace}
\newcommand{\backd}[0]{\hyperref[sec:discourse]{\olbl{\backtick d}}\index{\backtick d@\olbl{\backtick d}}\xspace}
\newcommand{\backi}[0]{\hyperref[sec:inf]{\olbl{\backtick i}}\index{\backtick i@\olbl{\backtick i}}\xspace}
\newcommand{\backposs}[0]{\hyperref[sec:possidiom]{\olbl{\backtick \$}}\index{\backtick \$@\olbl{\backtick \$}}\xspace}

\newcommand{\tg}[1]{\texttt{#1}}	% supersense tag name
\newcommand{\gfl}[1]{%\renewcommand\texttildelow{{\lower.74ex\hbox{\texttt{\char`\~}}}} % http://latex.knobs-dials.com/
\mbox{\textsmaller{\texttt{#1}}}}	% supersense tag symbol
 \newcommand{\tagdef}[1]{#1\hfill} % tag definition
\newcommand{\tagt}[2]{\ensuremath{\underset{\textrm{\textlarger{\tg{#2}}}\strut}{\w{#1}\rule[-.3\baselineskip]{0pt}{0pt}}}} % tag text (a word or phrase) with an SST. (second arg is the tag)
\newcommand{\glosst}[2]{\ensuremath{\underset{\textrm{#2}}{\textrm{#1}}}} % gloss text (a word or phrase) (second arg is the gloss)
\newcommand{\AnnA}[0]{\mbox{\textbf{Ann-A}}} % annotator A
\newcommand{\AnnB}[0]{\mbox{\textbf{Ann-B}}} % annotator B
\newcommand{\sys}[1]{\mbox{\textbf{#1}}}   % name of a system (one of our experimental conditions)
\newcommand{\dataset}[1]{\mbox{\textsc{#1}}}	% one of the datasets in our experiments
\newcommand{\datasplit}[1]{\mbox{\textbf{#1}}}	% portion one of the datasets in our experiments

\newcommand{\fnf}[1]{\textsc{\textsf{#1}}} % FrameNet frame
\newcommand{\fnr}[1]{\textbf{\textsf{#1}}} % FrameNet role (frame element name)
\newcommand{\fnrel}[1]{\textsl{#1}} % FrameNet frame relation type
\newcommand{\fnst}[1]{\textsl{#1}} % FrameNet semantic type
\newcommand{\fnlu}[1]{\textsf{#1}} % FrameNet lexical unit (predicate)
\newcommand{\pbf}[1]{\mbox{\textsf{#1}}} % PropBank frame (roleset)
\newcommand{\pbr}[1]{\textbf{\textsf{#1}}} % PropBank role (numbered or modifier argument label)
\newcommand{\vpred}[1]{\textbf{#1}} % verb predicate


%\newcommand{\lex}[1]{\textsmaller{\textsf{\textcolor{slate}{\textbf{#1}}}}}	% example lexical item
\newcommand{\lex}[1]{\textit{#1}} % lexical item/lexical example
\newcommand{\pex}[1]{\textit{#1}} % phrasal example - don't index by default

%\newcommand{\w}[1]{\textit{#1}}	% word
\newcommand{\gap}[0]{\ \ } % space around gap contents
\newcommand{\tat}[0]{\textasciitilde}

\newcommand{\shortlong}[2]{#1} % short vs. long version of the paper
\newcommand{\confversion}[1]{#1}
%\newcommand{\finalversion}[1]{#1}
\newcommand{\finalversion}[1]{}
\newcommand{\futureversion}[1]{}
\newcommand{\shortversion}[1]{#1}
\newcommand{\considercutting}[1]{#1}
\newcommand{\longversion}[1]{} % ...if only there were more space...
\newcommand{\subversion}[1]{#1} % for the submission version only
\newcommand{\draftnotice}[1]{} % for the draft version only
%\newcommand{\subversion}[1]{}

\newenvironment{ggroup}{{}}{{}}

\newcommand{\shortdef}[1]{\begin{mdframed}\noindent\textlarger{#1}\end{mdframed}}

\newenvironment{history}{\begin{mdframed}[linecolor=ltltgray,backgroundcolor=ltltgray]\small\noindent\textit{History.}}{\end{mdframed}}



\newcommand{\hierA}[1]{\textcolor{red}{\hyperref[sec:#1]{#1}}}
\newcommand{\hierB}[1]{\textcolor{blue}{\hyperref[sec:#1]{#1}}}
\newcommand{\hierC}[1]{\textcolor{mdgreen}{\hyperref[sec:#1]{#1}}}
\newcommand{\hierD}[1]{\textcolor{orange}{\hyperref[sec:#1]{#1}}}

\newcommand{\hierAdef}[1]{\section{\psstdef{#1}}\label{sec:#1}}
\newcommand{\hierBdef}[1]{\subsection{\psstdef{#1}}\label{sec:#1}}
\newcommand{\hierCdef}[1]{\subsubsection{\psstdef{#1}}\label{sec:#1}}
\newcommand{\hierDdef}[1]{\paragraph{\psstdef{#1}}\label{sec:#1}}

\hyphenation{WordNet}
\hyphenation{WordNets}
\hyphenation{FrameNet}
\hyphenation{SemCor}
\hyphenation{SemEval}
\hyphenation{ParsedSemCor}
\hyphenation{VerbNet}
\hyphenation{PennConverter}
\hyphenation{an-aly-sis}
\hyphenation{an-aly-ses}
\hyphenation{news-text}
\hyphenation{base-line}
\hyphenation{de-ve-lop-ed}
\hyphenation{comb-over}
\hyphenation{per-cept}
\hyphenation{per-cepts}
\hyphenation{post-edit-ing}
\hyphenation{shriv-eled}
\hyphenation{Huddle-ston}

\title{Adposition and Case Supersenses v2:\\ Guidelines for English}

\newcommand{\emldisplay}[2]{\texttt{\href{mailto:#1}{#2}}}
\newcommand{\eml}[1]{\textsmaller{\emldisplay{#1}{#1}}}

\author{\hspace{.2cm}\textbf{Nathan Schneider}\hspace{.2cm} \\ 
  \hspace{.2cm}Georgetown University\hspace{.2cm} \\
     \hspace{.2cm}\eml{nathan.schneider@georgetown.edu}\hspace{.2cm} \and
\textbf{Jena D. Hwang} \quad
\textbf{Archna Bhatia} \\
 	IHMC \\
     {\smaller \{\emldisplay{jhwang@ihmc.us}{jhwang},\emldisplay{abhatia@ihmc.us}{abhatia}\}\texttt{@ihmc.us}} \and
\textbf{Na-Rae Han} \\
	\hspace{.75cm}University of Pittsburgh\hspace{.75cm} \\
    \eml{naraehan@pitt.edu} \and 
\textbf{Vivek Srikumar} \\
	\hspace{1.25cm}University of Utah\hspace{1.25cm} \\
    \eml{svivek@cs.utah.edu} \and
\textbf{Tim O'Gorman} \quad \textbf{Sarah R. Moeller} \\
  University of Colorado Boulder \\
    {\smaller \{\emldisplay{timothy.ogorman@colorado.edu}{timothy.ogorman},\emldisplay{samo9533@colorado.edu}{samo9533}\}\texttt{@colorado.edu}} \and
\textbf{Omri Abend} \\
  Hebrew University of Jerusalem \\
    \eml{oabend@cs.huji.ac.il} \and
\textbf{Austin Blodgett \quad \textbf{Jakob Prange}} \\
  Georgetown University \\
    {\smaller \{\emldisplay{ajb341@georgetown.edu}{ajb341},\emldisplay{jp1724@georgetown.edu}{jp1724}\}\texttt{@georgetown.edu}}
}

\date{\today \nss{insert arXiv date here}}

\begin{document}
\maketitle
\begin{abstract}
\noindent 
This document offers a detailed linguistic description of SNACS \citep[Semantic Network of Adposition and Case Supersenses;][]{schneider-18}, 
an inventory of 50~semantic labels (``supersenses'')
that characterize the use of adpositions and case markers 
at a somewhat coarse level of granularity, 
as demonstrated in the STREUSLE~4.1 corpus (\url{https://github.com/nert-gu/streusle/}).
Though the SNACS inventory aspires to be universal, this document is specific to English; 
documentation for other languages will be published separately.

Version~2 is a revision of the supersense inventory proposed for English by 
\citet{schneider-15,schneider-16} %and documented in PrepWiki\footnote{\url{http://tiny.cc/prepwiki}} 
(henceforth ``v1''), which in turn was based on previous schemes.
The present inventory was developed after extensive review of the 
v1 corpus annotations for English, 
plus previously unanalyzed genitive case possessives \citep{blodgett-18},
as well as consideration of adposition 
and case phenomena in Hebrew, Hindi, Korean, and German. 
\Citet{hwang-17} present the theoretical underpinnings of the v2 scheme.
\Citet{schneider-18} summarize the scheme, its application to English corpus data, 
and an automatic disambiguation task.
\end{abstract}


\tableofcontents


\section{Overview}

%\nss{v1 is documented in PrepWiki (\url{http://tiny.cc/prepwiki}).
%v2 will be documented in Xposition (URL TBD).}

This document details version~2 of a scheme for annotating English 
prepositions and related grammatical markers with semantic class categories 
called \emph{supersenses}. 
The motivation and general principles for this scheme are laid out in 
publications cited in the abstract. 
This document focuses on the technical details, giving definitions, 
descriptions, and examples for each supersense and a variety of 
prepositions and constructions that occasion its use.

\subsection{What counts as an adposition?}

``Adposition'' is the cover term for prepositions and postpositions. 
Briefly, we consider an affix, word, or multiword expression to be adpositional if it:
\begin{itemize}
  \item mediates a semantically asymmetric figure--ground relation between two concepts, and
  \item is a grammatical item that can mark an NP. 
  We annotate \emph{tokens} of these items even where they mark clauses (as a subordinator) 
  or are intransitive.\footnote{Usually a coordinating conjunction, \p{but}  
  only receives a supersense when it is prepositional, as described 
  under \psst{PartPortion}.}
  We also include always-intransitive grammatical items whose core meaning is spatial and highly schematic, 
  like \p{together}, \p{apart}, and \p{away}.
  % \item Is not a differential object marker (e.g., Hebrew \p{'et}, which marks direct objects if and only if 
  % they are definite).
\end{itemize}
% \ab{differential object markers are ignored because they don't have any lexical semantic content, is that the criterion?}
% 
% \nss{What about a word that matches the above criteria where it is used as an intransitive 
% predicate, e.g. \pex{She is \p{out}/\p{away}}?}\ab{I think that is an adverbial usage instead of adpositional.}

Inspired by \citet{cgel}, the above criteria are broad enough to include 
a use of a word like \p{before} whether it takes an NP complement, 
takes a clausal complement (traditionally considered a subordinating conjunction), 
or is intransitive (traditionally considered an adverb):
\begin{xexe}
  \ex It rained \p{before} the party. [NP complement]
  \ex It rained \p{before} the party started. [clausal complement]
  \ex It rained \p{before}. [intransitive]
\end{xexe}

Even though they are not technically adpositions, 
we also apply adposition supersenses to possessive case marking 
(the clitic \p{'s} and possessive pronouns),
and some uses of the infinitive marker \p{to}, as detailed in \cref{sec:cxns}.

\subsection{Inventory}

The v2 hierarchy is a tree with 50~labels.
They are organized into three major subhierarchies: 
\psst{Circumstance} (18~labels), \psst{Participant} (14~labels), 
and \psst{Configuration} (18~labels). 

\begin{minipage}{\textwidth}
\vspace{.2cm}
\begin{multicols}{3}
\begin{ggroup}
  \sffamily\color{gray}
\begin{forest}
  for tree={%
    folder,
    grow'=0,
    fit=band,
    inner ysep=.75,
  }
  [{\hierA{Circumstance}}
    [{\hierB{Temporal}}
      [{\hierC{Time}}
        [{\hierD{StartTime}}]
        [{\hierD{EndTime}}]
      ]
      [{\hierC{Frequency}}]
      [{\hierC{Duration}}]
      [{\hierC{Interval}}]
    ]
    [{\hierB{Locus}}
      [{\hierC{Source}}]
      [{\hierC{Goal}}]
    ]
    [{\hierB{Path}}
      [{\hierC{Direction}}]
      [{\hierC{Extent}}]
    ]
    [{\hierB{Means}}]
    [{\hierB{Manner}}]
    [{\hierB{Explanation}}
      [{\hierC{Purpose}}]
    ]
  ]
\end{forest}
\columnbreak

\begin{forest}
  for tree={%
    folder,
    grow'=0,
    fit=band,
    inner ysep=.75,
  }
  [{\hierA{Participant}}
    [{\hierB{Causer}}
      [{\hierC{Agent}}
        [{\hierD{Co-Agent}}]
      ]
    ]
    [{\hierB{Theme}}
      [{\hierC{Co-Theme}}]
      [{\hierC{Topic}}]
    ]
    [{\hierB{Stimulus}}]
    [{\hierB{Experiencer}}]
    [{\hierB{Originator}}]
    [{\hierB{Recipient}}]
    [{\hierB{Cost}}]
    [{\hierB{Beneficiary}}]
    [{\hierB{Instrument}}]
  ]
\end{forest}
\columnbreak

\begin{forest}
  for tree={%
    folder,
    grow'=0,
    fit=band,
    inner ysep=.75,
  }
  [{\hierA{Configuration}}
    [{\hierB{Identity}}]
    [{\hierB{Species}}]
    [{\hierB{Gestalt}}
      [{\hierC{Possessor}}]
      [{\hierC{Whole}}]
    ]
    [{\hierB{Characteristic}}
      [{\hierC{Possession}}]
      [{\hierC{PartPortion}}
        [{\hierD{Stuff}}]
      ]
    ]
    [{\hierB{Accompanier}}]
    [{\hierB{InsteadOf}}]
    [{\hierB{ComparisonRef}}]
    [{\hierB{RateUnit}}]
    [{\hierB{Quantity}}
      [{\hierC{Approximator}}]
    ]
    [{\hierB{SocialRel}}
      [{\hierC{OrgRole}}]
    ]
  ]
\end{forest}
\end{ggroup}
\end{multicols}
\end{minipage}

\begin{itemize}
\item Items in the \psst{Circumstance} subhierarchy are prototypically 
expressed as adjuncts of time, place, manner, purpose, etc.\ 
elaborating an event or entity.
\item Items in the \psst{Participant} subhierarchy are prototypically 
entities functioning as arguments to an event.
\item Items in the \psst{Configuration} subhierarchy are prototypically
entities or properties in a static relationship to some entity.
\end{itemize}

\subsection{Limitations}

This inventory is only designed to capture semantic relations 
with a figure--ground asymmetry. This excludes:
\begin{itemize}
  \item The semantics of coordination, where the two sides of the relation 
are on equal footing (see \cref{sec:coord}).
%(Note that sometimes a morpheme can 
%have symmetric as well as asymmetric interpretations: e.g., Korean \p{-wa}.)
  \item Aspects of meaning that pertain to information structure, discourse, 
or pragmatics (see \cref{sec:discourse}).
\end{itemize}
Moreover, this inventory only captures semantic distinctions 
that tend to correlate with major differences in syntactic distribution. 
Thus, while there are labels for locative (\psst{Locus}), ablative (\psst{Source}), 
allative (\psst{Goal}), and \psst{Path} semantics---and analogous temporal categories---%
finer-grained details of spatiotemporal meaning are for the most part lexical 
(viz.: the difference between \pex{\p{in} the box} and \pex{\p{on} the box}, 
or temporal \p{at}, \p{before}, \p{during}, and \p{after}) and are not represented here.\footnote{This is not to claim
that all members of a category can be grammatical in all the same contexts: 
\pex{\p{on} Saturday} and \pex{\p{at} 5:00} are both labeled \psst{Time}, 
though the prepositions are by no means interchangeable in American English. 
We are simply asserting that the different constructions specific to days of the week 
versus times of the day are minor aspects of the grammar of English.}

\subsection{Major changes from v1}

Changes that affect only a single label are explained below the relevant 
v2 labels.

\begin{itemize}
  \item \textbf{Removed multiple inheritance.} 
  The v1 network was quite tangled. The structure is greatly simplified 
  by analyzing some tokens as \emph{construals} \citep{hwang-17}.
  \item \textbf{Revised and expanded the \psst{Configuration} subhierarchy.}
  \item \textbf{Removed the locative concreteness distinction.}
  In v1, labels \sst{Location}, \sst{InitialLocation}, and \sst{Destination} 
  were reserved for concrete locations, and the respective supertypes 
  \psst{Locus}, \psst{Source}, and \psst{Goal} used to cover abstract locations.
  This distinction was found to be difficult and without apparent
  relevance to preposition system of English or the other languages considered. 
  The concrete labels were thus removed.
  \item \textbf{Removed the location/state/value distinction.}
  The v1 scheme attempted to make an elaborate distinction between 
  values, states, and other kinds of abstract locations. 
  However, the English preposition system does not seem particularly 
  sensitive to these distinctions. (We are not aware of any prepositions 
  that mark primarily values or primarily states; rather, productive 
  metaphors allow locative prepositions to be extended to cover these, 
  and there are cases where teasing apart abstract location vs.~state vs.~value 
  is difficult.) Therefore, \sst{State}, \sst{StartState}, \sst{EndState}, 
  \sst{Value}, and \sst{ValueComparison} were removed. %\footnote{\nss{In v1, 10 tokens of \sst{StartState} and 20 of \sst{EndState}, mostly involving values on a scale, now merged with \psst{Source} and \psst{Goal}, resp.}}
  \item \textbf{Revised the treatment of comparison and related notions.} 
  Removed \sst{Comparison/Contrast}, \sst{Scalar/Rank}, \sst{ValueComparison}; 
  moved \psst{Approximator} under \psst{Quantity}.
  \item \textbf{Greatly simplified the \psst{Path} subhierarchy.} See \cref{sec:Path}.
  \item \textbf{Simplified the \psst{Temporal} subhierarchy.} See \cref{sec:Temporal}.
  \item \textbf{Removed} \sst{Activity} (mostly replaced with \psst{Circumstance} and \psst{Topic}), 
  \sst{Reciprocation} (mostly merged with \psst{Explanation}), and \sst{Material} (merged with \psst{Source}).
  \item \textbf{Removed abstract labels} \psst{Affector}, \psst{Undergoer}, and \psst{Place}.
\end{itemize}

\subsection{Major changes from earlier versions of this document}

\begin{itemize}
  \item \emph{Since the \textbf{January 16, 2018} version:}
  \begin{itemize}
    \item Policy changes reflected in STREUSLE~4.0: 
    \begin{itemize}
      \item Rewrote \fullref{sec:genitives} and updated corresponding examples 
      to reflect a clarified policy on possessive construals. 
      Moved wearer from \psst{Gestalt} to \psst{Possessor}
      and attire from \psst{Characteristic} to \psst{Possession}.
      \item Added \fullref{sec:passives} and updated corresponding examples.
    \end{itemize}
    \item Policy changes that will be reflected in STREUSLE~4.1: 
    \begin{itemize}
      \item In \cref{sec:as-as}, changed the function of the first \p{as} in the \p{as}-\p{as} construction 
      to \psst{Extent} (was \psst{Identity}).
      \item Changed the function of \psst{Originator} possessives to \psst{Gestalt} (was \psst{Possessor}).
      \item Expanded documentation and removed inconsistencies around containers and collective nouns 
      (see \psst{Stuff}, \psst{Quantity}, \psst{Characteristic}, \psst{OrgRole}).
      \item Specified \rf{Manner}{ComparisonRef} for certain adverbial uses of \p{like}.
      \item Revised the definition of \psst{Recipient} to relax the requirement of animacy.
      \item Mentioned conditions as a subclass of \psst{Circumstance}.
      \item Renamed \sst{Part/Portion} to \psst{PartPortion} to avoid technical complications 
      of the slash.
    \end{itemize}
    \item Added \fullref{sec:constraints}.
    \item Added \fullref{sec:age}.
    \item A few additional examples and fixes.
    \item Added an index of construals by function.
    \item Changes from v1 had neglected to mention the removal of \sst{Affector}, \sst{Undergoer}, \sst{Place}, 
    \sst{Elements}, and \sst{Superset} (thanks to Ken Litkowski for pointing this out).
  \end{itemize}
  \item \emph{Since the \textbf{April 7, 2017} version:}
  \begin{itemize}
    \item Broadened and clarified \sst{DeicticTime}, moved it up a level in the hierarchy, 
      and renamed it to \psst{Interval}. Clarified the distinction between \psst{Interval} and \psst{Duration}.
    \item Clarified \psst{Locus}, \psst{Source}, \psst{Goal}, \psst{Path}, and 
      \psst{Direction}, especially with regard to (i)~intransitive prepositions, 
      (ii)~distance measurements, and (iii)~inherent parts.
    \item Significantly expanded the scope of \psst{Manner} to cover states of entities and depictives.
    \item Clarified \p{like} as \psst{ComparisonRef} with regard to categories and sets, 
    and \psst{PartPortion} with regard to elements and exceptions.
    \item Clarified \p{with} in regard to \psst{Topic} and \psst{Stimulus}.
    \item Added discussion of the ambiguity of temporal \p{over}: \psst{Duration} versus \rf{Time}{Duration}.
    \item Extensively clarified \psst{Purpose} and \psst{Beneficiary}, 
    and their relationship to \psst{ComparisonRef}, \psst{Recipient}, \psst{Experiencer}, and \psst{Stimulus}.
    \item Clarified that goods and services are \psst{Theme}; expanded on \psst{Co-Theme} examples.
    \item \psst{Frequency} used for an iteration.
    \item Various selectional verbs and miscellaneous constructions.
    \item Added examples of \p{'s} possessive/genitive marking.
    \item Added section for special syntactic constructions (\cref{sec:cxns}).
    \item Added special labels (\cref{sec:special}).
    \item Added an index of adpositions and supersenses, and an index of construals.
    \item Revised the title, abstract, and introductory material.
  \end{itemize}
\end{itemize}

\hierAdef{Circumstance}

\shortdef{Macrolabel for labels pertaining to space and time,
and other relations that are usually semantically non-core properties of events.}

\psst{Circumstance} is used directly for:
\begin{itemize}
  \item \textbf{Contextualization}
\begin{exe}
    \ex \p*{In}{in} arguing for tax reform, the president claimed that loopholes allow 
    big corporations to profit from moving their headquarters overseas.
    \ex\label{ex:in-activity-Circumstance} 
      You crossed the line \p{in} sharing confidential information.\\{} 
      [but see \cref{ex:in-activity-Topic} under \psst{Topic}, which is syntactically parallel]
    \ex I found out \p{in} our conversation that she speaks 5~languages.
    \ex\rf{Circumstance}{Locus}:\begin{xlist}
      \ex I haven't seen them \p{in} that setting.
      \ex \p*{In}{in} that case, I wouldn't worry about it.
    \end{xlist}
    \ex We have to keep going \p{through} all these challenges. [metaphoric motion] (\rf{Circumstance}{Path})
    \ex Bipartisan compromise is unlikely \p{with} the election just around the corner.
    \ex\label{ex:as-while} \p*{As}{as} we watched, she transformed into a cat. 
    [`while', `unfolding at the same time as'; not simply providing a `when'---contrast~\cref{ex:as-when} under \psst{Time}]
  \end{exe}
  For these cases, the preposition helps situate 
  the background context in which the main event takes place. 
  The background context is often realized as a subordinate clause 
  preceding the main clause. 
  It may also be realized as an adjective complement:
  \begin{exe}
    \ex\begin{xlist}
      \ex My tutor was helpful \p{in} giving concrete examples and exercises.
      \ex You were correct \p{in} \choices{answering the question\\your answer}.
    \end{xlist}
  \end{exe}
  Relatedly, we use \psst{Circumstance} to analyze \pex{involved \p{in}}:
  \begin{exe}
    \ex\begin{xlist}
      \ex I was involved \p{in} a car accident. (\psst{Circumstance})
      \ex Many steps are involved \p{in} the process of buying a home. \\(\rf{Whole}{Circumstance})
    \end{xlist}
  \end{exe}
  \item \textbf{Setting events}
  \begin{exe}
    \ex\label{ex:settingevt} We are having fun \choices{\p{at} the party\\\p{on} vacation}. (\rf{Circumstance}{Locus})
  \end{exe}
  The object of the preposition is a noun denoting a containing event; 
  it thus may help establish the place, time, and/or reason for the governing scene, 
  but is not specifically providing any one of these, despite the locative preposition.
  These can be questioned (at least in some contexts) with \emph{Where?} or \emph{When?}. 
  \Cref{ex:settingevt} entails \cref{ex:settingevtpred}:
  \begin{exe}
    \ex\label{ex:settingevtpred} We are \choices{\p{at} the party\\\p{on} vacation}. (\rf{Circumstance}{Locus})
  \end{exe}
  which may be responsive to the questions \emph{Where are you?} and \emph{What are you doing?}.\footnote{When 
  the object of the preposition is not a (dynamic) event, as with \pex{We are \p{at} odds/\p{on} medication}, 
  \rf{Manner}{Locus} is used: see discussion of state PPs at \psst{Manner}.}
  
  \item \textbf{Occasions}
  \begin{exe}
    \ex I bought her a bike \p{for} Christmas.
    \ex I had peanut butter \p{for} lunch.
  \end{exe}
  These simultaneously express a \psst{Time} and some element of causality 
  similar to \psst{Purpose}.
  But the PP is not exactly answering a \pex{Why?}\ or \pex{When?}\ question. 
  Instead, the sentence most naturally answers a question like \pex{On what occasion was X done?}
  or \pex{Under what circumstances did X happen?}.
  \item Any other descriptions of event/state properties that are \textbf{insufficiently specified} 
  to fall under spatial, temporal, causal, or other subtypes like \psst{Manner}. E.g.:
  \begin{exe}
    \ex\label{ex:over-lunch} Let's discuss the matter \p{over} lunch. [compare \cref{ex:at-lunch}]
  \end{exe}

\item \textbf{Conditions}
\begin{exe}
  \ex You can leave \choices{\p{as\_long\_as}\\provided} your work is done.
  \ex Whether you can leave \choices{depends \p{on}\\is subject \p{to}} whether your work is done.
\end{exe}
\end{itemize}

\hierBdef{Temporal}

\shortdef{Abstract supercategory for temporal descriptions: 
\textbf{when}, \textbf{for how long}, \textbf{how often}, \textbf{how many times}, 
etc.\ something happened or will happen.}

\begin{history}
  The v1 category \sst{Age} (e.g., \pex{a child \p{of} five}) 
  was a mutual subtype of \psst{Temporal} and \sst{Attribute}. 
  Being quite specific and rare, for v2 it was merged with \psst{Characteristic}. 
  Combined with the changes to \psst{Time} subcategories (see below), 
  this reduced by~3 the number of labels in the \psst{Temporal} subtree, 
  bringing it to 7.
\end{history}

\hierCdef{Time}

\shortdef{\textbf{When} something happened or will happen, in relation to an 
explicit or implicit reference time or event.}

\begin{exe}
  \ex We ate \choices{\p{in} the afternoon\\\p{during} the afternoon\\\p{at} 2:00\\\p{on} Friday}.
  \ex\label{ex:at-lunch} Let's talk \choices{\p{at}\\\p{during}} lunch. [compare \cref{ex:over-lunch}]
\end{exe}
For a containing time period or event, \p{during} can be used and is unambiguously \psst{Time}---%
unlike \p{in}, \p{at}, and \p{on}, which can also be locative.\footnote{See \cref{sec:temploc} regarding the 
use of locational metaphors for temporal relations.}
\begin{exe}
  \ex\begin{xlist}
    \ex They will greet us \choices{\p{on}\\\p{upon}} our arrival.
    \ex\label{ex:onXOccasion} I succeeded \p{on} \choices{the fourth attempt\\several occasions}. [contrast \emph{on occasion}, \cref{ex:onOccasion}]
  \end{xlist}
  \ex\label{ex:as-when}\p*{As}{as} meaning `when' (contrast \cref{ex:as-while} under \psst{Circumstance}):\begin{xlist}
    \ex The lights went out \p{as} I opened the door.
    \ex A bee stung me \p{as} I was eating lunch.
  \end{xlist}
  \ex I will finish \p{after} \choices{tomorrow\\lunch\\you (do)}.
  \ex I will finish \p{by} \choices{tomorrow\\lunch}.
  \ex I will contact you \choices{\p{as\_soon\_as}\\once} it's ready.
\end{exe}

The preposition \p{since} is ambiguous:
\begin{exe}
  \ex {} [`after'] I bought a new car---that was \p{since} the breakup. (\psst{Time})
  \ex {} [`ever since'] I have loved you \p{since} the party where we met. (\psst{StartTime}) %\ab{are 9 \& 10 really different?\nss{yes, 10 cannot be paraphrased with `after'}}
  \ex {} [`because'] I'll try not to whistle \p{since} I know that gets on your nerves. (\psst{Explanation})
\end{exe}

Simple \psst{Time} is also used if the reference time is implicit and determined from 
the discourse:
\begin{exe}
  \ex We broke up last year, and I haven't seen her \p{since}. [since we broke up]
\end{exe}

However, \rf{Time}{Interval} is used for adpositions whose complement (object) 
is the amount of time between two reference points:
\begin{exe}
  \ex We left the party \p{after} an hour. [an hour after it started] (\rf{Time}{Interval})
  \ex We left the party an hour \p{ago}. [an hour before now] (\rf{Time}{Interval})
\end{exe}

The preposition \p{over} is also ambiguous:
\begin{exe}
  \ex The deal was negotiated \p{over} (the course of) a year. (\psst{Duration})
  \ex He arrived in town \p{over} the weekend. (\rf{Time}{Duration})
\end{exe}
See discussion under \psst{Duration}.

If the scene role is \psst{Time}, the PP can usually be questioned with \emph{When?}.

\psst{Time} is also used for special constructions for expressing clock times, e.g.~identifying 
a time via an offset:
\begin{exe}
  \ex\begin{xlist}
    \ex The alarm rang at \choices{a quarter \p{after}\\half \p{past}} 8. (\psst{Time})
    \ex\label{ex:quarterTo} The alarm rang at a quarter \p{to} 8. (\rf{Time}{Goal})
    \ex The alarm rang at a quarter \p{of} 8.\footnote{In some dialects, this is an alternate way to express the same meaning as \cref{ex:quarterTo}. 
    It seems that \p{to} and \p{of} construe the same time interval from opposite directions.} (\rf{Time}{Source})
  \end{xlist}
  \ex The alarm rang 15~minutes \p{before} 8. (\psst{Time}) [``15~minutes'' modifies the PP]
\end{exe}

\begin{history}
  In v1, point-like temporal prepositions (\p{at}, \p{on}, \p{in}, \p{as}) 
  were distinguished from displaced temporal prepositions (\p{before}, \p{after}, etc.)\ 
  which present the two times in the relation as unequal. 
  \sst{RelativeTime} inherited from \psst{Time} and was reserved for the 
  displaced temporal prepositions, as well as subclasses \psst{StartTime}, 
  \psst{EndTime}, \sst{DeicticTime}, and \sst{ClockTimeCxn}. 
  
  For v2, \sst{RelativeTime} was merged into \psst{Time}: the distinction 
  was found to be entirely lexical and lacked parallelism with the spatial hierarchy. 
  \sst{ClockTimeCxn} was also merged with \psst{Time}, the usages covered by the former 
  (expressions of clock time like \pex{ten \p{to} seven})
  being exceedingly rare and not very different semantically from 
  prepositions like \p{before}.
  \sst{DeicticTime} became \psst{Interval}.
\end{history}

\hierDdef{StartTime}

\shortdef{When the event denoted by the governor begins.}

Prototypical prepositions are \p{from} and \p{since} (but see note under \psst{Time} 
about the ambiguity of \p{since}):
\begin{exe}\ex\begin{xlist}
  \ex The show will run \p{from} 10 a.m. to 2 p.m.
  \ex a document dating \p{from} the thirteenth century
\end{xlist}\end{exe}

Note that simple \psst{Time} is used with verbs like \w{start} and \w{begin}: 
the event directly described by the PP is the starting, not the thing that started.
\begin{exe}
  \ex The show will start \p{at} 10 a.m. (\psst{Time})
\end{exe}

\hierDdef{EndTime}

\shortdef{When the event denoted by the governor finishes.}

Prototypical prepositions are \p{to}, \p{until}, \p{till}, \p{up\_to}, and \p{through}:
\begin{exe}
  \ex The show will run from 10 a.m. \p{to} 2 p.m.
  \ex Add the cider and boil \p{until} the liquid has reduced by half.
  \ex If we have survived \p{up\_to} now what is stopping us from surviving in the future?
  \ex They will be in London from March 24 \p{through} May 7.
\end{exe}

Note that simple \psst{Time} is used with verbs like \w{end} and \w{finish}: 
the event directly described by the PP is the ending, not the thing that ended.
\begin{exe}
  \ex The show will end \p{at} 2~p.m. (\psst{Time})
\end{exe}


\hierCdef{Frequency}

\shortdef{\textbf{At what rate} something happens or continues, 
or the instance of repetition that the event represents.}

\begin{exe}
  \ex Guests were arriving \p{at} a steady clip.
  \ex The risk becomes worse \p{by} the day.
  \ex\label{ex:onOccasion} I see them \choices{\p{on}\_occasion\\\p{from}\_time\_to\_time}. [contrast \emph{on \dots occasion}, \cref{ex:onXOccasion}]
  \ex\label{ex:dailyBasis} I see them \p{on}\_a\_~~daily~~\_basis. (\rf{Frequency}{Manner}) [cf.~\cref{ex:bipartisanBasis}]
  \ex I keep getting the same message \p{over} and \p{over} again. %\nss{weak or strong MWE?}
\end{exe}
\psst{Frequency} is also used when an iteration is specified with an obligatory 
ordinal number modifier. 
If the ordinal number is optional, the preposition presumably receives another label:
\begin{exe}
  \ex\begin{xlist}
    \ex The camcorder failed \p{for} the third time. (\psst{Frequency})
    \ex They won \p{for} the third year in$_{\psst{Manner}}$ a row. (\psst{Frequency})
    \ex We arrived \p{for} our (third) visit. (\psst{Purpose})
  \end{xlist}
\end{exe}

Contrast: \psst{RateUnit}

\hierCdef{Duration}

\shortdef{Indication of \textbf{how long} an event or state lasts
(with reference to an amount of time or 
time period\slash larger event that it spans).}

\begin{exe}
  \ex\label{ex:forDuration} I walked \choices{\p{for}\\\#\p{in}} 20~minutes.
  \ex\label{ex:GoalDuration} I walked to$_{\psst{Goal}}$ the store \choices{\p{in}/\p{within}\\\#\p{for}} 20~minutes. [see \cref{ex:inDuration}]
  \ex\label{ex:ExtentDuration} I walked a mile \choices{\p{in}/\p{within}\\\#\p{for}} 20~minutes.
  \ex\label{ex:AmbigDuration} I mowed the lawn \choices{\p{for}\\\p{in}/\p{within}} an hour.
\end{exe}
Note that the presence of a goal \cref{ex:GoalDuration} or 
extent of an event (\pex{a mile} in \cref{ex:ExtentDuration}) 
can affect the choice \psst{Duration} preposition, blocking \p{for}.
\Cref{ex:AmbigDuration} shows a direct object which can be interpreted 
either as something against which partial progress is made---licensing \p{for} 
and the inference that some of the lawn was not reached---or 
as defining the complete scope of progress, licensing \p{in}/\p{within} 
and the inference that the lawn was covered in its entirety.

The object of a \psst{Duration} preposition can also be a reference event 
or time period used as a yardstick for the extent of the main event:
\begin{exe}
  \ex\label{ex:EventDuration} I walked \p{for} the entire race. [the entire time of the race]
  \ex I walked \choices{\p{throughout}\\\p{through}\\well \p{into}} the night.
  \ex\label{ex:overDuration} The deal was negotiated \p{over} (the course of) a year.
\end{exe}
But \p{over} can also mark a time period that \emph{contains} the main event 
and is larger than it. While the path preposition \p{over} highlights that the 
object of the preposition extends over a period of time, it does not require that 
the main event extend over a period of time:
\begin{exe}
  \ex\label{ex:overTimeDuration} He arrived in town \p{over} the weekend. (\rf{Time}{Duration})
\end{exe}
Note that \p{during} can be substituted for \p{over} in \cref{ex:overTimeDuration} but not \cref{ex:overDuration}.

Some \p{for}-\psst{Duration}s measure the length of the specified event's \emph{result}:
\begin{exe}\ex \begin{xlist}
  \ex John went to the store \p{for} an hour. [he spent an hour at the store, not an hour going there]\footnote{This stands 
in contrast with \pex{John walked to the store \p{for} an hour}, where the most natural reading is that it took an hour to get to the store \citep[p.~230]{chang-98}.}
  \ex John left the party \p{for} an hour. [he spent an hour away from the party before returning]
\end{xlist}\end{exe}

A \psst{Duration} may be a stretch of time in which a simple event is repeated 
iteratively or habitually:
\begin{exe}\ex\begin{xlist}
  \ex I lifted weights \p{for} an hour. [many individual lifting acts collectively lasting an hour]
  \ex I walked to the store \p{for} a year. [over the course of a year, habitually went to the store by walking]
\end{xlist}\end{exe}

See further discussion at \psst{Interval}.

\hierCdef{Interval}

\shortdef{A marker that points retrospectively or prospectively in time, 
and if transitive, marks the time elapsed between two points in time.}

The clearest example is \p{ago}, which only serves to locate the \psst{Time}
of some past event in terms of its distance from the present:
\begin{exe}
  \ex\label{ex:ago} I arrived a year \p{ago}. \rf{Time}{Interval} \\{}
  [points backwards from the present: before now]
\end{exe}
The most common use of \psst{Interval} is in the construal \rf{Time}{Interval}: 
the time of an event is described via a temporal offset from some other time.

Another retrospective marker, \p{back}, can be transitive \cref{ex:backTrans}, 
or can be an intransitive modifier 
of a \psst{Time} PP \cref{ex:backIntrans}. 
Plain \psst{Interval} is used in the latter case:
\begin{exe}
  \ex\label{ex:backTrans} I arrived a year \p{back}.\footnote{While 
  \pex{a while \p{back}} and \pex{a few generations \p{back}} are generally accepted, 
  %\p{back} with smaller measurements,
  the use of \p{back} rather than \p{ago} for nearer and more precise temporal references,
  e.g.~\pex{10~minutes \p{back}}, appears to be especially associated with Indian English \citep[p.~7]{yadurajan-01}.} \rf{Time}{Interval}
  \ex\label{ex:backIntrans} I arrived \p{back} in$_{\psst{Time}}$ June. (\psst{Interval})
\end{exe}



(This category is unusual in primarily marking a construal for a different scene role. 
But this seems justified given the restrictive set of English temporal prepositions 
that can appear with a temporal offset, and the distinct ambiguity of \p{in}.
\psst{Interval} is designed as the temporal counterpart of \psst{Direction}, 
which can construe static distance measures; 
in fact, \sst{TimeDirection} was considered as a possible name, 
but \psst{Interval} seemed more straightforward for the most frequent class of usages.)

Other adpositions can also take an amount of intervening time as their \emph{complement} (object):
\begin{exe}
  \ex\label{ex:inAmbiguousTime} I will eat \p{in} 10~minutes.
    \begin{xlist}
      \ex\label{ex:inDuration} {} [`for no more than 10~minutes' reading]: \psst{Duration}\footnote{This usage of \p{in} has been classified under the terms \emph{frame adverbial} \citep{pustejovsky-91} and \emph{span adverbial} \citep{chang-98}.}
      \ex\label{ex:inInterval} {} [`10~minutes from now' reading]: \rf{Time}{Interval}\footnote{This usage of \p{in}, as well as \p{ago} \cref{ex:ago} and \p{back} \cref{ex:backTrans,ex:backIntrans},
      are \emph{deictic}, i.e., they are inherently relative to the speech time or deictic center. 
      (See also \citet[pp.~154--157]{klein-94}.)
      This was taken to be a criterion for the v1 category \sst{DeicticTime}, 
      but that was never well-defined in v1 and was broadened for this version.}
    \end{xlist}
  % \ex\begin{xlist}\nss{not sure what to do with these. remove for now}
  %   \ex What are the revenue projections 6~months \p{out}?
  %   \ex I've started watching a new TV series and am 3~episodes \p{in}.\nss{``3 episodes into the show''?}
  %\end{xlist}
  \ex\label{ex:AfterObj} The game started at 7:00, but I arrived \choices{\p{after}\\\p{within}} 20~minutes.
\end{exe}
Some adpositions license a temporal difference measure in \emph{modifier} position, which does not qualify:
\begin{exe}
  \ex To beat the crowds, I will arrive \uline{a while} \choices{\p{before} (it starts)\\\p{beforehand}}. (\psst{Time})
  \ex\label{ex:AfterMod} The game started at 7:00, but I arrived \uline{20~minutes} \choices{\p{after} (it started)\\\p{afterward}}. (\psst{Time})
\end{exe}
The preposition \p{after} can be used either way---contrast \cref{ex:AfterMod} with \cref{ex:AfterObj}.

Note that having \psst{Interval} as a separate category allows us to distinguish the sense of \p{in} 
in \cref{ex:inInterval} from both the \psst{Duration} sense \cref{ex:inDuration} 
and the \psst{Time} sense (\pex{\p{in} the morning}).

\paragraph{Versus \psst{Duration}.} 
The prepositions \p{in} and \p{within} are ambiguous between \psst{Interval} and \psst{Duration}.\footnote{By contrast, 
\p{after} seems to strongly favor \rf{Time}{Interval}. 
\pex{\p*{After}{after} a week, I had climbed all the way to the summit} is possible, 
but the conclusion that the climbing took a week may be an inference 
rather than something that is directly expressed.}
The distinction can be subtle and context-dependent.
The key test is whether the phrase answers a \emph{When?}\ question. 
If so, its scene role is \psst{Time}; otherwise, it is a \psst{Duration}.
%With a verb which refers to a punctual moment of culmination, 
%we use \psst{Interval} even though there may be an implicit 
%preparatory process with a duration:
\begin{exe}
  \ex \rf{Time}{Interval}:
  \begin{xlist}
    \ex I reached the summit \p{in} 3~days. [= 3~days later, I reached the summit.]
    \ex I was at the summit \p{within} 3~days. [= 3~days later, I was at the summit.]
    \ex I finished climbing \p{in} 3~days. [= 3~days later, I finished climbing.]
    \ex They had the engine fixed \p{in} 3~days. [= 3~days later, they had the engine fixed.]
  \end{xlist}
\end{exe}
\begin{exe}
  \ex \psst{Duration}:
    \begin{xlist}
      \ex I reached the summit \p{in} 3~days. [it took not more than 3~days]
      \ex I had climbed 1000 feet \p{in} [a total of] 3~days.
      \ex I fixed the engine \p{in} 3~days. [it took not more than 3~days] %\\
      %$\rightarrow$ I was fixing the engine \p{for} 3~days. (\psst{Duration})
    \end{xlist}
\end{exe}

%Though \p{for} generally marks \psst{Duration}s, it can mark an \psst{Interval} under negation:
With a negated event, we use \psst{Duration}:
\begin{exe}
  \ex I haven't eaten \choices{\p{in}\\\p{for}} hours. [hours have passed since the last time I ate]
  (\choices{\#When} haven't you eaten?)
\end{exe}

% Some temporal prepositions may be intransitive \cref{ex:IntrRefTime}, 
% or may take a temporal difference measure as the object \cref{ex:TimeDiff},
% provided that a reference time is salient in the discourse. 
% With such prepositions, the reference time may default to the speech time. 
% However, we limit \psst{DeicticTime} to those prepositions 
% whose \emph{inherent} reference point is the speech time. 
% The following contexts allow a \psst{Time} but not a \psst{DeicticTime}: 
% \begin{exe}
%   %\ex\label{ex:IntrRefTime} To beat the crowds at the game, I will arrive an hour \choices{\p{before}\\\p{beforehand}\\ \#\p{ago}}. (\psst{Time})
%   \ex\label{ex:IntrRefTime} To beat the crowds at the game, I will arrive a while \choices{\p{before}\\\p{beforehand}\\\#\p{ago}\\\#\p{back}}. (\psst{Time})
%   \ex\label{ex:TimeDiff} I took a seat in the waiting room. \choices{\p{After}\\\p{Within}\\??\p{In}} 15 minutes, the doctor saw me for an hour. (\psst{Time})
% \end{exe}
% To summarize: \psst{DeicticTime} covers preposition usages where `now' is inherent,
% and the more general \psst{Time} applies if the reference time is given in the 
% direct object or depends on the discourse.

\begin{history}
  Version~1 featured a label called \sst{DeicticTime}, under \sst{RelativeTime}, 
  which was meant to cover \p{ago} and temporal usages of other adpositions 
  (such as \p{in}) whose reference point is the utterance time or deictic center. 
  This concept proved difficult to apply and was (without good justification) 
  used as a catch-all for intransitive usages of temporal prepositions. 
  For v2, the new concept of \psst{Interval} is broader in that it drops the deictic 
  requirement (also covering \p{within}), while \psst{Time} has been clarified to include 
  intransitive usages of prepositions like \p{before} where the reference time 
  can be recovered from discourse context.
\end{history}


\hierBdef{Locus}

\shortdef{Location, condition, or value. May be abstract.}

\begin{exe}
  \ex I like to sing \choices{\p{at} the gym\\\p{on} Main St.\\\p{in} the shower}.
  \ex The cat is \choices{\p{on\_top\_of}\\\p{off}\\\p{beside}\\\p{near}} the dog.
  \ex There are flowers \choices{\p{between}\\\p{among}} the trees.
  %\ex the wheels \p{on} the bus % Whole ~> Locus
  \ex\label{ex:onRight} When you drive north, the river is \p{on} the right.
  \ex I read it \choices{\p{in} a book\\\p{on} a website}.
  \ex The charge is \p{on} my credit card.
  \ex We met \p{on} a trip to Paris.
  \ex The Dow is \p{at} \choices{a new high\\20,000}.
  %\ex I am now \p{off} work.\nss{\rf{Circumstance}{Locus}?}
  \ex That's \p{in} my price range.
  %\ex She was \p{in} a coma.\nss{\rf{Circumstance}{Locus}?}
\end{exe}
The \psst{Locus} may be a part of another scene argument:
part of a figure whose static orientation is described, 
or a focal part of a ground where contact with the figure occurs:\footnote{\psst{PartPortion} 
was considered but rejected for these cases. Instead we assume the verb 
semantics would stipulate that it licenses a \psst{Theme} as well as a (core) \psst{Locus} 
which must be a part of that \psst{Theme}.}
\begin{exe}
  \ex She was lying \p{on} her back.
  \ex\begin{xlist}
    \ex She kissed me \p{on} the cheek.
    \ex I want to punch you \p{in} the face.
  \end{xlist}
\end{exe}
Words that incorporate a kind of reference point are \psst{Locus} 
even without an overt object:
\begin{exe}
  \ex\begin{xlist}
    \ex The cat is \p{inside} the house.
    \ex The cat is \p{inside}.
  \end{xlist}
  \ex\begin{xlist}
    \ex All passengers are \p{aboard} the ship.
    \ex All passengers are \p{aboard}.
  \end{xlist}
\end{exe}
\psst{Locus} also applies to \p{in}, \p{out}, \p{off}, \p{away}, \p{back}, 
etc.\ when used to describe a location without an overt object:
\begin{exe}
  \ex\begin{xlist}
    \ex The doctor is \choices{\p{in}\\\p{out\_of}\\\p{away\_from}} the office.
    \ex The doctor is \choices{\p{in}\\\p{out}\\\p{away}}.
    \ex They are \p{out} to eat.
  \end{xlist}
\end{exe}
And to \p{around} meaning `nearby' or `in the area':
\begin{exe}
  \ex Will you be \p{around} in the afternoon?
  \ex She's the best doctor \p{around}!
\end{exe}

In a phenomenon called \textbf{fictive motion} \citep{talmy-96}, 
dynamic language may be used to describe static scenes. 
We use construal for these:
\begin{exe}
  \ex A road runs \p{through} my property. (\rf{Locus}{Path})
  \ex John saw Mary \choices{\p{through} the window\\\p{over} the fence}.\footnote{The scene establishes a static spatial arrangement of John, Mary, and the window\slash fence, 
  with only metaphorical motion. Yet this is a non-prototypical \psst{Locus}: it cannot be questioned with \emph{Where?}, for example. 
  Moreover, we understand from the scene that the object of the preposition is something with respect to which the viewer is navigating in order to see without obstruction.} (\rf{Locus}{Path})
  \ex The road extends \p{to} the river. (\rf{Locus}{Goal})
  \ex I saw him \p{from} the roof. (\rf{Locus}{Source})
  \ex\label{ex:protesters} Protesters were \choices{kept\\missing} \p{from} the area. (\rf{Locus}{Source})
  \ex\begin{xlist} 
    \ex We live \p{across\_from} you. (\rf{Locus}{Source})
    \ex We're just \p{across} the street from$_{\text{\rf{Locus}{Source}}}$ you. (\rf{Locus}{Path}) %\nss{what construal for My house is across\_from yours? cf. away\_from}
  \end{xlist}
\end{exe}
Construal is also used for prepositions licensed by scalar adjectives of distance, 
\cref{ex:adjDist}, and prepositions used with a cardinal direction, \cref{ex:cardinal}:
\begin{exe}
  \ex\label{ex:adjDist} \begin{xlist}
    \ex We are quite close \p{to} the river. (\rf{Locus}{Goal})
    \ex We are quite far \p{from} the river. (\rf{Locus}{Source})
  \end{xlist}
  \ex\label{ex:cardinal} \begin{xlist}
    \ex The river is \p{to} the north. (\rf{Locus}{Goal}) [cf.~\cref{ex:onRight}]
    \ex The river is north \p{of} Paris. (\rf{Locus}{Source})
  \end{xlist}
\end{exe}
See also \rf{Locus}{Direction} for static distance measurements, 
described under \psst{Direction}.

Qualitative states are analyzed as \rf{Manner}{Locus}, as described under \psst{Manner}.


\hierCdef{Source}

\shortdef{Initial location, condition, or value. May be abstract.}

For motion events, the initial location is where the thing in motion 
(the figure) starts out.
\psst{Source} also applies to abstract or metaphoric initial locations, 
including initial states in a dynamic event.
%and simple \psst{Source} is used for adpositions that generically mark starting points.

In English, a prototypical \psst{Source} preposition is \p{from}:
\begin{exe}
  \ex\label{ex:catBox} The cat jumped \choices{\p{from}\\\p{out\_of}} the box.
  \ex\label{ex:catLedge} The cat jumped \choices{\p{from}\\\p{off\_of}\\\p{off}} the ledge.
  \ex\label{ex:internet} I got it \choices{\p{from}\\\p{off}} the internet.
  \ex people \p{from} France
  \ex The temperature is rising \p{from} a low of 30 degrees.
  \ex I have arrived \p{from} work.
  \ex We discovered he was French \p{from} his attire. [indication]
  \ex I made it \p{out\_of} clay. [material]
  \ex\label{ex:coma} She \choices{awoke \p{from}\\came \p{out\_of}} a coma.
  \ex We are moving \p{off\_of} that strategy.
\end{exe}
The \psst{Source} use of \p{from} can combine with a specific locative PP:
\begin{exe}
  \ex I took the cat \p{from} behind$_{\psst{Locus}}$ the couch.
\end{exe}
Note that \p{away\_from} is ambiguous between marking a starting point (\psst{Source}) 
and a separate orientational reference point (\psst{Direction}):
\begin{exe}
  \ex At the sound of the gun, the sprinters ran \choices{\p{away\_from}\\\p{from}} the starting line. (\psst{Source})
  \ex The bikers ride parallel to the river for several miles, then 
  head east, \choices{\p{away\_from}\\\#\p{from}} the river. 
  (\psst{Direction}: bikers are never at the river)
\end{exe}
%
%Other prepositions specify more information about the nature of the figure's 
%position and trajectory vis-\`{a}-vis its initial location: 
%e.g., \p{off} and \p{off\_of} for motion away from the top of a surface, 
%and \p{out\_of} for motion through the boundary of a container. 
%We say this involves construal of a \psst{Source} as a \psst{Direction}:
% \begin{exe}
%   \exp{ex:catBox} The cat jumped \p{out\_of} the box. (\rf{Source}{Direction})
%   \exp{ex:catLedge} The cat jumped  the ledge. (\rf{Source}{Direction})
% \end{exe}
% \nss{possible objection: we don't distinguish \p{at} (generic \psst{Locus}) 
% vs. \p{on} and \p{in} (surface and container, resp.). So why distinguish their \psst{Source} counterparts?}
%
% If figurative motion language is entrenched and bleached such that the 
% image of a specific kind of motion is no longer salient, simple \psst{Source} is used. 
% The construal as \psst{Direction} is maintained only if the figurative path
% is somewhat salient:
%
Note, too, that \p{off(\_of)} and \p{out(\_of)} can also mark simple states:
\begin{exe}
  \ex I am \p{off} \choices{medications\\work}. (\rf{Manner}{Locus})
  \ex The lights are \choices{\p{off}\\\p{out}}. (\rf{Manner}{Locus})
  \ex Stay \p{out\_of} trouble. (\rf{Manner}{Locus})
\end{exe}
States are discussed at length under \psst{Manner}. 
There is also a (negated) possession sense of \p{out}/\p{out\_of}:
\begin{exe}
  \ex We are \p{out\_of} toilet paper. (\psst{Possession})
\end{exe}

Sometimes a specific \psst{Source} is implicit, and the preposition is intransitive. 
But if no specific referent is implied, another label may be more appropriate:
\begin{exe}
  \ex The cat was sitting on the ledge, then jumped \p{off}. (\psst{Source}: implicit `(of) it')
  \ex He was offered the deal, but walked \p{away}. (\psst{Source}: implicit `from it')
  \ex The bird flew \choices{\p{away}\\\p{off}}. (\psst{Direction}: vaguely away from the viewpoint)
\end{exe}

\psst{Source} is prototypically inanimate, 
though it can be used to construe animate \psst{Participant}s 
(especially \psst{Originator} and \psst{Causer}).
Contrasts with \psst{Goal}.

\paragraph{Agency as giving.}
When an \psst{Agent}'s action to help somebody is conceptualized as 
giving, and the nominalized action as the thing given, 
then \p{from} can mark the \psst{Agent} (metaphorical giver).
If the \p{from}-PP is adnominal, \rf{Agent}{Source} is used \cref{ex:AgentSource}.
However, if the \p{from}-PP is adverbial, and the verb relates to the metaphoric 
transfer rather than the event described by the action nominal, 
then the argument linking becomes too complicated for this scheme to express; 
simple \psst{Source} is used by default \cref{ex:AgentiveSource}:
\begin{exe}
  \ex\label{ex:AgentSource} The attention \p{from} the staff made us feel welcome. (\rf{Agent}{Source})
  \ex\label{ex:AgentiveSource}\psst{Source}:\begin{xlist} 
    \ex I received great care \p{from} this doctor.
    \ex I got a second chance \p{from} her.
    \ex I need a favor \p{from} you.
  \end{xlist}
\end{exe}

\hierCdef{Goal}

\shortdef{Final location (destination), condition, or value. May be abstract.}

Prototypical prepositions include \p{to}, \p{into}, and \p{onto}:
\begin{exe}
  \ex I ran \p{to} the store.
  \ex The cat jumped \p{onto} the ledge.
  %\ex a blow/bullet \p{to} the head. (\psst{Goal}? Theme ~> Goal for 'blow'?)
  \ex I touched my ear \p{to} the floor.
  \ex She sank \p{to} her knees.
  \ex Add vanilla extract \p{to} the mix.
  \ex Everyone contributed \p{to} the meeting.
  \ex The temperature is rising \p{to} a high of 40 degrees.
  \ex We have access \p{to} the library's extensive collections.
  \ex She slipped \p{into} a coma.
  \ex The drugs put her \p{in} a coma. (\rf{Goal}{Locus})
  \ex \textbf{Result} \citep[p.~1224]{cgel}: \begin{xlist}
    \ex We arrived at the airport only \p{to} discover that our flight had been canceled.
    \ex May you live \p{to} be 100!
  \end{xlist}
\end{exe}
For motion events, a \psst{Goal} must have been reached if the event 
has progressed to completion (was not interrupted).
\psst{Direction} is used instead for \p{toward(s)} and \p{for}, 
which mark an intended destination that is not necessarily reached:
\begin{exe}
  \ex\begin{xlist}
    \ex I headed \p{to} work. (\psst{Goal})
    \ex I headed \choices{\p{towards}\\\p{for}\\\#\p{to}} work but never made it there. (\psst{Direction})
  \end{xlist}
\end{exe}

% \paragraph{\emph{refer to} RESOURCE.}
% When \emph{refer to} means `mention' or `use a term for' \nss{TODO}
% When \emph{refer to} means `consult' or `advise to consult' a source of information such as a book,
% it is considered a multiword verbal expression (because the \p{to} cannot be omitted in adverbial questions\nss{Is it relevant that \emph{there} can be substituted?}), 
% so the \p{to} is not labeled with a supersense:
% \begin{exe}
%   \ex I referred\_to the dictionary. (\#Why did you refer?)
%   \ex I was referred\_to the dictionary. (\#Why were you referred?)
% \end{exe}
% When it means `make a referral', i.e.~`direct somebody to a business or service-provider', 
% and \p{to} marks the suggested business or service-provider, it is labeled \psst{Goal}.
% However, when \p{to} marks the recommendation-seeker, it is labeled \psst{Recipient}{Goal}:
% \begin{exe}
%   \ex\begin{xlist}
%     \ex I needed a hairdresser, so my friend referred me \p{to} \choices{Natasha\\Hair Inc.}. (\psst{Goal})
%     \ex I am a great hairdresser---please refer me \p{to} your friends! (\rf{Recipient}{Goal})
% \end{exe}

\paragraph{\emph{go to}.} A conventional way to express one's status as a student at some school is 
with the expression \pex{go \p{to} (name or kind of school)}.
Construal is used when \pex{go \p{to}} indicates student status, rather than 
(or in addition to) physical attendance:
\begin{exe}
  \ex\label{ex:student} I went \p{to} (school at$_{\psst{Locus}}$) UC Berkeley. (\rf{OrgRole}{Goal})
  \exp{ex:student} I went \p{to} UC Berkeley for the football game. (\psst{Goal})
\end{exe}
Going to a business as a customer, going to an attorney as a client, 
going to a doctor as a patient, etc.\ can also convey long-term status, 
but there is considerable gray area between habitual going and 
being in a professional relationship, so we simply use \psst{Goal}:
\begin{exe}
  \ex I go \p{to} Dr.~Smith for my allergies. (\psst{Goal})
\end{exe}

\paragraph{Locative as destination.}
English regularly allows canonically static locative prepositions to mark 
goals with motion verbs like \pex{put}.
We use the \rf{Goal}{Locus} construal to capture both the static and dynamic aspects of meaning:
\begin{exe}
  \ex\rf{Goal}{Locus}: \begin{xlist}
    \ex I put the lamp \p{next\_to} the chair.
    \ex I'll just hop \p{in} the shower.
    \ex I put my CV \p{on} the internet.
    \ex The cat jumped \p{on} my face.
    \ex The box fell \p{on} its side.
    \ex We arrived \p{at} the airport.
  \end{xlist}
\end{exe}

\paragraph{Application of a substance.}
\begin{xexe}
  \ex the paint that was applied \p{to} the wall (\psst{Goal})
  \ex the paint that was sprayed \p{onto} the wall (\psst{Goal})
  \ex the paint that was sprayed \p{on} the wall (\rf{Goal}{Locus}) % passive to avoid ambiguity with [the paint on the wall]NP
\end{xexe}
The wall is the endpoint of the paint, hence \psst{Goal} is the scene role. 
(Though the wall can be said to be affected by the action, we prioritize 
the motion aspect of the scene in choosing \psst{Goal} rather than \psst{Theme}.)

% Examined COCA first 100 results for "for London" and "for Paris".
% 'leave' is the dominant verb
%leave/flee/depart/embark/take off/set out/set sail/... for, board a plane for; head/make for; bound for; train/bus for

\psst{Goal} is prototypically inanimate, though it can be used to construe animate \psst{Participant}s 
(especially \psst{Recipient}).
Contrasts with \psst{Source}.

\hierBdef{Path}

\shortdef{The ground that must be covered in order for the motion to be complete.}

The ground covered is often a linear extent with or without 
specific starting and ending points:
\begin{exe}
  \ex The bird flew \p{over} the building.
  \ex The sun traveled \p{across} the sky.
  \ex Hot water is running \p{through} the pipes.
\end{exe}

It can also be a waypoint\slash something that must be passed or encircled. 
\begin{exe}
  \ex We flew to Rome \p{via} Paris.
  \ex I go \p{by} that coffee shop every morning.
  \ex The earth has completed another orbit \p{around} the sun.
\end{exe}
If this is a portal in the boundary of a container, 
it is often construed as \psst{Source}, \psst{Goal}, or \psst{Locus}:
\begin{exe}
  \ex The bird flew \p{in} the window. (\rf{Path}{Locus})
  \ex The bird flew \p{out} the window. (\rf{Path}{Source})
  \ex A cool breeze blew \p{into} the window. (\rf{Path}{Goal})
\end{exe}

The prepositions \p{around} and \p{throughout} can mark a region in which motion 
that follows an aimless or complex trajectory is contained. 
%and which it roughly ``covers'' via an . 
Construal is used for these, whether or not the region is explicit:
\begin{exe}\ex \rf{Locus}{Path}:\begin{xlist}
  \ex The kids ran \p{around}.
  \ex The kids ran \choices{\p{around}\\\p{throughout}} the kitchen.
  \ex The kids ran \p{around} in the kitchen.
\end{xlist}\end{exe}

See also: \psst{Instrument}, \psst{Manner}

\begin{history}
  The v1 hierarchy distinguished many different subcategories of path descriptions. 
  The labels \sst{Traversed}, \sst{1DTrajectory}, \sst{2DArea}, \sst{3DMedium}, 
  \sst{Contour}, \sst{Via}, \sst{Transit}, and \sst{Course} have all been merged
  with \psst{Path} for v2.
\end{history}

\hierCdef{Direction}

\shortdef{How motion or an object is aimed\slash oriented.}

A \psst{Direction} expresses the orientation of a stationary figure or of a figure's motion.
Prototypical markers\footnote{Known variously as \emph{adverbs}, \emph{particles}, 
and \emph{intransitive prepositions}.}
are \p{away} and \p{back}; \p{up} and \p{down}; 
\p{off}; and \p{out},
provided that no specific \psst{Source} or \psst{Goal} is salient:
\begin{exe}
  \ex The bird flew \choices{\p{up}\\\p{out}\\\p{away}\\\p{off}}.
  \ex I walked \p{over} to where they were sitting.
  \ex The price shot \p{up}.
\end{exe}

In addition, transitive \p{toward(s)}, \p{for}, and \p{at} can 
indicate where something is aimed or directed (but see discussion at \psst{Goal}):
\begin{exe}
  \ex The camera is aimed \p{at} the subject.
  \ex The toddler kicked \p{at} the wall.
\end{exe}

See discussion of \p{away\_from} at \psst{Source}.

\paragraph{Distance.}
\rf{Locus}{Direction} is used for expressions of static distance between two points:
\begin{exe}
  \ex 
    \begin{xlist}
      \ex The mountains are 3~km \choices{\p{away}\\\p{apart}}. (\rf{Locus}{Direction})
      \ex The mountains are 3~km \p{away\_from} our house. (\rf{Locus}{Direction})
    \end{xlist}
\end{exe}
This also applies to distances measured by \emph{travel time} (the amount of time 
is taken to be metonymic for the physical distance):
\begin{exe}
  \ex The mountains are an hour \choices{\p{away}\\\p{apart}}. (\rf{Locus}{Direction})
\end{exe}
Compare \psst{Extent}, which is the length of a path of motion or the amount of change.

\paragraph{Informal direction modifier in location description.}
\begin{exe}
  \ex They live (way) \choices{\p{out} past$_{\text{\rf{Locus}{Path}}}$ the highway.\\
    \p{over} by$_{\psst{Locus}}$ the school} (\rf{Locus}{Direction})
\end{exe}
Cf.~\cref{ex:backIntrans} at \psst{Interval}.

\hierCdef{Extent}

\shortdef{The size of a path, amount of change, or degree.}

This can be the physical distance traversed or the amount of change on a scale:
\begin{exe}
  \ex We ran \p{for} miles.
  \ex The price shot up \p{by} 10\%.
  \ex an increase \p{of} 10\% (\rf{Extent}{Identity})
\end{exe}
For static distance measurements, see \psst{Direction}.

For scalar \p{as} (see \cref{sec:as-as}), \psst{Extent} serves as the function (and sometimes also the role):
\begin{exe}
  \ex\begin{xlist}
    \ex I helped \p{as} much as I could. (\psst{Extent})
    \ex Your face is \p{as} red as a rose. (\rf{Characteristic}{Extent})
    \ex I stayed \p{as} long as I could. (\rf{Duration}{Extent})
  \end{xlist}
\end{exe}
    
\psst{Extent} also covers degree expressions, such as the following PP idioms:
\begin{exe}\ex\begin{xlist}
  \ex I'm not tired \p{at}\_all.
  \ex The food is mediocre \p{at}\_best.
  \ex You should \p{at}\_least try.
  \ex It is the worst \p{by}\_far.
  \ex We've finished \p{for}\_the\_most\_part.
  \ex It was a success \choices{\p{in}\_every\_respect\\\p{on}\_all\_levels}.
  \ex I hate it when they repeat a song \p{to}\_death.
\end{xlist}\end{exe}
Typically these are licensed by a verb or adjective.


\hierBdef{Means}

\shortdef{Secondary action or event that characterizes \textbf{how} 
the main event happens or is achieved.}

Prototypically a volitional action, though not necessarily \cref{ex:chlorophyll}. 
A volitional \psst{Means} will often modify an intended result, 
though the outcome can be unintended as well \cref{ex:oops}.
\begin{exe}
  \ex Open the door \p{by} turning the knob.
  \ex They retaliated \choices{\p{by} shooting\\\p{with} shootings}.
  \ex\label{ex:oops} The owners destroyed the company \p{by} growing it too fast.
  \ex\label{ex:chlorophyll} Chlorophyll absorbs the light \p{by} transfer of electrons.
\end{exe}

\psst{Means} is similar to \psst{Instrument}, which is used for causally supporting entities 
and is a kind of \psst{Participant}.
See also \psst{Manner}, \psst{Topic}.

Contrast with \psst{Explanation}, which characterizes \textbf{why} 
something happens. I.e., an \psst{Explanation} portrays the secondary event 
as the causal \emph{instigator} of the main event, whereas \psst{Means} 
portrays it merely as a \emph{facilitator}.

\begin{history}
  In v1, \psst{Means} was a subtype of \psst{Instrument}, 
  but with the removal of multiple inheritance for v2, 
  the former was moved directly under \psst{Circumstance} 
  and the latter directly under \psst{Participant}.
\end{history}

\hierBdef{Manner}

\shortdef{The style in which an event unfolds, the form that something takes, 
or the condition that something is in.}

% FN definition of Manner under Expend_resource: 
%Any description of the intentional act which is not covered by more specific FEs, 
%including secondary effects (quietly, loudly), and general descriptions comparing events 
%(the same way). In addition, it may indicate salient characteristics of an Agent 
%that also affect the action ( deliberately, eagerly, carefully). 

\psst{Manner} is used as the scene role for several kinds of descriptors 
which typically license some sort of \pex{How?} question:

\begin{itemize}
  \item The style in which an action is performed or an event unfolds, 
  expressed adverbially (canonical use of the term ``manner''):
  \begin{exe}
    \ex He reacted \choices{\p{with} anger\\\p{in} anger\\angrily}.\footnote{\pex{He reacted \p{out\_of} anger} is \rf{Explanation}{Source}.}
    \ex He reacted \p{with} nervous laughter. [contrast: \psst{Means}]
%His reaction was angry; His angry reaction: also Manner, though not prepositional
    \ex I made the decision \choices{\p{by} myself\\\p{without} anyone else\\\p{on}\_~~my~~\_own}. [see \cref{sec:refl}]
    \ex \rf{Manner}{ComparisonRef}:\begin{xlist}
      \ex You eat \p{like} a pig (eats).
      \ex You smell \p{like} a pig.
    \end{xlist}
    \ex\label{ex:smellOf} \choices{Your father smells\\The soup tastes} \p{of} elderberries. (\rf{Manner}{Stuff}) [also~\cref{ex:smellOfNotCmp}]
  \end{exe}
  
  \item An adverbial \textbf{depictive} characterizing a participant of an event:
  \begin{exe}
    \ex She entered the room \choices{\p{in} a stupor\\drunk}. (\rf{Manner}{Locus})
  \end{exe}

  \item The \textbf{form or shape} that something takes, including language of communication 
  and shape of motion:
  \begin{exe}
    \ex\label{ex:in-shape} \rf{Manner}{Locus}:\begin{xlist}
      \ex The clothes are (sitting) \p{in} a pile. [contrast adnominal use: \cref{ex:in-pile-adnominal} under \psst{Whole}]
      \ex The ribbon is (tied) \p{in} a bow.
      \ex The sand is \p{in} a pyramid shape.
      \ex\label{ex:pathmanner} They dance \p{in} a circle.
      \ex I read the book \p{in} French.
      \ex The book is \p{in} French.
      \ex music \p{in} C major
      \ex She loves teaching, and it shows \p{in} her smile.
    \end{xlist}
  \end{exe}
  
  \item \emph{What} + \p{like} (\emph{what he looks \p{like}}, etc.): see \cref{ex:whatlike} under \psst{ComparisonRef}.
  
  \item The \textbf{state or condition} that something is in. 
  The PP or intransitive preposition is used (especially predicatively) 
  to describe a qualitative state or condition of something (especially an entity) 
  that is not simply a relation of location, time, possession, quantity, causation, etc.\ 
  between governor and object.
  For example:
  
  \begin{itemize}
    \item With the noun \pex{state}, \pex{condition}, etc.:
    
    \begin{exe}
      \ex\rf{Manner}{Locus}:\begin{xlist}
        \ex The chairs are \p{in} excellent shape.
        \ex I'm \p{in} no condition to go outside.
      \end{xlist}
    \end{exe}

    \item Bodily/medical conditions presented as applying to the governor:
    
    \begin{exe}
      \ex John is \choices{\p{on} his back\\\p{on} antibiotics\\\p{on} the ventilator\\\p{in} pain\\\p{in} a coma}. (\rf{Manner}{Locus})
    \end{exe}
    
    \item Miscellaneous qualitative senses of specific prepositions used statively:
    
    \begin{exe}
      \ex John is \choices{\p{for}\\\p{against}} the war. [opinion] (\rf{Manner}{Beneficiary})
      \ex John is \p{into} sports. [hobbies/interests] (\rf{Manner}{Goal})
    \end{exe}
%Contrast with cases where the object is presented as a partial property of the entity-denoting governor: “The girl with pain / with a cold” is Characteristic

    \item Idiomatic PPs expressing states, for example:\footnote{Often the object of the preposition 
    is determinerless (\pex{\p{in} business}) \citep{baldwin-06} 
    or has a fixed determiner (\pex{\p{in} a hurry}).}
    \begin{exe}
    \ex \pex{\p{on} fire} (contrast \pex{\p{in} the fire}), 
      \pex{\p{on} time} (contrast \pex{\p{at} the time}), 
      \pex{\p{in} trouble}, \pex{\p{in} love}, \pex{\p{in} tune}, \pex{\p{in} a hurry}, 
      \pex{\p{at} odds}, \pex{\p{out\_of} business}, \pex{\p{out\_of} control} (\rf{Manner}{Locus})
    \ex They are \choices{\p{on}\_~~the~~\_way\\\p{on}\_~~their$_{\text{\backposs}}$~~\_way}  (\rf{Manner}{Locus})
    \end{exe}
    
    \item Intransitive prepositions expressing a qualitative state (not location, time, etc.):
    \begin{exe}
      \ex\rf{Manner}{Locus}:\begin{xlist}
        \ex The lights are \p{off}.
        \ex The party tonight is \p{on}. [scheduled to happen; not canceled]
        \ex Political TV shows are \p{in}. [in fashion]
      \end{xlist}
    \end{exe}
  \end{itemize}
  
  A few observations about these state PPs are in order.
  
  \begin{enumerate}
  \item In a reversal of the usual asymmetry between governor and adpositional object, 
  semantically, the PP defines the kind of scene that the governor participates in. 
  To an extent, this may be true of all predicative PPs, but the state PPs are often such 
  that the object of the preposition is neither an event nor a referential entity. 
  I.e., \pex{John is \p{in} a hurry} does not exactly express a relation 
  between the entities \pex{John} and \pex{a hurry}; rather, it expresses something 
  qualitative about the entity \pex{John}'s condition.
  
  \item The most idiomatic of the state PPs seem to resist questions of the form 
  \emph{What?}+NP-supercategory with a stranded preposition:
  \begin{exe}
    \ex More productive prepositional usages:\begin{xlist}
      \ex The party is \p{in} January.~$\rightarrow$ What month is the party \p{in}? [Or: When is the party?] (\psst{Time})
      \ex John is \p{on} aspirin.~$\rightarrow$ What medication is John \p{on}?\footnote{Or, colloquially, with a suspected mind-altering substance: \pex{What is John \emph{\p{on}}?!}} (\rf{Manner}{Locus})
    \end{xlist}
    \ex\label{ex:idiomPP} Less productive/more idiomatic preposition + NP combinations:\begin{xlist}
      \ex John is \p{in} \choices{a hurry\\a coma}.~$\nrightarrow$ What \_ is John \p{in}?\footnote{\pex{What condition/state is John \p{in}?} does work, but is quite vague.}  (\rf{Manner}{Locus})
      \ex John is \p{on} fire.~$\nrightarrow$ What \_ is John \p{on}? (\rf{Manner}{Locus})
    \end{xlist}
  \end{exe}

  \item Typically these states are binary: something is either \pex{\p{on} fire}\slash \pex{\p{on} time}, or not. 
  For some, the negation may be expressed by substituting a contrasting preposition: 
  an orchestra that is not \pex{\p{in} tune} is \pex{\p{out\_of} tune}.
  \end{enumerate}

  \paragraph{State PPs with complements.}
  The \rf{Manner}{Locus} construal is also used when there is effectively a preposition+NP+preposition  
  combination that links two arguments:
  \begin{exe}
    \ex\rf{Manner}{Locus}:\begin{xlist}
      \ex\label{ex:InLove} John is \p{in} love (with$_{\text{\rf{Stimulus}{Topic}}}$ Mary). [cf.~\cref{ex:InLoveWith}]
      \ex That is \p{at} odds with$_{\text{\rf{ComparisonRef}{Topic}}}$ our agreement.
    \end{xlist}
  \end{exe}
  
  \paragraph{Idiomatic PP with modifier slot: \pex{on a(n)\dots basis}.}
  There seems to be a construction \pex{\p{on} a(n) MODIFIER basis} where the 
  modifier phrase reflects the scene role being filled. 
  We use \psst{Manner} as the function:
  \begin{xexe}
    \ex\label{ex:bipartisanBasis} The legislation was passed \p{on}\_a\_~~bipartisan~~\_basis. (\psst{Manner})
    \ex I see them \p{on}\_a\_~~daily~~\_basis. (\rf{Frequency}{Manner}) [also \cref{ex:dailyBasis}]
  \end{xexe}

  \paragraph{Change-of-state PPs.}
  Occasionally, a PP will mark a start or end state, in which case we collapse 
  the state/location distinction, using \psst{Source} or \psst{Goal} as the scene role:
  \begin{exe}
    \ex John came \p{out\_of} a coma. (\psst{Source})
    \ex John slipped \p{into} a coma. (\psst{Goal})
    \ex The drugs put John \p{in} a coma. (\rf{Goal}{Locus})
    \ex They chopped the wood \p{in} pieces. (\rf{Goal}{Locus})
  \end{exe}
  



    
  \end{itemize}
  
  %\ex You can use it \p{as} a hammer. (\rf{Manner}{Identity}\nss{?})

% \nss{TODO: separate section about predicative PPs:}
% Other highly specialized preposition usages in free combination with objects
% We are out of eggs (Possession)
% But: We are rolling in dough; Romney was born in(to) money/wealth; Married into money; ?We are in money
% Maybe ‘wealth’ is interpreted as a state and that was extended to ‘money’/’dough’
% Married into the newspaper business
% The plane was on time; The plane arrived on time / in a timely fashion; The plane’s arrival was on time / timely; The plane’s on-time / timely arrival: all Manner ~> Locus. (Though the state expressed by the PP involves time, it does not directly answer a When? question, so it is not labeled Time.)
% Intransitive: The lights are off
% While somewhat borderline, we deem out of town as Locus as it answers a Where? question.\nss{noted under PP idioms}

\paragraph{Versus \psst{Circumstance}.}
State PPs like \pex{\p{at} odds} and \pex{\p{on} medication}, which receive the 
construal \rf{Manner}{Locus}, are similar to situating events like 
\pex{\p{at} the party} and \pex{\p{on} vacation}, which are analyzed as 
\rf{Circumstance}{Locus}. What matters for the scene role is whether the object 
of the preposition is an event or not.

\paragraph{Versus \psst{Characteristic}.} 
Note that, despite the prevailing use of the term `manner' for descriptors of \emph{events}, 
\psst{Manner} and \psst{Characteristic} each cover certain descriptors of \emph{entities}.
The use of \rf{Manner}{Locus} to cover the state that an entity is in was deemed necessary 
to account for the dual function of PPs like \emph{\p{in} French} as
predicate complements (\emph{the book is \p{in} French}) and 
adverbials (\emph{read it \p{in} French}).


When an entity is being described, 
it is difficult to semantically separate \textbf{states} from \textbf{attributes}. 
We currently make the distinction based on whether there is a construal of \emph{partiality}, 
using \psst{Characteristic} or its subtype \psst{Possession}
for partial-attributes like possessions, attire, and body part conditions
(\emph{the man \p{with} a hat}; \emph{He is \p{in} a suit}; \emph{the man \p{with} an ear infection}). 
\psst{Manner} is reserved for conditions where the object of the preposition cannot be 
localized with respect to the entity 
(\emph{the man \p{with} a cold}; \emph{He is \p{in} pain}).
However, the feasibility of this distinction may be worth revisiting in the future.\futureversion{\nss{One possibility
to consider for v3: merge Characteristic and Manner as Quality,
and treat the partiality difference as a matter of construal with different functions.
E.g. Quality construed as State/Predicate vs. Quality construed as Component.}}

\paragraph{Versus \psst{ComparisonRef}.}
See \psst{ComparisonRef}.

\begin{history}
  In v1, \psst{Manner} was positioned as an ancestor of all 
  categories that license a \emph{How?} question, including 
  \psst{Instrument}, \psst{Means}, and \sst{Contour}, as in \cref{ex:pathmanner}. 
  This criterion was deemed too broad, so \psst{Manner} has no 
  subtypes in v2.
\end{history}

\hierBdef{Explanation}

\shortdef{Assertion of \textbf{why} something happens or is the case.}

This marks a secondary event that is asserted as the reason for the main event or state. 

\begin{exe}
\ex  I went outside \p{because\_of} the smell. %(Why did you go outside?)
\ex  The rain is \p{due\_to} a cold front. %(Why is there rain?)
\ex  He reacted \p{out\_of} anger. (\rf{Explanation}{Source})
\ex\begin{xlist}
  \ex He thanked her \p{for} the cookies.
  \ex Thank you \p{for} being so helpful.
\end{xlist}
\end{exe}

When a preposition like \p{after} is used and the relation is temporal as well as causal, 
construal captures the overlap. While \p{since} and \p{as} can also be temporal, 
there are tokens where they cannot be paraphrased respectively with \p{after} and \emph{when}:
\begin{exe}
  \ex I joined a protest \p{after} the shameful vote in Congress. (\rf{Explanation}{Time})
  \ex Her popularity has grown \p{since} she announced a bid for president. (\rf{Explanation}{Time})
  \ex I will appoint him \choices{\p{since}\\\p{as}\\\#\p{after}\\\#when} he is most qualified for the job. (\psst{Explanation})
\end{exe}



Question test: \psst{Explanation} and its subtype \psst{Purpose} license
\pex{Why?} questions.

\hierCdef{Purpose}

\shortdef{Something that somebody wants to bring about,
asserted to be why something was done, is the case, or exists.}

Central usages of \psst{Purpose} explain the motivation behind an action.
Typically the governing event serves as a means for achieving or facilitating the \psst{Purpose}. 
Prototypical markers include \p{for} and infinitive marker \p{to}:
\begin{exe}
  \ex\begin{xlist}
      \ex He rose \p{to} make a grand speech.
      \ex He rose \p{for} a grand speech.
      \ex surgery \p{to} treat a leg injury
    \end{xlist}
\end{exe}
Something directly manipulated/affected can stand in metonymically 
for the desired event:
\begin{exe}\ex\begin{xlist}
  \ex I went to the store \p{for} eggs. [understood: `to acquire/buy eggs']
  \ex surgery \p{for} a leg injury [understood: `to treat a leg injury']
\end{xlist}\end{exe}

\noindent The following subcases serve to clarify the boundaries of \psst{Purpose}:
\begin{itemize}
\item	A desired outcome that is separate from, but typically a motivation for 
(hence subtype of \psst{Explanation}), the main event. 
It is possible to complete the main event without realizing the purpose.
\item \textbf{Inanimate} thing or event which is aided\slash facilitated\slash addressed\slash achieved\slash acquired 
as a consequence of the main event:
\begin{exe}
  \ex We hired a caterer \p{for} the party.
  \ex surgery \p{for} an ingrown toenail
  \ex Call the doctor \p{for} an appointment.
  \ex Go to the store \p{for} eggs.
\end{exe}
For \textbf{animates} who are aided or harmed as a consequence of the main event, see \psst{Beneficiary}.
\item	Something characterized as good/appropriate (or not) for some kind of use\slash activity\slash occasion 
or inanimate thing affected\slash addressed\slash etc., 
delimiting the applicability of a descriptor to that aspect of the thing:
\begin{exe}
  \ex \begin{xlist}
      \ex This place is great \p{for} ping-pong. 
      %[no commitment to evaluating the place in general]
      \ex This is a great place \p{for} ping-pong.
    \end{xlist}
  \ex This cleaner is good \p{for} hardwood floors.
  \ex\label{ex:greatForDinner} This restaurant is great \p{for} dinner.
\end{exe}
The evaluation is being delimited to a particular purpose:
\cref{ex:greatForDinner} is not claiming the restaurant is great \emph{in general}, 
just with respect to dinner.

For \textbf{animates} used in similar constructions, see \psst{Beneficiary}.
\end{itemize}

%●	Intended use of something. “a shoulder TO cry on”, “a shoulder FOR crying on": Characteristic ~> Purpose

Question test: \psst{Explanation} and its subtype \psst{Purpose}, 
when used adverbially, license \pex{Why?} questions. 
\psst{Purpose} usually licenses an \pex{in order to} 
or \pex{for the purpose of} paraphrase.

\paragraph{Goods and services.} See discussion at \psst{Theme}.

\paragraph{Inherent purposes.}
An \emph{entity} (typically an artifact) can be modified to explicate an intended use or affordance. 
We analyze such a relationship with the construal \rf{Characteristic}{Purpose}: 
the function of \psst{Purpose} reflects the \emph{intended use} aspect of the meaning, 
while the scene role of \psst{Characteristic}
highlights that the intended use can be understood as a \emph{static property} of the entity (part of its qualia structure).%
\footnote{In FrameNet 
as of v1.7, these sorts of purposes are labeled as \sst{Inherent\_purpose}. 
See, e.g., the example ``MONEY [to support yourself and your family]'' in the \textbf{Money} frame 
(\url{https://framenet2.icsi.berkeley.edu/fnReports/data/lu/lu13361.xml?mode=annotation}).}
\begin{exe}
  \ex\label{ex:charPurp} \rf{Characteristic}{Purpose}:
  \begin{xlist}
    \ex a shoulder \p{to} cry on
    \ex something \p{to} eat
    \ex cleaner \p{for} hardwood floors
%    \ex a good store \p{for} eggs [understood: `for acquiring/buying eggs']
%    \ex a good book \p{to} give to young readers
%    \ex a good book \p{for} young readers [understood: `for giving to young readers']
    %\ex physical therapy \p{for} a leg injury. [understood: `treating a leg injury']
  \end{xlist}
\end{exe}

Question test: \pex{What is this \_ for?}


\paragraph{Necessity.}
\psst{Purpose} marks the consequence enabled or prevented by some condition, 
typically a resource:
\begin{exe}
  \ex I need another course \p{in\_order\_to} graduate. (\psst{Purpose})
  \ex I need \$20 \choices{\p{for}\\\p{to} attend} the show. (\psst{Purpose})
  \ex The dough takes an hour \p{to} rise. (\psst{Purpose})
\end{exe}

\paragraph{Sufficiency and excess.}
See \psst{ComparisonRef}.


\paragraph{Versus \psst{Circumstance} for ritualized occasions.}
\psst{Purpose} applies to \p{for} 
when it marks a ritualized activity such as a meal or holiday/commemoration for which the main event describes a \textbf{preparation} stage:
%ORIGINAL HERE: provided that the main event is in preparation for the observance: 
\begin{exe}
  \ex \psst{Purpose}:
    \begin{xlist}
      \ex I walked to this restaurant \p{for} dinner. [walking is not a part of dinner]
      \ex I bought food \p{for} dinner.
      \ex We saved money \p{for} our annual vacation.
    \end{xlist}
\end{exe}
However, if the activity marked by \p{for} is interpreted as \textbf{containing} the main event, 
then we use \psst{Circumstance}: 
\begin{exe}
  \ex \psst{Circumstance}: 
    \begin{xlist}
      \ex We ate there \p{for} dinner.
      \ex I received a new bicycle \p{for} Christmas.
      \ex I always drink eggnog \p{for} Christmas. [at and in celebration of Christmastime]
      \ex We were wearing costumes \p{for} Halloween. 
    \end{xlist}
\end{exe}
If in doubt, \psst{Circumstance} is broader: e.g., \emph{We went there \p{for} dinner} 
if \emph{went} is ambiguous between journeying and attending.


\begin{history}
  In v1, the usages illustrated in \cref{ex:charPurp} were assigned a separate label, 
  \sst{Function}, which inherited from both \sst{Attribute} and \psst{Purpose}.
  The ability to use construal removes the need for a separate label.
\end{history}

\hierAdef{Participant}

\shortdef{Thing, usually an entity, that plays a causal role in an event.}

Not used directly---see subtypes.

\hierBdef{Causer}

\shortdef{Instigator of, and a core participant in, an event.}

\psst{Causer} is applied directly to inanimate things or forces conceptualized as entities, 
such as in a passive \p{by}-phrase (\cref{sec:passives}):
%Prototypical prepositions are \p{by} (prominently including passive-\p{by}), \p{of}, and \p{'s}:
\begin{exe}
  \ex the devastation of$_{\psst{Theme}}$ the town wreaked \p{by} the fire
  \ex\rf{Causer}{Gestalt}:\begin{xlist} 
    \ex the devastation \p{of} the fire on$_{\psst{Theme}}$ the town
    \ex the fire\p{'s} devastation of$_{\psst{Theme}}$ the town
  \end{xlist}
\end{exe}
The \psst{Causer} is sometimes construed as a \psst{Source}:
\begin{exe}
  \ex\rf{Causer}{Source}: \begin{xlist}
    \ex the devastation \p{from} the fire
    \ex fatalities \p{from} cancer
    \ex FDR suffered \p{from} polio.
  \end{xlist}
\end{exe}

See also: \psst{Instrument}

\hierCdef{Agent}

\shortdef{Animate instigator of an action (typically volitional).}

%Prototypical prepositions are \p{by} (prominently including passive-\p{by}), \p{of}, and \p{'s}:
This is most directly associated with the passive \p{by}-phrase (\cref{sec:passives}), 
but also permits other construals:
\begin{exe}
  \ex the decisive vote \p{by} the City Council
  \ex\rf{Agent}{Gestalt}:\begin{xlist}
    \ex the decisive vote \p{of} the City Council
    \ex the City Council\p{'s} decisive vote
    \ex they needed Joan\p{'s} help
    \ex It was \choices{the chairman\p{'s} fault\\the fault \p{of} the chairman}.
  \end{xlist}
  %\ex ?they needed help of hers
  %(\rf{Agent}{Possessor})
\end{exe}
When two symmetric \psst{Agent}s are collected in a single NP 
functioning as a set, it is marked as a \psst{Whole} construal:
\begin{exe}
  \ex There was a war \p{between} France and Spain. (\rf{Agent}{Whole})
  \ex a discussion \p{among} the board members (\rf{Agent}{Whole})
%  \ex Please talk \p{amongst} yourselves. \rf{Agent}{Whole}
% reflexives are weird. maybe Co-Agent

\end{exe}

Compare: \psst{Co-Agent}; 
see also: \psst{OrgRole}, \psst{Originator}, \psst{Source}, \psst{Stimulus}


\hierDdef{Co-Agent}

\shortdef{Second semantically core participant that would otherwise be labeled \psst{Agent}, 
but which is adpositionally marked in contrast with an \psst{Agent} 
occupying a non-oblique syntactic position (subject or object).
Typically, the \psst{Agent} and \psst{Co-Agent} engage in the event 
in a reciprocal fashion.}

\begin{exe}
  \ex I fought in a war \p{against} the Germans.
  \ex I \choices{talked\\argued} \p{with} my roommate about cleaning duties.
\end{exe}

See also: \psst{Accompanier}, \psst{SocialRel}

\hierBdef{Theme}

\shortdef{Undergoer that is a semantically core participant in an event or state, 
and that does not meet the criteria for any other label.}

Prototypical \psst{Theme}s undergo (nonagentive\footnote{We distinguish agentivity at the token level, 
unlike VerbNet, where the subject of motion verbs like \emph{arrive} is \psst{Theme} because it need not be agentive.}) motion, are transferred, 
or undergo an internal change of state (sometimes called \emph{patients}).
Adpositional \psst{Theme}s are usually, but not always, construed as something else:
\begin{exe}
  \ex\begin{xlist}
    \ex Quit \p{with} the whining!
    \ex She helped me \p{with} my taxes.
    \ex Don't \choices{bother\\waste time} \p{with} an extra trip.
    \ex I managed to cope \p{with} \choices{the heavy load\\my fear of heights}.
  \end{xlist}
  \ex There's nothing wrong \p{with} the engine.
  \ex Fill the bowl \p{with} water. (\rf{Theme}{Instrument})
  \ex\begin{xlist}
      \ex The food was covered \p{with} grease. (\rf{Theme}{Instrument})
      \ex The food was covered \p{in} grease. (\rf{Theme}{Locus})
    \end{xlist}
  \ex My hovercraft is full \p{of} eels.
  %\ex I wouldn't want to deprive you \p{of} this opportunity. - Theme or Possession?
  \ex \begin{xlist}
      \ex\label{ex:search-for} Sheldukher \choices{looked\\searched\\fumbled} \p{for} his laser pistol.\\{}
      [contrast with transitive verb plus \psst{Characteristic} in \cref{ex:search-obj-for}]
      \ex Sheldukher \choices{asked\\made a request} \p{for} his laser pistol.
      \ex There is a significant demand \p{for} new housing.
      \ex Let's wait \p{for} \choices{Steve\\more information\\the end of the party}.
    \end{xlist}
  \ex The mechanic made a repair \p{to} the engine. (\rf{Theme}{Goal})
  \ex
  {\setlength\multicolsep{0pt}%
  \begin{multicols}{2}
    \begin{xlist}
        %\ex the price \p{of} tea % moved to Gestalt
        \sn \psst{Theme}
        \ex the approach \p{of} the waves
        \ex the \choices{death\\murder} \p{of} a salesman
        
        %\sn the tea\p{'s} price % moved to Gestalt
        \sn \rf{Theme}{Gestalt}
        \sn the wave\p{s'} approach
        \sn the salesman\p{'s} \choices{death\\murder}
    \end{xlist}
  \end{multicols}}
  \ex \begin{xlist}
      \ex The mechanic worked \p{on} the engine.
      \ex We noshed \p{on} snacks.
      \ex Students spend a lot of money \p{on} textbooks.
    \end{xlist}
  \ex \begin{xlist}
      \ex There was an increase \p{in} oil prices.
      \ex I'm covered \p{in} bees! (\rf{Theme}{Locus})
    \end{xlist}
  \ex \begin{xlist}
      \ex The training saved us \p{from} almost certain death. (\rf{Theme}{Source})
      \ex They prevented us \p{from} boarding the plane. (\rf{Theme}{Source})
    \end{xlist}
\end{exe}

\paragraph{Goods and services.}
In a commercial scene, the preposition introducing the item or event 
incurring a cost receives \psst{Theme} as its scene role. 
If the object of the preposition denotes an event and the preposition 
is \p{to}, \p{for}, or similar, then the construal \rf{Theme}{Purpose} is used:
\begin{exe}
  \ex \begin{xlist}
    \ex They spent \$500 \p{on} the repairs. (\psst{Theme})
    \ex They charged/paid/owed \$500 \p{for} the bicycle. (\psst{Theme})
    \ex They charged/paid/owed \$500 \p{for} the repairs. (\rf{Theme}{Purpose})
    \ex They asked \$500 \p{to} make the repairs. (\rf{Theme}{Purpose})
    \ex \$500 \choices{\p{for}\\\p{to} make} the repairs was excessive. (\rf{Theme}{Purpose})
  \end{xlist}
\end{exe}

\paragraph{\emph{Between} and \emph{among}.}
When two symmetric undergoers are collected in a single NP 
functioning as a set, it is marked as a \psst{Whole} construal:
\begin{exe}
  \ex There was a collision in mid-air \p{between} two light aircraft. (\rf{Theme}{Whole})
  \ex Links \p{between} science and industry are important. (\rf{Locus}{Whole})
\end{exe}

\begin{history}
  In v1, following many thematic role inventories, 
  \sst{Patient} was a distinct label for undergoers that were 
  affected (undergoing an internal change of state). 
  It was merged into \psst{Theme} for v2 because the affectedness criterion can be subtle 
  and difficult to apply.
\end{history}

Compare: \psst{Co-Theme}

See also: \psst{Beneficiary}

\hierCdef{Co-Theme}

\shortdef{Second semantically core undergoer that would otherwise be labeled \psst{Theme}, 
but which is adpositionally marked in contrast with a \psst{Theme} 
occupying a non-oblique syntactic position (subject or object).}

Often, the \psst{Theme} and the \psst{Co-Theme} are similarly situated entities---rather than 
one being more figure-like and the other more ground-like---but the \psst{Co-Theme} 
is an oblique (adpositionally marked) argument.
This includes concrete scenes of combination, attachment, separation, and substitution 
of two similar entities. 

\begin{exe}
  \ex\begin{xlist}
    \ex His bicycle collided \p{with} hers.
    \ex Combine butter \p{with} vanilla.
    \ex\label{ex:repl-with} They replaced my old tires \p{with} new ones. [replacement; contrast \cref{ex:subst-for}]
  \end{xlist}
  \ex\begin{xlist}
    %\ex You can't create something \p{from} nothing. (\rf{Co-Theme}{Source})
    \ex The boys were separated \p{from} the girls. (\rf{Co-Theme}{Source})
    \ex Keep the dogs \p{from} the cats. (\rf{Co-Theme}{Source})
    \ex The shin bone is connected \p{to} the knee bone. (\rf{Co-Theme}{Goal})
  \end{xlist}
\end{exe}

By contrast, for similar scenes where the oblique argument is a ground-like entity 
(larger, less dynamic, more locational, etc.\ than the \psst{Theme}), 
that entity is typically a \psst{Locus}, \psst{Source}, or \psst{Goal}:
\begin{exe}
  \ex Dynamic:\begin{xlist}
    \ex Add vanilla \p{to} the mixture. (\psst{Goal})
    \ex Stir vanilla \p{into} the mixture. (\psst{Goal})
    \ex Detach the cable \p{from} the wall. (\psst{Source})
  \end{xlist}
  \ex Static:\begin{xlist}
    \ex The cable \choices{is attached\\connects} \p{to} the wall. (\rf{Locus}{Goal})
    \ex Protesters were \choices{kept\\missing} \p{from} the area. (\rf{Locus}{Source}) [repeated: \cref{ex:protesters}]
  \end{xlist}
\end{exe}

For creation or transformation of a whole entity (or a group of entities, such as ingredients) 
into another entity, \psst{Source} applies to the initial entity and \psst{Goal} to the result.

With abstract scenes, \psst{Co-Theme} is sometimes needed because another 
argument would be \psst{Theme}---e.g.~2-argument adjectives:
\begin{exe}
  \ex\begin{xlist}
    \ex You shouldn't confuse/associate Mozart \p{with} Rossini. (\psst{Co-Theme})
    \ex We are ready/eligible/due \p{for} an upgrade. (\rf{Co-Theme}{Purpose})
    \ex They prevented us \p{from} entering. (\rf{Co-Theme}{Source})
  \end{xlist}
\end{exe}

\begin{history}
  In v1, \sst{Co-Patient} was a distinct label, and the two shared a common supertype, 
  \sst{Co-Participant}. 
  See note at \psst{Theme}.
\end{history}

See also: \psst{InsteadOf}, \psst{Co-Agent}

\hierCdef{Topic}

\shortdef{Information content or subject matter in communication or cognition, 
or the matter something pertains to.}

A variety of prepositions---including the vast majority of occurrences of \p{about}---can 
mark a \psst{Topic}. The following subclasses warrant \psst{Topic} as the scene role:

\begin{itemize}
  \item \textbf{Communication} scenes: the content or subject matter of 
  speech, writing, art, performance, etc.
  \begin{xexe}
    \ex I \choices{gave a presentation\\spoke} \p{about}/\p{on} politics.
    \ex They wouldn't stop arguing \p{over} the plan.
    \ex I was accused \p{of} treason.
    \ex a picture \p{of} Whistler's mother
    \ex three \choices{copies\\versions} \p{of} the test
    \ex\rf{Topic}{Identity}---see discussion at \psst{Identity}:
      \begin{xlist}
        \ex the topic/issue/question \p{of} semantics
        \ex the idea \p{of} raising money
      \end{xlist}
    \ex The \choices{ratings\\reviews} \p{for} this film are atrocious.
    \ex I did not hazard a guess \p{as\_to} the cause.
  \end{xexe}
  \item \textbf{Cognition} scenes: the content or subject matter of thought and knowledge---belief, opinion,
  decision, learning, study, interest, expertise, skill, etc.
  \begin{exe}
    \ex\begin{xlist}
      \ex Try not to think \p{about} it.
      \ex We took a minute to \choices{think\\ponder} \p{over} the situation.
      \ex I plan \p{on} going again.
      \ex I am focused \p{on} the task at hand.
      \ex There is not enough research \p{on} the effects of global warming.
      \ex She was dumbfounded \p{as\_to} why the police had done that.
      \ex Think \p{of} all the possibilities!
      \ex I have no memory \p{of} the incident.
      \ex I am aware \p{of} the problem.
      \ex You can have your choice \p{of} chicken or fish.
      \ex I disagree \p{with} that statement.
      \ex I am familiar \p{with} this topic.
      \ex Are you interested \p{in} politics?
      \ex I'm confident \p{in} your abilities.
    \end{xlist}
  \ex\label{ex:Activity}\begin{xlist}
    \ex My daughter excels \choices{\p{in}\\\p{at}} sports.
    \ex\label{ex:cookieExpert} I'm \choices{an expert\\talented\\good} \p{at} baking cookies.
    \ex\label{ex:in-activity-Topic} 
      I wouldn't hestitate \p{in} seeing a doctor.\\{} 
      [but see \cref{ex:in-activity-Circumstance} under \psst{Circumstance}, which is syntactically parallel]
    \end{xlist}
  \end{exe}
  \item Relations of \textbf{regard}: the entity, issue, or aspect that the governing 
  predicate pertains to. The relation to the governor may be somewhat loose, 
  skirting the boundary between semantics and information structure.
  \begin{xexe}
    \ex Be reasonable \p{with} your expectations!
    \ex They are transparent \p{with} their fee.
    \ex The discount should apply \p{with} other restaurants too.
    \ex I approached the manager \p{about} the poor service. [implied communication]
    \ex I am a big baby \p{about} needles. [implied cognition]
    %\ex Don't hesitate \p{in} sending us a donation. -- 'hesitate' under cognition
    \ex The owner wouldn't budge \p{on} the price.
    \ex They came through \p{on} all of their promises.
    % they are going to mark you up |on| that feel good premise .
    % Hit or miss |on| the service .
    % They have a great lunch special with your choice |of| meat , chicken , steak , or pork 
    % The best darlington has to offer |in| contemporary sandwicheering .
    \ex She did not do the right thing \p{for} an item that was marked incorrectly.
    \ex I'm fast \p{at} baking cookies. [cf.~\cref{ex:cookieExpert}]
    \ex They have almost anything you could want \choices{\p{when\_it\_comes\_to}\\\p{in\_terms\_of}} spy and surveillance equipment .
  \end{xexe}
\end{itemize}

A few specific governors merit further discussion:

\paragraph{\pex{agree}.}
\begin{xexe}
  \ex Let us agree \p{on} the deal. (\psst{Topic})
  \ex Let us agree \p{to} the deal. (\rf{Topic}{Goal})
\end{xexe}

\paragraph{\pex{answer}, \pex{respond}, etc.}
\begin{exe}
  \ex\rf{Topic}{Goal}:\begin{xlist}
    \ex the answer \p{to} the question
    \ex my response \p{to} your question
  \end{xlist}
\end{exe}

For \w{respond \p{with}} and similar, it depends whether the object is an action, 
a device facilitating communication, or some aspect of transferred information:
\begin{xexe}
  \ex He responded to my kick \p{with} a punch. (\psst{Means})
  \ex He responded to my accusation \p{with} a lawsuit. (\psst{Means})
  \ex He responded to my accusation \p{with} dishonest emails. (\psst{Instrument})
  \ex He responded to my accusation \p{with} falsehoods. (\psst{Topic})
\end{xexe}

\paragraph{\pex{problem \p{with}}, \pex{experience \p{with}}, etc.} 
These are simply \psst{Topic}:
\begin{xexe}
  \ex \choices{There was\\We had} a problem \p{with} mice in the basement.
  \ex I have limited experience \p{with} numerical methods.
  \ex \choices{I had a bad experience\\my bad experience} \p{with} a vampire.
\end{xexe}

See also: \psst{Stimulus}

\begin{history}
  Previously, \lbl{Activity} covered usages such as in \cref{ex:Activity}, 
  but such usages were found to be infrequent and 
  \lbl{Activity} was deemed too narrow.
\end{history}

\hierBdef{Stimulus}%
%
\shortdef{That which is perceived or experienced (bodily, perceptually, or emotionally).}

\psst{Stimulus} does not seem to have any prototypical adposition 
in the languages we have looked at. In English, it can be construed in several ways:
\begin{exe}
  \ex My affection \p{for} you (\rf{Stimulus}{Beneficiary})
  \ex Scared \p{by} the bear (\rf{Stimulus}{Causer})
  %\ex We were listening \p{to} the music. (\rf{Stimulus}{Goal})
  \ex You should \choices{listen\\pay attention} \p{to} the music. (\rf{Stimulus}{Goal})
  \ex \rf{Stimulus}{Direction}:
    \begin{xlist}
      \ex We were looking \p{at} the photo.
      \ex\label{ex:AngryAt} I was angry \p{at} him. [cf.~\cref{ex:AngryWith}]
      \ex I startled \p{at} the noise.
    \end{xlist}
  \ex \rf{Stimulus}{Topic} is assigned to cases where the PP describes the topic or content of one's emotion: 
    \begin{xlist}
      \ex I care \p{about} you.
      \ex That's what I love \p{about} the show.
      \ex I took\_pride \p{in} the results.
      \ex I was \choices{proud \p{of}\\happy \p{with}} the results.
      \ex\label{ex:AngryWith} I was angry \p{with} him. [cf.~\cref{ex:AngryAt}]
      \ex\label{ex:InLoveWith} I was in$_{\text{\rf{Manner}{Locus}}}$ love \p{with} him. [cf.~\cref{ex:InLove}]
      \ex They bored me \p{with} their incessant talk about cats.
    \end{xlist}
  \ex\label{ex:StimBen} \rf{Stimulus}{Beneficiary}:
    \begin{xlist} % Also seems to be a metaphorical connection to Direction, but that would require multiple construal to express.
      \ex Her disdain \p{for} customers was apparent.
      \ex He has/feels compassion \choices{\p{towards}\\\p{for}} animals. % for TOWARDS multiple construal would be nice: Stimulus ~> Beneficiary ~> Direction
    \end{xlist}
  \ex I am \choices{thankful\\grateful} \p{for} your help. (\rf{Stimulus}{Explanation})
\end{exe}
See also: \psst{Topic}, \psst{Beneficiary}

Counterpart: \psst{Experiencer}

\hierBdef{Experiencer}

\shortdef{Animate who is aware of a bodily experience, perception, emotion, or mental state.}

\psst{Experiencer} does not seem to have any prototypical adposition 
in the languages we have looked at. In English, it can be construed in several ways:
\begin{exe}
  \ex\begin{xlist}
    \ex The anger \p{of} the students (\rf{Experiencer}{Gestalt})
    \ex The student\p{s'} anger (\rf{Experiencer}{Gestalt})
  \end{xlist}
  \ex\begin{xlist} 
    \ex Running is enjoyable \p{for} me (\rf{Experiencer}{Beneficiary})
    \ex The pizza was (too) salty \p{for} me (\rf{Experiencer}{Beneficiary})
  \end{xlist}
  \ex\begin{xlist}
    \ex It feels hot \p{to} me (\rf{Experiencer}{Goal})
    \ex That was astounding \p{to} me (\rf{Experiencer}{Goal})
  \end{xlist}
\end{exe}

Less canonically, \psst{Experiencer} applies to semi-pragmatic usages meaning `from the perspective of':\footnote{Interestingly, 
many uses of \p{for} carry an information structural association of delimiting the scope of an assertion. 
\pex{\p*{For}{for} John, the party was not fun at all} makes no commitment regarding how fun the party was to others. 
\pex{This food is good \p{for}$_{\psst{Purpose}}$ dinner\slash \p{for}$_{\psst{Beneficiary}}$ folks with dietary constraints} 
and \pex{He is short \p{for}$_{\psst{ComparisonRef}}$ a basketball player} also have this property. 
As the present scheme targets semantic relations, it is not equipped to formalize pragmatic aspects of the meaning.}
\begin{exe}
  \ex \begin{xlist}
    \ex \p*{For}{for} John, the party was not fun at all. (\rf{Experiencer}{Beneficiary})
    \ex \p*{For}{for} John, there was no reason to attend. (\rf{Experiencer}{Beneficiary})
  \end{xlist}
\end{exe}

Elsewhere, the term \emph{cognizer} is sometimes used for one whose 
mental state is described.

Counterpart: \psst{Stimulus}

\hierBdef{Originator}

\shortdef{Animate who is the initial possessor or creator/producer of something,
including the speaker/communicator of information. 
Excludes events where transfer/communication is not framed as unidirectional.}

A ``source'' in the broadest sense of a starting point/condition. 
Contrasts with \psst{Recipient} if there is transfer/communication.

English construals:\footnote{If we consider subject position as an \psst{Agent} construal 
and direct object position as a \psst{Theme} construal, then we can add examples like 
\pex{\uline{She} talked to her editor} (\rf{Originator}{Agent}) and 
\pex{They robbed \uline{her} of her life savings} (\rf{Originator}{Theme}).
\psst{Originator} does not apply to the subject of events like \pex{exchange} or \pex{talk/chat (with)}, 
which involve a back-and-forth between 
\psst{Agent} and \psst{Co-Agent} (or a plural \psst{Agent}).}
\begin{exe}
  \ex \rf{Originator}{Agent} (passive-\p{by} or adnominal \p{by}):
  \begin{xlist}
    \ex\label{ex:worksBy} works \p{by} Shakespeare [cf.~\cref{ex:worksOf,ex:worksGen}]
    \ex The telephone was invented \p{by} Alexander Graham Bell.
    \ex The story was \choices{given\\told} to$_{\text{\rf{Recipient}{Goal}}}$ her \p{by} her editor.
  \end{xlist}
  \ex \rf{Originator}{Source}:
  \begin{xlist}
    \ex\label{ex:worksOf} works \p{of} Shakespeare [cf.~\cref{ex:worksBy,ex:worksGen}]
    \ex The story was obtained \p{from} an anonymous White House employee.
    \ex I bought it \p{from} this company.
    \ex I heard the news \p{from} Larry.
  \end{xlist}
  \ex \rf{Originator}{Gestalt}: \begin{xlist}
    \ex \label{ex:worksGen} Shakespeare\p{'s} works [cf.~\cref{ex:worksOf,ex:worksBy}]
    \ex Rodin\p{'s} sculptures
    \ex the store\p{'s} fresh produce
    \ex the restaurant\p{'s} food
    \ex John\p{'s} \choices{question\\speech}
  \end{xlist}
\end{exe}

\paragraph{\emph{learn from.}} If the source of learning is an individual 
(or group of individuals, organization, etc.)\ who provides information, 
\rf{Originator}{Source} applies. Otherwise, it is simply \psst{Source}:
\begin{exe}
  \ex We learned a lot \p{from} Miss Zarves. (\rf{Originator}{Source})
  \ex We learned a lot \p{from} that \choices{book\\experience}. (\psst{Source})
\end{exe}


% \nss{Do we need to limit further the kinds of events that have an Originator 
% and Recipient? Only communicative events where the message is core? 
% (``Say'', ``tell'', ``inform'', but not ``talk'' or ``negotiate''. 
% Otherwise we have ``talk with him'' as \rf{Originator}{Co-Agent}; 
% ``negotations by the parties'' as \rf{Orignator}{Agent}; 
% and ``negotations between the parties'' as Originator \tat> Agent \tat> Whole!)}

\begin{history}
  \psst{Originator} merges v1 labels \sst{Donor/Speaker} and \sst{Creator}, 
  which were difficult to distinguish in the case of authorship.
  %(e.g., \pex{the operas \p{of} Puccini}).
  \sst{Donor/Speaker} was a subtype of \sst{InitialLocation}, which 
  inherited from \sst{Location} and \psst{Source}. 
  \sst{Creator} was a subtype of \psst{Agent}.
  Moving \psst{Originator} directly under \psst{Participant} 
  puts it in a neutral position with respect to its possible construals.
\end{history}

\hierBdef{Recipient}

\shortdef{The party (usually animate) that is the endpoint of (actual or intended) transfer of a thing or message, 
becoming the final \psst{Possessor} or \psst{Gestalt}.
Excludes events where transfer/communication is not framed as unidirectional.}
A ``goal'' in the broadest sense of an ending point/condition. 
Contrasts with \psst{Originator}.

English construals:\footnote{If subject position is viewed as an \psst{Agent} construal, 
then active subject with a transfer verbs like \pex{get} or \pex{receive} is \rf{Recipient}{Agent}.
If direct object position is viewed as a \psst{Theme} construal, 
then \pex{She informed \uline{her editor}} are \rf{Recipient}{Theme}.}
\begin{exe}
  \ex She \choices{gave the story\\spoke} \p{to} her editor. (\rf{Recipient}{Goal})
  \ex What title did you give \p{to} your essay? [inanimate] (\rf{Recipient}{Goal})
  \ex news \p{for} our readers (\rf{Recipient}{Direction})
  \ex He is yelling \p{at} me to get ready! (\rf{Recipient}{Direction}\footnote{While \emph{yell \p{at}} 
often has a connotation of shouting criticism towards somebody, 
and criticism would suggest \psst{Beneficiary},
the \psst{Recipient} aspect of the meaning is more explicit and essential:
yelling from a distance at someone does not imply criticism, 
and criticism about someone who is absent is not yelling at them.})
  \ex The news was not well received \p{by} the White House. (\rf{Recipient}{Agent})
  \ex Timmy\p{'s} piano lesson (\rf{Recipient}{Gestalt})
  \ex I'll have to check \p{with} my supervisor. (\rf{Recipient}{Co-Agent})
\end{exe}

\psst{Recipient} does not apply to events like \pex{exchange/talk/chat (\p{with})}, 
which involve a back-and-forth between 
\psst{Agent} and \psst{Co-Agent} (or a plural \psst{Agent} subject):
\begin{exe}
  \ex She \choices{swapped stories\\chatted} \p{with} her friends. (\psst{Co-Agent})
\end{exe}

See also: \psst{Beneficiary}

\begin{history}
  In v1, \psst{Recipient} was the counterpart to \sst{Donor/Speaker}:
  \psst{Recipient} was a subtype of \sst{Destination}, which 
  inherited from \sst{Location} and \psst{Goal}. 
  Moving \psst{Recipient} directly under \psst{Participant} 
  puts it in a neutral position with respect to its possible construals.
\end{history}

\hierBdef{Cost}

\shortdef{An amount (typically of money) that is linked to an item or service 
that it pays for\slash could pay for, or given as the amount earned or owed.} 

The governor may be an explicit commercial scenario:
\begin{exe}
  %\ex I paid/owed John \$10 for the book. %#nonprep
  \ex I \choices{bought\\sold} the book \p{for} \$10.
  \ex I got a refund \p{of} \$10.
  \ex\rf{Cost}{Locus}: \begin{xlist}
    \ex The book is \choices{priced\\valued} \p{at} \$10.
    \ex I bought it \p{at} a great price/rate.
  \end{xlist}
\end{exe}
Or the \psst{Cost} may be specified as an adjunct with a non-commerical governor:
\begin{exe}
  \ex You can ride the bus \p{for} \choices{free\\\$1}.
\end{exe}
\psst{Cost} is \emph{not} used with general scenes of possession or transfer, 
even if the thing possessed or transferred happens to be an amount of money:
\begin{exe}
  \ex I bestowed the winner \p{with} \$100. (\psst{Co-Theme})
\end{exe}

\begin{history}
  This category was not present in v1, which had the broader category \sst{Value}. 
  VerbNet \citep{verbnet,palmer-17} has a similar category called \sst{Asset}; we chose the name 
  \psst{Cost} to emphasize that it describes a relation rather than an entity type 
  (it does not apply to money with a verb like \pex{possess} or \pex{transfer}, 
  for instance).
\end{history}

\hierBdef{Beneficiary}

\shortdef{Animate or personified undergoer that is (potentially) 
advantaged or disadvantaged by the event or state.}

This label does not distinguish the polarity of the relation 
(helping or hurting, which is sometimes termed \emph{maleficiary}).

\begin{exe}
  \ex Vote \choices{\p{for}\\\p{against}} Pedro!
  \ex Junk food is bad \p{for} your health.
  \ex My parrot died \p{on} me.
  \ex\begin{xlist} 
    \ex These are clothes \p{for} children.
    \ex These are children\p{'s} clothes. (\rf{Beneficiary}{Possessor})
  \end{xlist}
  \ex Fortunately \p{for} the turkey('s future), he received a presidential pardon.
\end{exe}

\noindent Specific subclasses include:
\begin{itemize}
\item	Animate who will potentially experience a benefit or harm as a result of something 
but is not an experiencer or recipient of the main predicate itself. 
(May be an experiencer or recipient of the result.)
\item	Animate target of emotion or behavior, discussed below. %(rude TO women, compassion FOR animals)
\item	Animate who someone supports or opposes (e.g., \emph{vote \p{for}}, 
\emph{cheer \p{for}}, \emph{Hooray \p{for}}). % but "Well done TO John" is Recipient ~> Goal)
\item Intended user/usee: 
  \begin{exe}
    \ex (We sell) clothes \p{for} children
    \ex a gallows \p{for} criminals
    \ex This is the car \p{for} you! [advertising idiom]
  \end{exe}
\item	Something characterized as good/appropriate (or not) for some kind of 
\textbf{animate user or usee}, delimiting the applicability of a descriptor 
to that kind of individual: 
  \begin{exe}
    \ex \begin{xlist}
      \ex This place is great \p{for} young children.
      \ex This is a great place \p{for} young children.
    \end{xlist}
  \end{exe}
\end{itemize}
The first and last items above have analogues with \psst{Purpose}. 
The key difference is that \psst{Beneficiary} applies to an animate participant, 
whereas \psst{Purpose} applies to an intended consequence or one of its inanimate participants.

\paragraph{Targets of behavior versus emotion.}
A preposition can mark an individual in the context of evaluating how someone else is treating them, 
with a noun or adjective governor. 
If behavior is more salient than emotion, then \psst{Beneficiary} is the scene role. 
If emotion is highly salient, then \psst{Stimulus} is the scene role.
\begin{exe}
  \ex Behavior-focused:
    \begin{xlist}
      \ex She exhibits rudeness \p{towards} customers. (\rf{Beneficiary}{Direction})
      \ex He is \choices{rude\\condescending} \p{to} women. (\rf{Beneficiary}{Goal})
      \ex He is gentle and compassionate \p{with} animals. (\rf{Beneficiary}{Theme})
    \end{xlist}
  \ex Emotion-focused, repeated from \cref{ex:StimBen}:
    \begin{xlist} % Also seems to be a metaphorical connection to Direction, but that would require multiple construal to express.
      \ex Her disdain \p{for} customers was apparent. (\rf{Stimulus}{Beneficiary})
      \ex He has/feels compassion \choices{\p{towards}\\\p{for}} animals. (\rf{Stimulus}{Beneficiary})
    \end{xlist}
\end{exe}
Note that the emotion-focused examples can describe private emotional states directly, 
while the behavior-focused examples are behavior-based judgments or inferences about emotional states.

An obligation directed at somebody is analyzed like targeted behavior:
\begin{exe}
  \ex We have a solemn responsibility \p{to} our armed forces. (\rf{Beneficiary}{Goal})
\end{exe}

Similar to the behavior-focused examples, inanimate causes can have the potential to positively or negatively
affect somebody. Ability and permission modalities are included here:
\begin{exe}
    \ex\begin{xlist}
      \ex The strategy is \choices{beneficial \\ risky \\ an option} \p{for} investors. (\psst{Beneficiary})
      \ex The strategy \choices{is helpful \\ poses a risk \\ is available} \p{to} investors. (\rf{Beneficiary}{Goal})
    \end{xlist}
\end{exe}

\paragraph{Versus \psst{Recipient}.}
\psst{Beneficiary} applies to the classic English benefactive construction 
where it is ambiguous between assistance and intended-transfer:
\begin{exe}
  \ex John baked a cake \p{for} Mary. [to help Mary out, and/or with the intention of giving her the cake]
\end{exe}
However, if transfer (or communication) is the main semantics of the scene 
and benefit or harm is no more than an inference, then the scene role is \psst{Recipient}:
\begin{exe}
  \ex a \choices{message\\gift} \p{for} my mother (\rf{Recipient}{Direction})
  \ex a \choices{package} \p{for} the front office (\rf{Recipient}{Direction})
\end{exe}

See also: \psst{Experiencer}, \psst{OrgRole}

\hierBdef{Instrument}

\shortdef{An entity that facilitates an action by applying intermediate causal force.}

Prototypically, an \psst{Agent} intentionally applies the \psst{Instrument} 
with the purpose of achieving a result:
\begin{exe}\ex\begin{xlist}
  \ex I broke the window \p{with} a hammer.
  \ex I destroyed the argument \p{with} my words.
\end{xlist}\end{exe}
Less prototypically, the action could be unintentional:
\begin{exe}
  \ex I accidentally poked myself in the eye \p{with} a stick.
\end{exe}
The key is that the \psst{Instrument} is not sufficiently ``independently causal'' 
to instigate the event.

However, to downplay the agency of the individual operating the instrument, 
the instrument can be placed in a passive \p{by}-phrase, 
which construes it as the instigator:
\begin{exe}\ex\label{ex:passiveInstrument}\begin{xlist}
  \ex The window was broken \p{by} the hammer. (\rf{Instrument}{Causer})
  \ex My headache was alleviated \p{by} aspirin. (\rf{Instrument}{Causer})
\end{xlist}\end{exe}
Note that the examples in \cref{ex:passiveInstrument} can be rephrased 
in active voice with the \psst{Instrument} as the subject.

A device serving as a mode of transportation or medium of communication 
counts as an \psst{Instrument}, but is often construed as a \psst{Locus} or \psst{Path}:
\begin{exe}
  \ex Communicate \p{by} \choices{phone\\email}. (\psst{Instrument})
  \ex Talk \p{on} the phone. (\rf{Instrument}{Locus})
  \ex Send it \choices{\p{over}\\\p{via}} email. (\rf{Instrument}{Path})
  \ex Travel \p{by} train. (\psst{Instrument})
  \ex Escape \p{with} a getaway car. (\psst{Instrument})
  \ex Escape \p{in} the getaway car. (\rf{Instrument}{Locus})
\end{exe}
This includes some expressions which incorporate the \psst{Instrument} 
in a noun:
\begin{exe}
  \ex ride \p{on} horseback (\rf{Instrument}{Locus})
  \ex hold \p{at} knifepoint (\rf{Instrument}{Locus})
\end{exe}
Other non-prototypical instruments that can be construed as paths 
include waypoints from \psst{Source} to \psst{Goal}, 
and people\slash organizations serving as intermediaries:
\begin{exe}
  \ex We flew to London \p{via} Paris. (\rf{Instrument}{Path})
  \ex I found out the news \p{via} Sharon. (\rf{Instrument}{Path})
  \ex Joan bought her house \p{through} a real estate agent. (\rf{SocialRel}{Instrument})
  \ex For my Honda I always got replacement parts \p{through} the dealership. (\rf{OrgRole}{Instrument})
\end{exe}

Conversely, roadways count as \psst{Path}s but can be construed as \psst{Instrument}s:
\begin{exe}
  \ex Escape \p{through} the tunnel. (\psst{Path})
  \ex Escape \p{by} tunnel. (\rf{Path}{Instrument})
\end{exe}

Compare \psst{Means}, which is used for facilitative events rather than entities.
See also \psst{Topic}.

\hierAdef{Configuration}

\shortdef{Thing, usually an entity or property, that is involved 
in a static relationship to some other entity.}

Not used directly---see subtypes.

\hierBdef{Identity}

\shortdef{A category being ascribed to something, 
or something belonging to the category denoted by the governor.}

Prototypical prepositions are \p{of} (where the governor is the category) 
and \p{as} (where the object is the category):
\begin{exe}
  \ex\label{ex:stateof} the state \p{of} Washington [as opposed to the city]
  \ex The liberal state \p{of} Washington has not been receptive to Trump's message.
  \ex \p*{As}{as} a liberal state, Washington has not been receptive to Trump's message.
  \ex\label{ex:ascolleague} I like Bob \p{as} a colleague. [but not as a friend]
  \ex What a gem \p{of} a restaurant! [exclamative idiom: both NPs are indefinite]
  \ex the problem/task/hassle \p{of} raising money
  \ex the age \p{of} eight
  \ex They did a great job \p{of} cleaning my windows.
  \ex\label{ex:shell} \rf{Topic}{Identity}, with a governing noun in the domain of communication or cognition:
    \begin{xlist}
      \ex the topic/issue/question \p{of} semantics
      \ex the idea \p{of} raising money
    \end{xlist}
\end{exe}
Something may be specified with a category in order to disambiguate it \cref{ex:stateof}, 
or to provide an interpretation or frame of reference with which that entity is to be considered.
In some cases, like \cref{ex:shell}, the category is a \emph{shell noun} \citep{schmid-00} 
requiring further specification.

Categorizations may be situational rather than permanent/definitional:
\begin{exe}\ex\label{ex:assituational}\begin{xlist}
  \ex She appears \p{as} Ophelia in \emph{Hamlet}.
  \ex He is usually a bartender, but today he is working \p{as} a waiter.
\end{xlist}\end{exe}

Paraphrase test: ``(thing) IS (category) [in the context of the event]'': 
``Washington is a liberal state'', ``opening a new business is a hassle'', 
``She is Ophelia'', etc. Note that \p{as}+category may attach syntactically 
to a verb, as in \cref{ex:ascolleague} and  \cref{ex:assituational}, 
rather than being governed by the item it describes.

If the object of the preposition is a property (as opposed to a category), 
the scene role is \psst{Characteristic}:
\begin{exe} 
  \ex Adnominal: \rf{Characteristic}{Identity}\begin{xlist}
    \ex a car \p{of} high quality
    \ex a man \p{of} honor
    \ex a business \p{of} that sort [contrast with \psst{Species}, \cref{sec:Species}]
  \end{xlist}
  \ex Secondary predicate adjective: \rf{Characteristic}{Identity}\begin{xlist}
    \ex She described him \p{as} sad.
    \ex He strikes me \p{as} sad.
  \end{xlist}
\end{exe}

See also: \psst{ComparisonRef}

\begin{history}
  Generalized from v1, where it was called \sst{Instance} and restricted 
  to the ``(category) \p{of} (thing)'' formulation. 
  The relevant usages of \p{as} were labeled \sst{Attribute}.
\end{history}

\hierBdef{Species}

\shortdef{A category qualified by \w{sort}, \w{type}, \w{kind}, \w{species}, \w{breed}, etc. 
Includes \w{variety}, \w{selection}, \w{range}, \w{assortment}, etc.\ 
meaning `many different kinds'.}

\begin{exe}\ex\begin{xlist}
  \ex that sort \p{of} business
  \ex A good type \p{of} ant to keep is the red ant .
  \ex certain strains \p{of} \emph{Escherichia coli}
  \ex Modern breeds \p{of} these homing pigeons return reliably
  \ex Some poor sap applied the wrong brand \p{of} paint
  \ex This store offers a wide selection \p{of} footstools
\end{xlist}\end{exe}

\psst{Species} is \emph{not} used if the sort/variety noun 
is the object rather than the governor:
\begin{exe}
  \ex a business \p{of} that sort (\psst{Characteristic})
\end{exe}

\hierBdef{Gestalt}

\shortdef{Generalized notion of ``whole'' understood with reference to 
a component part, possession, set member, or characteristic. 
See \psst{Characteristic}.}

\psst{Gestalt}---the supercategory of \psst{Whole} and \psst{Possessor}---applies 
directly for entities and eventualities which can loosely be conceptualized as 
containing or possessing something else, but for which 
neither \psst{Whole} nor \psst{Possessor} is a good fit.

\paragraph{Properties.}
The holder of a property if the property is the governor:
\begin{exe}
    \ex {\setlength\multicolsep{0pt}%
    \begin{multicols}{2}
      \begin{xlist} 
        \ex the blueness \p{of} the sky
        \ex the size \p{of} the crowd
        \ex the price \p{of} the tea
        \ex the start time \p{of} the party
        
        \sn the sky\p{'s} blueness
        \sn the crowd\p{'s} size
        \sn the tea\p{'s} price
        \sn the party\p{'s} start time
      \end{xlist}
    \end{multicols}}
    \ex\label{ex:amountGestalt} the amount \p{of} time allowed [but see \cref{ex:QuantityGestalt}]
    \ex the food/service \p{at} this restaurant (\rf{Gestalt}{Locus})
\end{exe}

\paragraph{Containers.}
The construal \rf{Locus}{Gestalt} is used for a container denoted by the governor:
\begin{exe}
\ex the room\p{'s} 2 beds (\rf{Locus}{Gestalt})
\end{exe}

\paragraph{Discourse-associated item.}
A referent temporarily associated with another referent in the discourse 
and used to help identify it: 
\begin{exe}
  \ex Sam\p{'s} dog (= the dog that Sam mentioned seeing earlier in the conversation)
\end{exe}
%\item	Anything that is borderline between subcategories \psst{Possessor} and \psst{Whole}

\paragraph{Other possessive constructions.}
\psst{Gestalt} is the construal for many uses of possessive syntax
where the semantic criteria for \psst{Possessor} are not met. 
For instance, s-genitive marking of participant roles (\psst{Agent}, \psst{Experiencer}, 
etc.)\ are analyzed with \psst{Gestalt} as the function. 
Moreover, the s-genitive construction, 
unlike \p{of}, is never analyzed with \psst{Whole} as the function, 
so \rf{Whole}{Gestalt} is used. 
See \cref{sec:genitives} for discussion of possessive constructions.

\hierCdef{Possessor}

\shortdef{Animate who \textbf{has} something (the \psst{Possession}) 
which is not part of their body 
or inherent to their identity/character but could, in principle, be taken away.}

Prototypically expressed with the \emph{s-genitive} (\cref{sec:genitives}: \p{'s} and possessive pronouns), 
and \p{of} (the \emph{of-genitive}):

\begin{exe}
{\setlength\multicolsep{0pt}%
\begin{multicols}{2}
  \ex\begin{xlist}
  % of-genitive
  \ex the house \p{of} the Smith family
  \ex the corgis \p{of} Queen Elizabeth
  % s-genitive
  \sn the Smith family\p{'s} house
  \sn Queen Elizabeth\p{'s} corgis
  % \ex the rich \p{'s} money}
  % \ajb{\ex \choices{John \p{'s}\\\p{our}} dog
  % \ex ?the dog \p{of} John
  % \ex the dog \p{of} ours
\end{xlist}
\end{multicols}}
\end{exe}
\psst{Possessor} is not limited to cases of \emph{ownership}, but also includes temporary forms of possession, 
such when something is on loan to or under the control of the possessor:
\begin{exe}
  \ex John\p{'s} hotel room [the room John is staying in as a guest]
  \ex Mary\p{'s} delivery truck [the company truck that Mary drives as an employee]
\end{exe}
A wearer of attire is also in this category:
\begin{exe}
{\setlength\multicolsep{0pt}%
\begin{multicols}{2}
  % of-genitive
  \ex the cloak \p{of} He-Who-Must-Not-Be-Named
  % s-genitive
  \sn He-Who-Must-Not-Be-Named\p{'s} cloak
\end{multicols}}
\ex the cloak \p{on} He-Who-Must-Not-Be-Named (\rf{Possessor}{Locus})
\end{exe}

See \psst{Accompanier}, \psst{Beneficiary}, \psst{OrgRole}.

\hierCdef{Whole}

\shortdef{Something described with respect to its part, portion, subevent, subset, 
or set element. See \psst{PartPortion}.}

\begin{exe}
  
  \ex {\setlength\multicolsep{0pt}%
  \begin{multicols}{2} 
   \begin{xlist}
    % of-genitive
    \sn \psst{Whole}
    \ex	the new engine \p{of} the car
    \ex	the flaxen hair \p{of} the girl
    \ex\label{ex:layers}	the 3 layers \p{of} the cake
    \ex\label{ex:prongs}	the 3 prongs \p{of} the strategy
    \ex the tastiest bit \p{of} the cake
    \ex the southern tip \p{of} the island
    \ex the interior \p{of} the shopping bag
    \ex the end \p{of} the journey
    \ex the 14~episodes \p{of} a TV series
    
    % s-genitive
    \sn \rf{Whole}{Gestalt}
    \sn the car\p{'s} new engine
    \sn the girl\p{'s} flaxen hair
    \sn the cake\p{'s} 3 layers
    \sn the strategy\p{'s} 3 prongs
    \sn the cake\p{'s} tastiest bit
    \sn the island\p{'s} southern tip
    \sn the shopping bag\p{'s} interior
    \sn the journey\p{'s} end
    \sn a TV series\p{'s} 14~episodes
  \end{xlist}
  \end{multicols}}
  \ex	the south \p{of} France
    %\ex 2 \p{of} my 5 daughters % Quantity ~> Whole
  \ex\label{ex:rest} The \choices{remainder\\rest} \p{of} the cake %\nss{Maybe this should be 
    % \rf{Quantity}{Whole} after all, even though ``the rest of it'' is a dubious way to answer 
    % ``How much of it?''. ``The remaining 6 ounces of cake'' certainly specifies a quantity.\ab{but is that because of "6 ounces" or "remainder (\/remaining)"?   Since "remaining of the cake" is cake too (a physical object in this case), it has the property of having quantity, but "remaining of ..." does not have "quantity" as the sense in the foreground, it's part-whole relationship.}}
    %\ex	the tennis matches \p{of} a series
    %\ex	the beginning \p{of} the party
  \ex \rf{Whole}{Locus}: \begin{xlist}
    \ex	the 14~episodes \p{in} a TV series
    \ex	the new engine \p{in} the car
    \ex the escape key \p{on} the keyboard
    \ex the flaxen hair \p{on} the girl
  \end{xlist}
  \ex\label{ex:in-pile-adnominal}	the clothes \p{in} that pile are dirty (\rf{Whole}{Locus}) [but contrast adverbial/predicative \p{in}+shape: \cref{ex:in-shape} under \psst{Manner}]
  \ex There are several options to choose \p{from}. (\rf{Whole}{Source})
  \ex\label{ex:sets} Sets and ratios:
    \begin{xlist}
      \ex This is one \p{of} the \choices{worst\\better} retaurants in town. (\psst{Whole})
      \ex 2 \p{in} 10 American children are redheads. (\rf{Whole}{Locus})
      \ex 2 \p{out\_of} 10 American children are redheads. (\rf{Whole}{Source}) %\nss{should this be \psst{RateUnit}?}\ab{maybe}
      \ex \p*{Out\_of}{out\_of} the 10 children in the class, only Mary is a redhead. (\rf{Whole}{Source})
      \ex\label{ex:amongSet} \p*{Among}{among} the 10 children in the class, only Mary is a redhead. (\psst{Whole})
    \end{xlist}
\end{exe}

If the governor narrows the reference to a certain amount of the \psst{Whole}, 
the construal \rf{Quantity}{Whole} is used---see \cref{ex:QuantityWhole}. 
Note that this only applies if the governor is a measure term; 
it does not apply to distinctive parts like ``layers'' \cref{ex:layers} 
and ``prongs'' \cref{ex:prongs}, even if a count is specified.

Used to construe geographic and temporal ``containers'':
\begin{exe}
  \ex	Famous castles \p{of} the valley (\rf{Locus}{Whole})
  \ex \begin{xlist}
    \ex the \choices{15th\\Ides} \p{of} March (\rf{Time}{Whole})
    \ex March \p{of} 44~BC (\rf{Time}{Whole})
  \end{xlist}
\end{exe}

The prepositions \p{between} and \p{among} can impose \psst{Whole} construals 
by combining two or more items in the object NP (contrast with \cref{ex:amongSet}):
\begin{exe}
  \ex\label{ex:betweenParties}  The negotiations \choices{\p{between}\\\p{among}} the parties went well. (\rf{Agent}{Whole})
  \exp{ex:betweenParties} The negotiations \p{by} the parties went well. (\psst{Agent})
\end{exe}

\begin{history}
  In v1, \sst{Superset} was distinguished as a subtype of \psst{Whole} 
  for examples such as \cref{ex:sets}, but the distinction was dropped for v2 
  (as was \sst{Elements}: see \psst{PartPortion}).
\end{history}

\hierBdef{Characteristic}

\shortdef{Generalized notion of a part, feature, possession, 
or the contents or composition of something, 
understood with respect to that thing (the \psst{Gestalt}).}

Can be used to construe person-to-person relationships such as kinship, 
whose scene role should be \psst{SocialRel}. 
Labels \psst{Possession}, \psst{PartPortion}, and its subtype \psst{Stuff} 
are defined for some important subclasses.

\psst{Characteristic} applies directly to:
\begin{itemize}
\item	A property value: 
\begin{exe} 
  \ex Adnominal: \rf{Characteristic}{Identity}\begin{xlist}
    \ex a car \p{of} high quality
    \ex a man \p{of} honor
    \ex a business \p{of} that sort [contrast with \psst{Species}, \cref{sec:Species}]
  \end{xlist}
  \ex Secondary predicate adjective: \rf{Characteristic}{Identity}\begin{xlist}
    \ex She described him \p{as} sad.
    \ex He strikes me \p{as} sad.
  \end{xlist}
\end{exe}
\item	Role of a complex framal \psst{Gestalt} that has no obvious decomposition into parts: 
\begin{exe}\ex \begin{xlist}
  \ex the restaurant \p{with} \choices{a convenient location\\an extensive menu}
  \ex a party \p{with} great music
\end{xlist}\end{exe}
\item	That which is located in a container denoted by the governor: 
\begin{exe}
  \ex a room \p{with} 2 beds [beds are among the things in the room]
  \ex\rf{Characteristic}{Stuff} where the object of the preposition is construed as describing the contents in their entirety:\begin{xlist}
    \ex a shelf \p{of} rare books
    \ex a cardboard box \p{of} snacks
  \end{xlist}
\end{exe}
\item Member(s) forming a partial subset of an organizational collective denoted by the governor:
\begin{exe}
  \ex	A piano quintet is a chamber group \p{with} a piano (in it)\\ (\rf{OrgRole}{Characteristic})
\end{exe}
\item With a transitive verb like \emph{search}, \emph{examine}, or \emph{test}, 
the attribute of the \psst{Theme} that is being examined:
\begin{exe}
  \ex He examined the vase \p{for} damage.
  \ex\label{ex:search-obj-for} He searched the room \p{for} his laser pistol. [contrast intransitive \psst{Theme}, \cref{ex:search-for}]
  \ex He was tested \p{for} low blood sugar.
\end{exe}
\item The scale or dimension by which items are compared:
\begin{exe}
  \ex The children are \choices{sorted\\screened} \p{by} height
  \ex\begin{xlist}
    \ex She exceeds him \p{in} height
    \ex There is no difference \p{in} height
  \end{xlist}
\end{exe}
\item	Anything that is borderline between the \psst{Possession} and \mbox{\psst{PartPortion}} subcategories
\end{itemize}

Typically, one of ``\psst{Gestalt} \{HAS, CONTAINS\} \psst{Characteristic}'' is entailed. 
This does not help to distinguish subtypes.

\begin{history}
  The v1 label \sst{Attribute} was intended to apply to features of something, 
  but was vaguely defined. With the overhaul of the \psst{Configuration} 
  subhierarchy, \sst{Attribute} has primarily been replaced by 
  \psst{Characteristic} and its subtypes and \psst{Identity}.
\end{history}

\hierCdef{Possession}

\shortdef{That which some \psst{Possessor} (animate or personified, e.g.~an institution) 
\textbf{has}, and which is not part of their body or inherent to their identity/character 
but could, in principle, be taken away.}

Sometimes called \emph{alienable} possession. 
The possession may be concrete or abstract, and temporary or permanent.

Prototypical prepositions are \p{with} and \p{without}:
\begin{exe}
\ex	People \p{with} money
\end{exe}

Attire is included here as well:
\begin{exe}
  \ex the kid \p{with} \choices{a vest\\makeup} (on)
  \ex the kid \p{in} a vest (\rf{Possession}{Locus})
\end{exe}

Immediate concrete possession uses an \psst{Accompanier} construal:
\begin{exe}
  \ex Hagrid exited the shop \p{with} (= carrying) a snowy owl. (\rf{Possession}{Accompanier})
\end{exe}

There is also a (negated) possession sense of \p{out}/\p{out\_of}:
\begin{exe}
  \ex\begin{xlist} 
    \ex We are \p{out\_of} toilet paper.
    \ex Toilet paper? We are \p{out}.
  \end{xlist}
\end{exe}

Paraphrase test: ``\psst{Possessor} POSSESSES \psst{Possession}'', 
``\psst{Possessor} is IN POSSESSION OF \psst{Possession}'', or 
``\psst{Possessor} HAS ON \psst{Possession}''. 
The latter is especially appropriate for immediate concrete possession.

\hierCdef{PartPortion}

\shortdef{A part, portion, subevent, subset, or set element (e.g., an example or exception) 
of some \psst{Whole}.}

Anything directly labeled with \psst{PartPortion} 
is understood to be \textbf{incomplete} relative to the \psst{Whole}.
This includes body parts and partial food ingredients.

Prototypical prepositions include \p{with}, \p{without};
\p{such\_as}, \p{like} for exemplification; 
and \p{but}, \p{except}, \p{except\_for} for exceptions:
\begin{exe}
  \ex \begin{xlist}
    \ex	a car \p{with} a new engine
    \ex	a strategy \p{with} 3 prongs
    \ex	the girl \p{with} flaxen hair
    \ex	a man \p{with} a wooden leg named Smith
    \ex	a valley \p{with} a castle
    \ex	a quintet \p{with} 2 cellos
    \ex	a performance \p{with} a guitar solo
    \ex	a cake \p{with} 3 layers
    \ex	a sandwich \p{with} wheat bread
    \ex	soup \p{with} carrots (in it)
    \ex	a chicken sandwich \p{with} ketchup (on it)
  \end{xlist}
  \ex	Bread \p{without} gluten
\end{exe}
Some can be paraphrased with INCLUDES, but this is not determinative.

\paragraph{Elements and Exceptions.} 
\psst{PartPortion} is used for adpositions marking a member or non-member of a set:
\begin{exe}
  \ex\label{ex:suchAs}	strategies \p{such\_as} divide-and-conquer
  \ex Everyone \p{except}/\p{but} Bob plays trombone.
\end{exe}
Set-membership can be construed as comparison:
\begin{exe}
  \ex strategies \p{like} divide-and-conquer [same reading as \cref{ex:suchAs}]\\ (\rf{PartPortion}{ComparisonRef})
\end{exe}

\paragraph{Diverse Examples.}
In describing a set or whole, a sort of scanning with \p{from}\dots\p{to} can be used indicate diversity or coverage of 
the items/parts:
\begin{exe}
  \ex\label{ex:diverserange} Everyone \p{from}$_{\text{\rf{PartPortion}{Source}}}$ the peasants 
  \p{to}$_{\text{\rf{PartPortion}{Goal}}}$ the lord and lady gathered for the feast.
\end{exe}

\paragraph{\pex{Start \p{with}}, \pex{end \p{with}}, etc.}
Along similar lines as \cref{ex:diverserange}, \p{with} can be used with 
an aspectual verb to indicate an item in a sequence: 
\pex{start \p{with}}, \pex{continue \p{with}}, \pex{end \p{with}}, and similar.
Here the scene role \psst{PartPortion} applies 
(though note that it is a part with respect to another argument of the verb, 
not the verb itself):
\begin{exe}
  \ex\rf{PartPortion}{Means}:\begin{xlist} 
    \ex My teacher started the lesson \p{with} a quiz.
    \ex The lesson started \p{with} a quiz.
  \end{xlist}
  \ex The meal started \p{with} an appetizer. (\rf{PartPortion}{Instrument})
\end{exe}

\begin{history}
  In v1, instead of this category, there were separate categories 
  \sst{Elements} for set members, \sst{Comparison/Contrast} for exemplification,
  and \sst{Attribute} for other parts (grouped with properties, which are now \psst{Gestalt}).
  (\sst{Superset} was removed along with \sst{Elements}: see \psst{Whole}.)
\end{history}

\hierDdef{Stuff}

\shortdef{The members comprising a group/ensemble, 
or the material comprising some unit of substance. 
\psst{Stuff} is distinguished from other instances of \psst{PartPortion}
in fully covering (or ``summarizing'') the aggregate whole.}

Paraphrase test: ``\psst{Whole} CONSISTS OF \psst{Stuff}''

\begin{exe}
  \ex	\begin{xlist}    
    %\ex	A throng \p{of} tourists % Quantity ~> Stuff
    \ex	A clump \p{of} sand
    \ex	A piece \p{of} wood
    %\ex	A series \p{of} tennis matches % feels borderline with Quantity ~> Stuff
    \ex	An evening \p{of} Brahms
    \ex	A meal \p{of} salmon
  \end{xlist}
  \ex	A salad \choices{\p{of}\\\p{with}} mixed greens
  \ex\label{ex:bottleStuff} This bottle is \p{of} beer (and that one is of wine). (\rf{Characteristic}{Stuff}) [but see \cref{ex:bottleQuantity}]
  % \ex	\rf{Quantity}{Stuff}: see \cref{ex:QuantityStuff}
  %   \begin{xlist}
  %     \ex A bottle('s worth) \p{of} beer
  %     \ex A bag('s worth) \p{of} chips
  %   \end{xlist}
  \ex A group/throng \p{of} vacationers (\rf{Quantity}{Stuff}) [governor is collective noun not denoting an organization; more at \psst{Quantity}]
  \ex \rf{OrgRole}{Stuff}:
  	\begin{xlist}
      \ex An order \p{of} nuns
      \ex	A chamber group \choices{\p{of}\\\p{with}} 5 players
    \end{xlist}
\end{exe}

\psst{Stuff} has no specific counterpart under \psst{Whole}.

\hierBdef{Accompanier}

\shortdef{Entity that another entity is together with.}

Sometimes called \emph{comitative}.

Prototypical prepositions are \p{with}, \p{without}, \p{along\_with}, 
\p{together}, \p{together\_with}, and \p{in\_addition\_to}:
\begin{exe}
  \ex I'll have soup \choices{\p{with}\\\p{without}} salad.
  \ex She'll be \p{with} us in spirit.
\end{exe}
`Togetherness' is a subjective concept that goes beyond proximity; 
contrast \cref{ex:withMom} with \cref{ex:nextToMom}, which 
provide slightly different interpretations of the same spatial scene:
\begin{xexe}
  \ex\label{ex:withMom} The girl is standing \p{with} her mother. (\psst{Accompanier})
  \ex\label{ex:nextToMom} The girl is standing \p{next\_to} her mother. (\psst{Locus})
\end{xexe}

For an ``extra participant'' in an activity, 
where two parties perform the activity together 
(but the nature of the activity would not fundamentally 
change if they each performed it independently), 
a \psst{Co-Agent} construal is used:
\begin{exe}
  \ex Do you want to walk \p{with} me? (\rf{Accompanier}{Co-Agent})
\end{exe}
By contrast, if the nature of the scene fundamentally requires multiple participants, 
simple \psst{Co-Agent} is used. Often there is ambiguity:\footnote{Adding \p{together} 
seems to favor the (b)~readings: \pex{I fought \p{together\_with} them}, \pex{We fought \p{together}} 
can only mean we were on the same side. Contrastive stress can also force one reading: 
\pex{I fought \p*{WITH}{with} them (not \p*{AGAINST}{against} them)}.}
\begin{exe}
  \ex Do you want to talk \p{with} me? 
  \begin{xlist}
    \ex {}[\emph{The reading:} Should we have a conversation?] (\psst{Co-Agent})
    \ex {}[\emph{The reading:} Do you want to join me in talking to a third party?] 
      (\rf{Accompanier}{Co-Agent})
  \end{xlist}
  \ex I fought \p{with} them to reform the regulation.
  \begin{xlist}
    \ex {}[\emph{The reading:} I fought against them.] (\psst{Co-Agent})
    \ex {}[\emph{The reading:} I was on the same side as them.] (\rf{Accompanier}{Co-Agent})
  \end{xlist}
\end{exe}

If the object denotes an item that the governor has on hand in their possession, 
then the construal \rf{Possession}{Accompanier} is used:
\begin{exe}
  \ex I walked in \p{with} an umbrella. (\rf{Possession}{Accompanier})
\end{exe}

\paragraph{X\textsubscript{\emph{i}} \emph{bring}/\emph{take}/\dots~Y \p{with} PRON\textsubscript{\emph{i}}.} 
This construction repeats the subject argument in a \p{with}-PP, 
which is analyzed as \rf{Possessor}{Accompanier} or \psst{Accompanier}
depending on whether the scene involves possession (of something nonvolitional) or not:
\begin{exe}
  \ex\begin{xlist}
    \ex I brought my friend \p{with} me. (\psst{Accompanier}) [emphasizes that the (volitional) friend is accompanying the subject]
    \ex I brought my friend.
  \end{xlist}
  \ex\begin{xlist}
    \ex I brought my backpack \p{with} me. (\rf{Possessor}{Accompanier}) [emphasizes that the (nonvolitional) backpack is in the subject's immediate control]
    \ex I brought my backpack.
  \end{xlist}
\end{exe}

See also: \psst{Instrument}, \psst{Manner}

\hierBdef{InsteadOf}

\shortdef{A default or already established thing for which something else stands in 
or is chosen as an alternative.}

\begin{exe}
  \ex I ordered soup \choices{\p{instead\_of}\\\p{rather\_than}} salad.
  \ex \p*{Instead\_of}{instead\_of} ordering salad, I ordered soup.
  %\ex They replaced \uline{my old tires} with new ones. %#nonprep
  \ex The new shirts were gray \p{instead\_of} black.
  \ex\label{ex:subst-for} They substituted new tires \p{for} my old ones. [replacee; contrast \cref{ex:repl-with}]
\end{exe}
May be construed spatially:
\begin{exe}
  \ex I chose soup \p{over} salad. (\rf{InsteadOf}{Locus})
\end{exe}
But when \p{over} is used for a scene of liking or preference, see \psst{ComparisonRef}.

See also: \psst{Accompanier}, \psst{ComparisonRef}, \psst{Co-Theme}

\hierBdef{ComparisonRef}

\shortdef{The reference point in an explicit comparison (or contrast), i.e., 
an expression indicating that something is 
\textbf{similar/analogous to}, \textbf{different from}, or \textbf{the same as}
something else.}

The marker of the ``something else'' (the ground in the figure–ground relationship) 
is given the label \psst{ComparisonRef}:
\begin{exe}
  \ex \begin{xlist}
    \ex She is taller \p{than} me.
    \ex She is taller \p{than} I am.
    \ex She is taller \p{than} she is wide.
    \ex She is better at math \p{than} at drawing.
    \ex The shirt is more gray \p{than} black.
    %\ex She is greater in height \p{than} me.
  \end{xlist}
  \ex\label{ex:comparAs} \begin{xlist}
    \ex She is as tall \p{as} I am.
    \ex Your face is (as$_{\text{\rf{Characteristic}{Extent}}}$) red \p{as} a rose. (more on \p{as}-\p{as} comparatives: \cref{sec:as-as})
    \ex Your surname is the\_same \p{as} mine.
  \end{xlist}
  \ex Harry had never met anyone quite \p{like} Luna.
  \ex It was \choices{\p{as\_if}\\\p{like}} he had insulted my mother.
\end{exe}

The comparison is often made with respect to some dimension or attribute, the \psst{Characteristic}, 
which may or may not be scalar. 
The comparison may be figurative, employing simile, hyperbole, or spatial metaphor 
(\pex{close to} in the sense of `similar to'). 
The \psst{ComparisonRef} may even be a desirable or hypothetical/irrealis 
event or state (\pex{It was \p{as} it should have been}).

Prototypical prepositions include \p{than}, \p{as} (including the second item 
in the \p{as}--\p{as} construction), \p{like}, \p{unlike}. 
Prominent construals are \p{to} (\psst{Goal} for similar-thing) 
and \p{from} (\psst{Source} for dissimilar-thing).

\paragraph{\psst{Locus} construal.}
If something is preferred or appreciated \p{over} something else, \rf{ComparisonRef}{Locus} is used:
\begin{exe}
  \ex I prefer this restaurant \p{over} that one. (\rf{ComparisonRef}{Locus})\\{} 
  [paraphrase: I like this restaurant better \p{than} that one.]
\end{exe}
But for scenes of choice and substitution, see \psst{InsteadOf}.

\paragraph{\psst{Source} and \psst{Goal} construals.}
Resemblance and equivalence may be expressed with \p{to}, 
while difference may be expressed with \p{from}:
\begin{exe}
  \ex\rf{ComparisonRef}{Goal}:\begin{xlist}
    \ex Shall I compare thee \p{to} a summer's day?
    \ex Her height is \choices{equal\\close} \p{to} mine.
  \end{xlist}
  \ex\rf{ComparisonRef}{Source}:\begin{xlist}
    \ex We need to distinguish what is achievable \p{from} what is desirable.
    \ex Her height is different \p{from} mine.\footnote{American English. Interestingly, 
    \emph{different \p{to}} occurs in British English.}
  \end{xlist}
\end{exe}

\paragraph{\psst{Accompanier} construal.}
\begin{exe}
  \ex Don't compare me \p{with} my sister! (\rf{ComparisonRef}{Accompanier})
\end{exe}

\paragraph{Category as standard.} 
An indirect comparison can be made by relating something to a category 
to which it may or may not belong. 
The category stands for its members or prototypes. For example, in:
\begin{exe}
  \ex\label{ex:catAsStandard} He is short \p{for} a basketball player. (\psst{ComparisonRef})
\end{exe}
the category \pex{basketball player} serves as the standard against which \pex{he} is deemed short.

\paragraph{Sufficiency and excess.}
In a statement of sufficiency or excess, 
\rf{ComparisonRef}{Purpose} marks the consequence enabled or prevented by some condition.
This is similar to \textbf{necessity} (discussed under \psst{Purpose}), 
only here, the condition involves a comparison; the standard of comparison 
is implied by the consequence:
\begin{exe}
  \ex\rf{ComparisonRef}{Purpose}:\begin{xlist}
    \ex\label{ex:tooShort} He is \choices{too short\\not tall enough} \choices{\p{for}\\\p{to} play} basketball.
    \ex\label{ex:insufficient} His height is insufficient \p{for} basketball.
  \end{xlist}
\end{exe}
In these constructions, an adverb (\pex{too}, \pex{enough}, \pex{insufficiently}, etc.)\ 
or an adjective (\pex{insufficient}) licenses the PP or infinitival expressing the consequence.\footnote{See the Degree-Consequence construction \citep{bonial-18}.}
They amount to saying
\pex{He is shorter than the height he would have to be in order to play basketball}, 
which features separate constructions for comparison and necessity-for-purpose.

\paragraph{\rf{Manner}{ComparisonRef} construal.}
This applies to an analogy that describes the \emph{how} of an event 
(be it agentive or perceptual):
\begin{exe}
\ex \rf{Manner}{ComparisonRef}:\begin{xlist}
  \ex You eat \p{like} a pig (eats).
  \ex You smell \p{like} a pig.
\end{xlist}
\end{exe}
However, where an analogy is an external comment on an event 
rather than filling in a role of the event, it is simply \psst{ComparisonRef}. 
Contrast:
\begin{exe}
  \ex You ate a whole pie \p{like} my cousin did.
  \begin{xlist}
    \ex \emph{Role reading:} The way in which you ate a pie was similar. (\rf{Manner}{ComparisonRef})
    \ex \emph{External comment reading:} You ate a whole pie, and so did my cousin. (\psst{ComparisonRef})
  \end{xlist}
  %\ex I was elated, \p{like}/\p{as\_if} I had just won the lottery. (\psst{ComparisonRef})
\end{exe}

\paragraph{Analogy and non-analogy readings of \p{like}.}
In descriptions, adverbial \p{like}, \p{as\_if}, etc.\  %with an extracted object of \pex{what} 
can be ambiguous, especially in a scene of perception. 
For example:
\begin{exe}
  \ex This looks \p{like} a Van Gogh painting.
  \begin{xlist}
    \ex \emph{Analogy reading:} This looks similar to a Van Gogh painting. (\rf{Manner}{ComparisonRef})
    \ex \emph{Conclusion reading:} This looks to be a Van Gogh painting (it probably is one). (\rf{Theme}{ComparisonRef})
  \end{xlist}
  \ex It sounded \p{like}/\p{as\_if}
  \begin{xlist}
    \ex \dots he had drunk a gallon of helium. (\rf{Manner}{ComparisonRef}: analogy reading more likely)
    \ex \dots they weren't taking me seriously. (\rf{Theme}{ComparisonRef}: conclusion reading more likely)
  \end{xlist}
\end{exe}
Similarly for \pex{seem \p{like}}, \pex{feel \p{like}}, etc.

Another ambiguity can arise when \p{like} occurs with \pex{what} as its extracted object. 
In the following sentences, the most likely interpretation is not one of analogy between 
two things, but rather an open-ended description. 
(\pex{Who does it look \p{like}?}, by contrast, implicates an analogy to an individual.) 
We therefore treat \pex{\p{like} what} as a PP idiom, 
and label it \rf{Manner}{ComparisonRef}:
\begin{exe}
  \ex\label{ex:whatlike}\rf{Manner}{ComparisonRef}:\begin{xlist}
    \ex I know what\_~~Steve looks~~\_\p{like}. (I know how Steve looks.)
    \ex What\_~~does her hair look~~\_\p{like}? (How does her hair look?)
    \ex What\_~~is the party~~\_\p{like}? (How is the party?)
  \end{xlist}
\end{exe}
A \pex{how}-paraphrase is generally possible, though \pex{how} may suggest 
a positive or negative evaluation is available, whereas \pex{what} is more neutral.

Constrast unaccusative perception verb + \p{of} combinations:
\begin{exe}
  \ex\label{ex:smellOfNotCmp} \choices{Your father smells\\The soup tastes} \p{of} elderberries. (\rf{Manner}{Stuff}) [also~\cref{ex:smellOf}]
\end{exe}

\paragraph{Category exemplars and set members.} 
When governed by an NP naming a category or set, \p{like} is ambiguous 
between exemplifying a member, as in \cref{ex:likeSetMember} and \cref{ex:likeCatMember}, 
and merely indicating similarity, as in \cref{ex:likeSetSimilar} and \cref{ex:likeCatSimilar}:
\begin{exe}
  \ex Colbert frequently promotes comedians \p{like} himself.
    \begin{xlist}
      \ex\label{ex:likeSetSimilar} [\emph{Exclusive/restrictive reading:} \emph{similar to} himself (but not including himself)]
        (\psst{ComparisonRef})
      \ex\label{ex:likeSetMember} [\emph{Inclusive/nonrestrictive reading:} \emph{such as}/\emph{including} himself (he promotes himself, among others)]
        (\rf{PartPortion}{ComparisonRef})
    \end{xlist}
  \ex 
    \begin{xlist}
      \ex\label{ex:likeCatSimilar} I don't know anyone else \p{like} her. [anyone else \emph{similar to} her]\\ (\psst{ComparisonRef})
      \ex\label{ex:likeCatMember} It must be great to have a wonderful doctor \p{like} \choices{her\\she is}.\\ {}
      [It must be great to have her because she is a wonderful doctor]\\ (\rf{Identity}{ComparisonRef})
    \end{xlist}
\end{exe}

\hierBdef{RateUnit}

\shortdef{Unit of measure in a rate expression.}

This is for constructions using \p{per} or \p{by} to specify a unit:

\begin{exe} \ex \begin{xlist}
  \ex The cost is \$10 \p{per} item.
  \ex A fuel efficiency of 40 miles \p{per} gallon (of gas)
  \ex Pizza is sold \p{by} the slice.
  \ex They charge \p{by} the hour.
\end{xlist}\end{exe}

Paraphrase: The adposition can be paraphrased as ``for each/every''.

\begin{history}
  In v1, this fell under \sst{Value}.
\end{history}

\hierBdef{Quantity}

\shortdef{Something measured by a quantity denoted by the governor.}

The governor may be a precise or vague count/measurement. 
This includes nouns like ``lack'', ``dearth'', ``shortage'', ``excess'', or ``surplus''
(meaning a too-small or too-large amount).

Question test: the governor answers ``How much/many of (object)?''

The main preposition is \p{of}.

\begin{itemize}
\item Simple \psst{Quantity}:
\begin{exe}
  \ex\label{ex:bottleQuantity}	Pour me a bottle('s worth) \p{of} beer. [but see \cref{ex:bottleStuff}]
  \ex	I have 2 years \p{of} training.
  \ex	\begin{xlist}
    \ex I ate \choices{6 ounces\\a piece} \p{of} cake.
    \ex	An ounce \p{of} compassion
  \end{xlist}
  \ex	There's a dearth \p{of} cake in the house.
  \ex	This cake has thousands \p{of} sprinkles.
  \ex They number in the tens \p{of} thousands.
  \ex	\begin{xlist}
    \ex\label{ex:anumber} I have a \choices{number\\handful} \p{of} students.
    \ex	I have a lot \p{of} students.
    \ex	We did a lot \p{of} traveling.
    \ex	There is a lot \p{of} wet sand on the beach.
  \end{xlist}
  \ex	A pair \p{of} shoes
\end{exe}

\item If the measure includes a word like ``amount'', ``quantity'', or ``number'',\footnote{Excluding 
the expression ``a number'' meaning `several', as in \cref{ex:anumber}.} 
the construal \rf{Quantity}{Gestalt} is used 
(because the amount of something can be viewed as an attribute):
\begin{exe}
  \ex\label{ex:QuantityGestalt} \rf{Quantity}{Gestalt}:
  \begin{xlist}
    \ex	A generous amount \p{of} time
    \ex A large number \p{of} students
  \end{xlist}
\end{exe}
But if ``amount'', ``quantity'', etc. is used without a measure as its modifier, 
it is simply \psst{Gestalt}: see \cref{ex:amountGestalt}.

\item If the governor is a \textbf{collective noun} not denoting an organization, 
the construal \rf{Quantity}{Stuff} is used 
(note that a ``consisting of'' paraphrase is possible):
\begin{exe}
  \ex\label{ex:QuantityStuff} \rf{Quantity}{Stuff}:
  \begin{xlist}
    \ex Can you outrun a herd \p{of} wildebeest?
    \ex Put 3 bales \p{of} hay on the truck.
    \ex	\choices{A group\\2 groups\\A throng} \p{of} vacationers just arrived.
  \end{xlist}
\end{exe}
For organizational collectives, see \psst{OrgRole}.

\item Otherwise, if the object refers to \textbf{a specific item or set}, 
and the quantity measures a portion of that item 
(whether a quantifier, absolute measure, or fractional measure),
the construal \rf{Quantity}{Whole} is used:
\begin{exe}
  \ex\label{ex:QuantityWhole} \rf{Quantity}{Whole}:
  \begin{xlist}
    \ex	I ate 6 ounces \p{of} the cake in the refrigerator.
    \ex	I ate \choices{half\\50\%} \p{of} the cake.
    \ex	\choices{All/many/lots/a lot/\\some/few/both/none} \p{of} the town's residents 
    are students.
    \ex	I have seen all \p{of} the city. (= the whole city)
    \ex	A lot \p{of} the sand on the beach is wet.
    \ex	2 \p{of} the children are redheads.
    \ex 2 \p{of} the 10 children in the class are redheads.
  \end{xlist}
\end{exe}
However, simple \psst{Whole} is used if the portion is specified as 
``the rest'', ``the remainder'', etc., as in \cref{ex:rest}. %\nss{Reconsidering this: see \cref{ex:rest}}\ab{I see "half of it", "remainder of it" do have some shared meaning. Why the three "5 ounces of it", "half of it" and "remainder of it" seem different is that in the 1st case, the quantity of the proportion itself is specified, for the 2nd case, it is unspecified but can be determined from knowledge about the quantity of the whole thing and in the 3rd case, it is unspecified and cannot be %determined from knowing the quantity of the whole thing only, you also need to know how much has already been removed.}
\end{itemize}

\hierCdef{Approximator}

\shortdef{An ``operator'' that semantically takes a measurement, 
quantity, or range as an argument and ``transforms'' it in some way 
into a new measurement, quantity, or range.}

For instance:
\begin{exe}
  \ex We have \p{about} 3 eggs left.
  \ex We have \p{in\_the\_vicinity\_of} 3 eggs left.
  \ex We have \p{over} 3 eggs left.
  \ex We have \p{between} 3 and 6 eggs left.
\end{exe}
Similarly for \p{around}, \p{under}, \p{more\_than}, \p{less\_than}, \p{greater\_than}, 
\p{fewer\_than}, \p{at\_least}, and \p{at\_most}.\footnote{These constructions are 
markedly different from most PPs; it is even questionable whether these usages 
should count as prepositions. Without getting into the details here, 
even if their syntactic status is in doubt, 
we deem it practical to assign them with a semantic label in our inventory because they 
overlap lexically with ``true'' prepositions.}

\hierBdef{SocialRel}

\shortdef{Entity, such as an institution or another individual, 
with which an individual has a stable affiliation.}

Typically, \psst{SocialRel} applies directly to interpersonal relations 
(versus the subtype \psst{OrgRole} for relations involving an organization).
It does not have any prototypical adpositions. 
Construals include:
\begin{exe}
  \ex \begin{xlist}
      \ex\label{ex:workwithSR} I work \p{with} Michael. (\rf{SocialRel}{Co-Agent})
      \ex Joan has a class \p{with} Miss Zarves. (\rf{SocialRel}{Co-Agent})
    \end{xlist}
  \ex\rf{SocialRel}{Gestalt} 
  {\setlength\multicolsep{0pt}%
  \begin{multicols}{2}
    \begin{xlist}
      \ex Joan is the \choices{sister\\wife} \p{of} John.
      \ex Joan is a student \p{of} Miss Zarves.
      \ex the family \p{of} Miss Zarves
      
      \sn Joan is John\p{'s} \choices{sister\\wife}.
      \sn Joan is Miss Zarves\p{'s} student.
      \sn Miss Zarves\p{'s} family 
    \end{xlist}
  \end{multicols}}
  \ex Joan is studying \p{under} Prof.~Smith. (\rf{SocialRel}{Locus})
  \ex Joan is married \p{to} John. (\rf{SocialRel}{Co-Theme})
  \ex Joan is divorced \p{from} John. (\rf{SocialRel}{Co-Theme})
  \ex Joan bought her house \p{through} a real estate agent. [intermediary] (\rf{SocialRel}{Instrument})
\end{exe}

Note, however, that \emph{work \p{with}} is ambiguous between 
being in an established professional relationship \cref{ex:workwithSR}, 
and engaging temporarily in a joint productive activity:
\begin{exe}
  \ex\label{ex:workwithCA} I was working \p{with} Michael after lunch. (\psst{Co-Agent})
\end{exe}
It is up to annotators to decide from context which interpretation 
better fits the context.

\begin{history}
  Renamed from v1 label \sst{ProfessionalAspect}, which was borrowed from 
  \citet{srikumar-13,srikumar-13-inventory}.
  The name \psst{SocialRel} reflects
  a broader set of stative relations involving an individual 
  in a social context, including kinship and friendship.
  See also note under \psst{OrgRole}.
\end{history}

\hierCdef{OrgRole}

\shortdef{Either party in a relation between an 
organization\slash institution and an individual who has a stable affiliation 
with that organization, such as membership or a business relationship.}

Like its supertype \psst{SocialRel}, \psst{OrgRole} 
lacks any prototypical adposition, but participates in numerous construals:

\begin{exe}
  \ex \rf{OrgRole}{Gestalt} with the institution as possessor:
  {\setlength\multicolsep{0pt}%
  \begin{multicols}{2}
    \begin{xlist}
      \ex the chairman \p{of} the board
      \ex the president \p{of} France
      \ex \choices{employees\\customers} \p{of} Grunnings
      
      \sn the board\p{'s} chairman
      \sn France\p{'s} president
      \sn Grunnings\p{'s} \choices{employees\\customers}
    \end{xlist}
  \end{multicols}}
  \ex \rf{OrgRole}{Gestalt} with possessive marking on the individual:
    \begin{xlist}
      \ex \p{my} school/gym [that I attend]
      \ex \p{my} work [the place where I work]
      \ex \p{my} landscaping company [that I hired]
    \end{xlist}
  \ex \rf{OrgRole}{Possessor} if the individual is understood to possess authority 
    within or as a representative of the institution:
    \begin{xlist}
      \ex \p{my} small business [that I own or operate]
      \ex the president\p{'s} administration
    \end{xlist}
  \ex\begin{xlist}
    \ex Mr. Dursley works \p{for} Grunnings. (\rf{OrgRole}{Beneficiary})
    \ex Mr. Dursley works \p{at} Grunnings. (\rf{OrgRole}{Locus})
    \ex Mr. Dursley is \p{from} Grunnings. (\rf{OrgRole}{Source})
    \ex Mr. Dursley is \p{with} Grunnings. (\rf{OrgRole}{Accompanier})
    \ex Mr. Dursley is employed \p{by} Grunnings. (\rf{OrgRole}{Agent}) %\nss{or do we say `employ' is just a regular Agent/Theme verb?}
  \end{xlist}
  \ex I always do business \p{with} this company. (\rf{OrgRole}{Co-Agent})
  \ex\rf{OrgRole}{Accompanier}:\begin{xlist}
    \ex I bank \p{with} TSB.
    \ex my phone service \p{with} Verizon
  \end{xlist}
  \ex For my Honda I always got replacement parts \p{through} the dealership. [intermediary business] (\rf{OrgRole}{Instrument})
  \ex I serve \p{on} the committee. (\rf{OrgRole}{Locus})
  \ex\rf{OrgRole}{Stuff} if the governor is an organizational collective noun 
  and the object of the preposition describes the full membership:
  	\begin{xlist}
      \ex An order \p{of} nuns
      \ex	A chamber group \choices{\p{of}\\\p{with}} 5 players
    \end{xlist}
  \ex\rf{OrgRole}{Characteristic} if the governor is an organizational collective noun 
  and the object of the preposition denotes a subset of members:
  	\begin{xlist}
      \ex	A piano quintet is a chamber group \p{with} a piano (in it)
    \end{xlist}
\end{exe}

A family counts as an institution 
construed as a \psst{Whole} (set of its members) 
or as a \psst{Locus}:
\begin{exe}
  \ex I am the baby \p{of} the family. (\rf{OrgRole}{Whole})
  \ex people \p{in} my family (\rf{OrgRole}{Locus})
\end{exe}

For a relation between a unit and a larger institution, 
use \psst{Whole}:
\begin{exe}
  \ex the Principals Committee \p{of} the National Security Council (\psst{Whole})
\end{exe}

See also: \psst{Stuff}

\begin{history}
  \psst{OrgRole} is now distinguished within the broader \psst{SocialRel} category 
  following the precedent of the Abstract Meaning Representation \citep[AMR;][]{amr,amr-guidelines}. 
  In AMR, \texttt{have-org-role-91} captures relations between 
  an individual and an institution (such as an organization or family),
  whereas \texttt{have-rel-role-91} is used for relations between two individuals.
\end{history}

\section{Constraints on Role and Function Combinations}\label{sec:constraints}

The present scheme emerged out of extensive descriptive work with corpus data. 
Given the abundance of rare preposition usages, this document does not claim 
to cover every possible role\slash function combination for English, 
let alone other languages. 
Below are the few categorical restrictions that seem warranted for English.

\subsection{Supersenses that are purely abstract}

\psst{Participant}, \psst{Configuration}, and \psst{Temporal} are intended only to 
organize subtrees of the hierarchy, and not to be used directly. 

\subsection{Supersenses that cannot serve as functions}

\textbf{\psst{Experiencer}, \psst{Stimulus}, \psst{Originator}, \psst{Recipient}, \psst{SocialRel}, and \psst{OrgRole} 
can only serve as scene roles in English.} 
Though scenes of perception, transfer, and interpersonal\slash organizational relationships 
are fundamental in language, they always seem to exploit construals from other domains 
(motion, causation, possession, and so forth), at least insofar as 
English preposition\slash case marking is concerned. 

For example, \cref{ex:RecGoal} is clearly \psst{Recipient} at the scene level---Sam 
acquires possession of the box---but also 
fits the criteria for \psst{Goal} because Sam is an endpoint of motion 
(and \p{to} frequently marks \psst{Goal}s that are not \psst{Recipient}s). 
\Cref{ex:RecAgent} and \cref{ex:RecPoss} reflect \rf{Recipient}{Agent} and 
\rf{Recipient}{Gestalt} construals, respectively. 
\begin{xexe}
  \ex\label{ex:RecGoal} Give the box \p{to} Sam. (\rf{Recipient}{Goal})
  \ex\label{ex:RecAgent} the box received \p{by} Sam (\rf{Recipient}{Agent})
  \ex\label{ex:RecPoss} Sam\p{'s} receipt of the box (\rf{Recipient}{Gestalt})
\end{xexe}
Though the \psst{Goal} construal is arguably the most canonical expression of \psst{Recipient},
there is no preposition with a primary meaning of \psst{Recipient} independent of one of these other domains.

\textbf{Additional constraints on functions arise in the context of specific 
constructions (\cref{sec:cxns}).} For instance,
\begin{itemize}
  \item the s-genitive requires either \psst{Possessor} or \psst{Gestalt} as its function (\cref{sec:genitives})
  \item passive \p{by} requires \psst{Agent} or \psst{Causer} as its function (\cref{sec:passives})
\end{itemize}

\subsection{Supersenses that cannot serve as roles}

In the present scheme, there are no supersenses that are restricted to serving as functions.

\subsection{No temporal-locational construals}\label{sec:temploc}

\textbf{Temporal prepositions never occur with a function of \psst{Locus}, \psst{Path}, or \psst{Extent}.}

Languages routinely borrow from spatial language to describe time, 
and spatial cognition may underlie temporal cognition \citep[e.g.,][]{lakoff-80,nunez-06,casasanto-08}.
A liberal use of construal would treat \pex{arriving \p{in} the afternoon} as \rf{Time}{Locus}, 
\pex{sleeping \p{through} the night} as \rf{Duration}{Path}, 
\pex{running \p{for} 20~minutes} as \rf{Duration}{Extent}, and so forth.
However, for simplicity and practicality, we elect not to annotate \psst{Locus}, \psst{Path}, or \psst{Extent} 
construals on ordinary temporal adpositions. Thus:
\begin{xexe}
  \ex arriving \p{in} the afternoon (\psst{Time})
  \ex sleeping \p{through} the night (\psst{Duration})
  \ex running \p{for} 20~minutes (\psst{Duration})
\end{xexe}
\rf{Time}{Direction} is possible, however, as are other atemporal functions:
\begin{xexe}
  \ex Schedule the appointment \p{for} Monday. (\rf{Time}{Direction})
  \ex January \p{of} last year (\rf{Time}{Whole})
  \ex Will you attend Saturday\p{'s} class? (\rf{Time}{Gestalt})
  \ex It took a year\p{'s} work to finish the book. (\rf{Duration}{Gestalt})
\end{xexe}

Note that the above is qualified to `ordinary temporal adpositions'. 
\textbf{When the first argument of a comparative construction is marked with \p{as}, 
the function is always \psst{Extent}, even if the scene role is temporal.} 
See \cref{sec:as-as}.

\subsection{Construals where the function supersense is an ancestor or descendant of the role supersense}

Ordinarily, if a construal holds between two (distinct) supersenses, these are from different branches of the hierarchy.
In a few cases, however, one is the ancestor of the other.

\paragraph{Role is ancestor of function.}
\begin{itemize}
  \item Setting events or situations with a salient spatial metaphor are \rf{Circumstance}{Locus} or \rf{Circumstance}{Path}.
  \item Fictive motion (the extension of a normally dynamic preposition to a static spatial scene) 
  can warrant \rf{Locus}{Goal} or \rf{Locus}{Source}, as discussed under \psst{Locus}.
  \item Complete contents of containers are \rf{Characteristic}{Stuff}.
\end{itemize}

\paragraph{Function is ancestor of role.}
\begin{itemize}
  \item Some s-genitives are annotated as \rf{Whole}{Gestalt}: see \cref{sec:genitives}.
  \item When a locative PP is coerced to a goal, as with \emph{put}, \rf{Goal}{Locus} is used.
\end{itemize}



\section{Special Constructions}\label{sec:cxns}

This section discusses notable constructions that are not limited to a single supersense.

\subsection{Genitives/Possessives}\label{sec:genitives}

\Citet{blodgett-18} detail the application of this scheme to English possessive constructions:
the so-called \textbf{s-genitive}, as in \cref{ex:SGen}, and 
\textbf{of-genitive}, as in \cref{ex:OfGen}:
\begin{exe}
  \ex\label{ex:SGen} 
    \begin{xlist}
      \ex \choices{the Smith family\p{'s}\\\p{their}} house (\psst{Possessor})
      \ex \choices{the tea\p{'s}\\\p{its}} price (\psst{Gestalt})
    \end{xlist}
  \ex\label{ex:OfGen} 
    \begin{xlist}
      \ex the house \p{of} the Smith family (\psst{Possessor})
      \ex the price \p{of} the tea (\psst{Gestalt})
    \end{xlist}
\end{exe}
Note that the s-genitive is realized with case marking (clitic \p{'s} or possessive pronoun\footnote{For ease of indexing, 
\p{'s} or \p{s'} is preferred over possessive pronouns for s-genitive examples in this document.}) 
rather than a preposition, 
and the case-marked NP in the s-genitive alternates with the object of the preposition in the of-genitive.
(This may feel unintuitive: annotators looking at the s-genitive construction are often tempted to focus on 
the role occupied by the head noun rather than the case-marked noun.)

The s-genitive and of-genitive are particularly associated with 
\psst{Possessor} (which applies to a canonical form of possession) 
and the more general category \psst{Gestalt}; both supersenses are illustrated above \cref{ex:SGen,ex:OfGen}.
In addition, both genitive constructions can mark participant roles and other kinds of relations, 
including \psst{Whole} and \psst{SocialRel} relations. 
When the s-genitive is used, the \emph{function} is always either \psst{Gestalt} (most cases) 
or \psst{Possessor} (when the possession is sufficiently canonical).
While overlapping in scene roles with the s-genitive, 
\p{of} is considered compatible with some additional functions, 
including \psst{Whole}, \psst{Source}, and \psst{Theme}; thus of-genitives 
with such roles do not need to be construed as \psst{Gestalt} or \psst{Possessor}:
\begin{exe}
  \ex\rf{SocialRel}{Gestalt}:\begin{xlist}
    \ex the grandfather \p{of} Lord Voldemort
    \ex \choices{Lord Voldemort\p{'s}\\\p{his}} grandfather
  \end{xlist}
  \ex\begin{xlist}
    \ex the hood \p{of} the car (\psst{Whole})
    \ex the nose \p{of} He-Who-Must-Not-Be-Named (\psst{Whole})
    \ex \choices{the car\p{'s} hood\\\p{its}} (\rf{Whole}{Gestalt})
    \ex \choices{He-Who-Must-Not-Be-Named\p{'s} nose\\\p{his}} (\rf{Whole}{Gestalt})
  \end{xlist}
  \ex\begin{xlist}
    \ex the arrival \p{of} the queen (\psst{Theme})
    \ex \choices{the queen\p{'s} arrival\\\p{her}} (\rf{Theme}{Gestalt})
  \end{xlist}
  \ex \choices{Shakespeare\p{'s}\\\p{his}} works (\rf{Originator}{Gestalt})
  \ex These are children\p{'s} clothes.\footnote{Cannot readily be paraphrased with \p{their} because \w{children} is not referential, 
  but rather refers to a kind. This construction has been termed the \emph{descriptive genitive} \citep[pp.~322, 327--328]{quirk-85}.} [clothes intended for use and possession by children] (\rf{Beneficiary}{Possessor})
\end{exe}

The literature on the genitive alternation examines the factors that condition 
the choice of construction; important factors include the length and animacy of the possessed NP.
In addition, \p{of} participates in certain constructions that are not really possessives---%
e.g.~\pex{this sort \p{of} sweater} (\psst{Species}).

Certain idioms require an s-genitive argument that does not participate in 
any transparent semantic relationship; for these, \backposs is used (\cref{sec:possidiom}).

%\subsubsection{Construct State Genitive in Hebrew}

\subsection{Passives}\label{sec:passives}

The construction for passive voice (in verbs and nominalizations thereof) 
involves an optional \p{by}-PP;
the object of \p{by} alternates with the subject in active voice. 
While a variety of scene roles can be expressed with this phrase, 
the \emph{functions} associated with passive \p{by} are limited to 
\psst{Agent} and \psst{Causer}:
\begin{xexe}
  \ex the decisive vote \p{by} the City Council (\psst{Agent})
  \ex the devastation wreaked \p{by} the fire (\psst{Causer})
  \ex This story was told \p{by} my grandmother. (\rf{Originator}{Agent})
  \ex The news was not well received \p{by} the White House. (\rf{Recipient}{Agent})
  \ex Mr. Dursley is employed \p{by} Grunnings. (\rf{OrgRole}{Agent})
  \ex The window was broken \p{by} the hammer. (\rf{Instrument}{Causer})
  \ex scared \p{by} the bear (\rf{Stimulus}{Causer})
\end{xexe}

\subsection{Comparatives and Superlatives}

Various constructions express a comparison between two arguments. 

\paragraph{\psst{ComparisonRef} for second argument.}
When the second argument (the point of reference) 
is adpositionally marked, \psst{ComparisonRef} is used, regardless of 
its complement's syntactic type: 
\begin{exe}
  \ex\label{ex:comparisonrefArg}\begin{xlist} 
        \ex Your face is as red \p{as} \choices{a rose\\mine is}. (\psst{ComparisonRef})
        \ex Your face is redder \p{than} \choices{a rose\\mine is}. (\psst{ComparisonRef})
    \end{xlist}
\end{exe}
See further examples at \psst{ComparisonRef}.

\subsubsection{\p*{As}{as}-\p{as} comparative construction}\label{sec:as-as}

\paragraph{\psst{Extent} argument.} 
In an \p{as}-\p{as} comparison, the scene role of the first argument 
(the object of the first \p{as}) is the role that would be operative 
if the construction were removed and only the first argument remained: 
e.g., \pex{I stayed as long as I could} $\rightarrow$ \pex{I stayed long}.
The function of the first \p{as} is always \psst{Extent} 
to reflect that it marks the degree on a scale:
\begin{exe}
  \ex\begin{xlist}
    \ex I helped \p{as} much as I could. (\psst{Extent})
    \ex Your face is \p{as} red as a rose. (\rf{Characteristic}{Extent})
    \ex I helped \p{as} carefully as I could. (\rf{Manner}{Extent})
    \ex I stayed \p{as} long as I could. (\rf{Duration}{Extent})
    \ex I helped \p{as} often as I could. (\rf{Frequency}{Extent})
    \ex I've eaten (twice) \p{as} much (food) as you. [amount of something]\\ 
    (\rf{Approximator}{Extent})
  \end{xlist}
\end{exe}

\paragraph{Second argument: \psst{ComparisonRef}.} 
See \cref{ex:comparisonrefArg} above.

\subsubsection{Superlatives}\label{sec:superlative}

\psst{Whole} is used for the superset or gestalt licensed by a superlative:
\begin{exe}
  \ex the youngest \p{of} the children (\psst{Whole})
\end{exe}
See more at \psst{Whole}.


\subsection{Infinitive Clauses}\label{sec:inf}

In its function as infinitive marker, \p{to} is not generally considered to be a preposition. 
Nevertheless, we consider all uses of \p{to} for adposition supersense annotation 
because infinitive clauses (infinitivals) can express similar semantic relations 
as prepositional phrases. Most notably, infinitival purpose adjuncts alternate 
with \p{for}-PP purpose adjuncts:
\begin{exe}
  \ex\label{ex:infPurpose}\psst{Purpose}:\begin{xlist}
    \ex\begin{xlist}
      \ex Open the door \p{to} let in some air.
      \ex Open the door \p{for} some air.
    \end{xlist}
    \ex\begin{xlist}
      \ex I flew to headquarters \p{to} meet with the principals.
      \ex I flew to headquarters \p{for} a meeting with the principals.
    \end{xlist}
  \end{xlist}
\end{exe}
Thus, from a practical point of view, we might as well treat infinitival \p{to} 
as capable of marking a \psst{Purpose}.

The following is an exhaustive list of semantic analyses that we consider for infinitivals: 
\begin{itemize}
  \item \textbf{Purpose adjuncts}, generally adverbial, as in \cref{ex:infPurpose}. 
  These are labeled \psst{Purpose}. They can generally be paraphrased with \p{in\_order\_to}.
  
  \item \textbf{Inherent purposes}, generally adnominal, as in \cref{ex:charPurp}, 
  described under \psst{Purpose}. 
  These are labeled \rf{Characteristic}{Purpose}.
  
  \item In a \textbf{commercial scene}, that which costs money; labeled \rf{Theme}{Purpose}.
  Repeated from the discussion under \psst{Theme}:
  \begin{exe}
    \ex\begin{xlist}
      \ex They asked \$500 \p{to} make the repairs. (\rf{Theme}{Purpose})
      \ex \$500 \p{to} make the repairs was excessive. (\rf{Theme}{Purpose})
    \end{xlist}
  \end{exe}
  
  \item Constructions of \textbf{sufficiency and excess}---\pex{too short \p{to} ride}, 
  \pex{not tall enough \p{to} ride}, etc., where the assertion of sufficiency or excess 
  licenses an infinitival, labeled \rf{ComparisonRef}{Purpose}. 
  See discussion at \psst{ComparisonRef}.
\end{itemize}

Infinitival tokens not covered by this list are labeled \backi (\cref{sec:specialinf}).

\paragraph{Infinitival with \p{for}-subject.}
In \cref{ex:infPurpose}, the infinitive clause has no local subject---rather, 
an argument of the matrix clause doubles as the subject of the infinitive clause 
(control). However, a separate subject can be introduced with \p{for}, 
in which case \p{for}+NP is treated as a dependent of the infinitive verb 
and labeled \backi:
\begin{exe}
  \ex\begin{xlist}
    \ex I opened the door [\p{for}$_{\text{\backi}}$ Steve \p{to}$_{\psst{Purpose}}$ take out the trash].
    \ex It cost \$500 [\p{for}$_{\text{\backi}}$ the mechanic \p{to}$_{\text{\rf{Theme}{Purpose}}}$ make the repairs].
  \end{xlist}
\end{exe}

% \subsection{With Absolutes}
% 
% \nss{TODO}

\subsection{PP Idioms}

Many PPs exhibit some amount of lexicalization or idiomaticity.
This is especially true of PPs that tend to be used predicatively.
In general it is extremely difficult to establish tests to distinguish idiomatic PPs 
from fully productive combinations. 
However, the usual criteria apply for the supersense analysis.

For example, if the PP answers a \emph{Where?}\ question, 
it qualifies as \psst{Locus}; 
whereas qualitative states usually have \psst{Manner} as the scene role:
\begin{exe}
  \ex He is \p{out\_of} town. (\psst{Locus})
  \ex The company is \p{out\_of} business. (\rf{Manner}{Locus})
\end{exe}
See further discussion at \psst{Manner}.

\subsubsection{Reflexive PP Idioms}\label{sec:refl}

Certain idiomatic constructions involve a preposition that requires a reflexive 
direct object.

\paragraph{PERFORM-ACTIVITY \emph{for} oneself.}
\begin{itemize}
  \item When something is done for one's own benefit rather than someone else's:
    \begin{exe}
      \ex I took a vacation \p{for} myself (\psst{Beneficiary})
    \end{exe}
  \item When something is done in a way that affords direct rather than second-hand information:
    \begin{exe}
      \ex You should try out the restaurant \p{for} yourself! (\rf{Agent}{Beneficiary})
    \end{exe}
\end{itemize}
\paragraph{PERFORM-ACTIVITY \emph{by} oneself.}
\begin{itemize}
  \item When something is done without accompaniment (the negation would be \emph{\p{with} others}):
    \begin{exe}
      \ex I had lunch (all) \p{by} myself [`alone'] (\psst{Accompanier}\footnote{Though \emph{myself} 
      is not literally accompanying \emph{I}, the PP as a whole describes the nature of accompaniment (or lack thereof).}) 
    \end{exe}
  \item When something is accomplished without assistance:
    \begin{exe}
      \ex I made the decision (all) \p{by} myself. (\psst{Manner})
      \ex The computer rebooted all \p{by} itself. (\psst{Manner})
    \end{exe}
\end{itemize}
\paragraph{BE \emph{by} oneself.}
Alone; unaccompanied:
\begin{exe}
  \ex I am \p{by} myself right now. (\psst{Accompanier})
\end{exe}

\subsection{Ages}\label{sec:age}

An individual's age is a temporal property, licensing both \psst{Time} and \psst{Characteristic} prepositions:
\begin{exe}
  \ex\begin{xlist} 
    \ex a child \p{of} (age) 5 (years) (\psst{Characteristic})
    \ex Martha was already reading \choices{\p{at}/\p{by}/\p{before}} (the age of$_{\text{\psst{Identity}}}$) 5 (years). (\psst{Time})
  \end{xlist}
\end{exe}

\section{Special Labels}\label{sec:special}

For annotating data, there needs to be a way to indicate that \emph{none} 
of the adposition supersenses apply to a particular token. 

\subsection{DISCOURSE (\backd)}\label{sec:discourse}

Discourse connectives and other markers that transition between ideas 
or convey speaker attitude/hedging/emphasis/attribution but do not belong 
to propositional content. Examples include:

\begin{exe}
\ex \p{according\_to}; \w{\p{after}\_all}, \w{\p{of}\_course}, \w{\p{by}\_the\_way}; 
\w{\p{for}\_chrissake} (interjection); 
\w{\p{above}\_all}, \w{\p{to}\_boot}; % \w{more\_often\_than\_not}, <-- not prepositional as a whole
\w{\p{in}\_other\_words}, \w{\p{on}\_the\_other\_hand}; 
\w{\p{in} my experience}, \w{\p{in}\_my\_opinion}
\end{exe}

This label also covers ``additive focusing markers'' 
\citep[p.~592]{cgel} with a meaning similar to `also' or `too',
where an item is added to something already established in the discourse:
\begin{exe}
  \ex\begin{xlist}
    \ex I shot the sheriff \p{as}\_well.
    \ex They serve coffee, and tea \p{as}\_well.
  \end{xlist}
\end{exe}
It also covers topicalization markers:
\begin{exe}
  \ex \p*{As\_for}{as\_for} the sheriff, well, I shot 'im.
\end{exe}
Finally, \backd applies to adpositions relating a metalinguistic mention of 
a speech act to the speech content itself---whether the adposition 
introduces this speech act mention, as in \cref{ex:toSumItUp},
or links the discourse expression to a subordinate statement, as in \cref{ex:sumItUpWith}.
\begin{exe}
  \ex\begin{xlist}
    \ex\label{ex:toSumItUp} \p*{To}{to} sum it up: It was a terrible experience.
    \ex\label{ex:sumItUpWith} I will sum it up \p{with}: It was a terrible experience.
  \end{xlist}
\end{exe}

\subsection{COORDINATOR (\backc)}\label{sec:coord}

Coordinating conjunctions and similar expressions where 
the two elements in the relation are semantically on an equal footing, 
rather than in a figure/ground relationship: 
\begin{exe}
  \ex They serve coffee \p{as\_well\_as} tea. [`They serve coffee and also tea']
\end{exe}

\subsection{OTHER INFINITIVE (\backi)}\label{sec:specialinf}

As described in \cref{sec:inf}, infinitive clauses are analyzed with a supersense 
if and only if they serve as a purpose adjunct, or in certain purpose-related constructions 
(inherent purpose, action that costs money in a commercial scene, 
that which something is sufficient or excessive for).
The special label \backi is reserved for all other uses of infinitival \p{to}, 
as well as \p{for} whenever it introduces the subject of an infinitive clause.\footnote{Essentially, 
our position is that these uses of infinitivals are more like syntactically core elements 
(subject, object) than obliques, and thus should be excluded from semantic annotation 
under the present scheme.}

Infinitivals warranting \backi include:
\begin{exe}\ex\begin{xlist}
  \ex I want \p{to} meet you. [complement of control verb]
  \ex I would\_like \p{to} try the fish. [\pex{would\_like} is a polite alternative to \pex{want}]
  \ex It seems \p{to} be broken. [complement of raising verb]
  \ex You have an opportunity \p{to} succeed. [complement of noun]
  \ex I'm ready \p{to} leave. [complement of adjective]
  \ex I'm glad \p{to} hear you're engaged! [complement of emotion adjective]
  \ex You're great/a pleasure \p{to} work with. [complement of evaluative adjective or noun]
  \ex These new keys are expensive \p{to} copy. [tough-movement]
  \ex My plan is \p{to} eat at noon. [infinitival as NP]
  \ex It's impossible \p{to} get an appointment. [infinitival as NP, with cleft]
  \ex I know how \p{to} lead. [complement of wh-word]
  \ex I have nothing \p{to} hide. [complement of indefinite pronoun]
  \ex Do you have time \p{to} help me? [with resource, not necessity]
  \ex They took\_the\_time \p{to} listen to my concerns. [complement of verbal idiom]
\end{xlist}\end{exe}

Multiword auxiliaries---such as quasi-modals \pex{have\_to} `must', \pex{ought\_to} `should', etc., 
as well as \pex{have\_yet\_to}---subsume the infinitival \p{to}, so no label on \p{to} is required:
\begin{exe}
  \ex You have\_to choose a date.
\end{exe}

Whenever \p{for} introduces a subject of an infinitival clause, the \p{for} token is labeled 
\backi (regardless of whether \p{to} receives a semantic label; see \cref{sec:inf}):
\begin{exe}\ex\begin{xlist}
  \ex I need [\p{for}$_{\text{\backi}}$ you \p{to}$_{\text{\backi}}$ help me].
  \ex I opened the door [\p{for}$_{\text{\backi}}$ Steve \p{to}$_{\psst{Purpose}}$ take out the trash].
\end{xlist}\end{exe}

\subsection{OPAQUE POSSESSIVE SLOT IN IDIOM (\backposs)}\label{sec:possidiom}

Semantic supersenses are used where possible for genitive\slash possessive 
constructions, as discussed in \cref{sec:genitives}. 
However, there are a few idioms which require a possessive pronoun 
that does not participate transparently in any semantic relation; 
these are designated with the special label \backposs:
\begin{exe}\ex\begin{xlist}
  \ex I am eating on\_~~\p{my}~~\_own today.
  \ex She tried \p{her} best.
  \ex He's not \p{your} average baseball player.
  \ex Billy knows \p{his} ABCs!
\end{xlist}\end{exe}

\bibliographystyle{plainnat}
\bibliography{psst2.bib}

%\printbibliography[maxnames=99]

\printindex
\printindex[construals]
\printindex[revconstruals]

\end{document}
