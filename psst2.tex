\PassOptionsToPackage{usenames}{color}
\documentclass[11pt,letterpaper]{article}
\usepackage{comment}
\usepackage{relsize} % relative font sizes (e.g. \smaller). must precede ACL style
%\usepackage{style/acl2012}

\usepackage[linkcolor=blue]{hyperref}

\usepackage[round]{natbib}
\begin{comment}
\usepackage[style=authoryear-comp,natbib=true,hyperref=true]{biblatex}

% tell biblatex not to quote titles in the bibliography
\DeclareFieldFormat{title}{#1} % don't italicize titles by default
\DeclareFieldFormat[book]{title}{\mkbibemph{#1}\isdot} % but do italicize books
\DeclareFieldFormat[article]{title}{#1\isdot}
\DeclareFieldFormat[inbook]{title}{#1\isdot}
\DeclareFieldFormat[incollection]{title}{#1\isdot}
\DeclareFieldFormat[inproceedings]{title}{#1\isdot}
\DeclareFieldFormat[patent]{title}{#1\isdot}
\DeclareFieldFormat[thesis]{title}{#1\isdot}
\DeclareFieldFormat[unpublished]{title}{#1\isdot}
% ...and for articles, to use number(issue) instead of number.issue
\renewbibmacro*{journal+issuetitle}{%
  \usebibmacro{journal}%
  \setunit*{\addspace}%
  \iffieldundef{series}
    {}
    {\newunit
     \printfield{series}%
     \setunit{\addspace}}%
  \printfield{volume}%
%  \setunit*{\adddot}%
  \printfield{number}%
  \setunit{\addcomma\space}%
  \printfield{eid}%
  \setunit{\addspace}%
  \usebibmacro{issue+date}%
  \newunit\newblock
  \usebibmacro{issue}%
  \newunit}
\DeclareFieldFormat[article]{number}{\mkbibparens{#1}}
% use ``pages'' instead of ``pp.''
\DefineBibliographyStrings{english}{%
    pages  =  {pages} % for multiple page numbers
}
% but for articles, just use a colon
\DeclareFieldFormat[article]{pages}{:#1} %TODO. check with LeCun citation
% and don't put a colon after In
\renewbibmacro*{in:}{%
  \bibstring{in}%\addcolon
  \setunit{\space}}

\DeclareFieldFormat{label}{#1\isdot}
\renewbibmacro*{year+labelyear}{%
  \iffieldundef{year}
    {}
    {\printtext{%
       \printfield{year}%
       \printfield{labelyear}%
       \setunit{\adddot}}}}

\bibliography{features.bib}
\end{comment}


%\usepackage{times}
%\usepackage{latexsym}


\usepackage[boxed]{algorithm2e}
\renewcommand\AlCapFnt{\small}
\usepackage[small,bf,skip=5pt]{caption}
\usepackage{sidecap} % side captions
\usepackage{rotating}	% sideways

% Italicize subparagraph headings
\usepackage{titlesec}
\titleformat*{\subparagraph}{\itshape}
\titlespacing{\subparagraph}{%
  1em}{%              left margin
  0pt}{% space before (vertical)
  1em}%               space after (horizontal)

%\usepackage{lingmacros}
% Lists

\usepackage{enumitem} % customizable lists
\setitemize{noitemsep,topsep=0em,leftmargin=*}
\setenumerate{noitemsep,leftmargin=0em,itemindent=13pt,topsep=0em}

\usepackage{adjustbox}
\newcommand{\choices}[1]{\adjustbox{stack=ct}{#1}}  % for placing alternatives 
% inline with an example. e.g. \choices{to\\from\\with}
% ct = horizontally-centered, top

\usepackage{textcomp}
% \usepackage{arabtex} % must go after xparse, if xparse is used!
%\usepackage{utf8}
% \setcode{utf8} % use UTF-8 Arabic
% \newcommand{\Ar}[1]{\RL{\novocalize #1}} % Arabic text

\usepackage[procnames]{listings}

\usepackage{amssymb}	%amsfonts,eucal,amsbsy,amsthm,amsopn
\usepackage{amsmath}

%\usepackage{mathptmx}	% txfonts
\usepackage{fourier}
\usepackage[scaled=.87]{helvet}
\usepackage[scaled=.8]{beramono}
\usepackage[T1]{fontenc}
\usepackage[utf8x]{inputenc}

\usepackage{MnSymbol}	% must be after mathptmx

\usepackage{latexsym}





% Tables
\usepackage{array}
\usepackage{multirow}
\usepackage{booktabs} % pretty tables
\usepackage{multicol}
\usepackage{footnote}


\usepackage{url}
\usepackage[usenames]{color}
\usepackage{xcolor}

% colored frame box
\newcommand{\cfbox}[2]{%
    \colorlet{currentcolor}{.}%
    {\color{#1}%
    \fbox{\color{currentcolor}#2}}%
}

\usepackage[normalem]{ulem} % \uline
\usepackage{colortbl}
\usepackage{graphicx}
\usepackage{subcaption}
\usepackage{mdframed}

%\usepackage{tikz-dependency}
\usepackage{tikz}
\usepackage[edges]{forest}
%\usepackage{tree-dvips}
\usetikzlibrary{arrows,positioning,calc} 

\DeclareMathOperator*{\argmax}{arg\,max}
\DeclareMathOperator*{\argmin}{arg\,min}



% Author comments
\usepackage{color}
\newcommand\bmmax{0} % magic to avoid 'too many math alphabets' error
\usepackage{bm}
\definecolor{orange}{rgb}{1,0.5,0}
\definecolor{mdgreen}{rgb}{0,0.6,0}
\definecolor{mdblue}{rgb}{0,0,0.7}
\definecolor{dkblue}{rgb}{0,0,0.5}
\definecolor{dkgray}{rgb}{0.3,0.3,0.3}
\definecolor{slate}{rgb}{0.25,0.25,0.4}
\definecolor{gray}{rgb}{0.5,0.5,0.5}
\definecolor{ltgray}{rgb}{0.7,0.7,0.7}
\definecolor{purple}{rgb}{0.7,0,1.0}
\definecolor{lavender}{rgb}{0.65,0.55,1.0}

% Settings for algorithm listings
\makeatletter
\lst@AddToHook{EveryPar}{%
  \label{lst:\thelstnumber}% make a label for each line number except the first (assumes only one listing in the document)
}
\makeatother
\lstset{
% basicstyle=\rmshape,
  numbers=left,
  numberstyle=\tt\color{gray},
  firstnumber=2,
  stepnumber=5,
  xleftmargin=3em,
  language=Python,
  upquote=true,
  showstringspaces=false,
  formfeed=\newpage,
  tabsize=1,
  stringstyle=\color{mdgreen},
  commentstyle=\itshape\color{lavender},
  basicstyle=\small\smaller\ttfamily,
  morekeywords={lambda,with,as,assert},
  keywordstyle=\bfseries\color{magenta},
  procnamekeys={def},
  procnamestyle=\bfseries\color{orange},
  aboveskip=0.5cm,
  belowskip=0.5cm
}
\renewcommand{\lstlistingname}{Algorithm}


\newcommand{\ensuretext}[1]{#1}
\newcommand{\cjdmarker}{\ensuretext{\textcolor{green}{\ensuremath{^{\textsc{CJ}}_{\textsc{D}}}}}}
\newcommand{\nssmarker}{\ensuretext{\textcolor{magenta}{\ensuremath{^{\textsc{NS}}_{\textsc{S}}}}}}
\newcommand{\nasmarker}{\ensuretext{\textcolor{red}{\ensuremath{^{\textsc{NA}}_{\textsc{S}}}}}}
\newcommand{\bomarker}{\ensuretext{\textcolor{blue}{\ensuremath{^{\textsc{B}}_{\textsc{O}}}}}}
\newcommand{\jbmarker}{\ensuretext{\textcolor{orange}{\ensuremath{^{\textsc{J}}_{\textsc{B}}}}}}
\newcommand{\dbmarker}{\ensuretext{\textcolor{purple}{\ensuremath{^{\textsc{D}}_{\textsc{B}}}}}}
\newcommand{\arkcomment}[3]{\ensuretext{\textcolor{#3}{[#1 #2]}}}
%\newcommand{\arkcomment}[3]{}
\newcommand{\cjd}[1]{\arkcomment{\cjdmarker}{#1}{green}}
\newcommand{\nss}[1]{\arkcomment{\nssmarker}{#1}{magenta}}
\newcommand{\nas}[1]{\arkcomment{\nasmarker}{#1}{red}}
\newcommand{\bo}[1]{\arkcomment{\bomarker}{#1}{blue}}
\newcommand{\jb}[1]{\arkcomment{\jbmarker}{#1}{orange}}
\newcommand{\db}[1]{\arkcomment{\dbmarker}{#1}{purple}}
\newcommand{\params}{\mathbf{\theta}}
\newcommand{\wts}{\mathbf{w}}
\newcommand{\g}{\mathbf{g}}
\newcommand{\f}{\mathbf{f}}
\newcommand{\x}{\mathbf{x}}
\newcommand{\y}{\mathbf{y}}
\newcommand{\overbar}[1]{\mkern 1.5mu\overline{\mkern-1.5mu#1\mkern-1.5mu}\mkern 1.5mu} % \bar is too narrow in math
\newcommand{\cost}{c}

\newcommand{\citeposs}[2][]{\citeauthor{#2}'s (\citeyear[#1]{#2})}
\newcommand{\Citeposs}[2][]{\Citeauthor{#2}'s (\citeyear[#1]{#2})}

\usepackage{nameref}
\usepackage{cleveref}

% use \S for all references to all kinds of sections, and \P to paragraphs
% (sadly, we cannot use the simpler \crefname{} macro because it would insert a space after the symbol)
\crefformat{part}{\S#2#1#3}
\crefformat{chapter}{\S#2#1#3}
\crefformat{section}{\S#2#1#3}
\crefformat{subsection}{\S#2#1#3}
\crefformat{subsubsection}{\S#2#1#3}
\crefformat{paragraph}{\P#2#1#3}
\crefformat{subparagraph}{\P#2#1#3}
\crefmultiformat{part}{\S\S#2#1#3}{and~#2#1#3}{, #2#1#3}{, and~#2#1#3}
\crefmultiformat{chapter}{\S\S#2#1#3}{and~#2#1#3}{, #2#1#3}{, and~#2#1#3}
\crefmultiformat{section}{\S\S#2#1#3}{and~#2#1#3}{, #2#1#3}{, and~#2#1#3}
\crefmultiformat{subsection}{\S\S#2#1#3}{and~#2#1#3}{, #2#1#3}{, and~#2#1#3}
\crefmultiformat{subsubsection}{\S\S#2#1#3}{and~#2#1#3}{, #2#1#3}{, and~#2#1#3}
\crefmultiformat{paragraph}{\P\P#2#1#3}{and~#2#1#3}{, #2#1#3}{, and~#2#1#3}
\crefmultiformat{subparagraph}{\P\P#2#1#3}{and~#2#1#3}{, #2#1#3}{, and~#2#1#3}
% for \label[appsec]{...}
\crefname{appsec}{appendix}{appendices}
\Crefname{appsec}{Appendix}{Appendices}

\newcommand{\creflastconjunction}{, and\nobreakspace} % Oxford comma for lists

\newcommand*{\Fullref}[1]{\hyperref[{#1}]{\Cref*{#1}: \nameref*{#1}}}
\newcommand*{\fullref}[1]{\hyperref[{#1}]{\cref*{#1}: \nameref{#1}}}
% \newcommand{\fref}[1]{figure~\ref{#1}}
% \newcommand{\ffref}[2]{figures~\ref{#1} and~\ref{#2}}
% \newcommand{\Fref}[1]{Figure~\ref{#1}}
% \newcommand{\FFref}[2]{Figures~\ref{#1} and~\ref{#2}}
% \newcommand{\tref}[1]{table~\ref{#1}}
% \newcommand{\ttref}[2]{tables~\ref{#1} and~\ref{#2}}
% \newcommand{\Tref}[1]{Table~\ref{#1}}
% \newcommand{\aref}[1]{algorithm~\ref{#1}}
% \newcommand{\Aref}[1]{Algorithm~\ref{#1}}
\newcommand{\fnref}[1]{\autoref{#1}} % don't use \cref{} due to bug in (now out-of-date) cleveref package w.r.t. footnotes
%\newcommand{\eref}[1]{\eqref{#1}}

% Space savers
% From http://www.eng.cam.ac.uk/help/tpl/textprocessing/squeeze.html
%\addtolength{\dbltextfloatsep}{-.5cm} % space between last top float or first bottom float and the text.
%\addtolength{\intextsep}{-.5cm} % space left on top and bottom of an in-text float.
%\addtolength{\abovedisplayskip}{-.5cm} % space before maths
%\addtolength{\belowdisplayskip}{-.5cm} % space after maths
%\addtolength{\topsep}{-.5cm} %space between first item and preceding paragraph
%\setlength{\belowcaptionskip}{-.25cm}

\usepackage{gb4e} % linguistic examples. put after all other package imports


% customize \paragraph spacing
\makeatletter
\renewcommand{\paragraph}{%
  \@startsection{paragraph}{4}%
  {\z@}{.2ex \@plus 1ex \@minus .2ex}{-1em}%
  {\normalfont\normalsize\bfseries}%
}
\makeatother



% Special macros
\newcommand{\w}[1]{\textit{#1}}	% word
\newcommand{\p}[1]{\textbf{\textsf{#1}}} % preposition type
\newcommand{\lbl}[1]{\textsc{#1}} % class label
\newcommand{\sst}[1]{\lbl{#1}} % supersense tag label
\newcommand{\nsst}[1]{\sst{n:#1}} % noun supersense tag label
\newcommand{\vsst}[1]{\sst{v:#1}} % verb supersense tag label
\newcommand{\psst}[1]{\textcolor{mdgreen}{\hyperref[sec:#1]{\sst{#1}}}} % preposition supersense tag label
\newcommand{\olbl}[1]{\textcolor{purple}{\textrm{#1}}} % other label: `i, `d, etc.
%\newcommand{\nsst}[1]{\sst{#1~\textroundcap{\vphantom{-}}~}} % noun supersense tag label
%\newcommand{\vsst}[1]{\sst{#1\raisebox{-1.5pt}{\textasciicaron}}} % verb supersense tag label
%\newcommand{\psst}[1]{\sst{#1\raisebox{2pt}{\rotatebox{180}{\textsublhalfring{\phantom{.}}}}}} %\textcorner % preposition supersense tag label

\newcommand{\rf}[2]{\psst{#1}$\leadsto$\psst{#2}}
\newcommand{\rff}[3]{\psst{#1}$\leadsto$\psst{#2}$\leadsto$\psst{#3}}


\newcommand{\tg}[1]{\texttt{#1}}	% supersense tag name
\newcommand{\gfl}[1]{%\renewcommand\texttildelow{{\lower.74ex\hbox{\texttt{\char`\~}}}} % http://latex.knobs-dials.com/
\mbox{\textsmaller{\texttt{#1}}}}	% supersense tag symbol
 \newcommand{\tagdef}[1]{#1\hfill} % tag definition
\newcommand{\tagt}[2]{\ensuremath{\underset{\textrm{\textlarger{\tg{#2}}}\strut}{\w{#1}\rule[-.3\baselineskip]{0pt}{0pt}}}} % tag text (a word or phrase) with an SST. (second arg is the tag)
\newcommand{\glosst}[2]{\ensuremath{\underset{\textrm{#2}}{\textrm{#1}}}} % gloss text (a word or phrase) (second arg is the gloss)
\newcommand{\AnnA}[0]{\mbox{\textbf{Ann-A}}} % annotator A
\newcommand{\AnnB}[0]{\mbox{\textbf{Ann-B}}} % annotator B
\newcommand{\sys}[1]{\mbox{\textbf{#1}}}   % name of a system (one of our experimental conditions)
\newcommand{\dataset}[1]{\mbox{\textsc{#1}}}	% one of the datasets in our experiments
\newcommand{\datasplit}[1]{\mbox{\textbf{#1}}}	% portion one of the datasets in our experiments

\newcommand{\fnf}[1]{\textsc{\textsf{#1}}} % FrameNet frame
\newcommand{\fnr}[1]{\textbf{\textsf{#1}}} % FrameNet role (frame element name)
\newcommand{\fnrel}[1]{\textsl{#1}} % FrameNet frame relation type
\newcommand{\fnst}[1]{\textsl{#1}} % FrameNet semantic type
\newcommand{\fnlu}[1]{\textsf{#1}} % FrameNet lexical unit (predicate)
\newcommand{\pbf}[1]{\mbox{\textsf{#1}}} % PropBank frame (roleset)
\newcommand{\pbr}[1]{\textbf{\textsf{#1}}} % PropBank role (numbered or modifier argument label)
\newcommand{\vpred}[1]{\textbf{#1}} % verb predicate


%\newcommand{\lex}[1]{\textsmaller{\textsf{\textcolor{slate}{\textbf{#1}}}}}	% example lexical item
\newcommand{\lex}[1]{\textit{#1}} % lexical item/lexical example
\newcommand{\pex}[1]{\textit{#1}} % phrasal example - don't index by default

%\newcommand{\w}[1]{\textit{#1}}	% word
\newcommand{\gap}[0]{\ \ } % space around gap contents
\newcommand{\tat}[0]{\textasciitilde}

\newcommand{\shortlong}[2]{#1} % short vs. long version of the paper
\newcommand{\confversion}[1]{#1}
%\newcommand{\finalversion}[1]{#1}
\newcommand{\finalversion}[1]{}
\newcommand{\futureversion}[1]{}
\newcommand{\shortversion}[1]{#1}
\newcommand{\considercutting}[1]{#1}
\newcommand{\longversion}[1]{} % ...if only there were more space...
\newcommand{\subversion}[1]{#1} % for the submission version only
\newcommand{\draftnotice}[1]{#1} % for the draft version only
%\newcommand{\subversion}[1]{}

\newenvironment{ggroup}{{}}{{}}

\newcommand{\shortdef}[1]{\begin{mdframed}\noindent\textlarger{#1}\end{mdframed}}

\newenvironment{history}{\color{gray}\begin{mdframed}\small\noindent\textit{History.}}{\end{mdframed}}



\newcommand{\hierA}[1]{\textcolor{red}{\hyperref[sec:#1]{#1}}}
\newcommand{\hierB}[1]{\textcolor{blue}{\hyperref[sec:#1]{#1}}}
\newcommand{\hierC}[1]{\textcolor{mdgreen}{\hyperref[sec:#1]{#1}}}
\newcommand{\hierD}[1]{\textcolor{orange}{\hyperref[sec:#1]{#1}}}

\newcommand{\hierAdef}[1]{\section{\psst{#1}}\label{sec:#1}}
\newcommand{\hierBdef}[1]{\subsection{\psst{#1}}\label{sec:#1}}
\newcommand{\hierCdef}[1]{\subsubsection{\psst{#1}}\label{sec:#1}}
\newcommand{\hierDdef}[1]{\paragraph{\psst{#1}}\label{sec:#1}}

\hyphenation{WordNet}
\hyphenation{WordNets}
\hyphenation{FrameNet}
\hyphenation{SemCor}
\hyphenation{SemEval}
\hyphenation{ParsedSemCor}
\hyphenation{VerbNet}
\hyphenation{PennConverter}
\hyphenation{an-aly-sis}
\hyphenation{an-aly-ses}
\hyphenation{news-text}
\hyphenation{base-line}
\hyphenation{de-ve-lop-ed}
\hyphenation{comb-over}
\hyphenation{per-cept}
\hyphenation{per-cepts}
\hyphenation{post-edit-ing}
\hyphenation{shriv-eled}
\hyphenation{Huddle-ston}

\title{\draftnotice{{\it\small WORKING DRAFT}\\[5pt] 
Adposition Supersenses v2}}

\author{Nathan Schneider\\ \textsmaller{\texttt{\href{mailto:nschneid@cs.cmu.edu}{nathan.schneider@georgetown.edu}}}}

\date{}

\begin{document}
\maketitle
\begin{abstract}
\noindent 
This document describes an inventory of 50~semantic labels 
designed to characterize the use of adpositions and case markers 
at a somewhat coarse level of granularity. 
Version~2 is a revision of the supersense inventory proposed for English by 
\citet{schneider-15,schneider-16} (henceforth ``v1''), which in turn was based on previous schemes.
The present inventory was developed after extensive review of the 
v1 corpus annotations for English, as well as consideration of adposition 
and case phenomena in Hebrew, Hindi, and Korean.
Examples in this document are limited to English; 
an online multilingual lexical resource is forthcoming.
\end{abstract}

\section{Overview}

v1 is documented in PrepWiki (\url{http://tiny.cc/prepwiki}).

v2 will be documented in Xposition (URL TBD).

\subsection{What counts as an adposition?}

``Adposition'' is the cover term for prepositions and postpositions. 
Briefly, we consider a word or multiword expression to be an adposition if it:
\begin{itemize}
  \item Mediates a semantically asymmetric figure--ground relation between two concepts
  \item Is a grammatical item that can mark an NP, and in some cases may mark clauses (as a subordinator) 
  or be intransitive. 
  We also include always-intransitive grammatical items whose core meaning is spatial and highly schematic, 
  like English \p{together}, \p{apart}, \p{away}, and \p{forward}.
  \item Is not a differential object marker (e.g., Hebrew \p{`et}, which marks direct objects if and only if 
  they are definite).
\end{itemize}
\nss{What about a word that matches the above criteria where it is used as an intransitive 
predicate, e.g. \pex{She is \p{out}/\p{away}}?}

\subsection{Inventory}

The v2 hierarchy is a tree with 50~labels.
They are organized into three major subhierarchies: 
\psst{Circumstance} (18~labels), \psst{Participant} (14~labels), 
and \psst{Configuration} (18~labels). 

\begin{multicols}{3}
\begin{ggroup}
  \sffamily\color{gray}
\begin{forest}
  for tree={%
    folder,
    grow'=0,
    fit=band,
    inner ysep=.75,
  }
  [{\hierA{Circumstance}}
    [{\hierB{Temporal}}
      [{\hierC{Time}}
        [{\hierD{StartTime}}]
        [{\hierD{EndTime}}]
        [{\hierD{DeicticTime}}]
      ]
      [{\hierC{Frequency}}]
      [{\hierC{Duration}}]
    ]
    [{\hierB{Locus}}
      [{\hierC{Source}}]
      [{\hierC{Goal}}]
    ]
    [{\hierB{Path}}
      [{\hierC{Direction}}]
      [{\hierC{Extent}}]
    ]
    [{\hierB{Means}}]
    [{\hierB{Manner}}]
    [{\hierB{Explanation}}
      [{\hierC{Purpose}}]
    ]
  ]
\end{forest}
\columnbreak

\begin{forest}
  for tree={%
    folder,
    grow'=0,
    fit=band,
    inner ysep=.75,
  }
  [{\hierA{Participant}}
    [{\hierB{Causer}}
      [{\hierC{Agent}}
        [{\hierD{Co-Agent}}]
      ]
    ]
    [{\hierB{Theme}}
      [{\hierC{Co-Theme}}]
      [{\hierC{Topic}}]
    ]
    [{\hierB{Stimulus}}]
    [{\hierB{Experiencer}}]
    [{\hierB{Originator}}]
    [{\hierB{Recipient}}]
    [{\hierB{Cost}}]
    [{\hierB{Beneficiary}}]
    [{\hierB{Instrument}}]
  ]
\end{forest}
\columnbreak

\begin{forest}
  for tree={%
    folder,
    grow'=0,
    fit=band,
    inner ysep=.75,
  }
  [{\hierA{Configuration}}
    [{\hierB{Identity}}]
    [{\hierB{Species}}]
    [{\hierB{Gestalt}}
      [{\hierC{Possessor}}]
      [{\hierC{Whole}}]
    ]
    [{\hierB{Characteristic}}
      [{\hierC{Possession}}]
      [{\hierC{Part/Portion}}
        [{\hierD{Stuff}}]
      ]
    ]
    [{\hierB{Accompanier}}]
    [{\hierB{InsteadOf}}]
    [{\hierB{ComparisonRef}}]
    [{\hierB{RateUnit}}]
    [{\hierB{Quantity}}
      [{\hierC{Approximator}}]
    ]
    [{\hierB{SocialRel}}
      [{\hierC{OrgRole}}]
    ]
  ]
\end{forest}
\end{ggroup}
\end{multicols}

\begin{itemize}
\item Items in the \psst{Circumstance} subhierarchy are prototypically 
expressed as adjuncts of time, place, manner, purpose, etc.\ 
elaborating an event or entity.
\item Items in the \psst{Participant} subhierarchy are prototypically 
entities functioning as arguments to an event.
\item Items in the \psst{Configuration} subhierarchy are prototypically
entities or properties in a static relationship to some entity.
\end{itemize}

\subsection{Major changes from v1}

Changes that affect only a single label are explained below the relevant 
v2 labels.

\begin{itemize}
  \item \textbf{Removed multiple inheritance.}
  \item \textbf{Revised and expanded the \psst{Configuration} subhierarchy.}
  \item \textbf{Removed the locative concreteness distinction.}
  In v1, labels \sst{Location}, \sst{InitialLocation}, and \sst{Destination} 
  were reserved for concrete locations, and the respective supertypes 
  \sst{Locus}, \sst{Source}, and \sst{Goal} used to cover abstract locations.
  This distinction was found to be difficult and did not seem to be 
  relevant to grammatical constructions. The concrete labels were thus removed.
  \item \textbf{Removed the location/state/value distinction.}
  The v1 scheme attempted to make an elaborate distinction between 
  values, states, and other kinds of abstract locations. 
  However, the English preposition system does not seem particularly 
  sensitive to these distinctions. (We are not aware of any prepositions 
  that mark primarily values or primarily states; rather, productive 
  metaphors allow locative prepositions to be extended to cover these, 
  and there are cases where teasing apart abstract location vs.~state vs.~value 
  is difficult.) Therefore, \sst{State}, \sst{InitialState}, \sst{EndState}, 
  \sst{Value}, and \sst{ValueComparison} were removed.
  \item \textbf{Revised the treatment of comparison and related notions.} 
  \sst{Comparison/Contrast}, \sst{Scalar/Rank}, \sst{ValueComparison}; 
  moved \psst{Approximator} under \psst{Quantity}
  \item \textbf{Greatly simplified the \psst{Path} subhierarchy.} See \cref{sec:Path}.
  \item \textbf{Simplified the \psst{Temporal} subhierarchy.} See \cref{sec:Temporal}.
\end{itemize}

\hierAdef{Circumstance}

\hierBdef{Temporal}

\hierCdef{Time}

\hierDdef{StartTime}

\hierDdef{EndTime}

\hierDdef{DeicticTime}

\hierCdef{Frequency}

\hierCdef{Duration}

\hierBdef{Locus}

\hierCdef{Source}

\shortdef{Initial location, condition, or value. May be abstract.}

Prototypically inanimate, though it can be used to construe animate \psst{Participant}s 
(especially \psst{Originator}).
Contrasts with \psst{Goal}.

\hierCdef{Goal}

\shortdef{Final location, condition, or value. May be abstract.}

Prototypically inanimate, though it can be used to construe animate \psst{Participant}s 
(especially \psst{Recipient}).
Constrats with \psst{Source}.

\hierBdef{Path}

\hierCdef{Direction}

\hierCdef{Extent}

\hierBdef{Means}

\hierBdef{Manner}

\hierBdef{Explanation}

\hierCdef{Purpose}




\hierAdef{Participant}

\hierBdef{Causer}

\hierCdef{Agent}

\hierDdef{Co-Agent}

\hierBdef{Theme}

\hierCdef{Co-Theme}

\hierCdef{Topic}

\hierBdef{Stimulus}%
%
\shortdef{That which is perceived or experienced (bodily, perceptually, or emotionally).}

\psst{Stimulus} does not seem to have any prototypical adposition 
in the languages we have looked at. In English, it can be construed in several ways:
\begin{exe}
  \ex My affection \p{for} you (\rf{Stimulus}{Beneficiary})
  \ex Scared \p{by} the bear (\rf{Stimulus}{Causer})
  \ex I startled \p{at} the noise (\rf{Stimulus}{Goal})
  \ex I care \p{about} you (\rf{Stimulus}{Topic})
\end{exe}

Counterpart: \psst{Experiencer}

\hierBdef{Experiencer}

\shortdef{Animate who is aware of a bodily experience, perception, emotion, or mental state.}

\psst{Experiencer} does not seem to have any prototypical adposition 
in the languages we have looked at. In English, it can be construed in several ways:
\begin{exe}
  \ex The anger \p{of} the students (\rf{Experiencer}{Possessor})
  \ex Running is enjoyable \p{for} me (\rf{Experiencer}{Beneficiary})
  \ex It feels hot \p{to} me (\rf{Experiencer}{Recipient})
\end{exe}

Elsewhere, the term \emph{cognizer} is sometimes used for one whose 
mental state is described.

Counterpart: \psst{Stimulus}

\hierBdef{Originator}

\shortdef{Animate who is the initial possessor or creator/producer of something,
including the speaker/communicator of information.}
A ``source'' in the broadest sense of a starting point/condition. 
Contrasts with \psst{Recipient}.

Typically construed as \psst{Agent} (with \pex{give}, \pex{tell}, \pex{create}: 
subject or passive-\p{by}; adnominal \p{by} as in \pex{works \p{by} Shakespeare}) 
or \psst{Source} (\pex{obtain/hear \p{from}}; adnominal \p{of} as in \pex{works \p{of} Shakespeare}). 
Occasionally construed as \psst{Theme} (\pex{rob \uline{her} of her life savings}: direct object).

\begin{history}
  \psst{Originator} merges v1 labels \sst{Donor/Speaker} and \sst{Creator}, 
  which were difficult to distinguish in the case of authorship.
  %(e.g., \pex{the operas \p{of} Puccini}).
\end{history}

\hierBdef{Recipient}

\shortdef{Animate who is the (actual or intended) final possessor of a thing or message.}
A ``goal'' in the broadest sense of an ending point/condition. 
Contrasts with \psst{Originator}.

Typically construed as \psst{Goal} (\pex{give/talk \p{to}}), 
\psst{Agent} (with \pex{receive}: subject or passive-\p{by}), 
or \psst{Theme} (with \pex{inform}: direct object).


\hierBdef{Cost}

\shortdef{An amount (typically of money) that is linked to an item or service 
that it pays for\slash could pay for, or given as the amount earned or owed.} 

The governor may be an explicit commercial scenario:
\begin{exe}
  %\ex I paid/owed John \$10 for the book. %#nonprep
  \ex I \choices{bought\\sold} the book \p{for} \$10.
  \ex The book is \choices{priced\\valued} \p{at} \$10.
  \ex I got a refund \p{of} \$10.
\end{exe}
Or the \psst{Cost} may be specified as an adjunct with a non-commerical governor:
\begin{exe}
  \ex You can ride the bus \p{for} \choices{free\\\$1}.
\end{exe}
\psst{Cost} is \emph{not} used with general scenes of possession or transfer, 
even if the thing possessed or transferred happens to be an amount of money:
\begin{exe}
  \ex I bestowed the winner \p{with}$_{\text{\psst{Co-Theme}}}$ \$100.
\end{exe}

\begin{history}
  This category was not present in v1, which had the broader category \sst{Value}. 
  VerbNet has a similar category called \sst{Asset}; we chose the name 
  \psst{Cost} to emphasize that it describes a relation rather than an entity type 
  (it does not apply to money with a verb like \pex{possess} or \pex{transfer}, 
  for instance).
\end{history}

\hierBdef{Beneficiary}

\shortdef{Animate or personified undergoer that is (potentially) 
advantaged or disadvantaged by the event or state.}

This label does not distinguish the polarity of the relation 
(helping or hurting, which is sometimes termed \emph{maleficiary}).

\begin{exe}
  \ex Vote \choices{\p{for}\\\p{against}} Pedro!
  \ex Junk food is bad \p{for} your health.
  \ex My parrot died \p{on} me.
\end{exe}

\hierBdef{Instrument}

\shortdef{An entity that facilitates an action by applying intermediate causal force.}

Prototypically, an \psst{Agent} manipulates the \psst{Instrument} with the purpose of achieving a result.
This includes a device serving as a mode of transportation or medium of communication 
(often construed as a \psst{Locus} or \psst{Path}).
Less prototypically, the action could be unintentional 
(\pex{I accidentally poked myself in the eye \p{with} a stick}). 
The key is that the \psst{Instrument} is not sufficiently ``independently causal'' to instigate the event.
Other non-prototypical instruments include waypoints from \psst{Source} to \psst{Goal}, 
and people that relay information from \psst{Originator} to \psst{Recipient}.

Compare \psst{Means}, which is used for facilitative events rather than entities.

\hierAdef{Configuration}

\hierBdef{Identity}

\hierBdef{Species}

\hierBdef{Gestalt}

\hierCdef{Possessor}

\hierCdef{Whole}

\hierBdef{Characteristic}

\hierCdef{Possession}

\hierCdef{Part/Portion}

\hierDdef{Stuff}

\hierBdef{Accompanier}

\hierBdef{InsteadOf}

\hierBdef{ComparisonRef}

\shortdef{The reference point in an explicit comparison (or contrast), i.e., 
an expression indicating that something is 
\textbf{similar/analogous to}, \textbf{different from}, or \textbf{the same as}
something else.}

The marker of the ``something else'' (the ground in the figure–ground relationship) 
is given the label \psst{ComparisonRef}:
\begin{exe}
  \ex \begin{xlist}
    \ex She is taller \p{than} me.
    \ex She is taller \p{than} I am.
    \ex She is taller \p{than} she is wide.
    \ex She is better at math \p{than} at drawing.
    \ex The shirt is more gray \p{than} black.
    %\ex She is greater in height \p{than} me.
  \end{xlist}
  \ex \begin{xlist}
    \ex She is as tall \p{as} I am.
    \ex Your face is as$_{\text{\psst{Characteristic}}}$ red \p{as} a rose.
    \ex Your face is red \p{as} a rose.
    \ex Your surname is the\_same \p{as} mine.
  \end{xlist}
  \ex Harry had never met anyone quite \p{like} Luna.
  \ex It was \choices{\p{as\_if}\\\p{like}} he had insulted my mother.
\end{exe}

The comparison is often made with respect to some dimension or attribute, the \psst{Characteristic}, 
which may or may not be scalar. 
The comparison may be figurative, employing simile, hyperbole, or spatial metaphor 
(\pex{close to} in the sense of `similar to'). 
The \psst{ComparisonRef} may even be a desirable or hypothetical/irrealis 
event or state (\pex{It was \p{as} it should have been}).

Prototypical prepositions include \p{than}, \p{as} (including the second item 
in the \p{as}--\p{as} construction), \p{like}, \p{unlike}. 
Prominent construals are \p{to} (\psst{Goal} for similar-thing) 
and \p{from} (\psst{Source} for dissimilar-thing).

\hierBdef{RateUnit}

\hierBdef{Quantity}

\hierCdef{Approximator}

\hierBdef{SocialRel}

\hierCdef{OrgRole}

\bibliographystyle{plainnat}
\bibliography{psst2.bib}


%\printbibliography[maxnames=99]


\end{document}
